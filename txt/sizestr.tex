	\chapter{Размерная структура поселений {\it Macoma~balthica}}

		\section{Белое море}

	\subsection*{Эстуарий реки Лувеньги}
На данном участке размерную структуру поселения маком в среднем горизонте литорали (СГЛ) отслеживали на протяжении $20$~лет ($1992 - 2012$).
За все время наблюдения максимальная длина особи, отмеченная в поселении составляла $18$~мм.

Характер размерно-частотного распределения особей неоднократно менялся на протяжении периода наблюдений (приложение~\ref{app:White_sizestr_hist}, рис.~\ref{ris:size_str_estuary_Luv}).
С $1993$ до $1997$ года в размерной структуре поселения выделялось три модальных класса, причем за все $5$~лет один из них попадал на особей до $4$~мм , второй на $7 - 9$ мм и третий --- это особи длиной более $10$~мм. 
В $1998$ году размерная структура поселения стала мономодальной, так как практически не осталось крупных особей, но
появилось много моллюсков длиной $1-2$~мм. 
В дальнейшем до $2002$ года оставалось мономодальное распределение особей по размерам, и происходило смещение модального класса --- в $2002$ году это были особи размером $6-7$ мм. 

В $2003$ году можно было выделить два пика: моллюски длиной $1-2$ мм и $7-9$~мм, то есть размерная структура поселения вновь стала бимодальной. 
В дальнейшем до $2012$~года размерная структура маком в данном поселении остается бимодальной. 
Первый модальный класс сохраняется --- особи длиной $1-2$~мм, а второй модальный класс варьирует, его составляют в разные годы особи длиной от $9$ до $12$~мм.
Количественное соотношение особей двух модальных классов менялось. Чаще ($2004$, $2007 - 2010$~ года) преобладали мелкие моллюски, но в отдельные годы ($2006$, $2012$) доля крупных была выше, либо представительство крупных и мелких доминирующих классов было сравнимым ($2005$, $2011$~годы).



	\subsection*{Остров Горелый}

На данном участке размерную структуру поселения маком отслеживали на протяжении $20$~лет ($1992 - 2012$) в пределах трех горизонтов литорали и у нуля глубин.
За все время наблюдения максимальная длина особи, отмеченная в поселении составляла $20$~мм.

В верхнем горизонте литорали (ВГЛ) размерная структура поселения до $1997$ года (приложение~\ref{app:White_sizestr_hist}, рис.~\ref{ris:size_str_Goreliy_high}) представляла собой бимодальное распределение с модальными классами $2-5$~мм и $7-13$~мм.
В $1998$ году появилось значительное количество особей длиной $1-4$~мм. 
В дальнейшем можно было наблюдать смещение по оси размеров данного модального класса. 
В $2001$ году в поселении вновь сформировалась бимодальная размерная структура (модальные классы $1-3$ и $5-6$ мм), и в дальнейшем такое распределение сохранялось до $2007$ года.
В $2008 - 2009$ годах распределение было мономодальное с модальным классом $1-2$~мм.
Интересно отметить, что с $2002$ по $2009$ год доминирующим размерным классом в поселении были особи длиной $1-2$~мм.
В $2011-2012$ году восстановилась бимодальная размерная структура с модальными классами $1-4$ и $9-11$~мм.

%\bigskip
В среднем горизонте литорали (СГЛ) до $1996$ года в этой зоне выделялась бимодальная размерная структура (приложение~\ref{app:White_sizestr_hist}, рис.~\ref{ris:size_str_Goreliy_mid}) (модальные классы --- моллюски длиной $1-4$~мм и $6-13$~мм). 
В $1997$ году распределение было практически равномерное при общей низкой численности. 
В $1998$ году появилось значительное количество моллюсков длиной до $1$ мм. 
После чего наблюдалось смещение модального класса до $2003$ года. 
До $2001$ года размерная структура поселения оставалась мономодальной, но в $2002-03$ годах появился еще один модальный класс -- моллюски длиной до $2$ мм. 
Таким образом, после $2002$ года  в поселении вновь восстановилась бимодальная размерная структура, которая сохраняется вплоть до $2007$ года.
В $2008$ году распределение особей по размерам становится мономодальным за счет элиминирования особей крупных размеров. 
В $2011-2012$ году восстанавливается бимодальное распределение.


%\bigskip
В нижнем горизонте литорали (НГЛ) в $1992$ году в связи с малой численностью моллюсков сложно говорить о характерной размерной структуре поселения (приложение~\ref{app:White_sizestr_hist}, рис.~\ref{ris:size_str_Goreliy_midlow}).
В $1993$ году фактически можно выделить только один пик ($2-3$~мм), хотя и было очень незначительное повышение при длине $9-10$~мм. 
Но с $1994$ по $1996$ год было представлено бимодальное распределение с модальными классами $1-3$ мм и $9-11$~мм.
В $1997$ году численность моллюсков значительно снизилась, и распределение по размерам было практически равномерное. 
В $1998-1999$ году в значительных количествах появились особи длиной $2-3$~мм и можно было наблюдать смещение модального класса по оси размеров вплоть до $2003$ года, когда его значение становится $5-6$~мм. 
Кроме того, с $2002$ года можно было выделить еще один модальный класс -- особи длиной $1-2$~мм, то есть размерная структура поселения вновь стала бимодальной, каковой и оставалась до конца периода наблюдений.


%\bigskip
У нуля глубин в $1992$ году моллюсков практически не было (приложение~\ref{app:White_sizestr_hist}, рис.~\ref{ris:size_str_Goreliy_low}), но в $1993$ году можно говорить о бимодальной размерной структуре поселения, которая сохранялась до $1997$ года. 
В $1998-1999$ году произошло элиминирование крупных особей на фоне появления значительного количества особей длиной $1-2$~мм. 
В $2001-2003$ годах в поселении восстановилась бимодальная структура и в $2003$ году модальные классы образовывали особи длиной до $1$ мм и $8.1-9.0$ мм. 
С $2003$ до $2007$ года преобладали особи длиной $9-12$~мм, а с $2008$ появляется второй модальный пик --- особи размером $1-3$~мм.


		\subsection*{Материковая литораль в районе поселка Лувеньга}
На данном участке размерную структуру поселения маком отслеживали на протяжении $10$~лет ($1992 - 2004$) в пределах четырех биотопов.
За все время наблюдения максимальная длина особи, отмеченная в поселении составляла $24$~мм.

В зоне верхнего пляжа размерная структура поселения (приложение~\ref{app:White_sizestr_hist}, рис.~\ref{ris:size_str_2razrez_high}) в $1993$ году была мономодальная, но с $1994$ по $1997$ годы стала бимодальной с модальными классами $2-5$ и $6-10$~мм.
В $1998$ году появилось значительное число особей размером менее $1$ мм, после чего до $2002$ года прослеживалось смещение модального класса. 
В $2002$ году в поселении восстановилась бимодальная структура (модальные классы -- $1-2$ мм и $5-6$ мм).

%\bigskip
В поясе фукоидов размерная структура поселения (приложение~\ref{app:White_sizestr_hist}, рис.~\ref{ris:size_str_2razrez_fucus}) в $1992-1997$ году характеризовалась наличием двух модальных классов: $1-6$ и $7-12$~мм. 
С $1998$ по $2000$ года размерная структура поселения была мономодальной, причем все $3$ года пик формировали особи длиной $1-2$~мм. 
В $2002$ году вновь выделялось два модальных класса: $1-2$ и $7-8$~мм.

%\bigskip
В поясе зостеры до $1998$ года в размерной структуре поселения пояса зостеры выделялись незначительные пики и можно говорить о равномерном распределении моллюсков(приложение~\ref{app:White_sizestr_hist}, рис.~\ref{ris:size_str_2razrez_zostera}). 
После $1998$ года она стала мономодальной, причем пик формировали моллюски длиной $1-2$ мм.


%\bigskip
В зоне нижнего пляжа до $1999$ года размерная структура поселения была полимодальная, хотя эти пики нельзя было четко выделить (приложение~\ref{app:White_sizestr_hist}, рис.~\ref{ris:size_str_2razrez_low}). 
В $1999-2000$ годах практически не осталось крупных особей, но появилось значительное число моллюсков размером $1-2$~мм. 

		\subsection*{Южная губа о.~Ряшкова}

На данном участке наблюдения проводили с $2001$ года, размерную структуру поселения у нуля глубин отслеживали в течение $12$ лет.
Максимальный размер маком в данном поселении составил $23$~мм в $2003$ году, однако в другие годы максимальный размер не превышал $16$~мм.

В Южной губе на протяжении всего периода наблюдений размерная структура (приложение~\ref{app:White_sizestr_hist}, рис.~\ref{ris:size_str_YuG}) поселения была мономодальной с преобладанием особей длиной $1-3$~мм. 



		\subsection*{Западная Ряшкова салма}

На литорали о.~Ряшкова в Западной Ряшковой салме наблюдения проводили с $1994$ по $2012$ год ($18$~лет). Наблюдения проводили в среднем горизонте литорали.
Максимальный размер моллюсков, отмеченный в поселении составил $20$~мм.

На данном участке  до $1998$ года размерная структура была полимодальной(приложение~\ref{app:White_sizestr_hist}, рис.~\ref{ris:size_str_ZRS}). 
В $1999$ году крупные особи в основном элиминировали, и размерная структура стала мономодальной с доминированием моллюсков длиной $1-2$~мм.
В дальнейшем $2001$ года до конца наблюдений размерная структура была бимодальной с модальными классами $1-3$ и $9-11$~мм.


		\subsection*{о.~Ломнишный}

На литорали острова Ломнишный наблюдения проводили с $2007$ года в течение $6$~лет у нуля глубин. 
Максимальный размер особи, отмеченный в поселении составлял $17$~мм.

Размерная структура на данном участке в течение всего периода наблюдений была мономодальной (приложение~\ref{app:White_sizestr_hist}, рис.~\ref{ris:size_str_Lomnishniy}).
В основном доминировали особи длиной $1-3$~мм, за исключением $2009-2010$ годов, когда доминировали особи длиной $5$ и $7$~мм, соответственно.


\bigskip
Таким образом, наиболее распространенный вариант динамики размерной структуры в поселениях {\it M.~balthica} в Белом море это чередование бимодальной и мономодальной размерных структур.
Мономодальная структура обычно формируется на фоне практически полной элиминации крупных особей при пополнении поселения новой генерацией маком.
В дальнейшем, если новое пополнение происходит быстрее, чем предыдущая генерация элиминирует, то формируется бимодальная размерная структура.

Среди 6 мониторинговых участков в Кандалакшском заливе Белого моря для двух из них --- в Южной губе острова Ряшков и на о. Ломнишный --- динамика размерной структуры принципиально отличалась, и мы ежегодно видим мономодальное распределение особей по размерам с доминированием мелких особей.

\afterpage{\clearpage}

		\section{Баренцево море}

		\subsection{Губы Кольского залива}

На участке Абрам-мыс (рис.~\ref{ris:Barents_sizestr}) были представлены особи длиной от $2$ до $16$~мм. 
В среднем горизонте литорали характер распределения был мономодальный с преобладанием моллюсков длиной $10-13$~мм. 
В нижнем горизонте литорали к аналогичному пику (особи длиной $12-14$~мм) добавляется второй — моллюски длиной $2-3$~мм.

На участке в Пала-губе (рис.~\ref{ris:Barents_sizestr}) также в среднем горизонте распределение особей по размерам было мономодальным, а на нижнем --- бимодальным. 
Однако при этом наблюдалась обратная ситуация: в среднем горизонте литорали доминировали особи самой мелкой размерной группы --- $3-4$~мм, в то время как  в нижнем горизонте кроме таких особей  хорошо представлена размерная группа $10-12$~мм.

		\subsection{Губы побережья Восточного Мурмана}

В губе Гаврилово (прил.~\ref{app:Barents_sizestr_hist}, рис.~\ref{ris:Barents_sizestr}) распределение особей практически равномерное. 
В среднем горизонте литорали несколько преобладают особи длиной $15-20$~мм. 
В нижнем горизонте литорали представлены лишь единичные особи различных возрастов.	

Во всех горизонтах губы Ярнышной (прил.~\ref{app:Barents_sizestr_hist}, рис.~\ref{ris:Barents_sizestr}) доминировали особи длиной $4-6$~мм. 
На всех участках можно отметить присутствие относительно крупных моллюсков (особи длиной более $14$~мм), однако их представленность на порядок варьирует в разных горизонтах.

В губе Шельпино (прил.~\ref{app:Barents_sizestr_hist}, рис.~\ref{ris:Barents_sizestr})  представлены единичные особи длиной от $6$ до $16$~мм. 
В среднем горизонте литорали некоторое превышение формируют особи длиной $15$~мм, однако и они остаются немногочисленны.

В губе Порчниха (прил.~\ref{app:Barents_sizestr_hist}, рис.~\ref{ris:Barents_sizestr}) были представлены особи длиной от $4$ до $21$~мм. 
Распределение особей по размерам было полимодальным. 
Выделяется по крайней мере три моды: $4-7$~мм, $9-12$~мм и $18-20$~мм. 
Несущественное превышение численности  отмечено для особей длиной $13-15$~мм.

В губе Ивановская (прил.~\ref{app:Barents_sizestr_hist}, рис.~\ref{ris:Barents_sizestr}) были обнаружены макомы длиной от $2$ до $13$~мм. 
Количество особей в каждой размерной группе колебалось от $20$ до $30$ экземпляров, лишь моллюсков длиной $2$~мм было отмечено около $50$. 
Распределение особей по размерам было практически равномерным при некотором превышении доли особей длиной $2$ и $10$~мм. 

		\subsection{Дальний пляж губы Дальне-Зеленецкой (Восточный Мурман)}
На данном участке ни в один год в пробах не было отмечено особей {\it M.~balthica} с длиной раковины менее $2$~мм (прил.~\ref{app:Barents_sizestr_hist}, рис.~\ref{ris:size_str_DZ}). 
Максимальный размер моллюсков в разные годы колебался от $18$ до $20$~мм. 

С $2002$ до $2004$ года размерная структура маком в данном поселении была полимодальной. 
Можно говорить по крайней мере о трех модальных группах.
Доминировали все эти годы особи размером $8 - 14$~мм.
В $2005$ году размерная структура фактически мономодальная с преобладанием крупных особей длиной больше $12$~мм, и встречаются единичные моллюски размером $3 - 4$~мм.
В $2006$ году добавляется вторая модальная группа~--- особи длиной $3 - 5$~мм.
После $2007$ года восстанавливается полимодальное распределение особей по размерам.


\bigskip

Таким образом, на исследованных участках был представлены все возможные варианты размерной структуры: мономодальное (участки: Абрам-мыс СГЛ, Пала-губа СГЛ, губа Гаврилово СГЛ), бимодальное (участки: Абрам-мыс НГЛ, Пала-губа НГЛ, губа Ярнышная, губа Дальне-Зеленецкая СГЛ, губа Порчниха СГЛ) и практически равномерное (участки: губа Гаврилово НГЛ, губа Дальне-Зеленецкая ВГЛ и НГЛ, губа Шельпино ВГЛ и СГЛ, губа Ивановская ВСЛ) распределение особей по размерам. 

Мономодальное распределение особей по размерам наблюдается либо при доминировании мелких особей длиной $3-5$~мм, либо при доминировании крупных --- $12-18$~мм.
При бимодальном распределении обычно первую моду формировали мелкие макомы длиной $2-5$~мм, а вторую --- моллюски длиной более $10$~мм.

\afterpage{\clearpage}
