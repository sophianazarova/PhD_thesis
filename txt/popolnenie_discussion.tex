Для видов с планктонной личинкой пополнение молодью является основным фактором, влияющим на динамику поселений.
Поэтому мы рассматривали количественные характеристики фомирования спата и обилие годовалых особей, как отражения разичных стадий процесса пополнения.

%обсуждение
По нашим данным численность спата была одного порядка на трех исследованных участках ($4-5$~тыс.~экз./м$^2$), но в проливе Подпахта была выше на порядок (более $10$~тыс.~экз./м$^2$). 
Интересно отметить, что высокая плотность спата была отмечена именно в Подпахте, т.е. на участке с минимальной численностью взрослых особей. 

Учитывая имеющиеся оценки смертности спата нам показалось интересным посмотреть соотношение спата этого года и сеголетков. 
Сеголетками (возраст $1+$) считали особей длиной $1,1-2,0$~мм в соответствии с работой Н.В.Максимовича с соавторами, выполненной в исследованной акватории (\cite{Maximovich_et_al_1992}). 
Смертность спата маком за сезон оценивается в $59.1$\% (\cite{Burkovskiy_et_al_1998}). Полученные расчетные величины представлены в табл.~\ref{tab:spat_rasschet} и \ref{tab:N1_rasschet}.

\begin{table}[p]
\caption{Предположительное пополнение исследованных поселений \textit{Macoma balthica} в $2005$ году, рассчитанные на основе оценки смертности спата.}
\label{tab:spat_rasschet}
\begin{center}
\begin{tabular}{|l|c|c|}
\hline
                & $2005$ (расчет) & $2006$ \\ 
возраст         & $N_{sp}$      & $N_{1+}$  \\ \hline
Сухая салма     & $473$           & $194$  \\ \hline
бухта Лисья     & $415$           & $170$  \\ \hline
пролив Подпахта & $166$           & $68$   \\ \hline
бухта Клющиха   & $351$           & $144$  \\ \hline
\end{tabular}
\end{center}
	\footnotesize{Примечание: $N_{sp}$.~--- численность спата, экз./м$^2$, $N_{1+}$~--- численность сеголетков, экз./м$^2$.}

\end{table}


\begin{table}[p]
\caption{Предположительная эффективность пополнения исследованных поселений Macoma balthica к 2007 году, рассчитанные на основе оценки смертности спата.}
\label{tab:N1_rasschet}
\begin{center}
\begin{tabular}{|l|c|c|}
\hline
                & $2006$  & $2007$ (расчет) \\
возраст         & $N_{sp}$      & $N_{1+}$  \\ \hline
Сухая салма     & $4980$  & $2042$          \\ \hline
бухта Лисья     & $4040$  & $1656$          \\ \hline
пролив Подпахта & $4240$  & $1738$          \\ \hline
бухта Клющиха   & $10060$ & $4125$         \\ \hline
\end{tabular}
\end{center}
	\footnotesize{Примечание: $N_{sp}$.~--- численность спата, экз./м$^2$, $N_{1+}$~--- численность сеголетков, экз./м$^2$.}
\end{table}

Таким образом, расчетные величины обилия спата в $2005$ году на порядок отличаются от величин, показанных для $2006$ года. 
Возможно, это связано с значительными межгодовыми различиями в пополнении поселений, показанных для других участков (рис.~\ref{ris:dynamic_1year_Kandalaksha}). 
Также это может быть связано с тем, что приведенная оценка сделана для смертности за сезон, и смертность за последующую зиму может значительно занижать нашу оценку пополнения в $2005$ году. 

Размерная структура спата на всех исследованных участках характеризуется мономодальностью. 
Подобные данные были получены М.А. Зубахой с соавторами (2000), однако в данной работе было показано, что мономодальное распределение спата формируется в конце лета. 
Изначально при оседании формируется бимодальная размерная структура, связанная с двумя пиками численности личинок в планктоне, и затем за счет различной скорости роста личинок два пика постепенно сливаются (\cite{Zubakha_et_al_2000}).
На исследованных участках максимальный размер плантиграды имели на участке в бухте Клющиха ($0,4 - 1,5$~мм с модой $0,75$~мм), а минимальный в проливе Подпахта ($0,35 - 0,8$~мм с модой $0,5$~мм). 
Это хорошо согласуется с данными Е. Олафссона, который показал, что на песчаных грунтах нет подавления роста спата взрослыми особями, наблюдаемого на илисто-песчаном грунте (\cite{Olafsson_1989}). 
Участок в бухте Клющиха отличался отсутствием алевритов и пелитов (табл.~\ref{tab:grunt_granulometriya_White}), в то время как остальные характеризовались значительным заилением. 
В $1998$ году на участке Сухая салма к $25$~августа моду формировали особи длиной $0,55 - 0,75$~мм с небольшим преобладанием группы $0,65$~мм (\cite{Zubakha_et_al_2000}). 
По нашим данным к $20$~августа структура поселения была с выраженным пиком при длине спата $0,65$~мкм. 
Разброс размеров в $1998$ году был от $0,35$ до $0,95$~мм, а в $2006$ от $0,3$ до $0,85$~мм, то есть в $2006$ году особи были более мелкие, не смотря на более поздние сроки сбора материала. 

При анализе корреляции количества спата и обилием взрослых особей \textit{M.~balthica} было показано, что с биомассой достоверной корреляции нет, а есть отрицательная с численностью. 
Между тем, если предполагать трофическую или топическую конкуренцию, то следовало бы ожидать наличия именно отрицательной корреляции с биомассой, поскольку более крупные макомы имеют больший радиус облова и отбирают на себя больший поток энергии (\cite{Olafsson_1989, Zwarts_et_al_1994}). 
Тогда возникло предположение, что такая картина может объясняться взаимодействием спата с более мелкими, но более многочисленными макомами. 
Однако при анализе корреляции численности спата и количества маком различных размеров это показать не удалось.

Дисперсионный анализ показал, что численность спата сильно варьирует в зависимости от участка, и фактор участок определяет $45 \pm 6,8$\% вариации. 
Это может быть связано с сильной вариабельностью численности личинок в планктоне на различных участках (\cite{Maximovich_Shilin_2012}). 
Кроме того, поскольку в данном исследовании не проводилось наблюдение за динамикой оседания спата, то наблюдаемая картина является результирующей оседания и последующего перераспределения маком за счет биссусного дрифта (\cite{Armonies_Hellwig-Armonies_1992, Huxham_Richards_2003}).
Хотя для фактора численность взрослых маком сила влияния недостоверна, но поскольку анализ силы влияния фактора более слабый, чем дисперсионный анализ, то можно говорить о влиянии численности взрослых маком на количество спата. Но оценить влияние на имеющемся материале невозможно.

Интересные результаты получились при анализе влияния местообитания и численности взрослых маком на отдельные размерные группы маком. 
Для особей длиной более $5$~мм характер влияния факторов аналогичен таковому у спата, в то время как для особей длиной $1,1 - 5,0$~мм влияние фактора <<участок>> недостоверно, а влияние численности взрослых маком больше. 
Можно было бы предположить, что именно количество особей длиной $1,1 - 5,0$~мм определяет численность маком в поселении, однако корреляция между этими параметрами оказалась недостоверной.

Попытка выявить влияние на численность спата маком обилия макрозообентоса, что было показано в Ваттовом море (\cite{Flatch_2003}), не показала достоверной связи между данными показателями. 
Возможно, это связано с тем, что в Ваттовом море в исследованных поселениях обилие определялось количеством двустворчатого моллюска \textit{Cerastoderma edule} и пескожила \textit{Arenicola marina}, в то время как в исследованных нами поселениях не было столь крупных и активно изменяющих среду организмов. 

\par\bigskip
В данной работе мы оценивали успешность пополнения беломорских поселений \textit{M.~balthica} по численности годовалых особей.
Данный показатель варьировал в значительных пределах: от $0$ до $5,5$~тыс.~экз./м$^2$.
Таким образом, исследованные поселения демонстрируют характерную для Белого моря нерегулярность пополнения (\cite{Maximovich_et_al_1991, Maximovich_Gerasimova_2004, Gerasimova_Maximovich_2009}).

Считается, что пополнение локальных поселений массовых бентосных организмов с планктонной личинкой не зависит от количества половозрелых особей в нем, поскольку единый личиночный пул в планктоне формируется за счет всех половозрелых особей в гидрологически-замкнутой акватории (\cite{Maximovich_Shilin_2012}.
Мы попробовали на имеющихся материалах проверить данную гипотезу.
Поскольку для маком в Белом море показано (\cite{Semenova_1980, Maximovich_1985}), что ключевым фактором для возможности половозрелости является именно размер, а не возраст животного, и этот размер для макомы составляет 8 мм, мы оценивали корреляцию численности годовалых особей в поселении с численностью особей длиной более $8$~мм в предыдущий год (т.е.в год оседания).
Хотя была обнаружена низкая достоверная корреляция данных параметров, очевидно (рис.~\ref{ris:N1year_vs_N8mm}) что разброс данных величин достаточно высокий и влияние данного вактора невелико.
В пользу гипотезы о формировании общего личиночного пула на значительной акватории говорит и синхронность пополнения, наблюдаемая в ряде исследованных поселений (табл.~\ref{tab:Mantel_dynamic_N1y}).
Единственное поселение, для которого не показана синхроность с остальными --- на острове Ломнишном, наиболее удаленном поселении от остальных участков.
Однако прямой связи расстояния со степенью синхронности пополнения поселений обнаружено не было.

В Баренцевом море мы не проводили анализа пополнения, однако по данным размерной структуры можно сделать некоторые выводы.
На Дальнем пляже особи размером $2-3$~мм встречаются ежегодно, хотя бы в единичном количестве.
В данном районе такой размер характерен для маком возрастом $1+$ (\cite{Nazarova_et_al_2010}), таким образом, можно говорить о регулярном пополнении поселений молодью. 
Однако эффективность пополнения различается год от года. 
Наиболее успешные пополнения поселения молодью, по-видимому, происходили в $2005-2007$ годах, что и обусловило увеличение численности маком в $2006-2008$ годах на данном участке.
Таким образом, значительные межгодовые различия в эффективности пополнения поселений маком молодью характерны как для Белого, так и для Баренцева морей.

