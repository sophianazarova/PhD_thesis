%обзор литературы из ВКР
Большинство исследований, посвященных проблеме пополнения, выполнено в одном из районов Северного моря – так называемом море Ваддена. Было показано, что одним из ключевых факторов, влияющих на пополнение поселений Macoma balthica, является температура в зимний период. Пополнение после суровых зим было больше, чем после мягких. Было предложено два механизма (рис. 1), зависимые от зимней температуры: 1. количество яиц маком, выметанных в апреле больше после холодной зимы, поскольку при низкой температуре меньше уровень обмена, а, значит, меньше потери веса за зиму, и больше энергии остается на продукцию. 2. Биомасса Crangon crangon, одного из важных хищников для маком, была значительно выше после холодных зим. При проверке обеих гипотез, было показано, что второй механизм влияет значительно сильнее (Beukema et al., 1998). 	
В ряде других работ было показано влияние различных хищников на численность и распределение молоди маком. Так, для моря Ваддена именно обилием хищников объясняется формирование временных скоплений молоди маком на верхней литорали. При изучении факторов, обуславливающих такое распределение для M. balthica было показано, что обилие бентосных хищников больше на нижней литорали, и лишь молодь краба Carcinus maenas в значительных количествах встречается на верхней литорали. В полевых и лабораторных экспериментах было показано, что присутствие хищников значительно снижает численность спата, в то время как влияния на крупных особей обнаружено не было. По-видимому, за первый год макомы выходят из-под контроля бентосными хищниками за счет увеличения размеров теля (Hiddink et al., 2002). 
 Также при анализе динамики личинок различных беспозвоночных в планктоне было показано, что после суровых зим численность личинок краба Carcinus maenas значительно снижалась, и они появлялись в планктоне на 6-8 недель позже, чем после мягких зим. По-видимому, с этим временным несоответствием связано большее пополнение поселений маком после суровых зим (Strasser, Günther, 2001).
В более поздних исследованиях на море Ваддена было показано, что влияние суровых зим на пополнение Macoma balthica не столь широкомасштабно, как предполагалось ранее, и, по-видимому, существуют другие факторы, определяющие более локальные вариации в пополнении поселений (Strasser et al., 2003; Flatch, 2003).
Пресс хищников не объяснил эти различия, изменения сообществ и поступления биогенных элементов не объяснили картину, поскольку действовали лишь на отдельных участках. Наиболее вероятным фактором, по мнению авторов, является топографическая разница между двумя акваториями, где располагались исследованные участки. 
Предполагается, что в зависимости от закрытости акватории островами, и преобладающего направления ветров, будет идти более или менее эффективный перенос личинок и биссусный дрифт, а, значит, и пополнение поселения (Strasser et al., 2003).
Для другого участка моря Ваддена было показано, что комбинация эффектов высокого пресса хищников вместе с высоким обилием взрослой макрофауны обуславливает 95 процентное снижение количества спата теллинид (Macoma balthica и Tellina tenuis) после мягких зим (Flatch, 2003). 
Хотя влияние на пополнение поселения молодью плотности взрослых особей того же вида представляется достаточно логичным механизмом, существуют лишь отдельные работы, посвященные внутривидовым взаимодействиям у Macoma balthica. Так, в ряде работ показано, что плотность молоди не зависит от обилия взрослых маком (Olafsson, 1989; Vincent et al., 1989, Beukema et al., 2001; Richards et al., 2002). 
 Также было показано, что влияние плотности взрослых маком на рост спата зависит от типа грунта. На илисто-песчаном грунте, где и взрослые, и молодые моллюски питаются как собирающие детритофаги, рост спата подавляется при увеличении плотности взрослых особей. На песке, где молодь питаются как собирающие детритофаги, а взрослые – как фильтраторы, влияния на рост спата показано не было (Olafsson, 1989). 
Для Белого моря существуют лишь несколько работ, посвященных отдельным аспектам пополнения поселений маком. Так, И.В.Бурковским с соавторами показано, что макомы оседают вне плотных поселений взрослых (Бурковский и др., 1996 и 1998). Также показано, что важную роль в динамике численности личинок и спата влияет принос личинок с соседних акваторий. В течение лета формируется сначала бимодальная размерная структура спата, с двумя пиками личинок в планктоне, которая к концу августа сливается в мономодальную (Зубаха и др., 2000). 
Показана высокая смертность особей на всех этапах пополнения поселения. Так, смертность пелагических личинок оценивают в 36,4\% за сезон, а смертность спата – 59\% за сезон (Бурковский и др., 1998).

%обсуждение
Численность спата была одного порядка на трех участках (4-5 тыс. экз./кв.м), но в проливе Подпахта была выше на порядок (более 10 тыс. экз./кв.м). Интересно отметить, что высокая плотность спата была отмечена именно в Подпахте, т.е. на участке с минимальной численностью взрослых особей. 
Учитывая имеющиеся оценки смертности спата нам показалось интересным посмотреть соотношение спата этого года и сеголетков. Сеголетками (возраст 1+) считали особей длиной 1.1-2.0 мм в соответствии с работой Н.В.Максимовича с соавторами, выполненной в исследованной акватории (Максимович и др., 1992). Смертность спата маком за сезон оценивается в 59.1% (Бурковский и др.,1998). Полученные расчетные величины представлены в табл. 8 и 9.

Таблица 8. Предположительное пополнение исследованных поселений Macoma balthica в 2005 году, рассчитанные на основе оценки смертности спата.

2005 (расчет)
2006
возраст
Nсп.
N1+
Сухая салма
473
194
бухта Лисья
415
170
пролив Подпахта
166
68
бухта Клющиха
351
144
-------------------------------------------------------
Примечание: Nсп. – численность спата, экз./м2, N1+ – численность сеголетков, экз./м2.

Таблица 9. Предположительная эффективность пополнения исследованных поселений Macoma balthica к 2007 году, рассчитанные на основе оценки смертности спата.

2006
2007 (расчет)
Возраст
Nсп.
N1+
Сухая салма
4980
2042
бухта Лисья
4040
1656
пролив Подпахта
4240
1738
бухта Клющиха
10060
4125
-------------------------------------------------------
Примечание: обозначения как в табл. 8.

Таким образом, расчетные величины обилия спата в 2005 году на порядок отличаются от величин, показанных для 2006 года. Возможно, это связано с значительными межгодовыми различиями в пополнении поселений, показанных для других участков (Назарова, 2003). Также это может быть связано с тем, что приведенная оценка сделана для смертности за сезон, и смертность за последующую зиму может значительно занижать нашу оценку пополнения в 2005 году. Для определения более вероятного объяснения и определения корректности приведенной оценки смертности необходимы дополнительные наблюдения для сопоставления численности сеголетков в 2007 году с расчетной.
Размерная структура спата на всех исследованных участках характеризуется мономодальностью. Подобные данные были получены М.А. Зубахой с соавторами, однако в данной работе (Зубаха и др., 1998) было показано, что мономодальное распределение спата формируется в конце лета. Изначально при оседании формируется бимодальная размерная структура, связанная с двумя пиками численности личинок в планктоне, и затем за счет различной скорости роста личинок два пика постепенно сливаются (Зубаха и др., 1998).
На исследованных участках максимальный размер плантиграды имели на участке в бухте Клющиха (0,4 – 1,5 мм с модой 0,75 мм), а минимальный в проливе Подпахта (0,35 – 0,8 мм с модой 0,5 мм). Это хорошо согласуется с данными Е. Олафссона, который показал, что на песчаных грунтах нет подавления роста спата взрослыми особями, наблюдаемого на илисто-песчаном грунте (Olafsson, 1989). В нашем случае, хотя гранулометрического анализа грунтов не проводили, но визуально и при промывке проб  участок в бухте Клющиха отличался преобладанием песчаной фракции, в то время как остальные характеризовались значительным заилением. 
В 1998 году на участке Сухая салма к 25 августа моду формировали особи длиной 0,55 – 0,75 мм с небольшим преобладанием группы 0,65 мм (Зубаха и др., 1998). По нашим данным к 20 августа структура поселения была с выраженным пиком при длине спата 0,65 мкм. Разброс размеров в 1998 году был от 0,35 до 0,95 мм (Зубаха и др., 1998), а в 2006 от 0,3 до 0,85 мм, то есть в 2006 году особи были более мелкие, не смотря на более поздние сроки сбора материала. 
При анализе корреляции количества спата и обилием взрослых особей M. balthica было показано, что с биомассой достоверной корреляции нет, а есть отрицательная с численностью. Между тем, если предполагать трофическую или топическую конкуренцию, то следовало бы ожидать наличия именно отрицательной корреляции с биомассой, поскольку более крупные макомы имеют больший радиус облова и отбирают на себя больший поток энергии (Olafsson, 1989; Zwartz, 1994; Буковская, 1995; Николаева, 1997). Тогда возникло предположение, что такая картина может объясняться взаимодействием спата с более мелкими, но более многочисленными макомами. Однако при анализе корреляции численности спата и количества маком различных размеров это показать не удалось.
Дисперсионный анализ показал, что численность спата сильно варьирует в зависимости от участка, и фактор участок определяет 45 + 6,8\% вариации. Это может быть связано с сильной вариабельностью численности личинок в планктоне на различных участках (Максимович, Шилин, 1990). Кроме того, поскольку в данном исследовании не проводилось наблюдение за динамикой оседания спата, то наблюдаемая картина является результирующей оседания и последующего перераспределения маком за счет биссусного дрифта (Armonies, Hellwig-Armonies, 1992; Huxham, Richards, 2003).
Хотя для фактора численность взрослых маком сила влияния недостоверна, но поскольку анализ силы влияния фактора более слабый, чем дисперсионный анализ, то можно говорить о влиянии численности взрослых маком на количество спата. Но оценить влияние на имеющемся материале невозможно.
Интересные результаты получились при анализе влияния местообитания и численности взрослых маком на отдельные размерные группы маком. Для особей длиной более 5 мм характер влияния факторов аналогичен таковому у спата, в то время как для особей длиной 1,1-5,0 мм влияние фактора «участок» недостоверно, а влияние численности взрослых маком больше. Можно было бы предположить, что именно количество особей длиной 1,1-5,0 мм определяет численность маком в поселении, однако корреляция между этими параметрами оказалась недостоверной.
Попытка выявить влияние на численность спата маком обилия макрозообентоса, что было показано в море Ваддена (Flatch, 2003) не показала достоверной связи между данными показателями. Возможно, это связано с тем, что в море Ваддена в исследованных поселениях обилие определялось количеством двустворчатого моллюска Cerastoderma edule и пескожила Arenicola marina, в то время как в исследованных нами поселениях не было столь крупных и активно изменяющих среду организмов. 
