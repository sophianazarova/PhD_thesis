% для компиляции в lualatex!!
%\documentclass[12pt, a4paper]{article}
\documentclass[12pt, a4paper]{disser}
\usepackage[english,russian]{babel}
\usepackage[warn]{mathtext}
%\usepackage[T2A]{fontenc}
%\usepackage[utf8]{inputenc}

\usepackage{xecyr} % Продукт Вашего покорного слуги ;)

%\setmainfont{DejaVu Serif}
\setmainfont{Liberation Serif}

\usepackage{color}
\usepackage{amssymb,amsmath}
\usepackage{graphicx}
\usepackage{multicol}

\textheight=24cm           % высота текста
\textwidth=16cm            % ширина текста
\oddsidemargin=0pt         % отступ от левого края
\topmargin=-1.5cm          % отступ от верхнего края
\parindent=24pt            % абзацный отступ
\parskip=0pt               % интервал между абзацами
\tolerance=2000            % терпимость к "жидким" строкам
\flushbottom               % выравнивание высоты страниц
%\def\baselinestretch{1.5} % печать с большим интервалом

%\title{}
%\author{\copyright~~С.А.~Назарова \thanks{e-mail:~sophia.nazarova@gmail.com}}
%\date{}


\begin{document}

    \section{Характеристика района исследования}
        \subsection{Географическое и физиономическое описание}
            \subsubsection{Баренцево море}

            \paragraph{Северное Нагорное}

            \paragraph{Абрам­мыс}
Участок  в районе  Абрам­мыса  находится в третьем колене Кольского залива, максимально  удаленном   от   моря. 
Абрам­мыс     ­   район   города   Мурманска,   расположенный   на противоположной стороне от основного городского массива, напротив порта. 
Исследованный участок   литорали   находился   в   $1,5$~км   к   выходу   из   залива   от   причала,   куда   приходит пассажирский катер. 
Ширина   литорали   на   данном   участке   составляет   $45$~м.   
Верхний   горизонт   литорали представлен  каменисто-галичной  россыпью. 
В среднем  горизонте литорали на поверхности илисто-песчаного   грунта   располагаются   валуны,   покрытые   фукоидами   ({\it Fucus  vesiculosus}), которые   формируют   практически   сплошной   покров   с   отдельными   «окнами»   грунта (проективное  покрытие фукоидов $90~\%$).  
При приближении  к нижнему горизонту литорали количество   валунов   уменьшается,   и   проективное   покрытие   фукоидов   составляет   здесь   не более $10~\%$.

    \paragraph{Ретинское}

    \paragraph{Пала-губа}
Пала-губа   представляет   собой   глубоко   вдающуюся   в   берег   губу   длинным   узким «горлом», за которым следует расширение, формирующее несколько губ второго порядка. 
В «горле» расположен остров Шалим, и, таким образом, губа соединяется с Кольским заливом узкими   проливами.   
В   основной   части   Пала­губы   расположено   несколько   более   мелких островков. 
Исследованный участок располагался в длинной узкой губе (бухта Дровяная), закрытой на выходе островом Зеленый.
В кут губы впадает крупный ручей, формирующийся на литорали во время отлива оформленное русло, положение которого за два года наблюдений не изменилось.
Ширина литорали на данном участке составляет  $130$~м. 
Верхний горизонт литорали представлен   каменисто­валунной   россыпью,   которая   на   границе   со   средним   горизонтом становится более разреженной, и покрыта зарослями фукоидов ({\it Fucus vesiculosus}). 
Средний и нижний   горизонты   представлены   двумя   илисто­песчаными   пляжами,   разделенными каменисто­валунной грядой на месте резкого локального увеличения угла уклона свала. 
На нижней литорали грунт более заилен, и на поверхности располагаются агрегации {\it Mylius edulis} («мидиевые щетки»).

    \paragraph{Печенга}

    \paragraph{Губа Гаврилово}
Гаврилово – наиболее западная губа из исследованных нами участков на Восточном Мурмане. 
Эта   губа   с  достаточно   широким   входом,   свободно   открывающаяся  в  Баренцево море.   
Восточную   ее   часть   несколько   закрывает   от   прибоя   мыс,   формирующий   «горло», несколько суженное относительно основной части. 
В восточной части кута губа формирует узкий отрог длиной   около $200$~м, по которому течет ручей, распадающийся в центральной части   губы  в  среднем   горизонте   литорали   на   два   рукава,   и   сливающиеся   ниже   обратно   в единое русло.
Ширина литорали  в  данной губе составляет  $500$~м  (без  учета отрога, дно которого полностью обнажается в отлив) Верхний горизонт литорали на данном участке представлен каменисто-галечной   россыпью.   
Средний   горизонт   литорали   представляет   собой   обширную илисто-песчаную   отмель   с   отдельными   камнями   и   валунами.  
В   основном   камни   и   валуны сконцентрированы   вдоль   русла   ручья.   
Нижний   горизонт   литорали   представлен   песчаным пляжем. 

    \paragraph{Губа Ярнышная}
Губа Ярнышная представляет собой одну из крупнейших губ Восточного Мурмана, ее длина составляет около $5$~км. 
Вход в губу свободно открыт в Баренцево море. 
Берега губы сильно изрезаны. 
В кут губы Ярнышной впадает два крупных ручья \textemdash\ Ярнышный и Бобровый. 
По мере продвижения в кут губы, скальная и каменистая литораль переходит в каменисто-песчаную и илисто-песчаную. 
Исследованный участок расположен в юго­восточной части кута губы в районе впадения ручья Ярнышный.
На   участке   исследования   средний   горизонт   литорали   представлен   илисто-песчаным пляжем   с   отдельными   валунами,   поросшими   фукоидами   ({\it Fucus   vesiculosus}).   
В   среднем   и нижнем горизонте литорали вдоль русла ручья были остатки умершего плотного поселения  {\it Mytilus eduls} («мидиевая банка»), поэтому в период исследования в данном биотопе грунт был черный с запахом сероводорода.

    \paragraph{Губа Дальне-Зеленецкая}
Исследованный   участок   был   расположен   на   литоральной   отмели   Дальний   Пляж, поскольку именно он был в 1970х годах выбран как модель для описания литоральной фауны мягких   грунтов   на   Баренцевом   море.   
\textcolor{red}{Физико-географическое   описание   участка   по литературным данным представлено в главе «литературный обзор».}
На   границе   верхней   литорали   расположен   валунно­галечный   пляж,   нижняя   часть которого заросла фукоидами ({\it Fucus vesiculosus}). 
Ниже по литорали в юго­восточной части пляжа   тянется   узкая   (около   $10-15$~м   шириной)   полоса   крупного   песка,   в   которой представители макробентоса практически отсутствуют.
Средний   горизонт   литорали    ­   это   обширный   илисто-песчаный   пляж,   в  пределах которого визуально выделяется три зоны: с преобладанием пескожилов  {\it Arenicola marina}, с преобладанием   мелких   полихет­трубкостроителей   (в   первую   очередь, {\it Fabricia   sabella})   и переходная   зона   между   этими   сообществами.   
Нижняя   литораль   представлена   каменисто-песчаным пляжем с зарослями бурых ({\it Fucus vesiculosus}, {\it Fucus serratus}) и красных ({\it Palmaria  palmata}) водорослей на камнях.

    \paragraph{Губа Шельпино}
Шельпино   представляет   собой   большую   губу   с   широким   горлом,   в   котором расположен   один   крупный   и   несколько   мелких   островов.   
В   юго-восточной   части   губа продолжается   длинным   (около   $400$~м)   узким   отрогом,   полностью   обнажающимся   в   отлив. 
Именно в этом отроге и происходил пробоотбор. 
По   литорали   отрога   протекает   небольшой   ручей,   не   формирующий   четкого   русла. 
Летом вдоль ручья развиваются массовые скопления зеленой водоросли рода  {\it Enteromorpha}. 
Верхняя и средняя литораль представляют собой песчаный пляж с отдельными камнями и валунами. 
В среднем горизонте на камнях появляются водоросли. 
Нижний горизонт литорали оккупирован плотным поселением мидий {\it Mytilus edulis} на грунте.

    \paragraph{Губа Порчниха}
Порчниха  \textemdash\  крупная   губа,   закрытая   от   моря   островом   Большой   Олений.   
Кутовая часть разделена скальным мысом на две части. 
Одна из них направлена на юг, вторая на запад. 
Наши   исследования   проводились   в   западной   части   губы.   
В   эту   часть   губы   впадает полноводный ручей, имеющий на литорали оформленное русло. 
Верхний горизонт литорали представлен   гравийной   россыпью.   
Средний   горизонт   \textemdash\   илисто-песчаным   пляжем   с отдельными   лежащими   на   поверхности   камнями,   поросшими   бурыми   водорослями  {\it Fucus vesiculosus}.   
При   этом   в   грунте   также   присутствует   гравий   и   крупная   галька,   полностью погруженная в песок. 
Нижний горизонт литорали представлен плотным поселением   {\it Fucus  vesiculosus}.

    \paragraph{Губа Ивановская}
Губа Ивановская с $2009$ года является памятником природы областного значения. 
Это сама восточная из исследованных нами акваторий в Баренцевом море. 
Длина губы составляет около $20$~км. 
Вход в губу закрывает  остров Нокуев.
В связи с  закрытостью губы и ее размерами приливно­отливная волна   распространяется   в   губе   медленно   и   задержка   приливов   и   отливов   в   куту   губы относительно прилегающей морской акватории достигает нескольких часов. 
Губа   разделена   поперечными   рядами   на   три   части,   называемых   «ковшами». 
Исследования   проводили   во   втором   ковше   на   северном   берегу.   
Исследованный   участок представлял   собой   верхнюю   сублитораль   (глубина   $0,8$~м)   с   небольшим   уклоном   свала. 
Физиономически участок представлял собой илисто­песчаный «пляж» с отдельными камнями, лишенными растительности. 
Ниже исследованного участка начинался пояс взморника {\it Zostera  sp.} 

        \subsection{Характеристики грунта}
            \subsubsection{Баренцево море}
Анализ   гранулометрического   состава   грунта   позволяет   косвенно   оценивать  интенсивность   гидродинамики   и,   следовательно,   условия   питания   моллюсков   на 
исследованных   участках.   
Кроме   того,   наличие   доступного   детрита   можно   оценивать   с помощью определения концентрации органических веществ в грунте.
По соотношению частиц различного размера в грунте на всех участках преобладает (более $50$~\%) песчаная фракция (табл.~\ref{tab:grunt_general_Barents}). 
    \begin{table}[ht]
    \caption{Соотношение основных включений в грунте на участках}
    \label{tab:grunt_general_Barents}
    \begin{tabular}{|*{4}{p{0.25\textwidth}|}} \hline
    Участок  &  гравий &  песок &  алевриты и пелиты 
        \\ \hline
    Абрам-мыс &  $1,13$ & $52,41$ & $44,16$
        \\ \hline
    Пала-губа &   $0$ &  $99,00$ &  $1,0$
        \\ \hline
    Гаврилово &  $0,04$ & $98,41$ &  $0,74$
        \\ \hline
    Ярнышная &   $3,09$ & $95,02$ & $0,99$
        \\ \hline
    Дальнезеленецкая &  $0,31$ & $98,27$ & $0,82$
        \\ \hline
    Шельпино &  $30,10$ & $67,62$ & $1,60$
        \\ \hline
    Порчниха & $25,63$ & $74,78$ & $1,68$
        \\ \hline
    Ивановская  & $17,22$ & $70,50$ & $11,09$
        \\ \hline
    \end{tabular}

    {\footnotesize Примечание: указана доля частиц, \%}
    \end{table}


Гравий присутствует на всех участках, кроме Пала-губы.  
Доля  гравия может достигать $30$~\%. 
Интересно, что участки со значительным ($> 10$\textcolor{red}{\%?}) содержанием   гравия  \textemdash\  наиболее   восточные   из   всех   изученных.   
Доля   илистых   фракций обычно   невелика,   лишь   на   литорали   Абрам-мыса   и   в   сублиторали   губы   Ивановская   она превышает   $10$~\%.   
Из   всех   исследованных   участков   только   Абрам-мыс   представляет   собой типичную илисто­песчаную отмель, поскольку доля песка и алевритов и пелитов практически одинаковая и близка к $50$~\%.
Более детальное рассмотрение гранулометрического состава грунта показало, что по соотношению различных песков участки неоднородны (табл.~\ref{tab:grunt_granulometriya_Barents}).
    \begin{table}[ht]
    \caption{Гранулометрический состав грунта на исследованных участках в Баренцевом море}
    \label{tab:grunt_granulometriya_Barents}
    \begin{tabular}{|p{0.14\textwidth}|*{8}{p{0.09\textwidth}|}} \hline
    & крупный и средний гравий  &  мелкий гравий &  очень мелкий гравий & очень крупный песок & крупный песок &  средний песок & мелкий песок & алевриты и пелиты \\
        Участок &   $>10$ &  $10-5$ &   $5-3$ &  $3-1$ & $1-0,5$ &   $0,5-0,25$ &    $0,25-0,1$ &    $<0,1$
        \\ \hline
    Абрам-мыс &  $0$ &  $0,77$ &  $0,35$ &  $2,84$ &  $6,82$ &  $6,74$ & $36,01$ &  $44,16$
        \\ \hline
        Пала-губа  &  $0$ &  $0$ &  $0$ &  $24,45$ &  $13,91$ &  $26,00$ &  $34,63$ &  $1,00$
        \\ \hline
        Гаврилово &  $0$ &  $0$ &  $0,04$ &   $4,58$ &   $23,80$ &  $58,42$ &  $11,61$ &  $0,74$
        \\ \hline
        Ярнышная  &  $0,20$ &  $0,17$  &  $2,72$ &  $32,03$ &  $29,66$ &  $19,02$ &  $14,31$  &  $0,99$
        \\ \hline
        Дальнезеленецкая &  $0$ &  $0,08$ &    $0,22$ &    $7,81$ &    $36,20$ &  $38,26$ &   $16,00$ &   $0,82$
        \\ \hline
    Шельпино  &  $16,06$ &   $10,28$ &   $3,77$ &  $7,96$  &  $22,76$ &  $22,45$ &    $14,46$ &  $1,60$ 
        \\ \hline
    Порчниха  &  $7,48$ &   $11,62$ &  $6,54$ &   $26,17$ &  $16,84$ &  $12,74$ &  $19,03$ &  $1,68$ 
        \\ \hline
    Ивановская &  $6,06$ &    $7,10$ &   $4,06$ &   $16,70$ &  $9,27$ &   $8,88$ &   $35,65$ &  $11,09$
        \\ \hline
    \end{tabular}

    {\footnotesize Примечание: указана доля частиц, \%}
    \end{table}


Содержание   органических   веществ   в   грунте   было   невелико,   и   на   всех   участках   не превышало $2$~\% (табл.~\ref{}).
    \begin{table}[ht]
    \caption{Содержание органических веществ в грунте на исследованных участках в Баренцевом море}
    \label{tab:grunt_granulometriya_Barents}
    \begin{tabular}{|*{9}{p{0.09\textwidth}|}} \hline
    участок & Абрам-мыс &   Пала-губа &  Гаврилово  & Ярнышная &   Дальнезеленецкая &  Шельпино &   Порчниха &   Ивановская
        \\ \hline
     &  $1,58$ &    $0,12$ &   $0,50$ &   $0,65$ &   $0,39$ &   $0,82$ &   $0,70$ & $1,38$
        \\ \hline
    \end{tabular}

    {\footnotesize Примечание: указано содержание органических веществ в грунте, \%}
    \end{table}

\end{document}
