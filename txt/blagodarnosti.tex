\chapter*{Благодарности}
%\addcontentsline{toc}{chapter}{Благодарности}
В заключение я хочу поблагодарить администрацию Кандалакшского заповедника и лично \fbox{А.\:С.~Корякина} за поддержку наших экспедиций на Белом и Баренцевом морях.
Я благодарна администрации СПбГУ, биологического факультета и кафедры ихтиологии и гидробиологии за возможность работы на Морской биологической станции СПбГУ.

На Баренцевом море мы работали вместе с сотрудниками Мурманского морского биологического института, Мурманского государственного технического университета и Полярного научно-исследовательского института морского рыбного хозяйства и океанографии: М.В.~Макаров, С.В~Малавенда, С.\:С.~Малавенда, О.~Тюкина, И.\:П.~Прокопчук, которые оказывали нам всяческую поддержку.  

Эта работа не могла бы состоятся без моих коллег по экспедициям: Беломорской экспедиции ГИПС ЛЭМБ, студенчской Баренцевоморской экспедиции СПбГУ, Беломорской экспедиции кафежры ихтиологи и гидробиологии СПбГУ. 
Отдельное спасибо руководителям экспедиций: А.\:В.~Полоскину, И.\:А.~Коршуновой, Д.\:А.~Аристову, Е.\:А. Генельт-Яновскому, М.В.~Иванову за возможность работы в экспедиционных командах и помощь в сборе материала.

Я благодарю А.\:В.~Полоскина, Д.\:А.~Аристова, К.\:В.~Шунькину, А.\:В.~Герасимову (кафедра ихтиологии и гидробиологии СПбГУ), А.\:Д.~Наумова (ББС ЗИН РАН) за предоставленные материалы.

Постоянные обсуждения с Ю.\:Ю.~Тамберг и В.\:М.~Хайтовым значительно улучшили мои навыки в статистической обработке материала и помогло мне в работе.
На этапе обработки данных неоценимую помощь идеями и разъяснениями мне оказали В.\:М. Хайтов и Д.\:А. Аристов.


Кроме того, я не могу не поблагодарить руководителей Лаборатории экологии морского бентоса И.\:А. Коршуновой, А.\:В.~Полоскину, \fbox{Е.\:А. Нинбургу} и В.\:М. Хайтову, которые 13 лет назад убедили меня, что морская биология это очень интересно и вложили много сил в мое обучение и воспитание. 
Без них меня бы тут просто не было.

И мой низкий поклон моему научному руководителю Н.\:В. Максимовичу за конструктивную помощь на всех этапах работы, жесткие споры и долгие беседы, ехидные комментарии и  неизменно доброе отношение.

\vspace{3ex}

Данная работа выполнена при частичной финансовой поддержке грантов Санкт-Петер\-бург\-ского государственного университета (1.\:0.\:134.\:2010, 1.\:42.\:527.\:2011, 1.\:42.\:282.\:2012, 1.\:38.\:253.\:2014) и Российкого фонда фундаментальных исследований (12-04-01507, 13-04-10131 К). 

