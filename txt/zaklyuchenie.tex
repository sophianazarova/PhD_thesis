	\chapter*{Заключение}		
\addcontentsline{toc}{chapter}{Заключение}	% Добавляем его в оглавление

Двустворчатый моллюск \textit{Macoma balthica} является типичным представителем литоральной фауны в Белом и Баренцевом морях. 
В Белом море данный вид формирует плотные скопления, причем поселения в Кандалакшском заливе характеризуются максимальной средней численностью в европейской части ареала вида.
В Баренцевом море \textit{M.~balthica} на литорали Западного Мурмана и Кольского залива также формирует плотные поселения, в то время как на литорали Восточного Мурмана численность данного вида редко превышает $100$~экз./м$^2$.

Динамика размерной структуры поселений {\it M.~balthica} в Белом и Баренцевом представлена двумя типами.
Более распространенный вариант: чередование бимодального и мономодального распределение особей по размерам. 
При этом первый пик формируют молодые особи (обычно длиной до 5~мм), а в случае бимодальной добавляется второй модальный класс из взрослых особей (в Белом море длиной 9 -- 12~мм, в Баренцевом 10 -- 17~мм). 
В Баренцевом море часто новое пополнение происходит до ухода старшей генерации и наблюдается три модальных группы. 
Такой тип динамики связан с различной успешностью ежегодного пополнения поселений молодью и, по-видимому, наличием внутривидовой конкуренции между взрослыми и молодыми особями.
В некоторых условиях формируется более редкий тип динамики с ежегодным повторением мономодальной размерной структуры. 
Возможно, это связано со специфическими условиями гидродинамики, в которых происходт разделение молодых и старых особей по способу питания и, таким образом, снижение внутривидовой конкуренции и возможность большего успеха ежегодного пополнения поселения молодью. 
Другое возможное объяснение --- формирование такого типа динамики в поселениях, находящихся под прессом хищников, которые уменьшают численность взрослых особей.

Макомы в Баренцевом море гетерогенны по скорости роста. 
Более высокая скорость роста была отмечена у особей {\it M.~balthica} обитающих в среднем горизонте литорали.
Также показано увеличение среднего годового прироста в более восточных поселениях на Мурманском побережьи.
Анализ скорости роста в европейской части ареала показывает снижение данного показателя в краевых популяциях, причем на севере это снижение более выражено.

Динамика численности поселений {\it M.~balthica} в Белом море характеризуется значительными колебаниями, связанными в первую очередь с численностью сеголетков. Изменения плотности поселений маком оказываются синхронными в пределах значительной акватории.
Численность маком оказывается выше после холодных зим, таким образом, по-видимому, основное влияние оказывают ледовые условия.
Предположительно, в более холодные зимы устойчивый ледовый покров формируется раньше и надежнее, поэтому выживаемость спата в зимний период выше, что фиксируется в наших наблюдениях, как более эффективное пополнение поселения, приводящее к увеличению общей численности {\it M.~balthica}.

Обнаружение в поселениях, обитающих в присутствии хищников, плотностно-зависимых процессов второго порядка позволяет говорить о том, что традиционно-предполагаемое минимальное влияние хищников на бентосные популяции в Арктических морях не соответствует действительности, и оказываемое на конкретное поселение воздейсвтие может быть значимо.

Численность спата на порядок варьирует в пределах незначительной акватории.
Основное влияние оказывает топология местности.
Также, по-видимому, оседание спата снижено в поселениях в высокой численностью взрослых особей {\it M.~balthica}, хотя масштабы этого явления и конкретные механизмы остаются неизвестными.


