	\section{Обзор литературы}
\textcolor{red}{в конце каждого раздела надо сформулировать проверяемую гипотезу}
		\subsection{Вид {\it Macoma balthica}: традиционное представление и современные данные}
\textcolor{red}{Амфибореальный. Ареал распространения. Два подвида. Образуют гибриды. Нет данных об экологических различиях.}

		\subsection{Физико-географическое описание районов исследования}
\textcolor{red}{Климат. Температура. Соленость. Ледовый режим. Типы литорали.
ТЕмпературные данные Кольский меридиан и Декадная съемка Чупа.}

		\subsection{Поселения {\it Macoma balthica} в различных частях ареала}
\textcolor{red}{Биотопы, сообщества, обилие, биомассы.}

		\subsection{Динамика поселений {\it Macoma balthica} и влияющие на нее факторы}
\textcolor{red}{Сегерстрале в Балтике, Букма в Ваттовом море, и Рейзе там же, Каф. данные, Наумов.
Пополнение как ключевой фактор у бентосных видов с планктонной личинкой. 
Пополнение = нерест + оседание + первая зима.
Ваттово море - сильные/слабые зимы влияют на хищников, они влияют на ракушек. 
Влияние численности взрослых маком на спат.}

		\subsection{Рост {\it Macoma balthica} в различных частях ареала}
\textcolor{red}{Определение роста у двустворчатых моллюсков. 
методы сравнения ростовых кривых. 
Рост маком. 
Широтные измерения роста.}
