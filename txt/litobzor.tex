	\chapter{Обзор литературы}
%\textcolor{red}{в конце каждого раздела надо сформулировать проверяемую гипотезу}
%		\section{Вид {\it Macoma balthica}: традиционное представление и современные данные}
%\textcolor{red}{Амфибореальный. Ареал распространения. Два подвида. Образуют гибриды. Нет данных об экологических различиях.}



		\section{Физико-географическое описание районов исследования}
%\textcolor{red}{Климат. Температура. Соленость. Ледовый режим. Типы литорали.
%ТЕмпературные данные Кольский меридиан и Декадная съемка Чупа.}

Белое и Баренцево моря~--- арктические моря, однако литоральная фауна во многом сформирована бореальными видами (\cite{Zenkevich_1963}).
Условия обитания гидробионтов в них значительно отличаются в связи с географическим положением и особенностями гидрологии.
Рассмотрим их подробнее.

	\subsection{Белое море}

Белое море глубоко врезается в материк, и с этим связывают континентальность климата: лето относительно теплое, зима продолжительная и суровая. 
Зимой температура воздуха может опускаться до $-20 - -30^{\circ}C$, а летом подниматься до $+30^{\circ}C$, хотя обычно не превышает $15-20^{\circ}C$. 
В северных районах Белого моря температура воздуха в среднем ниже, чем в южных (\cite{Babkov_Golikov_1984}). 
Для губы Чупа минимальная температура воздуха наблюдается в январе (в среднем $-11^{\circ}C$), а максимальная в июле (в среднем $+14,7^{\circ}C$) (\cite{Babkov_1982}). 

Летом в вершинных частях заливов и на мелководье вода может прогреваться до $20 - 24^{\circ}C$. 
Зимой температура воды отрицательная, порядка $-1,5^{\circ}C$ (\cite{Babkov_Golikov_1984}).
Кандалакшский залив является наиболее прогреваемым участком. 
В западной его части среднегодовая температура воды составляет $4^{\circ}C$ (при разбросе от $3,2$ до $5,1^{\circ}C$), а амплитуда межсезонных колебаний составляет в среднем $14,8^{\circ}C$ (от $13,0$ до $16,5^{\circ}C$) (\cite{Kuznecov_1960}). 
В губе Чупа среднегодовая температура всей толщи воды составляет менее $2^{\circ}C$. 
Поскольку литораль находится в зоне влияния поверхностной водной массы, то зимой обитатели подвергаются воздействию отрицательных температур ($-1,5^{\circ}C$), в то время как летом вода на литорали прогревается до $+19,3^{\circ}C$ (\cite{Babkov_1982}). 

Другим важным для гидробионтов фактором является соленость воды. 
В Белом море среднегодовая соленость поверхностных вод составляет $23-25$\permil. 
По данным А.И.Бабкова и А.Н.Голикова (\cite*{Babkov_Golikov_1984}) в районе Кандалакши соленость может изменяться от $7$ до $26$\permil. 
Такие колебания связаны с обширным материковым стоком, частично с осадками и, в первую очередь, с весенним таянием льдов (\cite{Naumov_Fedyakov_1993}).
Вода в губе Чупа значительно распреснена, в первую очередь за счет стока рек Пулонга и Кереть, но также за счет ручьев. 
В верхнем $10$ метровом слое, то есть в слое, омывающем литораль, отмечены сезонные колебания солености более $10$\permil\ (от $15$ до $26$\permil), при этом максимальная соленость достигается в ноябре, а минимальная~--- в апреле (\cite{Babkov_1982}). 

В зимнее время для Белого моря характерен ледовый покров. 
При подвижках припая возможно истирание выступающих над поверхностью структур, в том числе живых организмов. 
Кроме того, возможен перенос организмов, вмерзших в лед или находящихся на примерзших водорослях.
 Время ледостава в разных районах Белого моря отличается. 
В губах Кандалакшского залива лед появляется в первой половине сентября и держится до второй половины мая. 
В губе Чупа формирование льда начинается в устьях рек и ручьев, а также в небольших закрытых губах, где на формирование льда мало оказывает влияние ветрового волнения. 
Неподвижный лед обычно формируется в первой половине декабря. 
Продолжительность ледостава в среднем составляет $5$~месяцев, но в суровые годы может доходить до $7$~месяцев (\cite{Babkov_Golikov_1984}). 

	\subsection{Баренцево море}

Баренцево море~--- окраинное море, характерной особенностью гидрологического режима которого является наличие двух  водных масс~--- арктической (полярные воды, большую часть года покрытых плавучими льдами) и субарктической (субполярных вод, свободных от плавучих льдов) (\cite{Adrov_1992}). 

Мурманским побережьем или Мурманом называют береговую линию Северного ледовитого океана от мыса Святой нос на востоке до реки Ворьемы на западе. 
Данный район разделяют на несколько областей: Западный Мурман~--- от реки Ворьемы до острова Кильдин или до Кольского залива, и Восточный Мурман~--- далее на восток до мыса Святой нос (\cite{Derugin_1915}).

Постоянный подток теплых атлантических вод препятствует образованию льда вдоль Мурманского побережья, и он встречается главным образом во внутренних частях губ и заливов.
Несколько большее количество льда образуется ежегодно в юго-восточном районе Мурмана, в то время как по Западному Мурману, как правило, не образуется сплошного припая. 
В основном, исключая некоторые опресненные закрытые бухты и заливы, влияние морского льда на распределение животных невелико, гораздо большее значение зимой играет сильное промораживание литорали во время отлива (\cite{Propp_1971}).

Приливы на Мурмане являются правильными полусуточными и образуются единой атлантической приливной волной. 
Далее она распространяется вдоль Мурмана на восток до Новой Земли. 
Высота приливной волны составляет $3$ метра. 

В среднем, соленость вод у Мурманского побережья составляет $33,2 - 33,6$\permil. 
Только весной во время сезонного увеличения берегового стока наблюдается краткое распреснение поверхностных слоев до $28 - 30$\permil, однако толщина опресненного слоя не превышает $2 - 3$~м.

Кольский залив~--- самый крупный из заливов Мурманского побережья Баренцева моря, лежит на границе Восточного и Западного Мурмана.
Географически в Кольском заливе выделяется три части, называемые коленами залива. 

Первое, северное или нижнее колено простирается от входа в Кольский залив до линии, соединяющей устье губы Средней и мыс Лас. 
Эта часть залива наиболее глубоководная (более $400$~м). 
Береговая линия северного колена Кольского залива чрезвычайно изрезана, и  здесь находятся самые крупные губы (\cite{Derugin_1915}), в том числе Пала-губа, ставшая объектом наших наблюдений .


Среднее колено (глубины до $200$~м) изогнуто в направлении к северо-западу и простирается на юг до мысов Пинагория и Мишукова. 
Второй участок наблюдений был расположен в районе границы северного и среднего колена Кольского залива (Ретинское).

Южная или верхняя часть наиболее мелкая (глубина около $50$~м), имеет направление с севера на юг, как и нижняя. 
В кут Кольского залива впадает две крупные реки~--- Тулома и Кола, и одна более мелкая~--- Лавна (\cite{Derugin_1915}).  
В районе самого узкого участка Кольского залива (Абрам-мыс) был расположен третий участок исследования в данном районе.
Последний участок, исследованный в Кольском заливе был расположен на западном берегу залива в черте города Мурманск (Северное Нагорное) в $3$~км от устья реку Туломы.

Воды Кольского залива неоднородны по своим свойствам. 
Это связано с несколькими причинами: большая протяженность залива, наличие глубоко вдающихся в побережье губ, влияние стока рек и ручьев. 
Гидрологическое лето начинается в поверхностных слоях воды в начале июля и продолжается до конца августа. 
Летом вода прогревается до $+8 - +18^{\circ}C$ в различных частях залива.

В  северном колене залива летом поверхностный слой значительно распреснен и соленость может достигать $8$\permil, причем толщина распресненного слоя может достигать $3-4$~метров. 
Глубже соленость не опускается ниже $30$\permil и у дна достигает $34$\permil. 
Зимой соленость поверхностного слоя также составляет $30 - 34$\permil. 

В южном колене в районе Абрам-мыса колебания солености на поверхности еще более заметны. 
Здесь сказывается не только сезонность стока, но и значительное влияние оказывает приливно-отливные течения. 
Летом во время прилива поверхностный слой толщиной до 3 метров обладает соленостью от $2$ до $16$\permil, в то время как на глубине $3$~метра соленость колеблется в пределах от $28$ до $31$\permil. 
В отлив мощность опресненного слоя увеличивается до $8$~метров, а поверхностная вода становится практически пресной (\cite{Derugin_1915}).

Таким образом, исследованные нами участки в Кольском заливе расположены в контрастных по географическим условиям его частях и позволяют относительно полно судить о данной акватории.

Фауна литораль Западного Мурмана наиболее богата по сравнению с остальным Мурманским побережьем. 
Традиционно, это связывают с более высокой среднегодовой температурой (температура воздуха в губах Западного Мурмана может быть на $0,4^{\circ}C$ выше по сравнению с Восточным Мурманом) и соленостью (выше $31$\permil\ в поверхностном слое) и закрытости губ Западного Мурмана от основной акватории моря (\cite{Guryanova_et_al_1930}). 
К сожалению, данный регион оказался для нас малодоступен при исследованиях, и мы располагаем лишь данными об обилии маком в губах Ура и Печенга.
Однако данные губы расположены в разных частях Западного Мурмана, что позволяет нам делать предвательные выводы о данном регионе.

Береговая линия Восточного Мурмана менее изрезана, чем Западного Мурмана. 
Побережье большинства небольших заливов и губ не защищено от прибойного воздействия (\cite{Guryanova_Ushakov_1929}).
Таким образом, Восточный Мурман на большем его протяжении не является благоприятным для развития литоральных инфаунных сообществ, однако существуют глубоко вдающиеся в побережье бухты, в которых обнаруживается меньшее волновое воздействие. 
Именно на литорали таких губ и заливов и формируются наиболее богатые инфаунные сообщества данного региона, включающие {\it M.~balthica}.

Наши исследования охватывают Восточный Мурман на значительном его протяжении: $6$~участков от губы Гаврилово до губы Ивановская (длина береговой линии более $150$ километров).
Обследованные бухты варьируют по длине, степени изолированности и наличию в них ручьев и небольших рек, влияющих на локальное опреснение.


География исследований охватывает в том числе Дальний пляж губы Дальне-Зеленецкой~--- исторически наиболее обследованной бухты на Мурмане.
Губа Дальне-Зеленецкая включает в себя две бухты~--- бухта Оскара и бухта, в кутовой части которой располагается литоральная отмель Дальнего Пляжа. 
Важной характеристикой губы является изолированность ее от интенсивного волнового воздействия за счет наличия островов на входе в губу.
	
При максимальных отливах протяженность литорали Дальнего пляжа с северо-запада на юго-восток составляет около $460$~м, а с юго-запада на северо-восток -- около $400$~м. 
	
В южной части отмели располагается дельта небольшого Зеленецкого ручья, вызывающего незначительное опреснение. 
Так, грунтовая вода, взятая у самого ручья, имеет соленость $32,9$\permil, а взятая на два метра в стороне от ручья~--- $34,07$\permil (\cite{Prigorovskiy_1948}). 
Гидрологический режим характеризуется тем, что в бухту заходят воды из более глубоких и холодных слоев открытого моря, что вызывает уменьшение температуры и повышение солености (\cite{Voronkov_et_al_1948}).

Волновая активность в губе не превышает $1,5 - 2$ балла (\cite{Alexeev_1976}). 
Наиболее сильному волновому воздействию подвержена южная и юго-восточная части отмели, где на галечно-валунном пляже располагается зона штормовых выбросов.
Придонная скорость течений, вызванных приливной волной, составляет $0,8$~м/сек. при глубине 0,3-0,5 метров и 0,06 м/сек. при глубине более $2$~метров.

Для песчаных отмелей характерна только одна граница~--- уровень высачивания, который делит пляж на две части, отличающиеся по условиям увлажненности донного осадка во время отлива (\cite{Streltsov_Agarova_1978}). 
Обширный, располагающийся ниже уровня высачивания и увлажненный во время отлива <<ватт>> простирается от отметок 1,25 до 2,1 м. над нулем глубин, сменяясь выше уровня высачивания узким $30$-метровым пляжем, где вода, занимавшая во время прилива интерстициальное пространство, вместе с грунтовыми водами вытекает на поверхность донного осадка. 
В западной части пляжа, самые верхние горизонты заняты валунной грядой (\cite{Agarova_et_al_1976}). 

Грунты отмели однообразны почти на всем ее протяжении. 
Мощность верхнего слоя ничтожна, и составляет $5 - 8$~см (\cite{Prigorovskiy_1948}). 
Для отмели процессы размыва преобладают над накоплением. 
Даже в зоне относительно высокой аккумуляции, в <<языках>> дельты ручья, мощность голоценовых отложений составляет всего $15 - 30$~см.

Максимальная концентрация песков (более $90$\% по массе) отмечена в юго-восточной оконечности у подножья террасы, сложенной древними морскими песками. Еще одной особенностью пляжа является повышенное содержание алевропелитов (\cite{Pavlova_1976}). 
Их локализация на пляже обусловлена эрозивной волноприбойной деятельностью, доминирующей при среднем уровне малой воды (\cite{Alexeev_1976}).

Органические вещества представлены гумусовыми соединениями и битумоидами местного и континентального происхождения (\cite{Gurevich_Yakovleva_1976}).
Наши мониторинговые работы в губе Дальне-Зеленецкая продолжают череду количественных гидробиологических исследований данного района (\cite{Prigorovskiy_1948, Matveeva_et_al_1955, Streltsov_et_al_1974, Agarova_et_al_1976, Zhukov_1984, Strelkov_et_al_2001}).

%%%%%%%%%%%%%%%%%%%%%%%%%%%%%%%%%%%%%%%%%%%%%%%%%%%%%%%%%%%%%%%%%%%%%%%%%%%

		\section{Экология вида}
Двустворчатый моллюск \textit{M.~balthica}~--- амфибореальный литоральный вид.
Это обычная литоральная форма в Белом море, у берегов Мурмана и далее на запад, вдоль атлантических берегов Европы~--- до Франции. 
По Атлантическому побережью Северной Америки макомы распространены от Лабрадора до штата Джорджия. 
В северной части Тихого океана~--- от Берингова моря до Японского, а по американскому побережью~--- до Калифорнии. 
В юго-восточной части Баренцева моря и в прилегающей части Карского моря они обитают  не на литорали, а на глубине нескольких метров. 
Моллюски заселяют всю основную часть Балтийского моря, далеко заходя во все заливы, где живут до глубины более 100 метров (\cite{Zacepin_Filatova_1968}).

В настоящее время вид {\it Macoma balthica} по результатам аллозимного анализа предлагают разделять на два подвида: {\it M.~b.~balthica}, обитающий в северной части Тихоокеанского региона, и {\it M.~b.~rubra} из Северо-Восточной Атлантики. 
Однако  в морях, связанных с  Атлантикой, существуют очаги распространения тихоокеанской формы. 
Так, в Балтийском и Баренцевом море Атлантическая и Тихоокеанская формы сосуществуют и образуют гибриды (\cite{Vainola_2003}). 
В Белом море встречается в основном {\it M.~b.~balthica}, и лишь в устье Онеги было обнаружено два экземпляра {\it M.~b.~rubra} (\cite{Nikula_et_al_2007}).
К настоящему моменту нет прямых данных о влиянии данных генетических особенностей на экологические характеристики особей, поэтому в данной работе рассматривается вид {\it Macoma balthica} sensu lato.

\textit{M.~balthica}~--- эвригалинный и эвритермный вид. 
Особи данного вида, обитающие в Белом море, способны выдерживать опреснение до $18$\permil\ и температуру до $25^{\circ}C$ (\cite{Naumov_2006}), но в Балтийском море макомы встречаются при солености $5$\permil (\cite{Karpevich_Shurin_1970}.
В экспериментах на моллюсках из Балтийского моря при температуре от $0$ до $22^oC$ и солености от $4$ до $25$\permil\ смертность взрослых особей оставалась менее 1\% в сутки (\cite{Karpevich_1968}.

Питаются макомы, собирая длинными червеобразными сифонами детрит с поверхности грунта (\cite{Naumov_2006}). 
Кроме того, показано, что особи \textit{M.~balthica} могут питаться как фильтраторы, когда в придонном слое воды находится взвешенный пищевой материал (Бубнова, 1972, Герасимова, 1988). 
Моллюски встречаются на илисто-песчаных грунтах, где обитают, закапываясь до глубины $2-6$~см. 
Они могут существовать и при длительной осушке: взрослые особи встречаются в местах с осушкой до 20 часов в сутки (\cite{Sveshnikov_1963}). 

\textit{M.~balthica} обитает на всех типах грунта: от чистого песка до жидкого ила. 
По данным Н.Л. Семеновой (1974) численность маком всегда ниже на чистом песке и увеличивается с увеличением заиления. 
Известно, что более богатым органическими веществами является более тонкий по гранулометрическому составу грунт (\cite{Bubnova_1972}), поэтому отмеченный выше характер распределения особей \textit{M.~balthica} может быть связан не столько с механическим составом грунта, сколько с обеспеченностью моллюсков пищей.

Макомы встречаются от самых верхних горизонтов литорали до глубины 140–160 метров (в Балтийском море). 
Было показано, что распределение макомы по литорали зависит главным образом от наличия заиленных пляжей (\cite{Semenova_1974}), где она находит подходящие условия для питания. 
	
Особи \textit{M.~balthica} обладают значительной подвижностью (\cite{Sveshnikov_1963}). 
Моллюски передвигаются в подповерхностном слое, и на грунте остается характерный след --- неглубокая извитая борозда (\cite{Naumov_2006}). 

Для мелких маком показан другой механизм передвижения, в первую очередь с целью расселения~---так называемый биссусный дрифт. 
Хотя во взрослом состоянии макомы не образуют биссуса, молодые особи могут выпускать длинные нити, которые позволяют даже слабому потоку поднимать особь над грунтом и переносить на некоторые расстояния. 
Показано, что способностью к биссусному дрифту обладают макомы размером до $4$~мм (\cite{Armonies_Hellwig-Armonies_1992}). 
Дальность этих миграций зависит от размера раковины макомы и длины биссусной нити, при этом некоторые особи мигрируют более чем на $10$ километров. 
В Северном море существуют поселения \textit{M.~balthica}, для которых показано пополнение не за счет личинок, а за счет переоседания особей из Ваттового моря (\cite{Beukema_deVlas_1989}).


		\section{Структура поселений {\it Macoma balthica}}

Наиболее изучены поселения \textit{M.~balthica} в Северном и Белом морях.

В Северном море в районе Ваттового моря \textit{M.~balthica} является одним доминирующих и широко распространенных видов зообентоса, как в литоральных, так и в сублиторальных местообитаниях.
Данная акватория характеризуется очень пологой литоральной зоной, которая формирует обширные (до нескольких километров шириной) илисто-песчаные отмели, именуемые ваттами, и образует обширные мелководья. 
Поселения маком встречаются здесь в широком диапазоне глубин и на разных типах грунтов (\cite{Beukema_et_al_1993, Hiddink_et_al_2002_predation_infauna, Hiddink_et_al_2002_predation_epifauna, Hiddink_2003}).
Максимальная средняя численность, описанная с Северном море составляет около $1600$~экз./м$^2$ (\cite{Reading_1979}), а биомасса может достигать $500$~г/м$^2$.

Оседание личинок маком в Северном море происходит весной главным образом на нижние горизонты литорали (\cite{Strasser_Gunter_2001}). 
Через несколько месяцев молодь моллюсков сдвигается к верхним горизонтам литорали (\cite{Armonies_Hellwig-Armonies_1992}). 
Особи в возрасте старше $1$ года распределены гораздо более равномерно, занимая практически все горизонты литорали и верхнюю сублитораль (\cite{Beukema_et_al_1993}). Поскольку локализация ювенильных и взрослых маком пространственно разделена, предполагается, что в ходе развития моллюски как минимум дважды мигрируют между различными горизонтами литорали. 

%В Балтийском море 

В Белом море особи \textit{M.~balthica} встречаются  повсеместно, за исключением Горла и Воронки. 
Моллюски обитают на различных глубинах: от верхнего горизонта литорали до  $4-5$~м. 
В эстуарных районах (дельта Северной Двины, Мезени) отмечен уход моллюсков в сублитораль до глубины $20$~м (\cite{Semenova_1974, Naumov_2006}).
По данным различных исследователей (\cite{Babkov_Golikov_1984, Naumov_2006}) для среднего и нижнего горизонта литорали с мягкими грунтами характерно формирование сообществ с доминированием \textit{M.~balthica}. 

Плотность поселений \textit{M.~balthica} может значительно изменяться как в пространстве, так и во времени. 
Максимальная плотность поселения характерна для нижнего горизонта литорали (\cite{Semenova_1974, Maximovich_et_al_1991}). 
По данным А.\:Д.~Наумова с соавторами максимальная плотность поселения для Белого моря отмечена в губе Чупа в биоценозе \textit{M.~balthica} и составила $2885$~экз./м$^2$ (\cite{Naumov_2006}).

%Также важными характеристиками поселений являются размерная и возрастная структура. Для M. balthica описано бимодальное и мономодальное распределение особей (Максимович и др., 1991; Николаева, 1997; Николаева, 1998; Назарова, 2003). Обе размерные структуры поселений, по-видимому, формируются в результате неравномерного пополнения поселений молодью.
%При массовом оседании личинок  \textit{M.~balthica}, в зависимости от выживаемости сеголеток, возможно два варианта развития поселения. 
%Если выживаемость хорошая, то можно наблюдать ежегодное смещение модального класса по оси размеров. При новом оседании личинок до полного отмирания особей первой генерации формируется бимодальное распределение. Другой описанный вариант~--- к следующему сезону сеголетки практически исчезают, и происходит новое оседание личинок. При повторении этой схемы наблюдается мономодальное распределение с доминированием по численности самых мелких особей (сеголеток) при практически полном отсутствии крупных особей. Естественно, при плохой выживаемости и отсутствии значительного оседания личинок поселение достаточно быстро отмирает (Максимович и др. 1991).

		\section{Динамика численности поселений {\it Macoma balthica} и влияющие на нее факторы}

%\textcolor{red}{Сегерстрале в Балтике, Букма в Ваттовом море, и Рейзе там же, Каф. данные, Наумов.
%Пополнение как ключевой фактор у бентосных видов с планктонной личинкой. 
%Пополнение = нерест + оседание + первая зима.
%Ваттово море - сильные/слабые зимы влияют на хищников, они влияют на ракушек. 
%Влияние численности взрослых маком на спат.}

\textit{M.~balthica}~--- бентосный инфаунный моллюск, взрослые особи которого перемещаются на относительно небольшие расстояния (не более метра) (\cite{Beukema_et_al_1993}).
Таким образом, вклад миграций в динамику взрослых особей пренебрежимо мал.
Основной процесс, определяющий динамику численности поселений маком~--- пополнение поселений молодью.
Процесс пополнения поселений состоит из нескольких этапов: формирование личиночного пула, оседание спата в поселение и первая зимовка.
Для Северного и Белого морей показано, что в пополнении поселений молодью выживаемость спата в первую зиму более важна, чем непоследственно количество осевших особей (\cite{Beukema_et_al_1998, Strasser_Gunter_2001, Maximovich_Gerasimova_2004}). 


В то же время в Северном море существуют поселения \textit{M.~balthica}, для которых показано пополнение не за счет личинок, а за счет переоседания особей из Ваттового моря (\cite{Beukema_deVlas_1989}). 
Для мелких маком показан специфический активный механизм передвижения, в первую очередь с целью расселения~--- так называемый биссусный дрифт. 
Хотя во взрослом состоянии макомы не образуют биссуса, молодые особи могут выпускать длинные нити, которые позволяют даже слабому потоку поднимать особь над грунтом и переносить на некоторые расстояния. 
Показано, что способностью к биссусному дрифту обладают макомы размером до $4$~мм (\cite{Armonies_Hellwig-Armonies_1992}). 
Дальность этих миграций зависит от размера раковины макомы и длины биссусной нити, при этом некоторые особи мигрируют более чем на $10$~километров. 

\subsection{Особенности жизненного цикла \textit{Macoma~balthica}}
При исследовании динамики популяций животных вопросы жизненного цикла и размножения играют важную роль, поскольку определяют потенциальное увеличение популяции за счет рождаемости. 

Исследователи приводят различные данные о возрасте и размере, при которых макомы достигают половой зрелости. 
Так, Л.П.~Флячинская пишет, что в Белом море \textit{M.~balthica} достигает половой зрелости к третьему году жизни при размере раковины $13-15$~мм (\cite{Flyachinskaya_1999}). 
Для Рижского залива отмечено созревание маком в возрасте $2-3$~года при размере $10-12$~мм (\cite{Karpevich_1968}). 
Для маком из бухты Уитстейбл (Англия) показан размер половозрелости $8$~мм (\cite{Caddy_1967}), а для бухты Ден-Хелдер (Голландия) точных указаний нет, но половозрелые особи находились в размерном классе $5-12$~мм (\cite{Lammens_1967}; цит. по \cite{Semenova_1980}).  
В работе Семеновой (\cite*{Semenova_1980}) высказывается идея, что ключевым фактором для возможности половозрелости является именно размер, а не возраст животного, и этот размер для макомы составляет $8$~мм. 
Это подтверждено и дальнейшими исследованиями на Белом море (\cite{Maximovich_1985}).

Время нереста различается в разных частях ареала. 
В бухте Ден-Хелдер (Голландия) нерест макомы длится два месяца в марте-апреле (\cite{Lammens_1967}; цит. по \cite{Semenova_1980}). 
В бухте Уитстейбл (Англия), на побережье Шотландии и в районе Датских ваттов (Северное море) нерест также продолжается два месяца, но сроки его более поздние~--- апрель-май (\cite{Caddy_1967, Stephen_1931}). 
Еще позже происходит нерест в Балтийском море (данные по Мекленбургской бухте) и в губе Дальне-Зеленцкой на Баренцевом море~--- с мая по август, т.е. в течение 4 месяцев (\cite{Oertzen_1972, Agarova_1974}). 
В Белом море сроки нереста очень сжатые~--- от $1-2$ недель до месяца в июне-начале июля (\cite{Semenova_1980, Maximovich_1985, Zubakha_et_al_2000}).

Считается, что триггером для начала нереста у макомы служит прогрев воды до $+10^{\circ}C$ (\cite{Maximovich_1985, Semenova_1980, Kaufman_1977}).  
Бьёкма и Меган (\cite{Beukema_Meehan_1985}) предполагают, что тригерная температура нереста является причиной, ограничивающей распределение моллюсков на юг~--- в южных широтах минимальная температура воды превышает $10^{\circ}C$. 
В этом случае южная граница ареала \textit{M.~balthica} должна совпадать с десяти градусной зимней изотермой, которая проходит около $45^{\circ}$с.ш. почти по всей северной Атлантике и круто изгибается к югу рядом с Американским побережьем. 
Действительно, распространение маком на юг по Американскому побережью дальше (до $37^{\circ}$с.ш.~--- штат Джорджия), чем по Европейскому (до $45^{\circ}$с.ш.~---  южная Франция).

В губе Дальне-Зеленецкой (Баренцево море) нерест маком наблюдался при температуре $2-9^{\circ}C$ (\cite{Agarova_1974}). 
Восточный Мурман характеризуется низкими среднегодовыми температурами, и наблюдаемый сдвиг тригерной температуры нереста можно объяснить эффектом смещения температур размножения видов на северных краях ареалов (\cite{Thorson_1946}). 
Это подтверждается еще и тем, что близкие к баренцевоморским температурные условия размножения \textit{M.~balthica} ($0-13^{\circ}C$) наблюдаются в Северной Канаде (\cite{Gilbert_1978}).

В Белом море проводили детальные исследования жизненного цикла маком. 
Гонады половозрелых маком созревают к концу мая, но нерест начинается, когда температура поверхностных слоев воды в море достигает $9-11^{\circ}C$  (\cite{Maximovich_1985, Flyachinskaya_1999}). 
Половые продукты выметываются в воду, где происходит оплодотворение. 
В лабораторных условиях при температуре $12^{\circ}C$ велигер формировался через $4,5$~суток после оплодотворения. 
Через $17-20$~суток на стадии педивелигера формировалась нога, и через 3$0-33$~суток происходил метаморфоз. 
В этот момент молодь оседает на субстрат, и ее называют <<плантиграда>> или <<спат>>, хотя второе название более распространено (\cite{Flyachinskaya_1999}). 
При оседании молоди маком, по-видимому, у особей не происходит выбора подходящего по характеру грунта, но затем происходит их перераспределение за счет биссусного дрифта (\cite{Armonies_Hellwig-Armonies_1992, Huxham_Richards_2003}). 


\subsection{Факторы, влияющие на пополнение поселений}
Большинство исследований, посвященных проблеме пополнения, выполнено в одном из районов Северного моря~--- так называемом Ваттовом море.	
Изначально было показано, что одним из ключевых факторов, влияющих на пополнение поселений \textit{M.~balthica}, является температура в зимний период, которая воздействует не напрямую, а через влияние на обилие хищников (\cite{Beukema_et_al_1998, Beukema_Dekker_2014, Dekker_Beukema_2014}).

В ряде других работ также было показано влияние различных хищников на численность и распределение молоди маком. 
Так, для Ваттового моря именно обилием хищников объясняется формирование временных скоплений молоди маком на верхней литорали. 
При изучении факторов, обуславливающих такое распределение для \textit{M.~balthica} было показано, что обилие бентосных хищников больше на нижней литорали, и лишь молодь краба \textit{Carcinus maenas} в значительных количествах встречается на верхней литорали. 
В полевых и лабораторных экспериментах было показано, что присутствие хищников значительно снижает численность спата, в то время как влияния на крупных особей обнаружено не было. 
По-видимому, за первый год макомы выходят из-под контроля бентосными хищниками за счет увеличения размеров тела (\cite{Hiddink_et_al_2002_predation_epifauna}). 

Также при анализе динамики личинок различных беспозвоночных в планктоне было показано, что после суровых зим численность личинок краба \textit{Carcinus maenas} значительно снижалась, и они появлялись в планктоне на $6-8$ недель позже, чем после мягких зим. 
По-видимому, с этим временным несоответствием связано большее пополнение поселений маком после суровых зим (\cite{Strasser_Gunter_2001}).

В более поздних исследованиях на Ваттовом море было показано, что влияние суровых зим на пополнение \textit{M.~balthica} не столь широкомасштабно, как предполагалось ранее, и, по-видимому, существуют другие факторы, определяющие более локальные вариации в пополнении поселений (\cite{Strasser_et_al_2003, Flatch_2003}).
Пресс хищников не объяснил эти различия, изменения сообществ и поступления биогенных элементов не объяснили картину, поскольку действовали лишь на отдельных участках. 
Наиболее вероятным фактором, по мнению данных авторов, является топографическая разница между двумя акваториями, где располагались исследованные участки. 
Предполагается, что в зависимости от закрытости акватории островами, и преобладающего направления ветров, будет идти более или менее эффективный перенос личинок и биссусный дрифт, а, значит, и пополнение поселения (\cite{Strasser_et_al_2003}).

Для другого участка Ваттового моря было показано, что комбинация эффектов высокого пресса хищников вместе с высоким обилием взрослой макрофауны обуславливает 95 процентное снижение количества спата теллинид (\textit{M.~balthica} и \textit{Tellina tenuis}) после мягких зим (\cite{Flatch_2003}). 

Хотя влияние на пополнение поселения молодью плотности взрослых особей того же вида представляется достаточно логичным механизмом, существуют лишь отдельные работы, посвященные внутривидовым взаимодействиям у \textit{M.~balthica}. 
Так, в ряде работ показано, что плотность молоди не зависит от обилия взрослых маком (\cite{Olafsson_1989, Vincent_et_al_1989, Beukema_et_al_2001, Richards_et_al_2002}). 

 Также было показано, что влияние плотности взрослых маком на рост спата зависит от типа грунта. 
На илисто-песчаном грунте, где и взрослые, и молодые моллюски питаются как собирающие детритофаги, рост спата подавляется при увеличении плотности взрослых особей. 
На песке, где молодь питаются как собирающие детритофаги, а взрослые~--- как фильтраторы, влияния на рост спата показано не было (\cite{Olafsson_1989}).

Для Белого моря существуют лишь несколько работ, посвященных отдельным аспектам пополнения поселений маком. 
Так, И.\:В.~Бурковским с соавторами показано, что макомы оседают вне плотных поселений взрослых (\cite{Burkovskiy_et_al_1998}). 
Также показано, что важную роль в динамике численности личинок и спата влияет принос личинок с соседних акваторий. 
В течение лета формируется сначала бимодальная размерная структура спата, с двумя пиками личинок в планктоне, которая к концу августа сливается в мономодальную (\cite{Zubakha_et_al_2000}). 
Показана высокая смертность особей на всех этапах пополнения поселения. Так, смертность пелагических личинок оценивают в $36,4$\% за сезон, а смертность спата~--- $59$\% за сезон (\cite{Burkovskiy_et_al_1998}).


		\section{Продолжительности жизни и рост {\it Macoma balthica} в различных частях ареала}
Данные о продолжительности жизни маком весьма противоречивы. 
Исследователи оценивают ее по-разному: от $3$ лет в Балтийском море до $15$ и даже до $30-35$ в Белом  (\cite{Segerstrale_1960, Semenova_1970, Maximovich_Kunina_1982}). 
Столь значительные расхождения в определении возраста связаны с особенностями применяемых методик, поэтому представляется важным рассмотреть этот вопрос подробно.

Современные методы определения возраста двустворчатых моллюсков разделяют на несколько типов: скульптурные, структурные, физико-химические и статистические. 
При этом первые три группы методов позволяют устанавливать возраст отдельных особей, в то время как статистические методы требуют изучения группы особей и дают вероятностную оценку возраста (\cite{Zolotarev_1989}). 

Принципиальной основой скульптурных методов определения возраста моллюсков является наличие на раковине неоднородностей скульптуры, связанных с периодичным (суточным, сезонным) изменением скорости роста особи. 
Для лучшего выделения наружных меток роста иногда створки просвечивают (\cite{Brousseau_1979}), поверхность раковины обрабатывают соляной кислотой для удаления периостракума (\cite{Tabunkov_1974}). 
С возрастом у многих моллюсков происходит изменение морфологии наружных колец. 
Обычно уменьшается их выраженность, увеличиваются различия в степени проявления однотипных элементов, что затрудняет или делает невозможным адекватную оценку возраста особи. 	

Другая проблема~--- возможность образования на поверхности раковины дополнительных меток роста. 
Они могут возникать при нересте, шторме, нападении хищников и носят непериодический характер. 
Однако обычно воздействие этих факторов непродолжительно и дополнительные кольца часто выражены слабее сезонных, что позволяет их различить при некотором навыке. 

Периодические изменения скорости роста моллюсков отражаются также на особенностях внутреннего строения раковин. 
Для анализа строения раковины изготовляют радиальные спилы или шлифы, после чего анализируют или непосредственно их, или приготовленные по ним ацетатные реплики.
Этот метод менее чувствителен к возрасту моллюсков и позволяет выделять годовые слои у старых особей со скоростью роста всего $0,1-0,25$~мм/год. 
Однако проблема дифференциации сезонных и прочих слоев роста остается.

Физико-химические методы определения возраста моллюсков более трудоемки и дороги, однако они позволяют определять возраст у моллюсков, у которых отсутствуют структурные возрастные образования. Данная группа методов основана на изучении неоднородности плотности, химического и изотопного состава. Наиболее часто используют определения стабильных изотопов кислорода, содержание магния и стронция, рентгенография створок (\cite{Zolotarev_1989})). 

Первые два метода достаточно точны, однако необходимость отбирать серию проб из створок, затрудняют их применение на некрупных объектах. 
Рентгенография выявляет сезонные изменения плотности скелетного вещества. 
Считается, что пара слоев с высокой и низкой плотностью образует годовой прирост (\cite{Ralph_Maxwell_1977}). 
Однако метод рентгенографии разработан пока недостаточно чтобы получить широкое применение. 

Таким образом, химические методы достаточно точны и дают объективные возрастные характеристики. 
Однако высокая стоимость и трудоемкость ограничивает их применение в массовых гидробиологических исследованиях, и до настоящего для \textit{M.~balthica} не проведена калибровка видимых колец и слоев химическими методами. 
В итоге скульптурные и структурные методы определения роста в настоящее время наиболее распространены из-за их доступности и относительной легкости процедуры. 
Неизбежная субъективность в интерпретации колец и слоев остановки роста ограничивает возможность сравнения данных, полученных разными исследователями. 
Однако в рамках одного исследования однотипность интерпретации колец и слоев позволяет сравнивать особей по получаемому относительному возрасту. 
Измерения меток зимних остановок роста, разделяющих кольца роста, позволяет восстановить размер особи в разном возрасте и реконструировать линейный рост.


Рост рассматривается как комплексный отклик организма на совокупность условий в локальном местообитании. 
Однако не менее интересной представляется попытка разложить всю совокупность условий на отдельные факторы, влияющие на ростовые характеристики. 
Одним из главных, определяющих рост факторов, является температура (\cite{Gilbert_1973, De_Wilde_1975, Bachelet_1980}). 
При повышении температуры происходит увеличение скорости метаболических процессов, в том числе темпов роста моллюсков в толерантных пределах.  
Для {\it M.~balthica} показано, что оптимальные условия роста~--- температура $0 - 10^{\circ} C$, а когда температура превышает $15^{\circ} C$ рост прекращается (\cite{De_Wilde_1975}). 
Ограничение роста при высоких температурах было отмечено и другими авторами, хотя на южной границе ареала (по-видимому, за счет физиологической адаптации) рост происходил и при более высоких температурах (\cite{Bachelet_1980}).

Изучение широтных измерений характера роста {\it M.~balthica} интересовали многих исследователей (\cite{Gilbert_1973, Bachelet_1980, Beukema_Meehan_1985, Wenne_Klusek_1985, Hummel_et_al_1998}).
Для сравнения использовали различные параметры: среднюю скорость роста роста моллюсков (отношение максимальной длины к возрасту особей), коэффициент $k$ уравнения Берталанфи, параметр $\omega$ ( произведение коэффициентов $L_{\infty}$ и $k$ из уравнения роста Берталанфи), годовой прирост.

Бьёкма и Меган (\cite{Beukema_Meehan_1985}) показали, что ростовые характеристики {\it M.~balthica} имеют выраженный широтный градиент.
В качестве параметра сравнения в этой работе был использован параметр $\omega$, который считается более адекватным для задач сравнения ростовых характеристик, чем сравнение параметров уравнения Берталанффи напрямую (\cite{Appeldoorn_1983}). 
Не смотря на широкое варьирование данного параметра, наблюдается уменьшение скорости роста в более северных популяциях маком.
В данной работе данные по российской части ареала {\it M.~balthica} ограничены работой Н.~Л.~Семёновой (\cite*{Semenova_1970}).

Хюммель с соавторами (\cite{Hummel_et_al_1998}) расширили географию исследования роста маком в северных морях, проанализировав годовой прирост моллюсков из Норвежского, Печорского, Баренцева и Карского морей.
Было показано, что группировки, генетически различные по результатам аллозимного анализа, отличались по величинам годового прироста. 
Макомы в популяциях с южной границы ареала росли медленнее, чем в центральной части ареала, а размах варьирования прироста в Белом море был сравним с таковым в европейских популяциях.
Печорские макомы, значительно отличающиеся генетически, также характеризовались более низкими годовыми приростами, однако дотигали при этом наибольших размеров.

Другим фактором, влияющим на процесс роста, является обилие пищи. 
Наблюдается достоверная связь между содержанием хлорофилла A на поверхности грунта, концентрацией фитопланктона и скоростью роста особей {\it M.~balthica} (\cite{Beukema_et_al_1977, Kube_et_al_1996}). 
С обилием пищи тесно связано влияние на рост моллюсков гранулометрического состава грунта и содержание в нем органических веществ. 
Чем меньше диаметр частиц грунта, тем больше площадь их поверхности и тем больше на них бактерий, соответственно более мелкодисперсный грунт оказывается для маком <<питательнее>>. 
Показано, что скорость роста особей на песчаном грунте ниже, чем на илистом (\cite{Wenne_Klusek_1985, Maximovich_et_al_1992}). 
Выявлена достоверная связь скорости роста моллюсков с долей мелкой фракции грунта и содержанием в нем органических веществ (\cite{Kube_et_al_1996}).

Соленость также оказывает влияние на рост моллюсков, хотя данные о характере этого влияния различны. 
Некоторые авторы отрицают влияние солености на скорость роста (\cite{Bachelet_1980}), другие авторы утверждают, что скорость роста и размеры моллюсков имеют тенденцию уменьшаться с уменьшением солености (\cite{Segerstrale_1960, Kube_et_al_1996}). 

Литературные данные о скорости роста моллюсков на различном мареографическом уровне противоречивы. 
Башле (\cite{Bachelet_1980}) обнаружил, что в эстуарии р.~Жиронда (южной границе ареала макомы в Европе) скорость роста моллюсков на верхней литорали значительно выше, чем на нижней. 
На верхней литорали моллюски достигают большего размера и дольше живут. 
Обратная связь найдена Грином (\cite{Green_1973}) и Харвеем и Винсентом (\cite{Harvey_Vincent_1990}) для канадских популяций {\it M.~balthica}. 
В качестве причины  таких различий авторы предполагают большее время питания на нижней литорали и негативное влияние высоких температур, ограничивающих рост, на верхней. 
Бьёкема и соавторы (\cite{Beukema_et_al_1977}) показали, что наибольшие скорости роста имеют моллюски со средней литорали, поскольку на верхней литорали скорость роста ограничивается временем питания, а на нижней~--- количеством пищи. 
В Белом море при сравнении темпов роста моллюсков из литоральных и сублиторального поселений, максимальный темп роста обнаружен в сублиторали. 
Однако различий в скорости роста между горизонтами литорали отчемено не было (\cite{Maximovich_et_al_1992}).
В Гданьском заливе скорость роста возрастала с глубиной~--- более высокая скорость роста обнаружена у моллюсков на глубине $35 - 75$~м, по сравнению с особями из мелководной ($5 - 6$~м) части залива (\cite{Wenne_Klusek_1985}). 
Обратная ситуация наблюдается в других частях Балтийского моря~--- минимальную скорость роста имеют моллюски с глубины $35$~м, максимальную с глубины $3$~м (\cite{Segerstrale_1960}). 

Таким образом, по-видимому сама по себе глубина обитания не оказывает влияние на темпы роста моллюсков. 
Кроме того, значительная подвижность маком затрудняет интерпретацию результатов. 
Скорость роста моллюсков определяются в первую очередь температурой и обилием пищи, а возникающая в ряде случаев зависимость от глубины может появляться за счет комбинирования этих параметров.

