	\section{Возрастная структура {\it Macoma balthica}}
		\subsection{Баренцево море)
	Максимальная продолжительность жизни маком, отмеченная нами, составляла 16 лет (рис. 7, 8). За два года наблюдений на всех участках не было отмечено сеголетков (особей возрастом 0+, что, по-видимому, было определено временем оседания молоди. Тем не менее, представленность на всех участках особей многих возрастов позволяет говорить о регулярном, хотя и не ежегодном пополнении исследованных поселений молодью. Рассмотрим характеристику возрастного состава поселений каждого из участков. 
	На литорали Абрам-мыса (рис. 7) были отмечены особи возрастом от 1 до 13 лет. В среднем горизонте литорали возрастная структура поселения M. balthica может быть обозначена как мономодальная с преобладанием особей возрастом 10+. В нижнем горизонте литорали кроме генераци возрастом 10+ были массово представлены особи возрастом 1+. 
	На участке в Пала-губе (рис. 7) во всех горизонтах литорали доминировали особи возрастом 2+. Однако в среднем горизонте литорали представлены моллюски возрастом от 1 до 8 лет, причем более старшие особи практически полностью отсутствуют. В нижнем горизонте литорали максимальный возраст был 14+, и кроме генерации третьего года жизни (2+) было представлено также несколько старших генераций, из которых наиболее многочисленны особи возрастом 8+. 
	В среднем горизонте губы Гаврилово (рис. 7) возраст особей колебался от 1 до 16 лет, при этом не были обнаружены особи возрастом 5,7 и 9 лет. Несколько преобладали особи возрастом 14+. На нижнем горизонте литорали были представлены лишт отдельные особи возрастом 3, 8, 9 и 12 лет, однако малая численноть не позволяет говорить о преобладании какой-либо генерации. 
	В губе Ярнышная (рис. 7) во всех горизонтах литорали возрастная структура была бимодальной. Однако, при рассмотрении модальных классов выясняется, что если в верхнем и нижнем горизонтах литорали доминируют особи 3-4 и 12-13 лет, то в среднем горизонте два пика создают особи возраста 2 и 4-8 лет.  
В губе Дальне-Зеленецкой (рис. 7) в верхнем и нижнем горизонтах литорали на фоне низкой численности формировалось практически равномерное распределение. В среднем горизонте литорали как в 2007, так и в 2008 г. первый пик формируют особи возраста 2+. В 2007 году второй пик — это генерация возрастом 6+. В 2008 г. особи этой генерации (7+) формируют лишь незначительное превышение, зато был отмечен пик, сформированный особями возрастом 11-13 лет. Отсутствие старших особей в сборах 2007 года связано скорее всего с агрегрованным распределением небольшого количества особей старших возрастов. По-видимому, при увеличении в 2008 г. объема выборки (табл. 1), такая агрегация попала в пробоотборник
	В губе Шельпино были отмечены лишь единичные особи разных взрастов. На верхнем горизонте литорали были представлены моллюски возрастом 5, 7, 8 и 12 лет. На среднем горизонте литорали особи были моложе: 2, 3 , 5, 9 и 10 лет.  
	На средней литорали губы Порчниха (рис. 7) были представлены моллюски возрастом от 1 до 10 лет. Отсутствовала генерация возратом 8+. Возрастная структура была бимодальной с преобладанием особей возрастом 3+ и 10+. 
	В сублиторали губы Ивановская (рис.8) были представлены макомы возрастом 1-10 лет. Распределение особей по возрастам было практически равномерным, отмечены лишь небольшие превышения в численности генераций возрастом 1+, 5+ и 8+. Меньше всего в поселении были представлены особи возраста 6+ и 10+. 
	При сравнении показателей размерной и возрастной структуры маком было выяснено, что характер распределения особей по размерам соответствует таковому по возрастам на всех участках. Если классифицировать представленные структуры по характеру распределения, то получается 3 варианта — мономодальное, бимодальное и равномерное (табл. 9). 
	При этом мономодальное распределение может формироваться либо за счет особей старших возрастов, либо, напротив, за счет молоди при полном отсутствии или низкой численности старших генераций.
	Бимодальное распределение формируется при  наличии в поселении на протяжении многих лет только двух массовых генераций, но преобладать по численности могут как более старые, так и более молодые особи. 
	Практически равномерное распределение в большинстве случаев описано на участках, где обилие маком чрезвычайно низко и в пробах попадаются единичные особи. Возможно, что при увеличении объема выборки получится выявить доминирующую генерацию. 
	Принципиально другая ситуация описана в сублиторали губы Ивановская. В данном случае практически равномерное распределение особей формируется при высоком обилии (численность более 1000 экз/кв.м, 280 особей в выборке), что позволяет нам говорить о надежности нашего описания. По-видимому, в условиях сублиторали губы Ивановская происходит ежегодное пополнение поселения молодью, причем смертность особей мало варьирует из года в год, что может быть связано с достаточной выравненностью условий в сублиторали относительно литорали. 

	Таким образом, зафиксированная максимальная продолжительность жизни M. balthica составила 16 лет. Во всех съемках не было отмечено моллюсков возрастом 0+. Характер размерной структуры значительно различался как от участка к участку, так и в зависимости от года исследования. На исследованных участках были представлены все типы размерно-возрастных структур: мономодальная, бимодальная и равномерное распределение, однако одинаковые структры могут формироваться при различном соотношении генераций.

