	\begin{enumerate}
%		\item Для Белого моря типичны поселения {\it Macoma balthica} с численностью  700 -- 800~экз./м$^2$ (при варьировании от 10 до 8500~экз./м$^2$). Варьирование обилия связано в первую очередь с численностью годовалых особей.
		\item В Кольском заливе Баренцева моря и Кандалакшском заливе  Белого моря значения биомассы (до 200 г/м$^2$) поселений {\it Macoma balthica} сопоставимы с аналогичным показателем в европейской части ареала, а плотность поселений нередко оказывается выше (до 8~тыс.~экз./м$^2$). Для литорали восточной части Мурманского побережья Баренцева моря типичны поселения {\it M.~balthica} с численностью менее 100 экз./м2 
		\item Плотность поселений спата {\it Macoma balthica} в Белом море может варьировать на порядок в пределах незначительной акватории, и достигать десятков тысяч экз./м$^2$.
		\item Беломорские и баренцевоморские поселения {\it M.~balthica} не различаются по средней скорости роста моллюсков, и отличаются по этому показателю минимальными характеристиками в пределах европейской части ареала вида. 
		\item Динамика размерной структуры поселений {\it Macoma balthica} в Белом и Баренцевом представлена двумя типами. \\
Наболее обычный вариант~--- чередование бимодального и мономодального распределений особей по размерам. При этом первый пик формируют молодые
особи (обычно длиной до 5 мм), а второй модальный класс состоит из взрослых особей (в Белом море длиной 9--12~мм, в Баренцевом море~--- 10--17~мм).
Как относительно редкое событие наблюдается мономодальная структура поселений с ежегодным преобладаем молоди.
		\item Динамика плотности поселений {\it Macoma balthica} в Кандалакшском заливе Белого моря демонстрирует элементы синхронности в поселениях, расположенных на расстоянии от 1 до 100~км, что происходит на фоне резкой межгодовой неравномерности пополнения поселений молодью.  

%		\item Для литорали восточной части Мурманского побережья Баренцева моря типичны поселения {\it Macoma balthica} с численностью  менее 100~экз./м$^2$, и эти поселения не достигают плотностей, которые показаны для поселений на литорали Западного Мурмана и в Кольском заливе. %(при варьировании от 30 до 3350~экз./м$^2$).
%		\item Среднее обилие {\it Macoma balthica} в поселениях Белого моря и Кольского залива Баренцева моря выше, чем в других частях ареала, а биомасса сравнима со значениями в центральной части ареала. 
%		\item Макомы в Баренцевом море гетерогенны по скорости роста: Максимальный годовой прирост отмечен у особей среднего размера (возраста) --- $6 - 9$~мм в среднем горизонте литорали. В пределах Восточного Мурмана средний годовой прирост особей {\it Macoma balthica} увеличивается в более восточных районах по сравнению с западными.
%		\item В пределах европейской части ареала особи {\it Macoma balthica} из поселений в Белом и Баренцевом морях характеризуются минимальными скоростями роста. При этом нет принципиальных различий в скорости роста беломорских и баренцевоморских маком.

%		\item Динамика численности годовалых особей {\it Macoma balthica} позволяет говорить о не ежегодном успехе пополнения их поселений в Белом море.
%		\item Динамика численности {\it Macoma balthica} в Кандалакшском заливе Белого моря демонстрирует элементы синхронности в поселениях, расположенных на расстоянии от 1 до 100~км. Кроме того, показано что численность маком оказывается выше в годы с холодными зимами.
%		\item Динамика размерной структуры поселений {\it Macoma balthica} в Белом и Баренцевом представлена двумя типами. \\
%Более распространенный вариант: чередование бимодального и мономодального распределение особей по размерам. При этом первый пик формируют молодые особи (обычно длиной до 5~мм), а в случае бимодального добавляется второй модальный класс из взрослых особей (в Белом море длиной 9 -- 12~мм, в Баренцевом 10 -- 17~мм). В Баренцевом море часто новое пополнение происходит до ухода старшей генерации и наблюдается три модальных группы. %Такой тип динамики связан с различной успешностью ежегодного пополнения поселений молодью и наличием внутривидовой конкуренции между взрослыми и молодыми особями.
%В некоторых условиях формируется более редкий тип динамики с ежегодным повторением мономодальной размерной структуры. %Возможно, это связано со специфическими условиями гидродинамики, в которых происходит разделение молодых и старых особей по способу питания и, таким образом, снижение внутривидовой конкуренции и возможность большего успеха ежегодного пополнения поселения молодью. Другое возможное объяснение --- формирование такого типа динамики в поселениях, находящихся под прессом хищников, которые уменьшают численность взрослых особей.
	\end{enumerate}

