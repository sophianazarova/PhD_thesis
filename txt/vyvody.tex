		\chapter*{Выводы}
\addcontentsline{toc}{chapter}{Выводы}	% Добавляем его в оглавление

	\begin{enumerate}
%		\item Для Белого моря типичны поселения {\it Macoma balthica} с численностью  700 -- 800~экз./м$^2$ (при варьировании от 10 до 8500~экз./м$^2$). Варьирование обилия связано в первую очередь с численностью годовалых особей.
		\item Для Белого моря типичны поселения {\it Macoma balthica} с численностью в сотни экз./м$^2$ (при варьировании от единичных особей до более 8~тыс.~экз./м$^2$). Варьирование обилия связано в первую очередь с численностью годовалых особей.
		\item Для литорали восточной части Мурманского побережья Баренцева моря типичны поселения {\it Macoma balthica} с численностью  менее 100~экз./м$^2$ (при варьировании от единичных особей до более 3~тыс.~экз./м$^2$). %(при варьировании от 30 до 3350~экз./м$^2$).
		\item Отдельные районы Кандалакшского залива Белого моря не различаются по средней численности особей {\it Macoma balthica}.
		\item Численность особей {\it Macoma balthica} в Баренцевом море на Восточном Мурмане ниже, чем на Западном и в Кольском заливе.
		\item Среднее обилие {\it Macoma balthica} в поселениях Белого моря и Кольского залива Баренцева моря выше, чем в других частях ареала, а биомасса сравнима со значениями в центральной части ареала. 
		\item Макомы в Баренцевом море гетерогенны по скорости роста: Максимальный годовой прирост отмечен у особей среднего размера (возраста) --- $6 - 9$~мм в среднем горизонте литорали. В пределах Восточного Мурмана средний годовой прирост особей {\it Macoma balthica} увеличивается в более восточных районах по сравнению с западными.
		\item Численность спата {\it Macoma balthica} в Белом море может варьировать на порядок в пределах незначительной акватории (от тысяч до десятков тысяч).
		\item Динамика численности годовалых особей {\it Macoma balthica} позволяет говорить о неежегодном успехе пополнения поселений в Белом море.
		\item Динамика численности {\it Macoma balthica} в Кандалакшском заливе Белого моря демонстрирует элементы синхронности в поселених, расположенных на расстоянии от 1 до 100~км. Кроме того, показана зависимость численности маком в текущий год от зимней температуры.
		\item Динамика размерной структуры поселений {\it Macoma balthica} в Белом и Баренцевом представлена двумя типами. \\
Более распространенный вариант: чередование бимодального и мономодального распределение особей по размерам. При этом первый пик формируют молодые особи (обычно длиной до 5~мм), а в случае бимодальной добавляется второй модальный класс из взрослых особей (в Белом море длиной 9 -- 12~мм, в Баренцевом 10 -- 17~мм). В Баренцевом море часто новое пополнение происходит до ухода старшей генерации и наблюдается три модальных группы. %Такой тип динамики связан с различной успешностью ежегодного пополнения поселений молодью и наличием внутривидовой конкуренции между взрослыми и молодыми особями.
В некоторых условиях формируется более редкий тип динамики с ежегодным повторением мономодальной размерной структуры. %Возможно, это связано со специфическими условиями гидродинамики, в которых происходт разделение молодых и старых особей по способу питания и, таким образом, снижение внутривидовой конкуренции и возможность большего успеха ежегодного пополнения поселения молодью. Другое возможное объяснение --- формирование такого типа динамики в поселениях, находящихся под прессом хищников, которые уменьшают численность взрослых особей.
	\end{enumerate}

