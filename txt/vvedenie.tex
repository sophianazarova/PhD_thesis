\chapter*{Введение}
\addcontentsline{toc}{chapter}{Введение}	% Добавляем его в оглавление

\paragraph {Актуальность темы и степень ее разработанности}
Двустворчатый моллюск {\it Macoma balthica} (Linnaeus, 1758) --- один из излюбленных модельных объектов в морских гидробиологических исследованиях. 
В классической биогеографии вид относят к амфибореальным. 
Это обычная литоральная форма в Белом море, у берегов Мурмана и далее на запад, вдоль атлантических берегов Европы --- до Франции. 
По Атлантическому побережью Северной Америки макомы распространены от Лабрадора до штата Джорджия. 
В северной части Тихого океана --- от Берингова моря до Японского, а по американскому побережью --- до Калифорнии. 
В юго-восточной части Баренцева моря и в прилегающей части Карского моря они обитают  не на литорали, а на глубине нескольких метров. 
Моллюски заселяют всю основную часть Балтийского моря, далеко заходя во все заливы, где живет до глубины более 100 метров (\cite{Zacepin_Filatova_1968}).

В настоящее время вид {\it Macoma balthica} по результатам аллозимного анализа предлагают разделять на два подвида: {\it M.~b.~balthica}, обитающий в северной части Тихоокеанского региона, и {\it M.~b.~rubra} из Северо-Восточной Атлантики. 
Однако  в морях, связанных с  Атлантикой, существуют очаги распространения тихоокеанской формы. 
Так, в Балтийском и Баренцевом море Атлантическая и Тихоокеанская формы сосуществуют и образуют гибриды (\cite{Vainola_2003}). 
В Белом море встречается в основном {\it M.~b.~balthica}, и лишь в устье Онеги было обнаружено два экземпляра {\it M.~b.~rubra} (\cite{Nikula_et_al_2007}).
К настоящему моменту нет прямых данных о влиянии данных генетических особенностей на экологические характеристики особей, поэтому в данной работе рассматривается вид {\it Macoma balthica} sensu lato.

{\it Macoma balthica} --- хорошо изученный вид в других частях ареала (см. например: \cite{Segerstrale_1960, Lavoie_1970, Gilbert_1978, Vincent_et_al_1989, Hiddink_et_al_2002_predation_epifauna, Hiddink_et_al_2002_predation_infauna, Beukema_et_al_2009}). 
Из арктических морей в настоящий момент поселения маком относительно хорошо изучены лишь в Белом море.

В Белом море макомы относятся к наиболее многочисленным обитателям илисто-песчаных пляжей. 
Эти моллюски являются одним из основных пищевых объектов для многих видов рыб и птиц Белого моря (\cite{Azarov_1963, Percov_1963, Golcev_et_al_1997, Bianki_et_al_2003}). 
Поэтому на территории Кандалакшского государственного природного заповедника {\it Macoma balthica} входит в список отслеживаемых видов кормовых беспозвоночных (\cite{Nazarova_2003}).

Также массовость и доступность для изучения позволяет использовать данный вид как удобную модель при анализе закономерностей развития поселений двустворчатых моллюсков. 
Именно поэтому локальные скопления маком Белого моря широко используются как объекты мониторинговых исследований, которые проводились и проводятся на всех крупных биологических стационарах на Белом море. 
В результате к настоящему моменту получены многолетние ряды данных, характеризующих  популяционные показатели маком на Белом море. 
При этом была отмечена существенность различий в организации локальных поселений маком (\cite{Semenova_1974, Maximovich_Kunina_1982, Maximovich_et_al_1991, Poloskin_1996, Nikolaeva_1998, Nazarova_2003, Nazarova_Poloskin_2005}).
 
Информации о поселениях маком в Баренцевом море значительно меньше. 
Детальные гидробиологические исследования сообществ мягких грунтов, в том числе  поселений {\it Macoma balthica}, на Мурмане относятся к 1970-м гг., однако основным полигоном для исследований стала лишь одна станция на литорали Дальнего пляжа губы Дальнезеленецкой (\cite{Agarova_et_al_1976}).
В 2002 году на Дальнем пляже была повторена количественная съемка бентоса и начат мониторинг сообществ (\cite{Genelt_Dalnezeleneckaya_2008})

Таким образом, к настоящему моменту данные по Баренцеву морю фрагментарны и не сформированы количественные представления о поселениях маком на Мурмане. 
По Белому морю информации значительно больше, но она относится к описанию отдельных локальных поселений, которые, на первый взгляд, весьма разнородны. 
Кроме того, до сих пор совершенно не изучен вопрос о факторах, влияющих на динамику поселений {\it Macoma balthica} в арктических морях. 
Данный вопрос подробно разобран для Ваттового моря (\cite{Hiddink_et_al_2002_predation_epifauna, Hiddink_et_al_2002_predation_infauna, Beukema_et_al_2009}), однако прямой перенос полученных результатов представляется невозможным из-за климатических различий между регионами.


\paragraph{Цели и задачи}
Целью данной работы стало изучение гетерогенности поселений {\it Macoma balthica} в условиях арктических морей.

Для достижения данной цели мы поставили следующие задачи.
  \begin{enumerate}
    \item Изучение размерной %и возрастной 
структуры в различных местообитаниях для описания эффектов внутрипопуляционной гетерогенности маком;
    \item изучение многолетней динамики поселений маком;
    \item изучение биотического и абиотического фона в поселениях;
%    \item изучение структуры сообществ макробентоса в изучаемых биотопах для выявления биотических взаимодействий видов;
%    \item изучение абиотических характеристик местообитаний (температура, соленость, осушка, грунт);
    \item изучение показателей линейного роста маком для шкалирования изученных поселений по степени оптимальности условий обитания;
%    \item изучение микрораспределения маком в местообитаниях для изучения хорологических аспектов формирования поселений маком;
    \item изучение численности спата для изучения механизмов, определяющих пополнение локальных поселений.
  \end{enumerate}

\paragraph{методология и методы исследования}

\paragraph{научная новизна}

\paragraph{теоретическая и практическая значимости работы}

\paragraph{положения, выносимые на защиту}

\paragraph{степень достоверности и апробацию результатов}
%Для решения поставленной цели были использованы мониторинговые наблюдения за 6 поселениями в Кандалакшском заливе.
%В Баренцевом море впервые проведены масштабные количественные описания поселений {\it M.~balthica}, всего 12 поселений.
%Исследования проводили общепринятыми гидробиологическими методами (\cite{Eleftheriou_2013}).



%\textcolor{red}{Тут должно быть что-то про:
%актуальность темы
%степень ее разработанности
%цели и задачи
%научная новизна
%теоретическая и практическая значимости работы
%методология и методы исследования
%положения, выносимые на защиту
%степень достоверности и апробацию результатов}
