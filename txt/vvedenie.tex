\section{Введение}
Двустворчатый моллюск {\it Macoma balthica} (Linnaeus, 1758) --- один из излюбленных модельных объектов в морских гидробиологических исследованиях. 
В классической биогеографии вид относят к амфибореальным. 
Это обычная литоральная форма в Белом море, у берегов Мурмана и далее на запад, вдоль атлантических берегов Европы --- до Франции. 
По Атлантическому побережью Северной Америки макомы распространены от Лабрадора до штата Джорджия. 
В северной части Тихого океана --- от Берингова моря до Японского, а по американскому побережью --- до Калифорнии. 
В юго-восточной части Баренцева моря и в прилегающей части Карского моря они обитают  не на литорали, а на глубине нескольких метров. 
Моллюски заселяют всю основную часть Балтийского моря, далеко заходя во все заливы, где живет до глубины более 100 метров (\cite{Zacepin_Filatova_1968}).

{\it Macoma balthica} --- хорошо изученный вид в других частях ареала (см. например: \cite{Segerstrale_1960, Lavoie_1970, Gilbert_1978 , Vincent_et_al_1989}). 
Из арктических морей в настоящий момент поселения маком хорошо изучены лишь в Белом море.

В Белом море макомы относятся к наиболее многочисленным обитателям илисто-песчаных пляжей. 
Эти моллюски являются одним из основных пищевых объектов для многих видов рыб и птиц Белого моря (\cite{Azarov_1963, Percov_1963, Golcev_et_al_1997, Bianki_et_al_2003}). 
Также массовость и доступность для изучения позволяет использовать данный вид как удобный модельный вид при анализе закономерностей развития поселений двустворчатых моллюсков. 
Именно поэтому локальные скопления маком Белого моря широко используются как объекты мониторинговых исследований, которые проводились и проводятся на всех крупных биологических стационарах на Белом море (\cite{Semenova_1974, Maximovich_Kunina_1982, Maximovich_et_al_1991, Poloskin_1996, Nikolaeva_1998, Nazarova_2003, Nazarova_Poloskin_2005}). 

В результате к настоящему моменту получены многолетние ряды данных, характеризующих  популяционные показатели маком на Белом море. 
При этом была отмечена существенность различий в организации локальных поселений маком. 
Стационарные во времени плотные поселения маком (\cite{Semenova_1974, Maximovich_et_al_1991}) с обилием 200-2500 экз./м2, описаны как поселения с ежегодно повторяющейся структурой (\cite{Maximovich_et_al_1991}).  
Поселения, отличающиеся значительные флуктуации численности {\it Macoma balthica} (\cite{Maximovich_1985, Nazarova_Poloskin_2005}), возникают как результат многолетней цикличности при смене генераций (\cite{Maximovich_et_al_1991, Nazarova_2003}).

Информации о поселениях маком в Баренцевом море значительно меньше. 
Детальные гидробиологические исследования сообществ мягких грунтов, в том числе  поселений {\it Macoma balthica}, на Мурмане относятся к 1970-м гг., однако основным полигоном для исследований стала лишь одна станция на литорали Дальнего пляжа губы Дальнезеленецкой (\cite{Agarova_et_al_1976}).
В 2002 году на Дальнем пляже была повторена количественная съемка бентоса и начат мониторинг сообществ (\cite{Genelt_Dalnezeleneckaya_2008})

В настоящее время вид {\it Macoma balthica} (sensu lato) по результатам аллозимного анализа  разделяют на два подвида: {\it M.~b.~balthica}, обитающий в северной части Тихоокеанского региона, и {\it M.~b.~rubra} из Северо-Восточной Атлантики. 
Однако  в морях, связанных с  Атлантикой, существуют очаги распространения тихоокеанской формы. 
Так, в Балтийском и Баренцевом море Атлантическая и Тихоокеанская формы сосуществуют и образуют гибриды (\cite{Vainola_2003}). 
В Белом море встречается {\it M.~b.~balthica}, и лишь в устье Онеги было обнаружено два экземпляра {\it M.~b.~rubra} (\cite{Nikula_et_al_2007}).




\textcolor{red}{Тут должно быть что-то про:
актуальность темы
степень ее разработанности
цели и задачи
научная новизна
теоретическая и практическая значимости работы
методология и методы исследования
положения, выносимые на защиту
степень достоверности и апробацию результатов}
