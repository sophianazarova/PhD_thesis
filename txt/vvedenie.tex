\paragraph {Актуальность темы и степень ее разработанности.}
Двустворчатый моллюск {\it Macoma balthica} (Linnaeus, 1758)~--- один из излюбленных модельных объектов в морских гидробиологических исследованиях. 
В классической биогеографии вид относят к амфибореальным. 
Это обычная литоральная форма в Белом море, у берегов Мурмана и далее на запад, вдоль атлантических берегов Европы~--- до Франции. 
По Атлантическому побережью Северной Америки макомы распространены от Лабрадора до штата Джорджия. 
В северной части Тихого океана~--- от Берингова моря до Японского, а по американскому побережью~--- до Калифорнии. 
В юго-восточной части Баренцева моря и в прилегающей части Карского моря они обитают  не на литорали, а на глубине нескольких метров. 
Моллюски заселяют всю основную часть Балтийского моря, далеко заходя во все заливы, где живут до глубины более 100 метров (\cite{Zacepin_Filatova_1968}).

В настоящее время вид {\it Macoma balthica} по результатам аллозимного анализа предлагают разделять на два подвида: {\it M.~b.~balthica}, обитающий в северной части Тихоокеанского региона, и {\it M.~b.~rubra} из Северо-Восточной Атлантики. 
Однако  в морях, связанных с  Атлантикой, существуют очаги распространения тихоокеанской формы. 
Так, в Балтийском и Баренцевом море Атлантическая и Тихоокеанская формы сосуществуют и образуют гибриды (\cite{Vainola_2003}). 
В Белом море встречается в основном {\it M.~b.~balthica}, и лишь в устье Онеги было обнаружено два экземпляра {\it M.~b.~rubra} (\cite{Nikula_et_al_2007}).
К настоящему моменту нет прямых данных о влиянии данных генетических особенностей на экологические характеристики особей, поэтому в данной работе рассматривается вид {\it Macoma balthica} sensu lato.


{\it Macoma balthica}~--- хорошо изученный вид в центральной части ареала (см. например: \cite{Segerstrale_1960, Lavoie_1970, Gilbert_1978, Vincent_et_al_1989, Hiddink_et_al_2002_predation_epifauna, Hiddink_et_al_2002_predation_infauna, Beukema_et_al_2009}). 
Из арктических морей в настоящий момент поселения маком относительно хорошо изучены лишь в Белом море.


В Белом море макомы относятся к наиболее многочисленным обитателям илисто-песчаных пляжей. 
Эти моллюски являются одним из основных пищевых объектов для многих видов рыб и птиц Белого моря (\cite{Azarov_1963, Percov_1963, Golcev_et_al_1997, Bianki_et_al_2003}). 
Поэтому на территории Кандалакшского государственного природного заповедника {\it Macoma balthica} входит в список отслеживаемых видов кормовых беспозвоночных (\cite{Nazarova_2003}).

Также массовость и доступность для изучения позволяет использовать данный вид как удобную модель при анализе закономерностей развития поселений двустворчатых моллюсков. 
Именно поэтому локальные скопления маком Белого моря широко используются как объекты мониторинговых исследований, которые проводились и проводятся на всех крупных биологических стационарах на Белом море. 
В результате к настоящему моменту получены многолетние ряды данных, характеризующих  популяционные показатели маком на Белом море. 
При этом была отмечена существенность различий в организации локальных поселений маком (\cite{Semenova_1974, Maximovich_Kunina_1982, Maximovich_et_al_1991, Poloskin_1996, Nikolaeva_1998, Nazarova_2003, Nazarova_Poloskin_2005}).
 
Информации о поселениях маком в Баренцевом море значительно меньше. 
Детальные гидробиологические исследования сообществ мягких грунтов, в том числе  поселений {\it Macoma balthica}, на Мурмане относятся к $1970$-м годам, однако основным полигоном для исследований стала лишь одна станция на литорали Дальнего пляжа губы Дальне-Зеленецкой (\cite{Agarova_et_al_1976}).
В $2002$ году на Дальнем пляже была повторена количественная съемка бентоса и начат мониторинг сообществ (\cite{Genelt_Dalnezeleneckaya_2008}).

Таким образом, к настоящему моменту данные по Баренцеву морю фрагментарны, а количественные представления о поселениях маком на Мурмане не сформированы. 
По Белому морю информации значительно больше, но она относится к описанию отдельных локальных поселений, которые на первый взгляд весьма разнородны. 
Кроме того, до сих пор совершенно не изучен вопрос о факторах, влияющих на динамику поселений {\it Macoma balthica} в арктических морях. 
Данный вопрос подробно разобран для Ваттового моря (\cite{Hiddink_et_al_2002_predation_epifauna, Hiddink_et_al_2002_predation_infauna, Beukema_et_al_2009}), однако прямой перенос полученных результатов представляется невозможным из-за климатических различий между регионами.


\paragraph{Цели и задачи.}
Целью данной работы является изучение гетерогенности поселений {\it Macoma balthica} в условиях арктических морей.

Для достижения данной цели были поставлены следующие задачи.
  \begin{enumerate}
    \item Изучение размерной %и возрастной 
структуры в различных местообитаниях для описания эффектов внутрипопуляционной гетерогенности маком;
    \item изучение многолетней динамики поселений маком;
    \item изучение биотического и абиотического фона в поселениях;
%    \item изучение структуры сообществ макробентоса в изучаемых биотопах для выявления биотических взаимодействий видов;
%    \item изучение абиотических характеристик местообитаний (температура, соленость, осушка, грунт);
    \item изучение показателей линейного роста маком для шкалирования изученных поселений по степени оптимальности условий обитания;
%    \item изучение микрораспределения маком в местообитаниях для изучения хорологических аспектов формирования поселений маком;
    \item изучение численности спата для изучения механизмов, определяющих пополнение локальных поселений.
  \end{enumerate}

\paragraph{Методология и методы исследования.}
Для решения поставленной цели в акватории Белого моря были использованы мониторинговые наблюдения за шестью поселениями в Кандалакшском заливе.
В Баренцевом море были проведены масштабные количественные описания поселений {\it M.~balthica}, всего $12$ поселений.
Полевые сборы проводили общепринятыми гидробиологическими методами (\cite{Eleftheriou_2013}) при помощи литоральных рамок (площадью от $1/30$ до $1/10$~м$^2$).
Для обработки данных использовали как традиционные методы статистического анализа (\cite{Tukey_1977, Mardia_et_al_1979, Chambers_Hastie_1991, Legendre_Legendre_2012, Hollander_et_al_2013}) так и относительно новые методы анализа многомерных данных (\cite{Clarke_et_al_2008}) и моделирования (\cite{Berryman_Turchin_2001}).

\paragraph{Научная новизна.}
В рамках данной работы впервые проведены масштабные количественные исследования поселений \textit{M.~balthica} на литорали Мурманского побережья Баренцева моря и получены характеристики их обилия и данные по изменчивости линейного роста маком в пределах Мурмана.
Впервые описана многолетняя динамика обилия поселений \textit{M.~balthica} в вершине Кандалакшского залива и показана синхронность пополнения поселений молодью.
Моделирование показало, что колебания численности маком зависят от зимней температуры.

\paragraph{Теоретическая и практическая значимости работы.}
В работе получены фундаментальные данные, описывающие поселения \textit{M.~balthica} в Белом и Баренцевом морях, при этом впервые дано количественное описание типичных поселений данного вида в Баренцевом море. 
Полученные данные могут быть использованы при оценке запасов кормовых беспозвоночных для хозяйственно-ценных видов рыб и птиц.
Проведено моделирование динамики численности \textit{M.~balthica} и показано влияние температуры на данный показатель, что может быть использовано для прогнозирования обилия маком. 
Проведенный анализ широтных изменений численности \textit{M.~balthica} показал, что распределение маком по данному показателю не соответствует широко-распространенной  <<гипотезе об обилии в центре>> (<<abindant-centre hypothesis>>, \cite{Sagarin_et_al_2006}), и может быть использован в критике данных представлений в биогеографических обзорах.
Результаты исследования могут быть использованы также в курсах лекций по гидробиологии, популяционной биологии, репродуктивной экологии морского бентоса и биогеографии в ВУЗах.

\paragraph{Положения, выносимые на защиту.}
\begin{enumerate}
\item \textit{Macoma balthica} на литорали Белого и Баренцева моря образуют разные по структуре поселения.
На литорали Кандалакшского залива Белого моря и в Баренцевом море (Западный Мурман и Кольский залив) вид  формирует плотные поселения, в которых численность особей значительно варьирует во времени и может достигать нескольких тысяч экз./м$^2$, но наиболее типичны поселения маком с плотностью в несколько сотен экз./м$^2$. 
При этом среднее обилие \textit{M.~balthica} в Кандалакшском заливе Белого моря и в Кольском заливе Баренцева моря наибольшее в пределах европейской части ареала вида.
На литорали Восточного Мурмана Баренцева моря \textit{M.~balthica} не формирует плотных поселений, и ее численность редко превышает 100~экз./м$^2$.

\item Характер динамики численности \textit{Macoma balthica} в Белом и Баренцевом морях определяется варьированием численности однолетних особей в поселениях, которое зависит от нерегулярности пополнения поселений молодью, обусловленной в первую очередь различным уровнем выживаемости на первом году жизни.
Беломорские поселения демонстрируют элементы синхронности процессов пополнения, что связано с влиянием температуры на выживаемость маком в первый год жизни  (численность однолетних особей после холодных зим с устойчивым ледоставом оказывается относительно выше) и спецификой условий в локальном местообитании.

\item Динамика размерной структуры поселений {\it Macoma balthica} в Белом и Баренцевом представлена двумя типами. 
Более распространенный вариант: чередование бимодального и мономодального характера распределения особей по размерам. 
При этом первый пик формируют молодые особи (обычно длиной до 5~мм), а в случае бимодальной добавляется второй модальный класс из взрослых особей (в Белом море длиной 9 -- 12~мм, в Баренцевом 10 -- 17~мм). 
В Баренцевом море часто новое пополнение происходит до ухода старшей генерации и наблюдается три модальных группы.
В некоторых условиях формируется более редкий тип динамики с ежегодным повторением мономодальной размерной структуры. 

\item Особи {\it Macoma balthica} в Белом и Баренцевом морях отличаются наименьшей скоростью роста в пределах европейской части ареала вида. 
При этом внутригрупповая вариация роста особей \textit{M.~balthica} в поселениях Белого и Баренцева моря практически полностью перекрывается.

\end{enumerate}


\paragraph{Степень достоверности и апробация результатов.}
Достоверность изложенных результатов определяется достаточным количеством обработанных проб, отобранных в биотопах, разнообразие которых отражает спектр типичных местообитаний \textit{M.~balthica} в исследованных акваториях.
Для обработки полученных данных использованы современные статистические методы, позволяющие верифицировать выдвигаемые гипотезы.

Полученные результаты были апробированы в ходе докладов на 
$46$-м (Ровинь, 2011), $49$-м (Санкт-Петербург, $2014$) и $50$-м (Хельголанд, $2015$) Европейских морских биологических симпозиумах (European marine biology symposium); 
$VI$ всероссийской школе по морской биологии <<Биоразнообразие сообществ морских и пресноводных экосистем России>> (Мурманск, $2007$); 
научных сессиях Беломорской биологической станции МГУ (Пояконда, $2004$, $2008$); 
научных сессиях Морской биологической станции СПбГУ (Санкт-Петербург, $2004$, $2008$, $2009$, $2010$); 
$X$ научном семинаре <<Чтения памяти К.М.~Дерюгина>> (Санкт-Петербург, $2008$),
а также на семинарах кафедры  ихтиологии и гидробиологии СПбГУ (Санкт-Петербург, $2003 - 2015$).

