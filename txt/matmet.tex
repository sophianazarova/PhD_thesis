% для компиляции в lualatex!!
%\documentclass[12pt, a4paper]{article}
\documentclass[12pt, a4paper]{disser}
\usepackage[english,russian]{babel}
\usepackage[warn]{mathtext}
%\usepackage[T2A]{fontenc}
%\usepackage[utf8]{inputenc}

\usepackage{xecyr} % Продукт Вашего покорного слуги ;)

\setmainfont{DejaVu Serif}

\usepackage{color}
\usepackage{amssymb,amsmath}
\usepackage{graphicx}
\usepackage{multicol}

\textheight=24cm           % высота текста
\textwidth=16cm            % ширина текста
\oddsidemargin=0pt         % отступ от левого края
\topmargin=-1.5cm          % отступ от верхнего края
\parindent=24pt            % абзацный отступ
\parskip=0pt               % интервал между абзацами
\tolerance=2000            % терпимость к "жидким" строкам
\flushbottom               % выравнивание высоты страниц
%\def\baselinestretch{1.5} % печать с большим интервалом

%\title{}
%\author{\copyright~~С.А.~Назарова \thanks{e-mail:~sophia.nazarova@gmail.com}}
%\date{}


\begin{document}

\chapter{Материал и методика}
	\section{География исследований}
		\subsection*{Баренцево море}
Материал  в акватории Баренцева моря  был  собран   в  августе  $2007--2008$  гг. в ходе   студенческой баренцевоморской экспедиции СПбГУ. 
Всего было исследовано $8$ участков --- $2$ в Кольском заливе (рис.~\ref{karta_Kolskiy})   и   $6$  в   прибрежной   зоне  Восточного  Мурмана (рис.~\ref{karta_Vost_Murman}).  
Участки   в   Кольском   заливе   были  расположены на побережье в районе Абрам-мыса и в Палa-губе, в районе города Полярный. 
Участки  осушной   зоны  на   Восточном   Мурмане   были   расположены   в   губах   Гавриловская,  Ярнышная, Дальнезеленецкая, Шельпинская, Порчниха и Ивановская.

Также в работе использованы материалы экспедиции по мониторингу Дальнего пляжа губы Дальнезеленецкой с 2002 года, любезно предоставленные Е.~А.~Генельт-Яновским. 



		\subsection*{Белое море}
Все беломорские материалы в данной работе относятся к акватории Кандалакшского залива. 

В вершине Кандалакшского залива наблюдения проводили на $5$ мониторинговых участках в рамках работы экспедиций Группы исследований прибрежных сообществ Лаборатории экологии морского бентоса (гидробиологии) СПбГДТЮ (рис.~\ref{karta_Kandalaksha}). 
Автор принимала участие в полевых сборах с $1999$ по $2007$ год.
Данные за другие годы взяты из архива ГИПС ЛЭМБ.

Три участка расположены в районе Лувеньгских шхер: эстуарий реки Лувеньги, Илистая губа острова Горелого и участок материковой литорали в 800 метрах западнее поселка Лувеньга.
Один участок был расположен на литорали острова Ряшков в Западной Ряшковой салме (Северный архипелаг).
Также в работе использованы данные Д.~А.~Аристова из Южной губы о.~Ряшков и с о.~Ломнишный(Северный архипелаг).

В районе губы Чупа исследования проводили на $4$ участках (рис.~\ref{karta_Chupa}) в ходе экспедиций кафедры ихтиологии и гидробиологии СПбГУ. 
Два участка были расположены на литорали острова Кереть --- в Сухой салме и бухте Клющиха. 
Один участок был расположен на материковой литорали пролива Подпахта и один --- в бухте Лисьей.


	\section{Описание сообществ, включающих {\it M.~balthica}}
\textcolor{red}{А что тут про Беломорских?..}

  На каждом участке в акватории Баренцева моря исследовали все  горизонты литорали, представленные мягкими грунтами.  
На каждом горизонте отбирали от $5$ до $87$ проб  (табл.  \ref{tab:}). Таким образом, всего было составлено $16$ описаний.

Как основное орудие сбора использовали литоральную рамку площадью $1/30$~м$^2$, из которой изымали грунт на глубину $5$~см. 
В случае, когда приходилось отбирать пробы из-под воды, использовали зубчатый водолазный дночерпатель площадью захвата $1/20$~м$^2$.
Отобранные пробы промывали на сите с диаметром ячеи $1$~мм. 
После промывки из   проб   выбирали   всех   особей  {\it Macoma   balthica}  и   представителей   сопутствующего макрозообентоса    для   определения   состава   сообщества.
   Представителей   сопутствующего макрозообентоса  определяли   до   минимально   возможного   таксона.



	\section{Изучение микрораспределения {\it M.~balthica}}

\textcolor{red}{квадраты на Белом}

\textcolor{red}{квадраты на Баренцевом}

\textcolor{red}{схема Траша на Баренцевом}
% из Генельт-Кобылков-Назарова, 2007 (что это было??)

Исследования были проведены в августе $2007$~г. на илисто-песчаной литорали кутовых участков губ Восточного Мурмана – Ярнышной и Дальне-Зеленецкой, и в октябре $2007$~г. на литорали Пала-губы (Кольский залив). 

В каждой точке отбиралось по $36$ проб площадью $1/30$~м$^2$, расположенных в пределах участка размером $7,5 \times 12$~м. 
Координаты каждой пробы были определены в декартовой системе координат в метрах, один из углов участка служил точкой отсчета. 
В дальнейшем пробы промывали на сите с диаметром ячеи $1$~мм. 
В лаборатории были выбраны и  подсчитаны все моллюски, ракообразные и приапулиды.

При дальнейшей обработке данных для каждого участках подсчитывали индекс структурности (отношение дисперсии к средней арифметической). 
Для анализа размеров агрегаций были построены коррелограммы, основанные на коэффициенте пространственной автокорреляции Морана.

\textcolor{red}{Пятиметровый лаг}

Наличие градиентов определяли с использованием корреляционного анализа Кенделла между координатами проб и обилием вида в каждой пробе. 
Все статистические анализы проводили с $95\%$ доверительной вероятностью ($P < 0,05$).


	
	\section{Изучение структуры поселений {\it Macoma balthica}}

  На каждом участке в акватории Баренцева моря исследовали все  горизонты литорали, представленные мягкими грунтами.  
На каждом горизонте отбирали от $5$ до $87$ проб  (табл.  \ref{tab:}). Всего было составлено $16$ описаний.

Как основное орудие сбора использовали литоральную рамку площадью $1/30$~м$^2$, из которой изымали грунт на глубину $5$~см. 
В случае, когда приходилось отбирать пробы из-под воды, использовали зубчатый водолазный дночерпатель площадью захвата $1/20$~м2.
Отобранные пробы промывали на сите с диаметром ячеи $1$~мм. 
После промывки из   проб   выбирали   всех   особей  {\it Macoma balthica}.

   Маком   измеряли  с точностью $0,1$~мм  и взвешивали с точностью $10$~мг.
 У всех особей измеряли максимальный линейный размер (длину). 
В дальнейшем по этим данным строили графики размерной структуры поселений. 

Кроме того, у моллюсков подсчитывали количество меток зимней остановки роста, которое принимали как возраст моллюсков ---­ число прожитых зим (например, $4+$ это  особи возрастом от $4$ до $5$ лет).   
Таким   образом   были   получены   оценки возрастной структуры поселений {\it M. balthica}.

	\section{Изучение динамики поселений {\it M.~balthica}}
В Белом море динамику поселений {\it M.~balthica} исследовали на $6$ участках в районе вершины Кандалакшского залива. 

Сборы проводили с 1992 по 2012 год ежегодно в июле-августе.

Структура материала представлена в таблице \ref{tab:material_Kandalaksha}.
\begin{table}
\begin{tabular}{|*{5}{p{0.2\textwidth}|}} \hline
участок & годы наблюдения & обследованные горизонты литорали & количество проб в однократной съемке & площадь пробоотборника  \\ \hline
о. Горелый Лувеньгских шхер & 1992 -- 2012 & ВГЛ, СГЛ, НГЛ & 1-3 & 1/30, 1/10 \\ \hline
Материковая литораль в районе пос. Лувеньга & 1992-2000, 2002, 2004 & ВГЛ, СГЛ, НГЛ & 12-20 & 1/30 \\ \hline
Эстуарий р. Лувеньги & 1992 -- 2012 & СГЛ & 3 & 1/10 \\ \hline
Литораль Западной Ряшковой салмы о. Ряшкова & 1994 -- 2012 & СГЛ & 2 & 1/10 \\ \hline
Южная губа о. Ряшкова & 2001 -- 2012 & НГЛ & 9-16 & 1/30 \\ \hline
о. Ломнишный & 2007 -- 2012 & НГЛ & 5-10 & 1/30  \\ \hline
\end{tabular}
\caption{Структура материала по динамике поселений{\it M.~balthica} вершины Кандалакшского залива}
\label{tab:material_Kandalaksha}
\end{table}

На каждом исследованном участке отбирали $3 -- 25$ проб площадью $1/30 -- 1/10$~м$^2$, которые затем промывали на сите с диаметром ячеи $0,5 -- 1$~мм. 
В пробах учитывали всех особей {\it Macoma balthica}, у которых в дальнейшем измеряли максимальный линейный размер (длину) с точностью ~0,1~мм. 

Для определения биомассы моллюсков взвешивали на электронных весах с точностью до $1$~мг. 
Для серий проб, где не проводили взвешивание моллюсков, биомассу определяли рассчетным методом с использованием аллометриеской зависимости сырой массы маком от длины их раковины \cite{Maximovich_et_al_1993}.

В дальнейшем рассчитывали показатели средней численности маком на квадратный метр (плотность поселения) и размерно-частотное распределение особоей.
Для построения размерно-частотного распределения шаг размерного класса составлял $1$~мм.

В дальнейшем при анализе мы работали с особями с длиной раковины более $1,0$~мм по двум причинам. 
Во-первых, для того чтобы сделать сравнимыми результаты с разных участков, где пробы промывались на ситах с разным диаметром ячеи. 
Во-вторых, пробы отбирали в середине лета, то есть к этому моменту молодь этого года частично осела, то есть оценка численности данной группы будет некорректна.
Мы считаем корректной такую редукцию материала, поскольку для Белого моря показано, что усешность пополнения поселений молодью в первую очередь зависит от выживаемости спата зимой (\textcolor{red}{тут ссылка на каких-то Максимовича-Герасимову. 2004 - БиНИИ? или 2012 - Hydrobiology}).

Для анализа динамики пополнения поселений молодью в $2012 - 2013$ годах у особей длиной менее $3$~мм были измерены длины колец зимней остановки роста. 
После определения размеров годовалых особей, по размерной было рассчитано их обилие в каждом году мониторингового наблюдения.
Всего было промерено \textcolor{red}{проверить, сколько промерено} 496 особей.





 
%методика из Назаровой-Полоскина
%Материал собран в августе 1992 - 2003 гг. Изучено три литоральных поселения маком: в Илистой губе острова Горелого (участок 1), в эстуарии реки Лувеньги (участок 2) и на материковой литорали к югу от поселка Лувеньга (участок 3). Сборы проведены пробоотборником площадью захвата 1/30 м2. Разовая выборка составляла от 9 до 25 проб с участка. Грунт выбирался до глубины 5 см и промывался на сите с диаметром ячеи 0.5мм.  Всех особей  M. balthica измеряли с точностью 0.1 мм. В каждый момент наблюдений определяли размерную структуру и плотность поселения маком. 



% методика из Аристова
%Оба участка закрыты от волнового воздейст-
%вия. Литораль в районе исследований представляет собой песчаный пляж с примесью
%ила с вкраплениями крупных валунов. Спуск в сублитораль пологий, отчетливо вы-
%раженный пояс фукоидов отсутствует. Население представлено типичными формами,
%такими как Arenicola marina, Macoma balthica, Mya arenaria, Hydrobia ulvae, Microspio sp.
%и др. (Д. А. Аристов, неопубликованные данные).
%В обеих точках производили сборы по следующей методике: в районе нуля глубин
%во время отлива в пределах участков случайным образом выбирали и обследовали не-
%сколько площадок. Поскольку радиусы индивидуальной активности A. islandica и пред-
%полагаемых жертв (двустворчатых моллюсков) существенно различаются, в пределах
%каждой площадки брали пару проб методом вложенных рамок. Первую пробу из пары
%(1/4 м2) брали для учета A. islandica. Грунт из пробы тщательно перебирали вручную,
%всех найденных представителей сем. Naticidae подсчитывали и определяли их видовую
%принадлежность. Вторая проба (1/30 м2) была взята для учета потенциальных жертв —
%двустворчатых моллюсков. Грунт из нее промывали на сите с диаметром ячеи 1 мм,
%а затем остаток разбирали в фотографической кювете с белым дном. Из грунта соби


% из Дерюгинских, 2007

В Баренцевом море динамику поселений маком исследовали на модельном участке --- литоральной отмели Дальний пляж губы Дальнезеленецкой. 
 Материал был собран в июле-августе $2002 -- 2008$~гг. в пределах от верхнего горизонта песчаной литорали ($+2,0$~м) до $+0,7$~м над нулем глубин. 

Учет видов-эдификаторов ({\it Arenicola marina} и {\it Fabricia sabella}), а также Pygospio elegans, проводился на трех станциях. Из этих станций две (№1 и 2) располагаются в пределах сообществ, имеющих наибольшую площадь, и одна (№3) в переходной зоне между этими сообществами.  На каждой станции в 5 рамках площадью ¼ м2 проводился визуальный учет Arenicola marina, а двух других видов – в 3 пробах с площади 1/245 м2. 
На каждой из этих трех станций, а также на 6 дополнительных станциях отбирали по 3-5 рамок 1/10 м2 для изучения структуры поселений двустворчатых моллюсков. Пробы промывали на сите с диаметром ячеи $1$~мм. 

У всех двустворчатых моллюсков измеряли длину раковины с точностью $0,1$~мм. 

Восстановление возрастной структуры Macoma balthica для 2002-06 годов было проведено по моллюскам из выборки 2007 года. Для этого у них были измерены кольца зимней остановки роста и рассчитаны размеры моллюсков каждой возрастной группы. В качестве границ размерно-возрастных классов принималась середина соответствующего размерного диапазона. В дальнейшем, в зависимости от длины, каждого моллюска из выборок 2002-06 гг. относили к определенному возрастному классу. Так как в 2007 году не были встречены особи с 8 видимыми кольцами зимней остановки роста, то все особи крупнее 17,7 мм (верхняя размерная граница возрастного класса 8+) были объединены в одну группу.



	\section{Изучение линейного роста {\it M.~balthica}}

Кроме того, у моллюсков измеряли расстояние от вершины до каждой ростовой метки для последующей реконструкции линейного роста. 

\end{document}
