\documentclass[a4paper,12pt,pdftex]{report}


%%% Поля и разметка страницы %%%
\usepackage{lscape} % Для включения альбомных страниц
\usepackage{geometry} % Для последующего задания полей
\usepackage{float}

%%% Кодировки и шрифты %%%
\usepackage{cmap} % Улучшенный поиск русских слов в полученном pdf-файле
\usepackage[T2A]{fontenc} % Поддержка русских букв
\usepackage[utf8]{inputenc} % Кодировка utf8
\usepackage[english, russian]{babel} % Языки: русский, английский
%\usepackage{pscyr} % Нормальные шрифты

\usepackage{extsizes} % Возможность сделать 14-й шрифт
\usepackage{anyfontsize}

%%% Математические пакеты %%%
\usepackage{amsthm,amsfonts,amsmath,amssymb,amscd} % Математические дополнения от AMS
\usepackage{icomma} % "Умная" запятая: $0,2$ --- число, $0, 2$ --- перечисление

%%% Оформление абзацев %%%
\usepackage{indentfirst} % Красная строка

%%% Цвета %%%
\usepackage[usenames]{color}
\usepackage{color}
\usepackage{colortbl}

%%% Таблицы %%%
\usepackage{longtable} % Длинные таблицы
\usepackage{multirow,makecell,array} % Улучшенное форматирование таблиц

%%% Общее форматирование
%\usepackage[singlelinecheck=off,center]{caption} % Многострочные подписи
\usepackage{caption2}%форматирование подписей к плавающим объектам
\renewcommand{\captionlabeldelim}{.}% после названия объекта ставим точку.
\usepackage{soul} % Поддержка переносоустойчивых подчёркиваний и зачёркиваний

%%% Библиография %%%
%\usepackage{cite}

%%% Гиперссылки %%%
%\usepackage[plainpages=false,pdfpagelabels=false]{hyperref}
\usepackage[linktocpage=true,plainpages=false,pdfpagelabels=false]{hyperref}

%%% Изображения %%%
\usepackage{graphicx} % Подключаем пакет работы с графикой

%%% Опционально %%%
% следующий пакет может быть полезен, если надо ужать текст, чтобы сам текст не править, но чтобы места он занимал поменьше
%\usepackage{savetrees}        

% этот пакет может быть полезен для печати текста брошюрой, сама с ним не разбиралась
%\usepackage[print]{booklet}

\newcommand{\R}{R} %Rlogo

\usepackage{wasysym}

%%%%%%%%%%%%%%%%%%%%%%%%%%%%%%%%%%%%%%%%%%%%%

%%% Макет страницы %%%


\geometry{top=1.5cm,bottom=1.5cm,left=1.5cm,right=1cm}
%\oddsidemargin=-13pt
%\topmargin=-66pt
%\headheight=12pt
%\headsep=38pt
%\textheight=732pt
%\textwidth=484pt
%\marginparsep=14pt
%\marginparwidth=43pt
%\footskip=14pt
%\marginparpush=7pt
%\hoffset=0pt
%\voffset=0pt
%\paperwidth=597pt
%\paperheight=845pt
\parindent=1cm %размер табуляции (для красной строки) в начале каждого абзаца
\renewcommand{\baselinestretch}{1.25}
%\newfloat{scheme}{tb}{sch}

%%% Общая информация %%%
\author{Назарова С.А.} % Фамилия И.О. автора

%%% Кодировки и шрифты %%%
%\renewcommand{\rmdefault}{ftm} % Включаем Times New Roman

%%% Выравнивание и переносы %%%
\sloppy
\clubpenalty=10000
\widowpenalty=10000

%%% Библиография %%%
%\makeatletter
%\bibliographystyle{utf8gost705u} % Оформляем библиографию в соответствии с ГОСТ 7.0.5
%\renewcommand{\@biblabel}[1]{#1.} % Заменяем библиографию с квадратных скобок на точку:
%\makeatother

%%% Изображения %%%
\graphicspath{{images/}} % Пути к изображениям

%%% Цвета гиперссылок %%%
\definecolor{linkcolor}{rgb}{0,0,0}
\definecolor{citecolor}{rgb}{0,0,0}
\definecolor{urlcolor}{rgb}{0,0,0}
\hypersetup{
    colorlinks, linkcolor={linkcolor},
    citecolor={citecolor}, urlcolor={urlcolor}
}
		% преамбула
%\input{packages}
%\input{styles}


\hyphenation{Даль-не-зе-ле-нец-кая}
 %файл с заданными частыми переносами

\setcounter{tocdepth}{1}%устанавливаем "глубину" оглавления до подсекций

%это движок biber
%\usepackage[backend=biber,bibencoding=utf8,sorting=ynt,maxcitenames=2,style=authoryear]{biblatex}

%biblatex-gost
\usepackage[backend=biber, style=gost-authoryear, bibstyle=gost-authoryear, language=auto, babel=other, bibencoding=utf8, bibdoi=false, biburl=false,  movenames=false, sortcites = false]{biblatex}
%\renewcommand*{\gostmedialanguage}{}

\addbibresource{sophia_base.bib}
%\addbibresource{disser.bib}

%\includeonly{chapters/ch2,chapters/ch3}

\begin{document} % конец преамбулы, начало документа
\input{names}			% Переопределение именований


%%титульный лист
\def\hrf#1{\hbox to#1{\hrulefill}} %команда для задания линеек
\newcommand{\sfs}{\fontsize{14pt}{15pt}\selectfont}
\sfs % размер шрифта и расстояния между строками
\thispagestyle{empty}

\vspace{10mm}
\begin{flushright}
	\Large На правах рукописи 
%  \textit{моя подпись}
\end{flushright}

\vspace{30mm}
\begin{center}
{\Large\bfНазарова София Александровна}
\end{center}

\vspace{30mm}
\begin{center}
{\bf \LARGE Организация поселений \\ \textit{ Macoma~balthica}~(Linnaeus,~1758) \\ в осушной зоне Белого и Баренцева морей
\par}
%{\bf \LARGE \textit{ Macoma~balthica}~(Linnaeus,~1758) в осушной зоне Белого и Баренцева морей: рост, организация и  долгосрочная динамика поселений
%\par}

\vspace{30mm}
{\Large
Специальность 03.02.10 --- <<Гидробиология>>
}

\vspace{15mm}
\LARGEАвтореферат\par
\Largeдиссертации на соискание учёной степени\par
кандидата биологических наук
\end{center}

\vspace{40mm}
\begin{center}
{\LargeСанкт-Петербург --- 2015}
\end{center}

\newpage
% оборотная сторона обложки
\thispagestyle{empty}
\noindent Работа выполнена в Федеральном государственном бюджетном образовательном учреждении высшего образования <<Санкт-Петербургский  государственный  университет>>

\begin{table} [h]  
  \begin{tabular}{ll}   
   \makecell[l]{\sfs Научный руководитель:\\~} &
   \makecell*[{{p{12cm}}}]{\textbf{\sfs Максимович Николай Владимирович} \\ \sfs
   доктор биологических наук, доцент}
      
\vspace{1mm} \\

   \makecell[l]{\sfs Официальные оппоненты: \vspace{6.65cm}} &
   \makecell[{{p{12cm}}}]{   
   \sfs \textbf{Денисенко Станислав Григорьевич,} \\
   \sfs доктор биологических наук, доцент, \\
   \sfs Федеральное государственное бюджетное учреждение науки Зоологический институт РАН,  
   \sfs старший научный сотрудник, зав. лабораторией морских исследований. \vspace{1mm} \\
   \sfs \textbf{Спиридонов Василий Альбертович,} \\
   \sfs доктор биологических наук, \\
   \sfs Федеральное государственное бюджетное учреждение науки Институт океанологии им.~П.\:П.~Ширшова, 
   \sfs старший научный сотрудник лаборатории прибрежных донных сообществ
   }

\vspace{1mm} \\

   \makecell[l]{\sfs Ведущая организация:\\~\\~\\~} &
   \makecell*[{{p{11cm}}}]{\sfs
   \textbf{Федеральное государственное бюджетное научное учреждение Полярный научно-исследовательский институт морского рыбного хозяйства и океанографии им.~Н.\:М.~Книповича}
   }
  \end{tabular}  
\end{table}

\noindent Защита состоится <<24>> марта 2016~г.~в~13 часов на~заседании диссертационного совета  Д.501.001.55 при  Московском государственном университете имени~М.\:В.~Ломоносова по адресу: 119899, Москва, Ленинские горы, дом~1, МГУ, корп.~12, Биологический факультет,  ауд.~389. Тел.~+7(495)939-25-73, эл.почта: dissovet\_00155@mail.ru


\vspace{5mm}
\noindent С  диссертацией  можно  ознакомиться  в  библиотеке  Биологического  факультета Московского  государственного  университета  имени~М.\:В.~Ломоносова и  на  сайте http://www.bio.msu.ru/

\vspace{5mm}
\noindentАвтореферат разослан <<\hrf{2em}>> \hrf{6em} 20\hrf{2em}.

\vspace{5mm}
\begin{table} [h]
  \begin{tabular}{p{8cm}cr}
    \begin{tabular}{p{8cm}}
      \sfs Ученый секретарь  \\
      \sfs диссертационного совета  \\
%      \sfs  Д.501.001.55,  \\
      \sfs кандидат биологических наук
    \end{tabular} 
    & \begin{tabular}{c}
       \includegraphics [height=2cm] {Kartasheva_sign.jpg} 
    \end{tabular} 
    & \begin{tabular}{r}
       \\
       \\
       \sfs Карташева Наталия Васильевна
    \end{tabular} 
  \end{tabular}
\end{table}
\newpage


\tableofcontents

%\linenumbers

%%Введение
\paragraph {Актуальность темы и степень ее разработанности.}
Двустворчатый моллюск {\it Macoma balthica} (Linnaeus, 1758)~--- один из излюбленных модельных объектов в морских гидробиологических исследованиях. 
Вид относят к амфибореальным формам. 
Это обычная литоральная форма в Белом море, у берегов Мурмана и далее на запад, вдоль атлантических берегов Европы~--- до Франции. 
По Атлантическому побережью Северной Америки макомы распространены от Лабрадора до штата Джорджия. 
В северной части Тихого океана~--- от Берингова моря до Японского, а по американскому побережью~--- до Калифорнии. 
В юго-восточной части Баренцева моря и в прилегающей части Карского моря они обитают  не на литорали, а на глубине нескольких метров. 
Моллюски заселяют всю основную часть Балтийского моря, далеко заходя во все заливы, где живут до глубины более 100 метров (\cite{Zacepin_Filatova_1968}).

В настоящее время вид {\it Macoma balthica} по результатам аллозимного анализа разделяют на два подвида: {\it M.~b.~balthica}, обитающий в северной части Тихоокеанского региона, и {\it M.~b.~rubra} из Северо-Восточной Атлантики. 
Однако  в морях, связанных с  Атлантикой, существуют очаги распространения тихоокеанской формы. 
Так, в Балтийском и Баренцевом море Атлантическая и Тихоокеанская формы сосуществуют и образуют гибриды (\cite{Vainola_2003}). 
В Белом море встречается в основном {\it M.~b.~balthica}, и лишь в устье р.~Онеги было обнаружено два экземпляра {\it M.~b.~rubra} (\cite{Nikula_et_al_2007}).
К настоящему моменту нет прямых данных о влиянии данных генетических особенностей на экологические характеристики особей, поэтому в данной работе рассматривается вид {\it Macoma balthica} sensu lato.


{\it Macoma balthica}~--- хорошо изученный вид в центральной части ареала (\cite{Segerstrale_1960, Lavoie_1970, Gilbert_1978, Vincent_et_al_1989, Hiddink_et_al_2002_predation_epifauna, Hiddink_et_al_2002_predation_infauna, Beukema_et_al_2009}, и др.). 
Из морей Северного Ледовитого океана в настоящий момент поселения маком относительно хорошо изучены лишь в Белом море.


В Белом море макомы относятся к наиболее многочисленным обитателям илисто-песчаных пляжей. 
Эти моллюски являются одним из основных пищевых объектов для многих видов рыб и птиц Белого моря (\cite{Azarov_1963, Percov_1963, Golcev_et_al_1997, Bianki_et_al_2003}). 
Поэтому на территории Кандалакшского государственного природного заповедника {\it Macoma balthica} входит в список отслеживаемых видов кормовых беспозвоночных, и является объектом мониторинга с $1992$~года (\cite{Nazarova_2003}).

Массовость и доступность для изучения также позволяет использовать данный вид как удобную модель при анализе закономерностей развития поселений двустворчатых моллюсков. 
Именно поэтому локальные скопления маком Белого моря широко используются как объекты мониторинговых исследований, которые проводились и проводятся на всех крупных биологических стационарах на Белом море. 
В результате к настоящему моменту получены многолетние ряды данных, характеризующих  популяционные показатели маком на Белом море. 
При этом была отмечена существенность различий в организации локальных поселений маком (\cite{Semenova_1974, Maximovich_Kunina_1982, Maximovich_et_al_1991, Poloskin_1996, Nikolaeva_1998, Nazarova_2003, Nazarova_Poloskin_2005}).

 
Информации о поселениях маком в Баренцевом море значительно меньше. 
Детальные гидробиологические исследования сообществ мягких грунтов, в том числе  поселений {\it Macoma balthica}, на Мурмане относятся к $1970$-м годам, однако основным полигоном для исследований стала лишь одна станция на литорали Дальнего пляжа губы Дальне-Зеленецкой (\cite{Agarova_et_al_1976}).
В $2002$ году на Дальнем пляже была повторена количественная съемка бентоса и начат мониторинг сообществ (\cite{Genelt_Dalnezeleneckaya_2008}).

Таким образом, к настоящему моменту данные по Баренцеву морю фрагментарны, а количественные представления о поселениях маком на Мурмане не сформированы. 
Информации о поселениях маком в Белом море значительно больше, однако до сих пор совершенно не изучен вопрос о факторах, влияющих на динамику поселений {\it Macoma balthica} в северной части ареала. Данный вопрос подробно разобран для Ваттового моря (\cite{Hiddink_et_al_2002_predation_epifauna, Hiddink_et_al_2002_predation_infauna, Beukema_et_al_2009}), однако прямой перенос полученных результатов представляется невозможным из-за климатических различий между регионами.


\paragraph{Цели и задачи.}
Целью данной работы является изучение организации поселений {\it Macoma balthica} в условиях осушной зоны Белого и Баренцева морей.

Для достижения данной цели в задачи вошло изучение:
  \begin{enumerate}
    \item структурных характеристик поселений \textit{M.~balthica} (показатели обилия, размерная структура);
    \item многолетней динамики поселений \textit{M.~balthica};
    \item биотического и абиотического фона биотопов;
    \item скорости линейного роста моллюсков;
    \item режима формирования спата.
  \end{enumerate}

\paragraph{Методология и методы исследования.}
Для достижения поставленной цели в акватории Белого моря были использованы мониторинговые наблюдения за шестью поселениями в Кандалакшском заливе.
В Баренцевом море были проведены масштабные количественные описания поселений {\it M.~balthica}, всего $12$ поселений.
Полевые сборы проводили общепринятыми гидробиологическими методами (\cite{Eleftheriou_2013}) при помощи литоральных рамок (площадью от $1/30$ до $1/10$~м$^2$).
Для обработки данных использовали как традиционные методы статистического анализа (\cite{Tukey_1977, Mardia_et_al_1979, Chambers_Hastie_1991, Legendre_Legendre_2012, Hollander_et_al_2013}) так и относительно новые методы анализа многомерных данных (\cite{Clarke_et_al_2008}) и моделирования (\cite{Berryman_Turchin_2001}).

\paragraph{Научная новизна.}
В рамках данной работы впервые проведены масштабные количественные исследования поселений \textit{M.~balthica} на литорали Мурманского побережья Баренцева моря и получены характеристики их обилия и данные по изменчивости линейного роста маком в пределах Мурмана.
Впервые описана многолетняя динамика обилия поселений \textit{M.~balthica} в вершине Кандалакшского залива и показана синхронность пополнения поселений молодью.
Моделирование показало, что колебания плотности поселений маком зависят от зимней температуры.

\paragraph{Теоретическая и практическая значимости работы.}
В работе получены фундаментальные данные, описывающие поселения \textit{M.~balthica} в Белом и Баренцевом морях, при этом впервые дано количественное описание типичных поселений данного вида в Баренцевом море. 
Полученные данные могут быть использованы при оценке запасов кормовых беспозвоночных для хозяйственно-ценных видов рыб и птиц.
Проведено моделирование динамики плотности поселений \textit{M.~balthica} и показано влияние температуры на данный показатель, что может быть использовано для прогнозирования обилия маком. 
Проведенный анализ широтных изменений численности \textit{M.~balthica} показал, что распределение маком по данному показателю не соответствует широко-распространенной  <<гипотезе об обилии в центре>> (<<abundant-centre hypothesis>>, \cite{Sagarin_et_al_2006}), и может быть использован в критике данных представлений в биогеографических обзорах.
Результаты исследования могут быть использованы также в курсах лекций по гидробиологии, популяционной биологии, репродуктивной экологии морского бентоса и биогеографии в ВУЗах.

\paragraph{Положения, выносимые на защиту.}
Основные черты организации поселений \textit{M.~balthica} в осушной зоне Белого и Баренцева морей можно описать следующим образом.
\begin{enumerate}
\item %Плотность поселений \textit{Macoma balthica} в типичных местообитаниях осушной зоны Кандалакшского залива Белого моря на порядок выше, чем Восточном Мурмане, и сравнима с показателями, характерными для Западного Мурмана и Кольского залива Баренцева моря. 
%\textit{Macoma balthica} на литорали Белого и Баренцева моря образуют разные по структуре поселения.
На литорали Кандалакшского залива Белого моря и в Баренцевом море (Западный Мурман и Кольский залив) \textit{Macoma balthica} формирует поселения, в которых плотность значительно варьирует во времени и может достигать нескольких тысяч экз./м$^2$, но наиболее типичны поселения маком с плотностью в несколько сотен экз./м$^2$. 
%При этом среднее обилие \textit{M.~balthica} в Кандалакшском заливе Белого моря и в Кольском заливе Баренцева моря наибольшее в пределах европейской части ареала вида.
На литорали Восточного Мурмана Баренцева моря вид не формирует плотных поселений, и значения данного показателя редко превышает 100~экз./м$^2$.

\item Организация поселений  \textit{Macoma balthica} в условиях осушной зоны Белого и Баренцева морей не имеет принципиальных различий:
	\begin{itemize}
		\item  в типичном случае в многолетней динамике поселений сменяются мономодальный (преобладание молоди) и бимодальный (добавление второго модального класса - группы особей старшего возраста) типы размерной структуры; 
		\item как относительно редкое событие наблюдаются мономодальная структура поселений с ежегодным преобладаем молоди;
	\end{itemize}
%Динамика размерной структуры поселений {\it Macoma balthica} в Белом и Баренцевом морях представлена двумя типами. 
%Более распространенный вариант: чередование бимодального и мономодального характера распределения особей по размерам. 
%При этом первый пик формируют молодые особи (обычно длиной до 5~мм), а в случае бимодальной добавляется второй модальный класс из взрослых особей (в Белом море длиной 9 -- 12~мм, в Баренцевом 10 -- 17~мм). 
%В Баренцевом море часто новое пополнение происходит до ухода старшей генерации и наблюдается три модальных группы.
%В некоторых условиях формируется более редкий тип динамики с ежегодным повторением мономодальной размерной структуры. 

\item Характер динамики плотности поселений \textit{Macoma balthica} определяется, в основном, неравномерностью  уровня ежегодного пополнения их молодью. 
%Характер динамики численности \textit{Macoma balthica} в Белом и Баренцевом морях определяется варьированием численности однолетних особей в поселениях, которое зависит от нерегулярности пополнения поселений молодью, обусловленной в первую очередь различным уровнем выживаемости на первом году жизни.
Беломорские поселения демонстрируют элементы синхронности процессов пополнения, что связано с влиянием температуры на выживаемость маком в первый год жизни  (численность однолетних особей после холодных зим с устойчивым ледоставом оказывается относительно выше) и спецификой условий в локальном местообитании.

\item Скорость роста особей \textit{Macoma balthica} в Белом и Баренцевом морях достоверно ниже, чем в других акваториях европейской части ареала. 
По характеру вариации средней скорости роста маком поселения Баренцева моря и Белого моря различий не имеют. 
%Особи {\it Macoma balthica} в Белом и Баренцевом морях отличаются наименьшей скоростью роста в пределах европейской части ареала вида. 
%При этом внутригрупповая вариация роста особей \textit{M.~balthica} в поселениях Белого и Баренцева моря практически полностью перекрывается.

\end{enumerate}


\paragraph{Апробация результатов.}
%Достоверность изложенных результатов определяется достаточным количеством обработанных проб, отобранных в биотопах, разнообразие которых отражает спектр типичных местообитаний \textit{M.~balthica} в исследованных акваториях.
%Для обработки полученных данных использованы современные статистические методы, позволяющие верифицировать выдвигаемые гипотезы.

Полученные результаты были апробированы в ходе докладов на 
$46$-м (Ровинь, 2011), $49$-м (Санкт-Петербург, $2014$) и $50$-м (Хельголанд, $2015$) Европейских морских биологических симпозиумах (European marine biology symposium); 
$VI$ всероссийской школе по морской биологии <<Биоразнообразие сообществ морских и пресноводных экосистем России>> (Мурманск, $2007$); 
научных сессиях Беломорской биологической станции МГУ (Пояконда, $2004$, $2008$); 
научных сессиях Морской биологической станции СПбГУ (Санкт-Петербург, $2004$, $2008$, $2009$, $2010$); 
$X$ научном семинаре <<Чтения памяти К.М.~Дерюгина>> (Санкт-Петербург, $2008$),
семинарах ЗИН РАНЖ: лаборатории морских исследований, лаборатории пресноводной и экспериментальной гидробиологии и Беломорской биологической станции <<Картеш>> (Санкт-Петербург, $2015$),
а также на семинарах кафедры  ихтиологии и гидробиологии СПбГУ (Санкт-Петербург, $2003 - 2015$).


\afterpage{\clearpage}

%%благодарности
\chapter*{Благодарности}
%\addcontentsline{toc}{chapter}{Благодарности}
В заключение я хочу поблагодарить администрацию Кандалакшского заповедника и лично \fbox{А.\:С.~Корякина} за поддержку наших экспедиций на Белом и Баренцевом морях.
Я благодарна администрации СПбГУ, биологического факультета и кафедры ихтиологии и гидробиологии за возможность работы на Морской биологической станции СПбГУ.

На Баренцевом море мы работали вместе с сотрудниками Мурманского морского биологического института, Мурманского государственного технического университета и Полярного научно-исследовательского института морского рыбного хозяйства и океанографии: М.В.~Макаров, С.В~Малавенда, С.\:С.~Малавенда, О.~Тюкина, И.\:П.~Прокопчук, которые оказывали нам всяческую поддержку.  

Эта работа не могла бы состоятся без моих коллег по экспедициям: Беломорской экспедиции ГИПС ЛЭМБ, студенчской Баренцевоморской экспедиции СПбГУ, Беломорской экспедиции кафежры ихтиологи и гидробиологии СПбГУ. 
Отдельное спасибо руководителям экспедиций: А.\:В.~Полоскину, И.\:А.~Коршуновой, Д.\:А.~Аристову, Е.\:А. Генельт-Яновскому, М.В.~Иванову за возможность работы в экспедиционных командах и помощь в сборе материала.

Я благодарю А.\:В.~Полоскина, Д.\:А.~Аристова, К.\:В.~Шунькину, А.\:В.~Герасимову (кафедра ихтиологии и гидробиологии СПбГУ), А.\:Д.~Наумова (ББС ЗИН РАН) за предоставленные материалы.

Постоянные обсуждения с Ю.\:Ю.~Тамберг и В.\:М.~Хайтовым значительно улучшили мои навыки в статистической обработке материала и помогло мне в работе.
На этапе обработки данных неоценимую помощь идеями и разъяснениями мне оказали В.\:М. Хайтов и Д.\:А. Аристов.


Кроме того, я не могу не поблагодарить руководителей Лаборатории экологии морского бентоса И.\:А. Коршуновой, А.\:В.~Полоскину, \fbox{Е.\:А. Нинбургу} и В.\:М. Хайтову, которые 13 лет назад убедили меня, что морская биология это очень интересно и вложили много сил в мое обучение и воспитание. 
Без них меня бы тут просто не было.

И мой низкий поклон моему научному руководителю Н.\:В. Максимовичу за конструктивную помощь на всех этапах работы, жесткие споры и долгие беседы, ехидные комментарии и  неизменно доброе отношение.

\vspace{3ex}

Данная работа выполнена при частичной финансовой поддержке грантов Санкт-Петер\-бург\-ского государственного университета (1.\:0.\:134.\:2010, 1.\:42.\:527.\:2011, 1.\:42.\:282.\:2012, 1.\:38.\:253.\:2014) и Российкого фонда фундаментальных исследований (12-04-01507, 13-04-10131 К). 



%%материал и методика
\chapter{Материал и методика}
	\section{География исследований}
Географическое распространение вида {\it M.~balthica} охватывает бореальную зону Атлантического и Тихого океанов.
В Европеской части ареала {\it M.~balthica} заходит в арктические моря, и встречается в Норвежском, Баренцевом, Белом и Карском морях.
Наиболее северной точкой считается Шпицберген (\cite{Zacepin_Filatova_1968}).
Таким образом, Белое и Баренцево моря являются краевой частью ареала для маком.
		\subsection{Белое море}
В вершине Кандалакшского залива наблюдения проводили на $6$ участках в рамках работы экспедиций Группы исследований прибрежных сообществ Лаборатории экологии морского бентоса (гидробиологии) СПбГДТЮ (рис.~\ref{ris:karta_White}). 
	\begin{figure}[p]
    \includegraphics[width=\textwidth]{../maps/White_sea1.pdf}
    \caption{Исследованные участки в Кандалакшском заливе Белого моря}
    \label{ris:karta_White}
	\end{figure}
Три участка расположены в районе Лувеньгских шхер: эстуарий реки Лувеньги, Илистая губа острова Горелого и участок материковой литорали в $800$ метрах западнее поселка Лувеньга (участки 1, 2 и 3).
Один участок был расположен на литорали острова Ряшков в Западной Ряшковой Салме (Северный архипелаг) (участок 4).
В работе использованы данные Д.\:А.~Аристова из Южной губы о.~Ряшков и с о.~Большой Ломнишный (Северный архипелаг) (рис.~\ref{ris:karta_White}, участки 5 и 6). 

В районе губы Чупа исследования проводили на $4$ участках (рис.~\ref{ris:karta_White}) в ходе экспедиций кафедры ихтиологии и гидробиологии СПбГУ. 
Два участка были расположены на литорали острова Кереть --- в Сухой Салме и бухте Клющиха (участки 7 и 8). 
Один участок был расположен на материковой литорали пролива Подпахта и один --- в бухте Лисьей (участки 9 и 10).

%Также в работе использованы данные ББС <<Картеш>> ЗИН РАН по обилию маком в губах Медвежья и Сельдяная (\cite{Varfolomeeva_Naumov_2013}) (рис.~\ref{ris:karta_White}, участки 11 и 12).

\afterpage{\clearpage}

		\subsection{Баренцево море}
Материал  в акватории Баренцева моря  был  собран    в ходе   студенческой баренцевоморской экспедиции СПбГУ. 
Всего было исследовано $8$ участков --- $2$ в Кольском заливе и   $6$  в   прибрежной   зоне  Восточного  Мурмана (рис.~\ref{ris:karta_Barents}).  
	\begin{figure}[p]
    \includegraphics[width=\textwidth]{../maps/Barents_sea1.pdf}
    \caption{Исследованные участки вдоль Мурманского побережья Баренцева моря}
    \label{ris:karta_Barents}
	\end{figure}
На   Восточном   Мурмане исследованные участки литорали  были   расположены   в   губах   Гавриловская (участок 13),  Ярнышная (участок 14), Дальнезеленецкая (участок 15), Шельпинская (участок 16), Порчниха (участок 17) и Ивановская (участок 18).
Участки литорали  в   Кольском   заливе   были  расположены на побережье в районе Абрам-мыса (участок 19) и в Палa-губе (участок 20), в районе города Полярный. 


Также в работе использованы данные К.\:В.~Шунькиной и Е.\:А.~Генельт-Яновского по обилию маком в губе Печенга и Ура-губе (Западный Мурман) (рис.~\ref{ris:karta_Barents}, участки 21 и 22), и в районе Северного Нагорного и Ретинского (Кольский залив) (рис.~\ref{ris:karta_Barents}, участки 23 и 24).

\afterpage{\clearpage}

    \section{Характеристика местообитаний}
Для всех участков было составлено физиономическое описание.
На каждом участке измеряли ширину литорали.
По горизонтам описывали визуально тип грунта, наличие валунов, бурых водорослей, взморника \textit{Zostera marina}, зеленых нитчатых водорослей. 
Также описывали поясность сообществ, если она была выражена.
Термогалинные характеристики для отдельных участков не были описаны, и мы использовали литературные данные о среднегодовых параметрах для акваторий, и опубликованные данные о динамике температур в исследованный период (\cite{KGZ_letopis, rp5_Kandalaksha, pinro}.

Показательной комплексной оценкой гидродинамики региона и условий питания детритофагов служат показатели состава грунта. 
Поэтому на ряде исследованных участков были отобраны образцы грунта. 
В Белом море грунт исследовали на обоих участках в губе Чупа и на трех участках в вершине Кандалакшского залива.
В Баренцевом море грунт исследовали на всех участках Восточного Мурмана и двух участках Кольского залива (Абрам-мыс и Пала-губа).

Пробы грунта отбирали на всех исследованных горизонтах каждого участка.
В экспедиции после отбора из грунта выбирали крупных животных (червей, раков, моллюсков, приапулид), образцы высушивали и упаковывали для отправки в город. 
В городе образцы досушивали в термостате при температуре $105^o$C до момента, когда масса образца переставала изменяться. 
Из каждого образца брали по три навески грунта для определения содержания органических веществ. 
Навески помещали в муфельную печь с температурой $450^o$C на $8$ часов. 
После сжигания навески повторно взвешивали, и по разнице масс определяли массовую долю органических веществ в грунте. 
По трем навескам рассчитывали среднюю массовую долю для каждого образца.

Оставшийся грунт использовали для определения гранулометрического состава. 
При описании гранулометрического состава грунта использовали классификацию И.\,Л.~Безрукова и А.\,Н.,~Лисицина для морских водоемов (таблица~\ref{tab:lisicyn_granulometriya}, \cite{Bezrukov_Lisicyn_1960}).
\begin{table}[p]
    \centering
    \caption{Классификация фракций грунта по размеру частиц (\cite{Bezrukov_Lisicyn_1960})}
    \label{tab:lisicyn_granulometriya}
\begin{tabular}{|l|l|}
    \hline
    Размер фракции,~мм & Название фракции         \\ \hline
     $> 10$    & Крупный и средний гравий  \\
    $10-5$               & Мелкий гравий         \\
    $5-3$                & Очень мелкий гравий   \\
    $3-1$                & Очень крупный песок   \\
    $1-0,5$              & Крупный песок         \\
    $0,5-0,25$           & Средний песок         \\
    $0,25-0,1$           & Мелкий песок          \\
    $< 0,1$           & алевриты       \\\hline
\end{tabular}
\end{table}
Для этого грунт взвешивали, после чего просеивали в сухом состоянии через колонку сит (диаметр ячеи: $10 - 5 - 3 - 1 - 0,5 - 0,25$~мм). 
Частицы размером менее $0,25$~мм просеивали через сито с диаметром ячеи $0,1$~мм с использованием струи воды, после чего оставшиеся на сите --- высушивали при температуре $105^o$C. 
Каждую фракцию частиц взвешивали, и определяли их массовую долю. 
Анализ частиц размером менее $0,1$~мм (алевритов)не проводили. 

\afterpage{\clearpage}

    \section{Описание сообществ, включающих {\it Macoma balthica}}
На 6 мониторинговых участках в Кандалакшском заливе Белого моря проводили качественное описание фауны в пределах обследованных горизонтов литорали.
Таким образом, всего составлено $12$~описаний.
На каждом участке в акватории Баренцева моря исследовали все  горизонты литорали, представленные мягкими грунтами.  
Таким образом, всего было составлено $16$ описаний.

Как основное орудие сбора использовали литоральную рамку площадью $1/30$~м$^2$, из которой изымали грунт на глубину $5$~см. 
В случае, когда приходилось отбирать пробы из-под воды, использовали зубчатый водолазный дночерпатель площадью захвата $1/20$~м$^2$.
Отобранные пробы промывали на сите с диаметром ячеи $1$~мм. 

После промывки из   проб   выбирали   всех   особей  {\it M.~balthica}  и   представителей   сопутствующего макрозообентоса    для   определения   состава   сообщества.
Представителей   сопутствующего макрозообентоса  определяли   до   минимально   возможного   таксона. Таксономию и номенклатуру сверяли по Всемирному регистру морских видов (\cite{WoRMS}).

Для сравнения видового состава сообщества использовали коэффициент Жаккара. 
Результаты визуализировали при помощи  кластерного анализа методом ближайшего соседа. 
Достоверность выделенных групп оценивали с помощью анализа сходства профилей (SIMPROF) (\cite{Clarke_et_al_2008}).
Для оценки влияния факторов использовали многомерное шкалирование MDS в сочетании с анализом сходства ANOSIM.
Анализы проводили в программе PaSt (\cite{Hammer_et_al_2001}) и \R{} (\cite{R_2014}).


%	\subsection{Изучение микрораспределения {\it Macoma balthica}}

%\textcolor{red}{квадраты на Белом}

%\textcolor{red}{квадраты на Баренцевом}
%При оценке распределения особей в губе Порчниха в 2007 г. было отобрано 32 пробы рамкой 1/30м2, причем пробы брались вплотную друг к другу 4 рядами по 8 шт.

% из Генельт-Кобылков-Назарова, 2007 (что это было??)
%Изучение распределения особей {\it Macoma baltica} было проведено в Баренцевом море по методике, описанной Трашем (\cite{Thrush_et_al_1989}) с изменением масштаба.
%Исследования были проведены в августе $2007$~г. на илисто-песчаной литорали кутовых участков губ Восточного Мурмана --- Ярнышной и Дальнезеленецкой, и в октябре $2007$~г. на литорали Пала-губы (Кольский залив). 
%Для Дальнезеленецкой губы съемка была повторена в августе $2008$ года на полигоне двойного размера.

%В каждой точке отбиралось по $36$ проб площадью $1/30$~м$^2$, расположенных в пределах участка размером $7,5 \times 12$~м. 
%Координаты каждой пробы были определены в декартовой системе координат в метрах, один из углов участка служил точкой отсчета. 
%В дальнейшем пробы промывали на сите с диаметром ячеи $1$~мм. 
%В лаборатории были выбраны и  подсчитаны все макомы.

%При дальнейшей обработке данных для каждого участках подсчитывали индекс структурности (отношение дисперсии к средней арифметической). 
%Для анализа размеров агрегаций были построены коррелограммы, основанные на коэффициенте пространственной автокорреляции Морана (\cite{ncf}).
%Достоверность коэффициентов определяли пермутационным методом.
%Наличие градиентов проверяли с использованием корреляционного анализа Кендалла между координатами проб и обилием вида в каждой пробе. 
%Все статистические анализы проводили в статистической среде \R{} (\cite{R_2014}) с $95\%$ доверительной вероятностью ($P < 0,05$).
%Для интерпретации результатов корреляционного анализа были использовали пузырькоые диаграммы.

\afterpage{\clearpage}	

	\section{Изучение структуры поселений {\it Macoma balthica}}
Для описания структуры поселений использовали данные всех доступных сборов.

В Белом море всего было обследовано $10$ участков в акватории Кандалакшского залива. 
На шести из них наблюдения проводили на всех горизонтах литорали, представленных мякими грунтами.
На четырех других были обследованы отдельные горизонты. 

Для Баренцева моря были использованы данные по обилию с $12$ участков. 
На каждом участке в акватории Баренцева моря исследовали все  горизонты литорали, представленные мягкими грунтами.  

Как основное орудие сбора использовали литоральную рамку площадью $1/30$~м$^2$, из которой изымали грунт на глубину $5$~см. 
В случае, когда приходилось отбирать пробы из-под воды, использовали зубчатый водолазный дночерпатель площадью захвата $1/20$~м2.
Отобранные пробы промывали на сите с диаметром ячеи $1$~мм или $0,5$ (на трех мониторинговых участках в районе Лувеньги и в Западной Ряшковой Салме, участки $7, 8 - 10$ на рис.~\ref{ris:karta_White}). 
После промывки из   проб   выбирали   всех   особей  {\it M.~balthica}.
Подробная информация о количестве проб и размере учетных площадок для каждого участка представлены в приложении~\ref{app:NB_table}.

В дальнейшем подсчитывали количество особей в пробах, которое пересчитывали в численность моллюсков на квадратный метр. 
Для всех средних значений расчитывали точность учета $d,\% = m/M$, где $m$~--- стандартная ошибка средней, $M$~--- средняя арифметическая.
Биомассу определяли путем взвешивания на весах с точностью 10 мг, либо, для части участков на Белом море, рассчетным методом.
Мы использовали формулу зависимости массы макомы от ее длины $W = 0,00016 \times L^{2,96}$, полученную для губы Чупа (\cite{Maximovich_et_al_1993}).


Изучение размерной структуры поселений маком проводили на всех участках.
Для этого у всех моллюсков в пробах под бинокуляром измеряли максимальный линейный размер (длину) с точностью $0,1$~мм.
По полученным данным строили размерно-частотное распределение с шагом $1$~мм.

%Кроме того, на части участков у моллюсков подсчитывали количество меток зимней остановки роста, которое принимали как возраст моллюсков --- число прожитых зим (например, $4+$ это  особи возрастом от $4$ до $5$ лет).   
%Таким   образом   были   получены   оценки возрастной структуры поселений {\it M. balthica}.

Для визуализации данных по обилию использовали графики Тьюки (Tukey boxplots, \cite{Tukey_1977}). 
Сравнение обилия проводили с помощью непараметрического теста Краскел-Уоллиса. 
Для выявления связи величин обилия с гранулометрическим составом грунта использовали непараметрическую корреляцию Спирмена (\cite{Hollander_et_al_2013}).
Классификацию размерных структур проводили с помощью анализа главных компонент (\cite{Mardia_et_al_1979}).
Все анализа проводили в статистической среде \R{} (\cite{R_2014}).

\afterpage{\clearpage}

	\section{Изучение динамики поселений {\it Macoma balthica}}
        \subsection{Белое море}
В Белом море динамику поселений {\it Macoma balthica} исследовали на $6$ участках в районе вершины Кандалакшского залива. 

Сборы проводили с 1992 по 2012 год ежегодно в июле-августе.
Автор принимала участие в полевых сборах с $1999$ по $2007$ год.
Данные за другие годы взяты из архива ГИПС ЛЭМБ.

Структура материала представлена в таблице \ref{tab:material_Kandalaksha}.
\begin{table}[p]
\caption{Структура материала по динамике поселений {\it Macoma balthica} вершины Кандалакшского залива}
\label{tab:material_Kandalaksha}
%\begin{tabular}{|*{5}{p{0.2\textwidth}|}} \hline
    \begin{tabularx}{\textwidth}{|*{5}{X|}} \hline
участок & годы наблюдения & обследованные горизонты литорали & количество проб в однократной съемке & площадь пробоотборника, м$^2$  \\ \hline
о. Горелый Лувеньгских шхер & 1992 -- 2012 & ВГЛ, СГЛ, НГЛ & 1-3 & 1/30, 1/10 \\ \hline
Материковая литораль в районе пос. Лувеньга & 1992-2000, 2002, 2004 & ВГЛ, СГЛ, НГЛ & 12-20 & 1/30 \\ \hline
Эстуарий р. Лувеньги & 1992 -- 2012 & СГЛ & 3 & 1/10 \\ \hline
Литораль Западной Ряшковой Салмы о. Ряшкова & 1994 -- 2012 & СГЛ & 2 & 1/10 \\ \hline
Южная губа о. Ряшкова & 2001 -- 2012 & НГЛ & 9-16 & 1/30 \\ \hline
о. Ломнишный & 2007 -- 2012 & НГЛ & 5-10 & 1/30  \\ \hline
\end{tabularx}
\end{table}

На каждом исследованном участке отбирали $3 - 25$ проб площадью $1/30 - 1/10$~м$^2$, которые затем промывали на сите с диаметром ячеи $0,5 - 1$~мм. 
В пробах учитывали всех особей {\it M.~balthica}, у которых в дальнейшем измеряли максимальный линейный размер (длину) с точностью ~0,1~мм. 

Для определения биомассы моллюсков взвешивали на электронных весах с точностью до $1$~мг. 
Для серий проб, где не проводили взвешивание моллюсков, биомассу определяли рассчетным методом с использованием аллометриеской зависимости сырой массы маком от длины их раковины (\cite{Maximovich_et_al_1993}). 

В дальнейшем рассчитывали показатели средней численности маком на квадратный метр (плотность поселения) и размерно-частотное распределение особоей.
Для построения размерно-частотного распределения шаг размерного класса составлял $1$~мм.

В дальнейшем при анализе мы работали с особями с длиной раковины более $1,0$~мм по двум причинам. 
Во-первых, для того чтобы сделать сравнимыми результаты с разных участков, где пробы промывались на ситах с разным диаметром ячеи. 
Во-вторых, пробы отбирали в середине лета, то есть к этому моменту молодь этого года частично осела, то есть оценка численности данной группы будет некорректна.
Мы считаем корректной такую редукцию материала, поскольку для Белого моря считается, что усешность пополнения поселений молодью в первую очередь зависит от выживаемости спата зимой (\cite{Maximovich_Gerasimova_2004}).

Для анализа динамики пополнения поселений молодью в $2012 - 2013$ годах у особей длиной менее $3$~мм были измерены длины колец зимней остановки роста. 
После определения размеров годовалых особей, по размерной было рассчитано их обилие в каждом году мониторингового наблюдения.
Всего было промерено 496 особей.



В работе использованы мониторинговые данные кафедры ихтиологии и гидробиологии СПбГУ по обоим участкам на острове Кереть (\cite{Maximovich_et_al_1991, Gerasimova_Maximovich_2013}) (рис.~\ref{ris:karta_White}, участки $1, 2$). 
Также в работе использованы многолетие данные ББС <<Картеш>> ЗИН РАН по обилию маком в губах Медвежья и Сельдяная (\cite{Varfolomeeva_Naumov_2013}) (рис.~\ref{ris:karta_White}, участки $11, 12$).


 
%методика из Назаровой-Полоскина
%Материал собран в августе 1992 - 2003 гг. Изучено три литоральных поселения маком: в Илистой губе острова Горелого (участок 1), в эстуарии реки Лувеньги (участок 2) и на материковой литорали к югу от поселка Лувеньга (участок 3). Сборы проведены пробоотборником площадью захвата 1/30 м2. Разовая выборка составляла от 9 до 25 проб с участка. Грунт выбирался до глубины 5 см и промывался на сите с диаметром ячеи 0.5мм.  Всех особей  M. balthica измеряли с точностью 0.1 мм. В каждый момент наблюдений определяли размерную структуру и плотность поселения маком. 



% методика из Аристова
%Оба участка закрыты от волнового воздейст-
%вия. Литораль в районе исследований представляет собой песчаный пляж с примесью
%ила с вкраплениями крупных валунов. Спуск в сублитораль пологий, отчетливо вы-
%раженный пояс фукоидов отсутствует. Население представлено типичными формами,
%такими как Arenicola marina, Macoma balthica, Mya arenaria, Hydrobia ulvae, Microspio sp.
%и др. (Д. А. Аристов, неопубликованные данные).
%В обеих точках производили сборы по следующей методике: в районе нуля глубин
%во время отлива в пределах участков случайным образом выбирали и обследовали не-
%сколько площадок. Поскольку радиусы индивидуальной активности A. islandica и пред-
%полагаемых жертв (двустворчатых моллюсков) существенно различаются, в пределах
%каждой площадки брали пару проб методом вложенных рамок. Первую пробу из пары
%(1/4 м2) брали для учета A. islandica. Грунт из пробы тщательно перебирали вручную,
%всех найденных представителей сем. Naticidae подсчитывали и определяли их видовую
%принадлежность. Вторая проба (1/30 м2) была взята для учета потенциальных жертв —
%двустворчатых моллюсков. Грунт из нее промывали на сите с диаметром ячеи 1 мм,
%а затем остаток разбирали в фотографической кювете с белым дном. Из грунта соби

\afterpage{\clearpage}

        \subsection{Баренцево море}
% из Дерюгинских, 2007 + новой статьи

В Баренцевом море динамику поселений маком исследовали на модельном участке --- литоральной отмели Дальний пляж губы Дальнезеленецкой. 
В работе использованы материалы экспедиции по мониторингу Дальнего пляжа губы Дальнезеленецкой с $2002$ года, любезно предоставленные Е.\:А.~Генельт-Яновским.
Автор принимала участие в полевых сборах в $2006 - 2008$~гг.

Материал был собран в июле-августе $2002 - 2008$~гг. в пределах от верхнего горизонта песчаной литорали ($+2,0$~м) до $+0,7$~м над нулем глубин. 

 В $2002$ году была заложена сетка из $8$ станций. 
 В пределах каждой станции отбирали $3 - 5$ проб площадью $1/10$~м$^2$, которые промывали на сите с диаметром ячеи $1$~мм. 
 У всех двустворчатых моллюсков измеряли длину раковины с точностью $0,1$~мм. 
 С 2004 года отбирали пробы на трех станциях из $8$, которые располагались в контрастных сообществах (\cite{Genelt_Dalnezeleneckaya_2008}). 

В качестве точки сравнения нами был выбран $1973$ год (\cite{Streltsov_et_al_1974, Agarova_et_al_1976}), поскольку в тот год была проведена основная количественная съемка на Дальнем пляже. 

 %Восстановление возрастной структуры {\it Macoma balthica} для 2002-06 годов было проведено по моллюскам из выборки 2007 года. Для этого у них были измерены кольца зимней остановки роста и рассчитаны размеры моллюсков каждой возрастной группы. В качестве границ размерно-возрастных классов принималась середина соответствующего размерного диапазона. В дальнейшем, в зависимости от длины, каждого моллюска из выборок 2002-06 гг. относили к определенному возрастному классу. Так как в 2007 году не были встречены особи с 8 видимыми кольцами зимней остановки роста, то все особи крупнее 17,7 мм (верхняя размерная граница возрастного класса 8+) были объединены в одну группу.

Сравнение средних проводили с помощью критериев Вилкоксона и Краскел-Уоллеса (\cite{Hollander_et_al_2013}).
При анализе трендов в динамике поселений использовали корреляционный анализ Мантеля (\cite{Legendre_Legendre_2012}) для удаления тренда из рядов. 
Также данный метод использовали для оценки синхронности динамик обилия моллюсков в разных поселениях.
Для выяления плотностнозависимых процессов были использованы частные автокорреляции (PRCF --- Partial rate correlation function) (\cite{Berryman_Turchin_2001}).
Для изучения влияния температуры на динамику обилия \textit{M.~balthica} использовали линейные модели (\cite{Chambers_Hastie_1991}).
Оценку корректности построенной модели проверяли с помощью критериев Дарбина-Уотсона (отсутствие автокорреляций), Шапиро-Уилка (нормальное распределение остатков) и Бройше-Пагана (гомогенность дисперсий).
Все рассчеты проводили в статистической среде \R{} (\cite{R_2014}).

\afterpage{\clearpage}

	\section{Изучение линейного роста {\it Macoma balthica}}
%из статьи про рост
Рост \textit{M.~balthica} в Белом море достаточно детально изучен (\cite{Semenova_1970, Maximovich_et_al_1992, Hummel_et_al_1998}), поэтому мы проводили специальные исследования только для Баренцева моря.

Рост изучали по материалам, полученным в августе $2007 - 2008$~гг. для $7$ участков в Баренцевом море: Абрам-мыс, Пала-губа, губы Гавриловская, Ярнышная, Дальнезеленецкая, Шельпино, Порчниха.
Станции для отбора проб располагали по горизонтам литорали. 

У всех особей {\it Macoma balthica} в пробах ($1/30$ или $1/20$~м$^2$, промывка на сите с диаметром ячеи 1~мм) измеряли длину (наибольший линейный размер) раковины и (по меткам роста) ее значения в период каждой зимней остановки роста с точностью 0,1 мм.
Полученные для каждой станции измерения особей были сведены в описание возрастной структуры по схеме, представленной в табл.~\ref{tab:rost_matrica_primer}. 
\begin{table}[p]
        \caption{Пример треугольной матрицы с данными по росту моллюсков и их возрастной структуре}
        \label{tab:rost_matrica_primer}
%    \begin{tabular}{|c|c|cc|cc|ccccccccc|}
        \begin{tabularx}{\textwidth}{|X|X|XX|XX|XXXXXXXXX|}
        \hline
        &    & \multicolumn{4}{c|}{$L$}               & \multicolumn{9}{c|}{$L_{k}$} \\ 
        $t$     & $N$  & $min$ & $max$ & $aver$ & $m_{L}$   & 1 к & 2к  & 3к  & 4к  & 5к  & 6к  & 7к  & 8к   & 9к   \\ \hline
        0+      & 0  &       &       &         &         &     &     &     &     &     &     &     &      &      \\
        1+      & 9  & 1,8   & 2,5   & 2,2     & 0,1     & 1,1 &     &     &     &     &     &     &      &      \\
        2+      & 76 & 1,6   & 7,9   & 3,1     & 0,1     &\cellcolor{yellow}0,7 & \cellcolor{yellow}2,0 &     &     &     &     &     &      &      \\
        3+      & 40 & 2,1   & 5,8   & 3,8     & 0,1     & 0,7 & 1,8 & 2,9 &     &     &     &     &      &      \\
        4+      & 34 & 2,1   & 8,5   & 5,4     & 0,2     & 0,7 & 1,8 & 3,1 & 4,6 &     &     &     &      &      \\
        5+      & 37 & 3,5   & 9,8   & 6,8     & 0,2     & 0,8 & 1,9 & 3,1 & 4,6 & 6,2 &     &     &      &      \\
        6+      & 44 & 4,6   & 11,5  & 8,2     & 0,2     & 0,8 & 1,8 & 2,9 & 4,1 & 5,5 & 7,3 &     &      &      \\
        7+      & 48 & 7,4   & 12    & 9,9     & 0,2     & 0,9 & 2,1 & 3,3 & 4,6 & 6,0 & 7,7 & 9,1 &      &      \\
        8+      & 61 & 8     & 13,7  & 10,6    & 0,1     & \cellcolor{red}0,7 & \cellcolor{red}2,0 & \cellcolor{red}3,4 & \cellcolor{red}4,6 & \cellcolor{red}6,1 & \cellcolor{red}7,5 & \cellcolor{red}8,9 & \cellcolor{red}9,9  &      \\
        9+      & 44 & 8,6   & 14,2  & 11,1    & 0,2     & -   & -   & 3,4 & 4,7 & 6,5 & 8,2 & 9,7 & 10,5 & 11,4 \\ \hline
                &    &       &       & $L_{k} aver$  &  & \cellcolor{blue}0,8 & \cellcolor{blue}1,9 & \cellcolor{blue}3,1 & \cellcolor{blue}4,5 & \cellcolor{blue}6,0 & \cellcolor{blue}7,7 & \cellcolor{blue}9,2 & \cellcolor{blue}10,2 & \cellcolor{blue}11,4 \\
                &    &       &       & $m_{Lк}$      &  & 0,0 & 0,0 & 0,1 & 0,1 & 0,2 & 0,2 & 0,3 & 0,4  &      \\
                &    &       &       & $L_{k} min$  &   & 0,7 & 1,8 & 2,9 & 4,1 & 5,5 & 7,3 & 8,9 & 9,9  &      \\
                &    &       &       &  $L_{k} max$ &   & 1,1 & 2,1 & 3,4 & 4,7 & 6,5 & 8,2 & 9,7 & 10,5 &     \\ \hline
    \end{tabularx}
    \footnotesize{Примечания: $t$ --- возраст моллюсков; 
        $N$ --- количество  особей  данного возраста, экз.; 
        $L min$  ---  минимальная   длина  особей   данного   возраста,   мм;   
        $L max$   ---   максимальная   длина   особей   данного   возраста,   мм; 
        $L aver$ --- средняя длина моллюсков данного возраста, мм; 
        $m_L$ --- ошибка средней, 
        $L_k$ 1к -- 13к --- длина особи к определенному возрасту, измеренная по меткам зимней остановки роста, мм;
        $L_k aver$ --- средняя длина данной метки остановки роста, мм; 
        $m_{L_k}$ --- ошибка средней; 
        $L_k min$ --- минимальная длина данной метки остановки роста, мм; 
        $L_k   max$   --   максимальная   длина   данной   метки   остановки   роста.   
        В   таблице   приведены средние длины данного кольца у моллюсков определенного возраста. \\[1em]
    Выделения: синий --- средневзвешенный возрастной ряд для маком в данном поселении;
        красный --- возрастной ряд отдельной генерации маком;
        желтый --- средний годовой прирост моллюсков в определнном возрасте}
\end{table}
Таким образом, всего было получено $14$ описаний, условно характеризующих отдельные поселения маком. 
Как видно из данных табл.~\ref{tab:rost_matrica_primer}, каждое из описаний содержало результаты реконструкции динамики средней длины раковины маком в генерациях. 
Эти данные мы использовали для сравнительного анализа характера линейного роста моллюсков в поселениях и расчета величин группового годового прироста особей в генерации (как разность средних длин раковин моллюсков в последовательные моменты зимней остановки роста).

Возрастные ряды аппроксимировали при помощи линейной модификации уравнения Берталанфи: $L_{t} = L_{max} \times (1 - e^{(-k(t - t_{0}))})$, где $L_{max}$, $k$, $t_{0}$ --- коэффициенты, $t$ --- возраст, а $L_{t}$ --- длина раковины моллюска в возрасте $t$.
Сравнительный анализ кривых роста произведен с учетом разброса эмпирических данных относительно регрессионной модели. 
В качестве меры расстояния использовали отношение величины статистики $F$ (частное от деления остаточной вариансы относительно кривой роста на сумму остаточных варианс относительно частных моделей роста) к $5$\%-ному квантилю $F$-распределения (\cite{Maximovich_1989}). 
Рассчеты проводили при помощи оригинального макроса к MS Excel, выполненного Т.С.~Ивановой.
При сравнении авторских данных с литературными источниками использовали как вышеописанную методику, так и сравнение параметра $\omega = L_{\infty} \times k$ (где $L_{\infty}$ и $k$ --- коэффициенты уравнения роста Берталанффи), который считается более адекватным для задач сравнения ростовых характеристик, чем сравнение параметров уравнения Берталанффи напрямую (\cite{Appeldoorn_1983, Beukema_Meehan_1985}). 

Структуру вариансы величин группового годового прироста анализировали при помощи двухфакторного дисперсионного анализа (\cite{Chambers_Hastie_1991}). 
Как факторы влияния рассматривали начальную для данного интервала среднюю длину раковины, местообитания (участок) и мареографический уровень положения станции (горизонт литорали).
В статистических расчетах ориентировались на уровень значимости критерия $\alpha < 0,05$.
Рассчеты проводили в Statistica for Windows.

\afterpage{\clearpage}

	\section{Изучение спата и пополнения поселений {\it Macoma balthica}}
Для исследования количественного формирования спата было выбрано $4$ участка, различных по абиотическим условиям (рис.~\ref{ris:karta_White}, участки 1-4). 
Выбор был обусловлен тем, что все эти участки уже изучали до этого, на двух из них ведется мониторинг силами сотрудников кафедры ихтиологии и гидробиологии. 
Все участки характеризуются мягкими грунтами, и по данным предшествующих наблюдений, на них существуют стабильные во времени плотные поселения маком.
Съемки проведены в июле и в конце августа $2006$ года.

В июле на среднем горизонте литорали было отобрано по 5 проб на каждом участке для учета маком старше $1$ года. 
Размер учетных площадок составлял от $0,1$ до $0,05$~м$^2$. 
Пробы промывали на сите с диаметром ячеи $1$ мм. 
В пробах проводился количественный учет макробентоса, и определялась его биомасса.
У всех особей \textit{M.~balthica} под бинокуляром измеряли максимальный линейный размер (длину) раковины с точностью $0,1$~мм. 
Биомассу маком в дальнейшем рассчитывали с использованием формулы аллометрической зависимости индивидуальной сырой массы маком от длины раковины (\cite{Maximovich_et_al_1993}).

В конце августа на этих же участках были отобраны пробы с учетной площади $0,01$ кв. м, которые фиксировали, а затем без промывки разбирали под бинокуляром.  
Из данных проб выбирали всех особей \textit{M.~balthica}, осевших в этом году, т.е не имевших кольца остановки роста. 
В дальнейшем у всех плантиград измеряли длину. 

Статистическую обработку материала проводили по стандартным формулам в программах MS Excel 2003 и Statistica 6.0. 
Для выявления связи численности спата с обилием маком и с обилием макрозообентоса использовался ранговый коэффициент корреляции Спирмена ({\cite{Hollander_et_al_2013}).

Для оценки влияния численности взрослых маком на размеры пополнения был проведен иерархический дисперсионный анализ  (\cite{Chambers_Hastie_1991}).
Фактор <<численность взрослых особей>> (N взр.) был представлен в двух градациях: <<высокая>> (более $1000$~экз./м$^2$) и <<низкая>> (менее $600$~экз./м$^2$).
В качестве вложенного фактора использовался <<участок>>: Сухая Салма и бухта Лисья как участки с высокой плотностью и бухта Клющиха и пролив Подпахта~--- с низкой. 

Аналогичный дисперсионный анализ был проведен для отдельных размерных классов взрослых маком для выявления характера влияния на разноразмерных особей факторов <<численность взрослых особей>> и <<участок>>. 
В анализе использовали данные по количеству взрослых маком размером от $2$ до $9$~мм с шагом $1$~мм, т.к. именно особи этих размеров представлены в выборках в достаточном для анализа количестве.
Каждый дисперсионный анализ сопровождался оценкой силы влияния факторов с ошибкой и достоверностью. 

\par\medskip
Для изучения динамики пополнения поселения численность годовалых особей в каждый год была восстановлена по данным размерной структуры.
Для этого по данным мониторингов $2012 - 2013$ годов были проведены измерения длин колец зимней остановки роста у особей длиной менее $3$~мм.
Для сравнения полученных данных использовали критерий Краскел-Уоллеса, и в случае достоверных отличий --- тест Тьюки (Tukey’s ‘Honest Significant Difference’) ({\cite{Hollander_et_al_2013}).
Для определения границ размерно-возрастных классов {\it Macoma balthica} возрастом $0+$, $1+$ и $2+$ были рассчитаны средние размеры особей каждого возраста.
Пограничный размер между двумя когортами рассчитывали как середину между средними размерами особей в когорте.

На основании полученных данных о размере годовалых маком была рассчитана их численность в каждом году наблюдения.
Для выявления связи между обилием годовалых особей с различными параметрами использовали коэффициент корреляции Спирмена {\cite{Hollander_et_al_2013}).
Гипотезу о синхронности пополнения поселений в акватории проверяли при помощи корреляции Мантеля (\cite{Legendre_Legendre_2012}).
Все рассчеты проводили в статистической среде \R{} (\cite{R_2014}).

\afterpage{\clearpage}

%%литобзор
	\chapter{Обзор литературы}
%\textcolor{red}{в конце каждого раздела надо сформулировать проверяемую гипотезу}
%		\section{Вид {\it Macoma balthica}: традиционное представление и современные данные}
%\textcolor{red}{Амфибореальный. Ареал распространения. Два подвида. Образуют гибриды. Нет данных об экологических различиях.}



		\section{Физико-географическое описание районов исследования}
%\textcolor{red}{Климат. Температура. Соленость. Ледовый режим. Типы литорали.
%ТЕмпературные данные Кольский меридиан и Декадная съемка Чупа.}

Белое и Баренцево моря~--- арктические моря, однако литоральная фауна во многом сформирована бореальными видами (\cite{Zenkevich_1963}).
Условия обитания гидробионтов в них значительно отличаются в связи с географическим положением и особенностями гидрологии.
Рассмотрим их подробнее.

	\subsection{Белое море}

Белое море глубоко врезается в материк, и с этим связывают континентальность климата: лето относительно теплое, зима продолжительная и суровая. 
Зимой температура воздуха может опускаться до $-20 - -30^{\circ}C$, а летом подниматься до $+30^{\circ}C$, хотя обычно не превышает $15-20^{\circ}C$. 
В северных районах Белого моря температура воздуха в среднем ниже, чем в южных (\cite{Babkov_Golikov_1984}). 
Для губы Чупа минимальная температура воздуха наблюдается в январе (в среднем $-11^{\circ}C$), а максимальная в июле (в среднем $+14,7^{\circ}C$) (\cite{Babkov_1982}). 

Летом в вершинных частях заливов и на мелководье вода может прогреваться до $20 - 24^{\circ}C$. 
Зимой температура воды отрицательная, порядка $-1,5^{\circ}C$ (\cite{Babkov_Golikov_1984}).
Кандалакшский залив является наиболее прогреваемым участком. 
В западной его части среднегодовая температура воды составляет $4^{\circ}C$ (при разбросе от $3,2$ до $5,1^{\circ}C$), а амплитуда межсезонных колебаний составляет в среднем $14,8^{\circ}C$ (от $13,0$ до $16,5^{\circ}C$) (\cite{Kuznecov_1960}). 
В губе Чупа среднегодовая температура всей толщи воды составляет менее $2^{\circ}C$. 
Поскольку литораль находится в зоне влияния поверхностной водной массы, то зимой обитатели подвергаются воздействию отрицательных температур ($-1,5^{\circ}C$), в то время как летом вода на литорали прогревается до $+19,3^{\circ}C$ (\cite{Babkov_1982}). 

Другим важным для гидробионтов фактором является соленость воды. 
В Белом море среднегодовая соленость поверхностных вод составляет $23-25$\permil. 
По данным А.И.Бабкова и А.Н.Голикова (\cite*{Babkov_Golikov_1984}) в районе Кандалакши соленость может изменяться от $7$ до $26$\permil. 
Такие колебания связаны с обширным материковым стоком, частично с осадками и, в первую очередь, с весенним таянием льдов (\cite{Naumov_Fedyakov_1993}).
Вода в губе Чупа значительно распреснена, в первую очередь за счет стока рек Пулонга и Кереть, но также за счет ручьев. 
В верхнем $10$ метровом слое, то есть в слое, омывающем литораль, отмечены сезонные колебания солености более $10$\permil\ (от $15$ до $26$\permil), при этом максимальная соленость достигается в ноябре, а минимальная~--- в апреле (\cite{Babkov_1982}). 

В зимнее время для Белого моря характерен ледовый покров. 
При подвижках припая возможно истирание выступающих над поверхностью структур, в том числе живых организмов. 
Кроме того, возможен перенос организмов, вмерзших в лед или находящихся на примерзших водорослях.
 Время ледостава в разных районах Белого моря отличается. 
В губах Кандалакшского залива лед появляется в первой половине сентября и держится до второй половины мая. 
В губе Чупа формирование льда начинается в устьях рек и ручьев, а также в небольших закрытых губах, где на формирование льда мало оказывает влияние ветрового волнения. 
Неподвижный лед обычно формируется в первой половине декабря. 
Продолжительность ледостава в среднем составляет $5$~месяцев, но в суровые годы может доходить до $7$~месяцев (\cite{Babkov_Golikov_1984}). 

	\subsection{Баренцево море}

Баренцево море~--- окраинное море, характерной особенностью гидрологического режима которого является наличие двух  водных масс~--- арктической (полярные воды, большую часть года покрытых плавучими льдами) и субарктической (субполярных вод, свободных от плавучих льдов) (\cite{Adrov_1992}). 

Мурманским побережьем или Мурманом называют береговую линию Северного ледовитого океана от мыса Святой нос на востоке до реки Ворьемы на западе. 
Данный район разделяют на несколько областей: Западный Мурман~--- от реки Ворьемы до острова Кильдин или до Кольского залива, и Восточный Мурман~--- далее на восток до мыса Святой нос (\cite{Derugin_1915}).

Постоянный подток теплых атлантических вод препятствует образованию льда вдоль Мурманского побережья, и он встречается главным образом во внутренних частях губ и заливов.
Несколько большее количество льда образуется ежегодно в юго-восточном районе Мурмана, в то время как по Западному Мурману, как правило, не образуется сплошного припая. 
В основном, исключая некоторые опресненные закрытые бухты и заливы, влияние морского льда на распределение животных невелико, гораздо большее значение зимой играет сильное промораживание литорали во время отлива (\cite{Propp_1971}).

Приливы на Мурмане являются правильными полусуточными и образуются единой атлантической приливной волной. 
Далее она распространяется вдоль Мурмана на восток до Новой Земли. 
Высота приливной волны составляет $3$ метра. 

В среднем, соленость вод у Мурманского побережья составляет $33,2 - 33,6$\permil. 
Только весной во время сезонного увеличения берегового стока наблюдается краткое распреснение поверхностных слоев до $28 - 30$\permil, однако толщина опресненного слоя не превышает $2 - 3$~м.

Кольский залив~--- самый крупный из заливов Мурманского побережья Баренцева моря, лежит на границе Восточного и Западного Мурмана.
Географически в Кольском заливе выделяется три части, называемые коленами залива. 

Первое, северное или нижнее колено простирается от входа в Кольский залив до линии, соединяющей устье губы Средней и мыс Лас. 
Эта часть залива наиболее глубоководная (более $400$~м). 
Береговая линия северного колена Кольского залива чрезвычайно изрезана, и  здесь находятся самые крупные губы (\cite{Derugin_1915}), в том числе Пала-губа, ставшая объектом наших наблюдений .


Среднее колено (глубины до $200$~м) изогнуто в направлении к северо-западу и простирается на юг до мысов Пинагория и Мишукова. 
Второй участок наблюдений был расположен в районе границы северного и среднего колена Кольского залива (Ретинское).

Южная или верхняя часть наиболее мелкая (глубина около $50$~м), имеет направление с севера на юг, как и нижняя. 
В кут Кольского залива впадает две крупные реки~--- Тулома и Кола, и одна более мелкая~--- Лавна (\cite{Derugin_1915}).  
В районе самого узкого участка Кольского залива (Абрам-мыс) был расположен третий участок исследования в данном районе.
Последний участок, исследованный в Кольском заливе был расположен на западном берегу залива в черте города Мурманск (Северное Нагорное) в $3$~км от устья реку Туломы.

Воды Кольского залива неоднородны по своим свойствам. 
Это связано с несколькими причинами: большая протяженность залива, наличие глубоко вдающихся в побережье губ, влияние стока рек и ручьев. 
Гидрологическое лето начинается в поверхностных слоях воды в начале июля и продолжается до конца августа. 
Летом вода прогревается до $+8 - +18^{\circ}C$ в различных частях залива.

В  северном колене залива летом поверхностный слой значительно распреснен и соленость может достигать $8$\permil, причем толщина распресненного слоя может достигать $3-4$~метров. 
Глубже соленость не опускается ниже $30$\permil и у дна достигает $34$\permil. 
Зимой соленость поверхностного слоя также составляет $30 - 34$\permil. 

В южном колене в районе Абрам-мыса колебания солености на поверхности еще более заметны. 
Здесь сказывается не только сезонность стока, но и значительное влияние оказывает приливно-отливные течения. 
Летом во время прилива поверхностный слой толщиной до 3 метров обладает соленостью от $2$ до $16$\permil, в то время как на глубине $3$~метра соленость колеблется в пределах от $28$ до $31$\permil. 
В отлив мощность опресненного слоя увеличивается до $8$~метров, а поверхностная вода становится практически пресной (\cite{Derugin_1915}).

Таким образом, исследованные нами участки в Кольском заливе расположены в контрастных по географическим условиям его частях и позволяют относительно полно судить о данной акватории.

Фауна литораль Западного Мурмана наиболее богата по сравнению с остальным Мурманским побережьем. 
Традиционно, это связывают с более высокой среднегодовой температурой (температура воздуха в губах Западного Мурмана может быть на $0,4^{\circ}C$ выше по сравнению с Восточным Мурманом) и соленостью (выше $31$\permil\ в поверхностном слое) и закрытости губ Западного Мурмана от основной акватории моря (\cite{Guryanova_et_al_1930}). 
К сожалению, данный регион оказался для нас малодоступен при исследованиях, и мы располагаем лишь данными об обилии маком в губах Ура и Печенга.
Однако данные губы расположены в разных частях Западного Мурмана, что позволяет нам делать предвательные выводы о данном регионе.

Береговая линия Восточного Мурмана менее изрезана, чем Западного Мурмана. 
Побережье большинства небольших заливов и губ не защищено от прибойного воздействия (\cite{Guryanova_Ushakov_1929}).
Таким образом, Восточный Мурман на большем его протяжении не является благоприятным для развития литоральных инфаунных сообществ, однако существуют глубоко вдающиеся в побережье бухты, в которых обнаруживается меньшее волновое воздействие. 
Именно на литорали таких губ и заливов и формируются наиболее богатые инфаунные сообщества данного региона, включающие {\it M.~balthica}.

Наши исследования охватывают Восточный Мурман на значительном его протяжении: $6$~участков от губы Гаврилово до губы Ивановская (длина береговой линии более $150$ километров).
Обследованные бухты варьируют по длине, степени изолированности и наличию в них ручьев и небольших рек, влияющих на локальное опреснение.


География исследований охватывает в том числе Дальний пляж губы Дальне-Зеленецкой~--- исторически наиболее обследованной бухты на Мурмане.
Губа Дальне-Зеленецкая включает в себя две бухты~--- бухта Оскара и бухта, в кутовой части которой располагается литоральная отмель Дальнего Пляжа. 
Важной характеристикой губы является изолированность ее от интенсивного волнового воздействия за счет наличия островов на входе в губу.
	
При максимальных отливах протяженность литорали Дальнего пляжа с северо-запада на юго-восток составляет около $460$~м, а с юго-запада на северо-восток -- около $400$~м. 
	
В южной части отмели располагается дельта небольшого Зеленецкого ручья, вызывающего незначительное опреснение. 
Так, грунтовая вода, взятая у самого ручья, имеет соленость $32,9$\permil, а взятая на два метра в стороне от ручья~--- $34,07$\permil (\cite{Prigorovskiy_1948}). 
Гидрологический режим характеризуется тем, что в бухту заходят воды из более глубоких и холодных слоев открытого моря, что вызывает уменьшение температуры и повышение солености (\cite{Voronkov_et_al_1948}).

Волновая активность в губе не превышает $1,5 - 2$ балла (\cite{Alexeev_1976}). 
Наиболее сильному волновому воздействию подвержена южная и юго-восточная части отмели, где на галечно-валунном пляже располагается зона штормовых выбросов.
Придонная скорость течений, вызванных приливной волной, составляет $0,8$~м/сек. при глубине 0,3-0,5 метров и 0,06 м/сек. при глубине более $2$~метров.

Для песчаных отмелей характерна только одна граница~--- уровень высачивания, который делит пляж на две части, отличающиеся по условиям увлажненности донного осадка во время отлива (\cite{Streltsov_Agarova_1978}). 
Обширный, располагающийся ниже уровня высачивания и увлажненный во время отлива <<ватт>> простирается от отметок 1,25 до 2,1 м. над нулем глубин, сменяясь выше уровня высачивания узким $30$-метровым пляжем, где вода, занимавшая во время прилива интерстициальное пространство, вместе с грунтовыми водами вытекает на поверхность донного осадка. 
В западной части пляжа, самые верхние горизонты заняты валунной грядой (\cite{Agarova_et_al_1976}). 

Грунты отмели однообразны почти на всем ее протяжении. 
Мощность верхнего слоя ничтожна, и составляет $5 - 8$~см (\cite{Prigorovskiy_1948}). 
Для отмели процессы размыва преобладают над накоплением. 
Даже в зоне относительно высокой аккумуляции, в <<языках>> дельты ручья, мощность голоценовых отложений составляет всего $15 - 30$~см.

Максимальная концентрация песков (более $90$\% по массе) отмечена в юго-восточной оконечности у подножья террасы, сложенной древними морскими песками. Еще одной особенностью пляжа является повышенное содержание алевропелитов (\cite{Pavlova_1976}). 
Их локализация на пляже обусловлена эрозивной волноприбойной деятельностью, доминирующей при среднем уровне малой воды (\cite{Alexeev_1976}).

Органические вещества представлены гумусовыми соединениями и битумоидами местного и континентального происхождения (\cite{Gurevich_Yakovleva_1976}).
Наши мониторинговые работы в губе Дальне-Зеленецкая продолжают череду количественных гидробиологических исследований данного района (\cite{Prigorovskiy_1948, Matveeva_et_al_1955, Streltsov_et_al_1974, Agarova_et_al_1976, Zhukov_1984, Strelkov_et_al_2001}).

%%%%%%%%%%%%%%%%%%%%%%%%%%%%%%%%%%%%%%%%%%%%%%%%%%%%%%%%%%%%%%%%%%%%%%%%%%%

		\section{Экология вида}
Двустворчатый моллюск \textit{M.~balthica}~--- амфибореальный литоральный вид.
Это обычная литоральная форма в Белом море, у берегов Мурмана и далее на запад, вдоль атлантических берегов Европы~--- до Франции. 
По Атлантическому побережью Северной Америки макомы распространены от Лабрадора до штата Джорджия. 
В северной части Тихого океана~--- от Берингова моря до Японского, а по американскому побережью~--- до Калифорнии. 
В юго-восточной части Баренцева моря и в прилегающей части Карского моря они обитают  не на литорали, а на глубине нескольких метров. 
Моллюски заселяют всю основную часть Балтийского моря, далеко заходя во все заливы, где живут до глубины более 100 метров (\cite{Zacepin_Filatova_1968}).

В настоящее время вид {\it Macoma balthica} по результатам аллозимного анализа предлагают разделять на два подвида: {\it M.~b.~balthica}, обитающий в северной части Тихоокеанского региона, и {\it M.~b.~rubra} из Северо-Восточной Атлантики. 
Однако  в морях, связанных с  Атлантикой, существуют очаги распространения тихоокеанской формы. 
Так, в Балтийском и Баренцевом море Атлантическая и Тихоокеанская формы сосуществуют и образуют гибриды (\cite{Vainola_2003}). 
В Белом море встречается в основном {\it M.~b.~balthica}, и лишь в устье Онеги было обнаружено два экземпляра {\it M.~b.~rubra} (\cite{Nikula_et_al_2007}).
К настоящему моменту нет прямых данных о влиянии данных генетических особенностей на экологические характеристики особей, поэтому в данной работе рассматривается вид {\it Macoma balthica} sensu lato.

\textit{M.~balthica}~--- эвригалинный и эвритермный вид. 
Особи данного вида, обитающие в Белом море, способны выдерживать опреснение до $18$\permil\ и температуру до $25^{\circ}C$ (\cite{Naumov_2006}), но в Балтийском море макомы встречаются при солености $5$\permil (\cite{Karpevich_Shurin_1970}.
В экспериментах на моллюсках из Балтийского моря при температуре от $0$ до $22^oC$ и солености от $4$ до $25$\permil\ смертность взрослых особей оставалась менее 1\% в сутки (\cite{Karpevich_1968}.

Питаются макомы, собирая длинными червеобразными сифонами детрит с поверхности грунта (\cite{Naumov_2006}). 
Кроме того, показано, что особи \textit{M.~balthica} могут питаться как фильтраторы, когда в придонном слое воды находится взвешенный пищевой материал (Бубнова, 1972, Герасимова, 1988). 
Моллюски встречаются на илисто-песчаных грунтах, где обитают, закапываясь до глубины $2-6$~см. 
Они могут существовать и при длительной осушке: взрослые особи встречаются в местах с осушкой до 20 часов в сутки (\cite{Sveshnikov_1963}). 

\textit{M.~balthica} обитает на всех типах грунта: от чистого песка до жидкого ила. 
По данным Н.Л. Семеновой (1974) численность маком всегда ниже на чистом песке и увеличивается с увеличением заиления. 
Известно, что более богатым органическими веществами является более тонкий по гранулометрическому составу грунт (\cite{Bubnova_1972}), поэтому отмеченный выше характер распределения особей \textit{M.~balthica} может быть связан не столько с механическим составом грунта, сколько с обеспеченностью моллюсков пищей.

Макомы встречаются от самых верхних горизонтов литорали до глубины 140–160 метров (в Балтийском море). 
Было показано, что распределение макомы по литорали зависит главным образом от наличия заиленных пляжей (\cite{Semenova_1974}), где она находит подходящие условия для питания. 
	
Особи \textit{M.~balthica} обладают значительной подвижностью (\cite{Sveshnikov_1963}). 
Моллюски передвигаются в подповерхностном слое, и на грунте остается характерный след --- неглубокая извитая борозда (\cite{Naumov_2006}). 

Для мелких маком показан другой механизм передвижения, в первую очередь с целью расселения~---так называемый биссусный дрифт. 
Хотя во взрослом состоянии макомы не образуют биссуса, молодые особи могут выпускать длинные нити, которые позволяют даже слабому потоку поднимать особь над грунтом и переносить на некоторые расстояния. 
Показано, что способностью к биссусному дрифту обладают макомы размером до $4$~мм (\cite{Armonies_Hellwig-Armonies_1992}). 
Дальность этих миграций зависит от размера раковины макомы и длины биссусной нити, при этом некоторые особи мигрируют более чем на $10$ километров. 
В Северном море существуют поселения \textit{M.~balthica}, для которых показано пополнение не за счет личинок, а за счет переоседания особей из Ваттового моря (\cite{Beukema_deVlas_1989}).


		\section{Структура поселений {\it Macoma balthica}}

Наиболее изучены поселения \textit{M.~balthica} в Северном и Белом морях.

В Северном море в районе Ваттового моря \textit{M.~balthica} является одним доминирующих и широко распространенных видов зообентоса, как в литоральных, так и в сублиторальных местообитаниях.
Данная акватория характеризуется очень пологой литоральной зоной, которая формирует обширные (до нескольких километров шириной) илисто-песчаные отмели, именуемые ваттами, и образует обширные мелководья. 
Поселения маком встречаются здесь в широком диапазоне глубин и на разных типах грунтов (\cite{Beukema_et_al_1993, Hiddink_et_al_2002_predation_infauna, Hiddink_et_al_2002_predation_epifauna, Hiddink_2003}).
Максимальная средняя численность, описанная с Северном море составляет около $1600$~экз./м$^2$ (\cite{Reading_1979}), а биомасса может достигать $500$~г/м$^2$.

Оседание личинок маком в Северном море происходит весной главным образом на нижние горизонты литорали (\cite{Strasser_Gunter_2001}). 
Через несколько месяцев молодь моллюсков сдвигается к верхним горизонтам литорали (\cite{Armonies_Hellwig-Armonies_1992}). 
Особи в возрасте старше $1$ года распределены гораздо более равномерно, занимая практически все горизонты литорали и верхнюю сублитораль (\cite{Beukema_et_al_1993}). Поскольку локализация ювенильных и взрослых маком пространственно разделена, предполагается, что в ходе развития моллюски как минимум дважды мигрируют между различными горизонтами литорали. 

%В Балтийском море 

В Белом море особи \textit{M.~balthica} встречаются  повсеместно, за исключением Горла и Воронки. 
Моллюски обитают на различных глубинах: от верхнего горизонта литорали до  $4-5$~м. 
В эстуарных районах (дельта Северной Двины, Мезени) отмечен уход моллюсков в сублитораль до глубины $20$~м (\cite{Semenova_1974, Naumov_2006}).
По данным различных исследователей (\cite{Babkov_Golikov_1984, Naumov_2006}) для среднего и нижнего горизонта литорали с мягкими грунтами характерно формирование сообществ с доминированием \textit{M.~balthica}. 

Плотность поселений \textit{M.~balthica} может значительно изменяться как в пространстве, так и во времени. 
Максимальная плотность поселения характерна для нижнего горизонта литорали (\cite{Semenova_1974, Maximovich_et_al_1991}). 
По данным А.\:Д.~Наумова с соавторами максимальная плотность поселения для Белого моря отмечена в губе Чупа в биоценозе \textit{M.~balthica} и составила $2885$~экз./м$^2$ (\cite{Naumov_2006}).

%Также важными характеристиками поселений являются размерная и возрастная структура. Для M. balthica описано бимодальное и мономодальное распределение особей (Максимович и др., 1991; Николаева, 1997; Николаева, 1998; Назарова, 2003). Обе размерные структуры поселений, по-видимому, формируются в результате неравномерного пополнения поселений молодью.
%При массовом оседании личинок  \textit{M.~balthica}, в зависимости от выживаемости сеголеток, возможно два варианта развития поселения. 
%Если выживаемость хорошая, то можно наблюдать ежегодное смещение модального класса по оси размеров. При новом оседании личинок до полного отмирания особей первой генерации формируется бимодальное распределение. Другой описанный вариант~--- к следующему сезону сеголетки практически исчезают, и происходит новое оседание личинок. При повторении этой схемы наблюдается мономодальное распределение с доминированием по численности самых мелких особей (сеголеток) при практически полном отсутствии крупных особей. Естественно, при плохой выживаемости и отсутствии значительного оседания личинок поселение достаточно быстро отмирает (Максимович и др. 1991).

		\section{Динамика численности поселений {\it Macoma balthica} и влияющие на нее факторы}

%\textcolor{red}{Сегерстрале в Балтике, Букма в Ваттовом море, и Рейзе там же, Каф. данные, Наумов.
%Пополнение как ключевой фактор у бентосных видов с планктонной личинкой. 
%Пополнение = нерест + оседание + первая зима.
%Ваттово море - сильные/слабые зимы влияют на хищников, они влияют на ракушек. 
%Влияние численности взрослых маком на спат.}

\textit{M.~balthica}~--- бентосный инфаунный моллюск, взрослые особи которого перемещаются на относительно небольшие расстояния (не более метра) (\cite{Beukema_et_al_1993}).
Таким образом, вклад миграций в динамику взрослых особей пренебрежимо мал.
Основной процесс, определяющий динамику численности поселений маком~--- пополнение поселений молодью.
Процесс пополнения поселений состоит из нескольких этапов: формирование личиночного пула, оседание спата в поселение и первая зимовка.
Для Северного и Белого морей показано, что в пополнении поселений молодью выживаемость спата в первую зиму более важна, чем непоследственно количество осевших особей (\cite{Beukema_et_al_1998, Strasser_Gunter_2001, Maximovich_Gerasimova_2004}). 


В то же время в Северном море существуют поселения \textit{M.~balthica}, для которых показано пополнение не за счет личинок, а за счет переоседания особей из Ваттового моря (\cite{Beukema_deVlas_1989}). 
Для мелких маком показан специфический активный механизм передвижения, в первую очередь с целью расселения~--- так называемый биссусный дрифт. 
Хотя во взрослом состоянии макомы не образуют биссуса, молодые особи могут выпускать длинные нити, которые позволяют даже слабому потоку поднимать особь над грунтом и переносить на некоторые расстояния. 
Показано, что способностью к биссусному дрифту обладают макомы размером до $4$~мм (\cite{Armonies_Hellwig-Armonies_1992}). 
Дальность этих миграций зависит от размера раковины макомы и длины биссусной нити, при этом некоторые особи мигрируют более чем на $10$~километров. 

\subsection{Особенности жизненного цикла \textit{Macoma~balthica}}
При исследовании динамики популяций животных вопросы жизненного цикла и размножения играют важную роль, поскольку определяют потенциальное увеличение популяции за счет рождаемости. 

Исследователи приводят различные данные о возрасте и размере, при которых макомы достигают половой зрелости. 
Так, Л.П.~Флячинская пишет, что в Белом море \textit{M.~balthica} достигает половой зрелости к третьему году жизни при размере раковины $13-15$~мм (\cite{Flyachinskaya_1999}). 
Для Рижского залива отмечено созревание маком в возрасте $2-3$~года при размере $10-12$~мм (\cite{Karpevich_1968}). 
Для маком из бухты Уитстейбл (Англия) показан размер половозрелости $8$~мм (\cite{Caddy_1967}), а для бухты Ден-Хелдер (Голландия) точных указаний нет, но половозрелые особи находились в размерном классе $5-12$~мм (\cite{Lammens_1967}; цит. по \cite{Semenova_1980}).  
В работе Семеновой (\cite*{Semenova_1980}) высказывается идея, что ключевым фактором для возможности половозрелости является именно размер, а не возраст животного, и этот размер для макомы составляет $8$~мм. 
Это подтверждено и дальнейшими исследованиями на Белом море (\cite{Maximovich_1985}).

Время нереста различается в разных частях ареала. 
В бухте Ден-Хелдер (Голландия) нерест макомы длится два месяца в марте-апреле (\cite{Lammens_1967}; цит. по \cite{Semenova_1980}). 
В бухте Уитстейбл (Англия), на побережье Шотландии и в районе Датских ваттов (Северное море) нерест также продолжается два месяца, но сроки его более поздние~--- апрель-май (\cite{Caddy_1967, Stephen_1931}). 
Еще позже происходит нерест в Балтийском море (данные по Мекленбургской бухте) и в губе Дальне-Зеленцкой на Баренцевом море~--- с мая по август, т.е. в течение 4 месяцев (\cite{Oertzen_1972, Agarova_1974}). 
В Белом море сроки нереста очень сжатые~--- от $1-2$ недель до месяца в июне-начале июля (\cite{Semenova_1980, Maximovich_1985, Zubakha_et_al_2000}).

Считается, что триггером для начала нереста у макомы служит прогрев воды до $+10^{\circ}C$ (\cite{Maximovich_1985, Semenova_1980, Kaufman_1977}).  
Бьёкма и Меган (\cite{Beukema_Meehan_1985}) предполагают, что тригерная температура нереста является причиной, ограничивающей распределение моллюсков на юг~--- в южных широтах минимальная температура воды превышает $10^{\circ}C$. 
В этом случае южная граница ареала \textit{M.~balthica} должна совпадать с десяти градусной зимней изотермой, которая проходит около $45^{\circ}$с.ш. почти по всей северной Атлантике и круто изгибается к югу рядом с Американским побережьем. 
Действительно, распространение маком на юг по Американскому побережью дальше (до $37^{\circ}$с.ш.~--- штат Джорджия), чем по Европейскому (до $45^{\circ}$с.ш.~---  южная Франция).

В губе Дальне-Зеленецкой (Баренцево море) нерест маком наблюдался при температуре $2-9^{\circ}C$ (\cite{Agarova_1974}). 
Восточный Мурман характеризуется низкими среднегодовыми температурами, и наблюдаемый сдвиг тригерной температуры нереста можно объяснить эффектом смещения температур размножения видов на северных краях ареалов (\cite{Thorson_1946}). 
Это подтверждается еще и тем, что близкие к баренцевоморским температурные условия размножения \textit{M.~balthica} ($0-13^{\circ}C$) наблюдаются в Северной Канаде (\cite{Gilbert_1978}).

В Белом море проводили детальные исследования жизненного цикла маком. 
Гонады половозрелых маком созревают к концу мая, но нерест начинается, когда температура поверхностных слоев воды в море достигает $9-11^{\circ}C$  (\cite{Maximovich_1985, Flyachinskaya_1999}). 
Половые продукты выметываются в воду, где происходит оплодотворение. 
В лабораторных условиях при температуре $12^{\circ}C$ велигер формировался через $4,5$~суток после оплодотворения. 
Через $17-20$~суток на стадии педивелигера формировалась нога, и через 3$0-33$~суток происходил метаморфоз. 
В этот момент молодь оседает на субстрат, и ее называют <<плантиграда>> или <<спат>>, хотя второе название более распространено (\cite{Flyachinskaya_1999}). 
При оседании молоди маком, по-видимому, у особей не происходит выбора подходящего по характеру грунта, но затем происходит их перераспределение за счет биссусного дрифта (\cite{Armonies_Hellwig-Armonies_1992, Huxham_Richards_2003}). 


\subsection{Факторы, влияющие на пополнение поселений}
Большинство исследований, посвященных проблеме пополнения, выполнено в одном из районов Северного моря~--- так называемом Ваттовом море.	
Изначально было показано, что одним из ключевых факторов, влияющих на пополнение поселений \textit{M.~balthica}, является температура в зимний период, которая воздействует не напрямую, а через влияние на обилие хищников (\cite{Beukema_et_al_1998, Beukema_Dekker_2014, Dekker_Beukema_2014}).

В ряде других работ также было показано влияние различных хищников на численность и распределение молоди маком. 
Так, для Ваттового моря именно обилием хищников объясняется формирование временных скоплений молоди маком на верхней литорали. 
При изучении факторов, обуславливающих такое распределение для \textit{M.~balthica} было показано, что обилие бентосных хищников больше на нижней литорали, и лишь молодь краба \textit{Carcinus maenas} в значительных количествах встречается на верхней литорали. 
В полевых и лабораторных экспериментах было показано, что присутствие хищников значительно снижает численность спата, в то время как влияния на крупных особей обнаружено не было. 
По-видимому, за первый год макомы выходят из-под контроля бентосными хищниками за счет увеличения размеров тела (\cite{Hiddink_et_al_2002_predation_epifauna}). 

Также при анализе динамики личинок различных беспозвоночных в планктоне было показано, что после суровых зим численность личинок краба \textit{Carcinus maenas} значительно снижалась, и они появлялись в планктоне на $6-8$ недель позже, чем после мягких зим. 
По-видимому, с этим временным несоответствием связано большее пополнение поселений маком после суровых зим (\cite{Strasser_Gunter_2001}).

В более поздних исследованиях на Ваттовом море было показано, что влияние суровых зим на пополнение \textit{M.~balthica} не столь широкомасштабно, как предполагалось ранее, и, по-видимому, существуют другие факторы, определяющие более локальные вариации в пополнении поселений (\cite{Strasser_et_al_2003, Flatch_2003}).
Пресс хищников не объяснил эти различия, изменения сообществ и поступления биогенных элементов не объяснили картину, поскольку действовали лишь на отдельных участках. 
Наиболее вероятным фактором, по мнению данных авторов, является топографическая разница между двумя акваториями, где располагались исследованные участки. 
Предполагается, что в зависимости от закрытости акватории островами, и преобладающего направления ветров, будет идти более или менее эффективный перенос личинок и биссусный дрифт, а, значит, и пополнение поселения (\cite{Strasser_et_al_2003}).

Для другого участка Ваттового моря было показано, что комбинация эффектов высокого пресса хищников вместе с высоким обилием взрослой макрофауны обуславливает 95 процентное снижение количества спата теллинид (\textit{M.~balthica} и \textit{Tellina tenuis}) после мягких зим (\cite{Flatch_2003}). 

Хотя влияние на пополнение поселения молодью плотности взрослых особей того же вида представляется достаточно логичным механизмом, существуют лишь отдельные работы, посвященные внутривидовым взаимодействиям у \textit{M.~balthica}. 
Так, в ряде работ показано, что плотность молоди не зависит от обилия взрослых маком (\cite{Olafsson_1989, Vincent_et_al_1989, Beukema_et_al_2001, Richards_et_al_2002}). 

 Также было показано, что влияние плотности взрослых маком на рост спата зависит от типа грунта. 
На илисто-песчаном грунте, где и взрослые, и молодые моллюски питаются как собирающие детритофаги, рост спата подавляется при увеличении плотности взрослых особей. 
На песке, где молодь питаются как собирающие детритофаги, а взрослые~--- как фильтраторы, влияния на рост спата показано не было (\cite{Olafsson_1989}).

Для Белого моря существуют лишь несколько работ, посвященных отдельным аспектам пополнения поселений маком. 
Так, И.\:В.~Бурковским с соавторами показано, что макомы оседают вне плотных поселений взрослых (\cite{Burkovskiy_et_al_1998}). 
Также показано, что важную роль в динамике численности личинок и спата влияет принос личинок с соседних акваторий. 
В течение лета формируется сначала бимодальная размерная структура спата, с двумя пиками личинок в планктоне, которая к концу августа сливается в мономодальную (\cite{Zubakha_et_al_2000}). 
Показана высокая смертность особей на всех этапах пополнения поселения. Так, смертность пелагических личинок оценивают в $36,4$\% за сезон, а смертность спата~--- $59$\% за сезон (\cite{Burkovskiy_et_al_1998}).


		\section{Продолжительности жизни и рост {\it Macoma balthica} в различных частях ареала}
Данные о продолжительности жизни маком весьма противоречивы. 
Исследователи оценивают ее по-разному: от $3$ лет в Балтийском море до $15$ и даже до $30-35$ в Белом  (\cite{Segerstrale_1960, Semenova_1970, Maximovich_Kunina_1982}). 
Столь значительные расхождения в определении возраста связаны с особенностями применяемых методик, поэтому представляется важным рассмотреть этот вопрос подробно.

Современные методы определения возраста двустворчатых моллюсков разделяют на несколько типов: скульптурные, структурные, физико-химические и статистические. 
При этом первые три группы методов позволяют устанавливать возраст отдельных особей, в то время как статистические методы требуют изучения группы особей и дают вероятностную оценку возраста (\cite{Zolotarev_1989}). 

Принципиальной основой скульптурных методов определения возраста моллюсков является наличие на раковине неоднородностей скульптуры, связанных с периодичным (суточным, сезонным) изменением скорости роста особи. 
Для лучшего выделения наружных меток роста иногда створки просвечивают (\cite{Brousseau_1979}), поверхность раковины обрабатывают соляной кислотой для удаления периостракума (\cite{Tabunkov_1974}). 
С возрастом у многих моллюсков происходит изменение морфологии наружных колец. 
Обычно уменьшается их выраженность, увеличиваются различия в степени проявления однотипных элементов, что затрудняет или делает невозможным адекватную оценку возраста особи. 	

Другая проблема~--- возможность образования на поверхности раковины дополнительных меток роста. 
Они могут возникать при нересте, шторме, нападении хищников и носят непериодический характер. 
Однако обычно воздействие этих факторов непродолжительно и дополнительные кольца часто выражены слабее сезонных, что позволяет их различить при некотором навыке. 

Периодические изменения скорости роста моллюсков отражаются также на особенностях внутреннего строения раковин. 
Для анализа строения раковины изготовляют радиальные спилы или шлифы, после чего анализируют или непосредственно их, или приготовленные по ним ацетатные реплики.
Этот метод менее чувствителен к возрасту моллюсков и позволяет выделять годовые слои у старых особей со скоростью роста всего $0,1-0,25$~мм/год. 
Однако проблема дифференциации сезонных и прочих слоев роста остается.

Физико-химические методы определения возраста моллюсков более трудоемки и дороги, однако они позволяют определять возраст у моллюсков, у которых отсутствуют структурные возрастные образования. Данная группа методов основана на изучении неоднородности плотности, химического и изотопного состава. Наиболее часто используют определения стабильных изотопов кислорода, содержание магния и стронция, рентгенография створок (\cite{Zolotarev_1989})). 

Первые два метода достаточно точны, однако необходимость отбирать серию проб из створок, затрудняют их применение на некрупных объектах. 
Рентгенография выявляет сезонные изменения плотности скелетного вещества. 
Считается, что пара слоев с высокой и низкой плотностью образует годовой прирост (\cite{Ralph_Maxwell_1977}). 
Однако метод рентгенографии разработан пока недостаточно чтобы получить широкое применение. 

Таким образом, химические методы достаточно точны и дают объективные возрастные характеристики. 
Однако высокая стоимость и трудоемкость ограничивает их применение в массовых гидробиологических исследованиях, и до настоящего для \textit{M.~balthica} не проведена калибровка видимых колец и слоев химическими методами. 
В итоге скульптурные и структурные методы определения роста в настоящее время наиболее распространены из-за их доступности и относительной легкости процедуры. 
Неизбежная субъективность в интерпретации колец и слоев остановки роста ограничивает возможность сравнения данных, полученных разными исследователями. 
Однако в рамках одного исследования однотипность интерпретации колец и слоев позволяет сравнивать особей по получаемому относительному возрасту. 
Измерения меток зимних остановок роста, разделяющих кольца роста, позволяет восстановить размер особи в разном возрасте и реконструировать линейный рост.


Рост рассматривается как комплексный отклик организма на совокупность условий в локальном местообитании. 
Однако не менее интересной представляется попытка разложить всю совокупность условий на отдельные факторы, влияющие на ростовые характеристики. 
Одним из главных, определяющих рост факторов, является температура (\cite{Gilbert_1973, De_Wilde_1975, Bachelet_1980}). 
При повышении температуры происходит увеличение скорости метаболических процессов, в том числе темпов роста моллюсков в толерантных пределах.  
Для {\it M.~balthica} показано, что оптимальные условия роста~--- температура $0 - 10^{\circ} C$, а когда температура превышает $15^{\circ} C$ рост прекращается (\cite{De_Wilde_1975}). 
Ограничение роста при высоких температурах было отмечено и другими авторами, хотя на южной границе ареала (по-видимому, за счет физиологической адаптации) рост происходил и при более высоких температурах (\cite{Bachelet_1980}).

Изучение широтных измерений характера роста {\it M.~balthica} интересовали многих исследователей (\cite{Gilbert_1973, Bachelet_1980, Beukema_Meehan_1985, Wenne_Klusek_1985, Hummel_et_al_1998}).
Для сравнения использовали различные параметры: среднюю скорость роста роста моллюсков (отношение максимальной длины к возрасту особей), коэффициент $k$ уравнения Берталанфи, параметр $\omega$ ( произведение коэффициентов $L_{\infty}$ и $k$ из уравнения роста Берталанфи), годовой прирост.

Бьёкма и Меган (\cite{Beukema_Meehan_1985}) показали, что ростовые характеристики {\it M.~balthica} имеют выраженный широтный градиент.
В качестве параметра сравнения в этой работе был использован параметр $\omega$, который считается более адекватным для задач сравнения ростовых характеристик, чем сравнение параметров уравнения Берталанффи напрямую (\cite{Appeldoorn_1983}). 
Не смотря на широкое варьирование данного параметра, наблюдается уменьшение скорости роста в более северных популяциях маком.
В данной работе данные по российской части ареала {\it M.~balthica} ограничены работой Н.~Л.~Семёновой (\cite*{Semenova_1970}).

Хюммель с соавторами (\cite{Hummel_et_al_1998}) расширили географию исследования роста маком в северных морях, проанализировав годовой прирост моллюсков из Норвежского, Печорского, Баренцева и Карского морей.
Было показано, что группировки, генетически различные по результатам аллозимного анализа, отличались по величинам годового прироста. 
Макомы в популяциях с южной границы ареала росли медленнее, чем в центральной части ареала, а размах варьирования прироста в Белом море был сравним с таковым в европейских популяциях.
Печорские макомы, значительно отличающиеся генетически, также характеризовались более низкими годовыми приростами, однако дотигали при этом наибольших размеров.

Другим фактором, влияющим на процесс роста, является обилие пищи. 
Наблюдается достоверная связь между содержанием хлорофилла A на поверхности грунта, концентрацией фитопланктона и скоростью роста особей {\it M.~balthica} (\cite{Beukema_et_al_1977, Kube_et_al_1996}). 
С обилием пищи тесно связано влияние на рост моллюсков гранулометрического состава грунта и содержание в нем органических веществ. 
Чем меньше диаметр частиц грунта, тем больше площадь их поверхности и тем больше на них бактерий, соответственно более мелкодисперсный грунт оказывается для маком <<питательнее>>. 
Показано, что скорость роста особей на песчаном грунте ниже, чем на илистом (\cite{Wenne_Klusek_1985, Maximovich_et_al_1992}). 
Выявлена достоверная связь скорости роста моллюсков с долей мелкой фракции грунта и содержанием в нем органических веществ (\cite{Kube_et_al_1996}).

Соленость также оказывает влияние на рост моллюсков, хотя данные о характере этого влияния различны. 
Некоторые авторы отрицают влияние солености на скорость роста (\cite{Bachelet_1980}), другие авторы утверждают, что скорость роста и размеры моллюсков имеют тенденцию уменьшаться с уменьшением солености (\cite{Segerstrale_1960, Kube_et_al_1996}). 

Литературные данные о скорости роста моллюсков на различном мареографическом уровне противоречивы. 
Башле (\cite{Bachelet_1980}) обнаружил, что в эстуарии р.~Жиронда (южной границе ареала макомы в Европе) скорость роста моллюсков на верхней литорали значительно выше, чем на нижней. 
На верхней литорали моллюски достигают большего размера и дольше живут. 
Обратная связь найдена Грином (\cite{Green_1973}) и Харвеем и Винсентом (\cite{Harvey_Vincent_1990}) для канадских популяций {\it M.~balthica}. 
В качестве причины  таких различий авторы предполагают большее время питания на нижней литорали и негативное влияние высоких температур, ограничивающих рост, на верхней. 
Бьёкема и соавторы (\cite{Beukema_et_al_1977}) показали, что наибольшие скорости роста имеют моллюски со средней литорали, поскольку на верхней литорали скорость роста ограничивается временем питания, а на нижней~--- количеством пищи. 
В Белом море при сравнении темпов роста моллюсков из литоральных и сублиторального поселений, максимальный темп роста обнаружен в сублиторали. 
Однако различий в скорости роста между горизонтами литорали отчемено не было (\cite{Maximovich_et_al_1992}).
В Гданьском заливе скорость роста возрастала с глубиной~--- более высокая скорость роста обнаружена у моллюсков на глубине $35 - 75$~м, по сравнению с особями из мелководной ($5 - 6$~м) части залива (\cite{Wenne_Klusek_1985}). 
Обратная ситуация наблюдается в других частях Балтийского моря~--- минимальную скорость роста имеют моллюски с глубины $35$~м, максимальную с глубины $3$~м (\cite{Segerstrale_1960}). 

Таким образом, по-видимому сама по себе глубина обитания не оказывает влияние на темпы роста моллюсков. 
Кроме того, значительная подвижность маком затрудняет интерпретацию результатов. 
Скорость роста моллюсков определяются в первую очередь температурой и обилием пищи, а возникающая в ряде случаев зависимость от глубины может появляться за счет комбинирования этих параметров.


\afterpage{\clearpage}

%%описание участков

    \section{Характеристика района исследования}
        \subsection{Географическое и физиономическое описание}
			\subsubsection{Белое море}

\paragraph{Участок материковой литорали, расположенный в 800 м к югу от поселка Лувеньга.}

Данный разрез имеет вид прямоугольника, длина которого ограничена $10$~метрами, а ширина равна ширине литорали в максимальный сизигийный отлив (72 метра). 
На данном участке пробы брались равномерно на протяжении всей ширины литорали. Описание разреза дано по работе А.~Полоскина (\cite*{Poloskin_1996}).

Верхняя часть литорали на разрезе представляет гравийно-мелкокаменистую осыпь со значительным наклоном дна, нижняя граница которой расположена в 
$10$~метрах от штормовых выбросов.

Ниже на литорали располагается пологий пляж с илистым песком с заметными вкраплениями крупного песка. 
Во время отлива здесь могут оставаться небольшие лужицы. 
В данном биотопе отмечены отдельные выбросы пескожилов {\it Arenicola marina} и кое-где тонкий мат зеленых нитчаток. 
В дальнейшем эта зона будет называться <<верхний пляж>>. 
На расстоянии $19$~метров от штормовых выбросов верхний пляж ограничивает валунная гряда.

За валунной грядой следует валунная россыпь с плотными поселениями фукоидов. 
Постепенно россыпь разреживается и между валунами появляются окна илисто-песчаного грунта. 
Плотность пояса фукоидов также постепенно уменьшается, и к $37$~метру от штормовых выбросов фукоиды и валуны практически полностью исчезают. 
В дальнейшем этот биотоп будет называться <<пояс фукоидов>>.

Ниже располагается следующий хорошо различимый биотоп --- пояс взморника {\it Zostera marina} (данное название сохранится за ним и далее). 
Плотное, почти со стопроцентным проективным покрытием, поселение этих растений на илисто-песчаном грунте простирается до $59$~метра от штормовых выбросов. 
Помимо взморника, в данном биотопе отмечено большое количество нитчатых водорослей с прикрепленных на них молодью мидий {\it Mytilus edulis}.

От $59$ до $72$~метра расположен участок, осушающийся только в сизигийный отлив на два с небольшим часа. 
Илисто-песчаный пляж данного биотопа служит местом обитания для поселений пескожила и большого количества мидиевых щеток. 
Данный биотоп будет именоваться <<нижний пляж>>. 


\paragraph{Участок в Илистой губе острова Горелого.}
Ширина литорали на данном участке составляет $24$~метра. 
Так как верхняя литораль характеризуется каменистым грунтом, то пробы брались только в среднем и нижнем горизонте литорали.
Верхняя часть литорали представляет собой гравийную россыпь, выходящую на приморский луг. 
Ниже (в среднем горизонте) следует илисто-песчаный пляж с редкими некрупными камнями и отдельными выбросами пескожилов.  
На расстоянии $15$~метров от линии штормовых выбросов появляются редкие вкрапления фукоидов (на границе среднего и нижнего горизонтов литорали) и увеличивается количество мелких камней, но  все же этот участок можно характеризовать как илисто-песчаный пляж. 
Плотность поселения {\it Arenicola marina} заметно увеличивается по сравнению со средним горизонтом.
На уровне $17-21$ метров от штормовых выбросов располагается валунная гряда с плотными поселениями фукоидов (нижний горизонт литорали). 
В данной зоне пробы отбирались на участках, не закрытых талломами водорослей. 
В районе нуля глубин на данном участке также характерен илисто-песчаный грунт с плотным поселением {\it Arenicola marina}.


\paragraph{Участок в эстуарии реки Лувеньги.}
На данном участке ширина литорали составляет $500$~метров. 
На всем протяжении это практически горизонтальный илисто-песчаный пляж с плотным поселением пескожилов. 
Так как этот участок расположен в эстуарии реки, то он характеризуется пониженной соленостью. 
В данном районе пробы брались на расстоянии $350$~метров от линии штормовых выбросов на нижнем горизонте литорали.


            \subsubsection{Баренцево море}

            \paragraph{Северное Нагорное}
Данный участок расположен в третьем колене Кольского залива, на южном его берегу в пределах одноименного района г.~Мурманск. 
Собственно литораль начинается за жилым массивом, в месте расположения опор моста через Кольский залив. 
Место сбора находилось в 600 м севернее моста. 
Ширина литорали на данном участке составляет $100$~м. 
Верхний горизонт литорали представлен небольшими валунами и россыпью гравия. 
Средний и нижний горизонты литорали представляют собой достаточно пологий илисто-песчаный склон с редкими валунами. 
Грунт достаточно сильно эвтрофицирован, очень вязкий. 
Между валунами встречаются поселения пескожила {\it Arenicola marina}.

            \paragraph{Абрам-мыс}
Участок  в районе  Абрам-мыса  находится в третьем колене Кольского залива, максимально удаленном от моря.
Абрам-мыс --- район   города   Мурманск,  расположенный   на противоположной стороне от основного городского массива, напротив порта. 
Исследованный участок   литорали   находился   в   $1,5$~км   к   выходу   из   залива   от   причала,   куда   приходит пассажирский катер. 
Ширина   литорали   на   данном   участке   составляет   $45$~м.   
Верхний   горизонт   литорали представлен  каменисто-галичной  россыпью. 
В среднем  горизонте литорали на поверхности илисто-песчаного   грунта   располагаются   валуны,   покрытые   фукоидами   ({\it Fucus  vesiculosus}), которые   формируют   практически   сплошной   покров   с   отдельными   «окнами»   грунта (проективное  покрытие фукоидов $90~\%$).  
При приближении  к нижнему горизонту литорали количество   валунов   уменьшается,   и   проективное   покрытие   фукоидов   составляет   здесь   не более $10~\%$.

    \paragraph{Ретинское}
Ретинское находится на западном берегу Кольского залива, напротив г.~Североморск. 
В береговую линию вдается небольшая, овальной формы губа. 
Ширина литорали составляет около $60$~м. 
Дно каменистое, между камнями --- илисто-песчаный грунт, достаточно промытый. 
На верхнем горизонте литорали располагаются крупные валуны, покрытые фукусами и балянусами, чуть ниже находятся крупные камни полностью покрытые фукоидами.
Средний и нижний горизонты литорали представлены среднего размера камнями, примерно половина из которых покрыта фукоидами. 

    \paragraph{Пала-губа}
Пала-губа   представляет   собой   глубоко   вдающуюся   в   берег   губу   длинным   узким «горлом», за которым следует расширение, формирующее несколько губ второго порядка. 
В «горле» расположен остров Шалим, и, таким образом, губа соединяется с Кольским заливом узкими   проливами.   
В   основной   части   Пала-губы   расположено   несколько   более   мелких островков. 
Исследованный участок располагался в длинной узкой губе (бухта Дровяная), закрытой на выходе островом Зеленый.
В кут губы впадает крупный ручей, формирующийся на литорали во время отлива оформленное русло, положение которого за два года наблюдений не изменилось.
Ширина литорали на данном участке составляет  $130$~м. 
Верхний горизонт литорали представлен   каменисто-валунной   россыпью,   которая   на   границе   со   средним   горизонтом становится более разреженной, и покрыта зарослями фукоидов ({\it Fucus vesiculosus}). 
Средний и нижний   горизонты   представлены   двумя   илисто-песчаными   пляжами,   разделенными каменисто-валунной грядой на месте резкого локального увеличения угла уклона свала. 
На нижней литорали грунт более заилен, и на поверхности располагаются агрегации {\it Mylius edulis} («мидиевые щетки»).

    \paragraph{Печенга}
Печенга расположена на Западном Мурмане, в $150$~км от границы с Норвегией. 
Собственно поселок находится на берегу сильно вдающейся в полуостров губы Печенга. 
Сбора материала производился в средней части этой губы, на удалении $1,5$~км от кута губы. 
Литораль на этом участке достигает ширины $50$~м. 
Верхний горизонт литорали представлен среднего размера валунами. 
На среднем горизонте валуны расположены более редко, а между ними находится россыпь достаточно крупного гравия. 
Нижний горизонт литорали илисто-песчаный. 

    \paragraph{Губа Гаврилово}
Гаврилово – наиболее западная губа из исследованных нами участков на Восточном Мурмане. 
Эта   губа   с  достаточно   широким   входом,   свободно   открывающаяся  в  Баренцево море.   
Восточную   ее   часть   несколько   закрывает   от   прибоя   мыс,   формирующий   «горло», несколько суженное относительно основной части. 
В восточной части кута губа формирует узкий отрог длиной   около $200$~м, по которому течет ручей, распадающийся в центральной части   губы  в  среднем   горизонте   литорали   на   два   рукава,   и   сливающиеся   ниже   обратно   в единое русло.
Ширина литорали  в  данной губе составляет  $500$~м  (без  учета отрога, дно которого полностью обнажается в отлив) Верхний горизонт литорали на данном участке представлен каменисто-галечной   россыпью.   
Средний   горизонт   литорали   представляет   собой   обширную илисто-песчаную   отмель   с   отдельными   камнями   и   валунами.  
В   основном   камни   и   валуны сконцентрированы   вдоль   русла   ручья.   
Нижний   горизонт   литорали   представлен   песчаным пляжем. 

    \paragraph{Губа Ярнышная}
Губа Ярнышная представляет собой одну из крупнейших губ Восточного Мурмана, ее длина составляет около $5$~км. 
Вход в губу свободно открыт в Баренцево море. 
Берега губы сильно изрезаны. 
В кут губы Ярнышной впадает два крупных ручья --- Ярнышный и Бобровый. 
По мере продвижения в кут губы, скальная и каменистая литораль переходит в каменисто-песчаную и илисто-песчаную. 
Исследованный участок расположен в юго-восточной части кута губы в районе впадения ручья Ярнышный.
На   участке   исследования   средний   горизонт   литорали   представлен   илисто-песчаным пляжем   с   отдельными   валунами,   поросшими   фукоидами   ({\it Fucus   vesiculosus}).   
В   среднем   и нижнем горизонте литорали вдоль русла ручья были остатки умершего плотного поселения  {\it Mytilus eduls} («мидиевая банка»), поэтому в период исследования в данном биотопе грунт был черный с запахом сероводорода.

    \paragraph{Губа Дальнезеленецкая}
Исследованный   участок   был   расположен   на   литоральной   отмели   Дальний   Пляж, поскольку именно он был в 1970х годах выбран как модель для описания литоральной фауны мягких   грунтов   на   Баренцевом   море.   
\textcolor{red}{Физико-географическое   описание   участка   по литературным данным представлено в главе «литературный обзор».}
На   границе   верхней   литорали   расположен   валунно-галечный   пляж,  нижняя   часть которого заросла фукоидами ({\it Fucus vesiculosus}). 
Ниже по литорали в юго-восточной части пляжа   тянется   узкая   (около   $10-15$~м   шириной)   полоса   крупного   песка,   в   которой представители макробентоса практически отсутствуют.
Средний   горизонт   литорали --- это   обширный   илисто-песчаный   пляж,   в  пределах которого визуально выделяется три зоны: с преобладанием пескожилов  {\it Arenicola marina}, с преобладанием   мелких   полихет-трубкостроителей   (в   первую   очередь, {\it Fabricia   sabella})   и переходная   зона   между   этими   сообществами.   
Нижняя   литораль   представлена   каменисто-песчаным пляжем с зарослями бурых ({\it Fucus vesiculosus}, {\it Fucus serratus}) и красных ({\it Palmaria  palmata}) водорослей на камнях.

    \paragraph{Губа Шельпино}
Шельпино   представляет   собой   большую   губу   с   широким   горлом,   в   котором расположен   один   крупный   и   несколько   мелких   островов.   
В   юго-восточной   части   губа продолжается   длинным   (около   $400$~м)   узким   отрогом,   полностью   обнажающимся   в   отлив. 
Именно в этом отроге и происходил пробоотбор. 
По   литорали   отрога   протекает   небольшой   ручей,   не   формирующий   четкого   русла. 
Летом вдоль ручья развиваются массовые скопления зеленой водоросли рода  {\it Enteromorpha}. 
Верхняя и средняя литораль представляют собой песчаный пляж с отдельными камнями и валунами. 
В среднем горизонте на камнях появляются водоросли. 
Нижний горизонт литорали оккупирован плотным поселением мидий {\it Mytilus edulis} на грунте.

    \paragraph{Губа Порчниха}
Порчниха  ---  крупная   губа,   закрытая   от   моря   островом   Большой   Олений.   
Кутовая часть разделена скальным мысом на две части. 
Одна из них направлена на юг, вторая на запад. 
Наши   исследования   проводились   в   западной   части   губы.   
В   эту   часть   губы   впадает полноводный ручей, имеющий на литорали оформленное русло. 
Верхний горизонт литорали представлен   гравийной   россыпью.   
Средний   горизонт   ---   илисто-песчаным   пляжем   с отдельными   лежащими   на   поверхности   камнями,   поросшими   бурыми   водорослями  {\it Fucus vesiculosus}.   
При   этом   в   грунте   также   присутствует   гравий   и   крупная   галька,   полностью погруженная в песок. 
Нижний горизонт литорали представлен плотным поселением   {\it Fucus  vesiculosus}.

    \paragraph{Губа Ивановская}
Губа Ивановская с $2009$ года является памятником природы областного значения. 
Это сама восточная из исследованных нами акваторий в Баренцевом море. 
Длина губы составляет около $20$~км. 
Вход в губу закрывает  остров Нокуев.
В связи с  закрытостью губы и ее размерами приливно-отливная волна   распространяется   в   губе   медленно   и   задержка   приливов   и   отливов   в   куту   губы относительно прилегающей морской акватории достигает нескольких часов. 
Губа   разделена   поперечными   рядами   на   три   части,   называемых   «ковшами». 
Исследования   проводили   во   втором   ковше   на   северном   берегу.   
Исследованный   участок представлял   собой   верхнюю   сублитораль   (глубина   $0,8$~м)   с   небольшим   уклоном   свала. 
Физиономически участок представлял собой илисто-песчаный «пляж» с отдельными камнями, лишенными растительности. 
Ниже исследованного участка начинался пояс взморника {\it Zostera  sp.} 

        \subsection{Характеристики грунта}
Анализ   гранулометрического   состава   грунта   позволяет   косвенно   оценивать  интенсивность   гидродинамики   и,   следовательно,   условия   питания   моллюсков   на исследованных   участках.   
Кроме   того,   наличие   доступного   детрита   можно   оценивать   с помощью определения концентрации органических веществ в грунте.

            \subsubsection{Белое море}
\textcolor{red}{тут надо осенью сделать анализ грунтов по заповеднику}

    \begin{table}[ht]
    \caption{Гранулометрический состав грунта на исследованных участках в Баренцевом море}
    \label{tab:grunt_granulometriya_White}
    \begin{tabularx}{\textwidth}{|p{0.2\textwidth}|*{6}{X|}} \hline
                            & Галечники      & Гравий & Псаммиты грубые & Псаммиты средние & Псаммиты мелкие & Алевриты и пелиты \\
                             & \textgreater10 & 10-1   & 1-0,5           & 0,5-0,25         & 0,25-0,1        & \textless0,1      \\ \hline
Эстуарий р.~Лувеньги           &                &        &                 &                  &                 &                   \\ \hline
о.~Горелый                     &                &        &                 &                  &                 &                   \\ \hline
материковая литораль, Лувеньга &                &        &                 &                  &                 &                   \\ \hline
Западная Ряшкова Салма         &                &        &                 &                  &                 &                   \\ \hline
Южная губа, о.~Ряшков          &                &        &                 &                  &                 &                   \\ \hline
о.~Ломнишный                   &                &        &                 &                  &                 &                   \\ \hline
Сухая Салма                    & 0,41           & 0,8    & 0,87            & 3,57             & 61,5            & 32,85             \\ \hline
бухта Клющиха                  & 0,1            & 0,1    & 0,3             & 9,9              & 89,6            & 0                 \\ \hline
   \end{tabularx}

    {\footnotesize Примечание: указана доля частиц, \%}
    \end{table}

            \subsubsection{Баренцево море}
Анализ грунта проводили на 8 участках из исследованных в Баренцевом море.
По соотношению частиц различного размера в грунте на всех участках преобладает (более $50$~\%) песчаная фракция (табл.~\ref{tab:grunt_general_Barents}). 
    \begin{table}[ht]
    \caption{Соотношение основных включений в грунте на участках литорали Баренцева моря}
    \label{tab:grunt_general_Barents}
%    \begin{tabular}{|*{4}{p{0.25\textwidth}|}} \hline
    \begin{tabularx}{\textwidth}{|*{4}{X|}} \hline
    Участок  &  гравий &  песок &  алевриты и пелиты 
        \\ \hline
    Абрам-мыс &  $1,13$ & $52,41$ & $44,16$
        \\ \hline
    Пала-губа &   $0$ &  $99,00$ &  $1,0$
        \\ \hline
    Гаврилово &  $0,04$ & $98,41$ &  $0,74$
        \\ \hline
    Ярнышная &   $3,09$ & $95,02$ & $0,99$
        \\ \hline
    Дальнезеленецкая &  $0,31$ & $98,27$ & $0,82$
        \\ \hline
    Шельпино &  $30,10$ & $67,62$ & $1,60$
        \\ \hline
    Порчниха & $25,63$ & $74,78$ & $1,68$
        \\ \hline
    Ивановская  & $17,22$ & $70,50$ & $11,09$
        \\ \hline
    \end{tabularx}

    {\footnotesize Примечание: указана доля частиц, \%}
    \end{table}


Гравий присутствует на всех участках, кроме Пала-губы.  
Доля  гравия может достигать $30$~\%. 
Интересно, что участки со значительным ($> 10$\textcolor{red}{\%?}) содержанием   гравия  ---  наиболее   восточные   из   всех   изученных.   
Доля   илистых   фракций обычно   невелика,   лишь   на   литорали   Абрам-мыса   и   в   сублиторали   губы   Ивановская   она превышает   $10$~\%.   
Из   всех   исследованных   участков   только   Абрам-мыс   представляет   собой типичную илисто-песчаную отмель, поскольку доля песка и алевритов и пелитов практически одинаковая и близка к $50$~\%.
Более детальное рассмотрение гранулометрического состава грунта показало, что по соотношению различных песков участки неоднородны (табл.~\ref{tab:grunt_granulometriya_Barents}).
    \begin{table}[ht]
    \caption{Гранулометрический состав грунта на исследованных участках в Баренцевом море}
    \label{tab:grunt_granulometriya_Barents}
    \begin{tabularx}{\textwidth}{|p{0.14\textwidth}|*{8}{X|}} \hline
    & круп\-ный и сред\-ний гравий  &  мел\-кий гра\-вий &  очень мел\-кий гра\-вий & очень круп\-ный песок & круп\-ный песок &  сред\-ний песок & мел\-кий песок & алеври\-ты и пели\-ты \\
        Участок &   $>10$ &  $10-5$ &   $5-3$ &  $3-1$ & $1-0,5$ &   $0,5-0,25$ &    $0,25-0,1$ &    $<0,1$
        \\ \hline
    Абрам-мыс &  $0$ &  $0,77$ &  $0,35$ &  $2,84$ &  $6,82$ &  $6,74$ & $36,01$ &  $44,16$
        \\ \hline
        Пала-губа  &  $0$ &  $0$ &  $0$ &  $24,45$ &  $13,91$ &  $26,00$ &  $34,63$ &  $1,00$
        \\ \hline
        Гаврилово &  $0$ &  $0$ &  $0,04$ &   $4,58$ &   $23,80$ &  $58,42$ &  $11,61$ &  $0,74$
        \\ \hline
        Ярнышная  &  $0,20$ &  $0,17$  &  $2,72$ &  $32,03$ &  $29,66$ &  $19,02$ &  $14,31$  &  $0,99$
        \\ \hline
        Даль\-не\-зе\-ле\-нец\-кая &  $0$ &  $0,08$ &    $0,22$ &    $7,81$ &    $36,20$ &  $38,26$ &   $16,00$ &   $0,82$
        \\ \hline
    Шельпино  &  $16,06$ &   $10,28$ &   $3,77$ &  $7,96$  &  $22,76$ &  $22,45$ &    $14,46$ &  $1,60$ 
        \\ \hline
    Порчниха  &  $7,48$ &   $11,62$ &  $6,54$ &   $26,17$ &  $16,84$ &  $12,74$ &  $19,03$ &  $1,68$ 
        \\ \hline
    Ивановская &  $6,06$ &    $7,10$ &   $4,06$ &   $16,70$ &  $9,27$ &   $8,88$ &   $35,65$ &  $11,09$
        \\ \hline
    \end{tabularx}

    {\footnotesize Примечание: указана доля частиц, \%}
    \end{table}


Содержание   органических   веществ   в   грунте   было   невелико,   и   на   всех   участках   не превышало $2$~\% (табл.~\ref{tab:grunt_granulometriya_Barents}).
    \begin{table}[ht]
    \caption{Содержание органических веществ в грунте на исследованных участках в Баренцевом море}
    \label{tab:grunt_granulometriya_Barents}
    \begin{tabularx}{\textwidth}{|*{9}{X|}} \hline
    участок & Абрам-мыс &   Пала-губа &  Гав\-ри\-ло\-во  & Яр\-ныш\-ная &   Даль\-не\-зе\-ле\-нец\-кая &  Шель\-пи\-но &   Порч\-ни\-ха &   Ива\-нов\-ская
        \\ \hline
     &  $1,58$ &    $0,12$ &   $0,50$ &   $0,65$ &   $0,39$ &   $0,82$ &   $0,70$ & $1,38$
        \\ \hline
    \end{tabularx}

    {\footnotesize Примечание: указано содержание органических веществ в грунте, \%}
    \end{table}



\afterpage{\clearpage}

%%роль в сообществах разнообразных местообитаний
		\chapter{Биотический фон в сообществах {\it Macoma balthica}}

	\section{Белое море}
Описание сообществ макробентоса проводили на 6 мониторинговых участках в Кандалакшском заливе отдельно на каждом мареографическом уровне. 
Таким образом, всего было получено $12$ таксономических списков.
Всего на  исследованных участках было обнаружено $57$ таксонов беспозвоночных (приложение~\ref{app:species}, таблица~\ref{tab:White_species}).
Из них только непосредственно {\it Macoma balthica} встречена во всех 12 описаниях.
$18$ таксонов из $57$ были представлены только в одном описании.
Количество таксонов в одном описании колебалось от $5$ в верхнем горизонте материковой литорали в Лувеньге до $42$ у нуля глубин в Южной губе о.~Ряшкова.
По количеству таксонов преобладали представители Polychaeta (22 таксона).

%Классификация участков по видовому составу была проведена при помощи кластеризации методом ближайшего соседа по коэффициенту Жаккара. 
%Достоверность кластеров оценивали с помощью анализа сходства профилей (SIMPROF) (\cite{Clarke_et_al_2008}).

При анализе фаун с выделением горизонтов было выделено 6 групп участков ($p<0,05$) (рис.~\ref{ris:cluster_white_species_tidal}). 
	\begin{figure}[p]
		\begin{center}
			\includegraphics{../White_Sea/soobshestvo/White_fauna_tidal_jaccard_single_1.pdf}
		\end{center}
	\caption{Классификация отдельных горизонтов литорали в Белом море по видовому составу}
	\label{ris:cluster_white_species_tidal}

	\footnotesize{Кластеризация по методу ближайшего соседа с использованием коэффициента Жаккара. По оси ординат --- коэффициент Жаккара. Цветом показаны кластеры, достоверно выделяющиеся при 5\% уровне значимости.}
	\end{figure}
Группировка станций по кластерам неоднородна. 
Три кластера демонстрируют сходство по географическому признаку (голубой, синий и, отчасти, фиолетовый на рис.~\ref{ris:cluster_white_species_tidal}), три по мареографическому признаку (красный, синий и голубой кластер на рис.~\ref{ris:cluster_white_species_tidal}), остальные не показывают явной приуроченности.


При анализе фаун отдельных участков было выделено три группы (рис.~\ref{ris:cluster_white_species_sites}.) 
	\begin{figure}[p]
		\begin{center}
			\includegraphics{../White_Sea/soobshestvo/White_fauna_sites_jaccard_single_1.pdf}
		\end{center}
	\caption{Классификация исследованных участков в Белом море по видовому составу}
	\label{ris:cluster_white_species_sites}

	\footnotesize{Кластеризация по методу ближайшего соседа с использованием коэффициента Жаккара. По оси ординат --- коэффициент Жаккара. Цветом показаны кластеры, достоверно выделяющиеся при 5\% уровне значимости.}
	\end{figure}
Первый кластер образуют сообщества в  Южной губе о.~Ряшкова и на о.~Ломнишный, которые близки как географически, так и мареографически (исследованы сообщества у нуля глубин).
В отдельный кластер попадает материковая литораль в районе Лувеньги, что связано, по-видимому, с максимальным биотопическим разнообразием на данном участке, поскольку здесь в пределах ограниченного участка представлены как илисто-песчаные пляжи верхней и нижней литорали, так и заросли фукоидов и взморника.
Участки на о.~Горелый, в эстуарии р.~Лувеньги и на островной литорали Западной Ряшковой салмы формируют третий кластер.
Он характеризуется наименьшим внутренним сходством, однако участки, где исследовали только средний горизонт литорали (Западная Ряшкова салма и эстуарий р.~Лувеньги) более сходны между собой, чем попадающий в тот же кластер о.~Горелый.

\afterpage{\clearpage}

	\section{Баренцево море}

Всего на исследованных участках нами было обнаружено $48$ таксонов беспозвоночных (приложение~\ref{app:species}, таблица~\ref{tab:Barents_species}). 
При этом в пределах каждого из горизонтов литорали были встречены все таксоны. 
Более трети таксонов ($17$ из $48$) - это редкие виды (встречены в одном описании), и лишь {\it Macoma balthica} встречается во всех описаниях. 
Количество таксонов на участке колебалось от $6$ (верхняя сублитораль губы Ивановская) до $22$ (средний горизонт литорали губы Дальне-Зеленецкая). 
По соотношению таксонов на всех участках преобладали Polychaeta.
	
%Классификация участков по видовому составу была проведена при помощи кластеризации методом ближайшего соседа по коэффициенту Жаккара. 
%Достоверность кластеров оценивали с помощью анализа сходства профилей (SIMPROF) (\cite{Clarke_et_al_2008}).

При анализе отдельных горизонтов литорали было выделено два кластера: сублитораль губы Ивановская и литораль всех остальных участков (рис.~\ref{ris:cluster_barents_species_tidal}). 
	\begin{figure}[p]
		\begin{center}
			\includegraphics{../Barenc_Sea/soobshestvo/Barents_fauna_tidal_jaccard_single_1.pdf}
		\end{center}
	\caption{Классификация отдельных горизонтов литорали в Баренцевом море по видовому составу}
	\label{ris:cluster_barents_species_tidal}

	\footnotesize{Кластеризация по методу ближайшего соседа с использованием коэффициента Жаккара. По оси ординат --- коэффициент Жаккара. Цветом показаны кластеры, достоверно выделяющиеся при 5\% уровне значимости.}
	\end{figure}

Возможно, что была выбрана слишком дробная единица анализа, и посмотрим как распределятся полные описания сообществ по изученных участкам литорали (рис.~\ref{ris:cluster_barents_species_sites}. 
	\begin{figure}[p]
		\begin{center}
			\includegraphics{../Barenc_Sea/soobshestvo/Barents_fauna_sites_jaccard_single_1.pdf}
		\end{center}
	\caption{Классификация исследованных участков в Баренцевом море по видовому составу}
	\label{ris:cluster_barents_species_sites}

	\footnotesize{Кластеризация по методу ближайшего соседа с использованием коэффициента Жаккара. По оси ординат --- коэффициент Жаккара. Цветом показаны кластеры, достоверно выделяющиеся при 5\% уровне значимости.}
	\end{figure}
Результат аналогичен, достоверно отличается только фауна губы Ивановская.


Для оценки влияния гранулометрического состава грунта на состав сообщества были выделены группы илисто-песчаная, песчаная и гравийно-песчаная литораль. 
В результате не было обнаружено достоверного влияния данного показателя на видовой состав сообщества ($R=0,053, p=0,36$).
	
Таким образом, таксономический состав сообществ на исследованных участках достаточно вариабелен, и по-видимому, сходство определяется географической близостью участков. 

%была еще тема на конференции в Мурманске. Не надо ли добавить оттуда и пересчитать все нафиг.

\subsection{Структура сообщества на литорали губы Дальне-Зеленецкая}
На литоральной отмели Дальний Пляж губы Дальне-Зеленецкой были проведены мониторинговые наблюдения за структурой сообщества.
В результате кластерного анализа по матрице коэффициентов Брей-Кертиса было выделено две достоверных (SIMPROF: p=0,05) группы (рис~\ref{ris:DZ_cluster_soobshestva}). 
	\begin{figure}[p]
		\begin{center}
			\includegraphics{../after_Deryuginskie/2_disser/station_bray_2002_SIMPROF_BW1.pdf}
		\end{center}
	\caption{Классификация станций на Дальнем пляже губы Дальнезеленецкая}
	\label{ris:DZ_cluster_soobshestva}

	\footnotesize{По оси ординат --- коэффициент Брея-Кертиса. Сплошными линиями показаны группы, достоверно выделяющиеся при 5\% уровне значимости.}
	\end{figure}

Сравнение двух выделенных зон по обилию видов-эдификаторов (рис.~\ref{ris:DZ_edifikatory}, A) показало что внутри выделенных групп станций их обилие различается. В группе А обилие полихет-трубкостроителей \textit{F.~sabella} максимально, и это подтверждает наше предположение, что эта группа соответствует сообществу полихет-трубкостроителей по терминологии Матвеевой с соавторами (\cite{Matveeva_et_al_1955}). 
	\begin{figure}[p]
	\begin{minipage}[b]{.46\linewidth}
	\begin{center}
	{\footnotesize А.{\it Fabricia sabella}}
		\includegraphics[width=\linewidth]{../after_Deryuginskie/2_disser/Fabricia_in_zones1.pdf}
	\end{center}
	\end{minipage}
	%
	\hfil %Это пружинка отодвигающая рисунки друг от друга
	%
	\begin{minipage}[b]{.46\linewidth}
	\begin{center}
	{\footnotesize Б. {\it Arenicola marina}}
		\includegraphics[width=\linewidth]{../after_Deryuginskie/2_disser/Arenicola_in_zones1.pdf}
	\end{center}
	\end{minipage}
	\caption{Обилие видов-эдификаторов в выделенных сообществах}
	{\footnotesize На графике: жирная горизонтальная линия --- медиана, границы <<ящика>> --- 1 и 3 квартили, <<усы>> --- $1,5$ интерквартильного расстояния, точки - значения выпадающие за $1,5$ интерквартильных расстояния}
	\label{ris:DZ_edifikatory}
	\end{figure}
Аналогично, среднее обилие пескожилов  \textit{A.~marina} в группе В на порядок превышает таковое в группе А (рис.~\ref{ris:DZ_edifikatory}, Б), и, несмотря на разнородность станций данной группы, можно говорить о их принадлежности к сообществу пескожилов по терминологии Матвеевой с соавторами.
Характерно, что два вида-эдификатора~--- \textit{Arenicola marina} и \textit{Fabricia sabella}~--- демонстрируют антагонизм в распределении на литорали.

Рассмотрим динамику массовых видов и доминантов на литорали Дальнего пляжа губы Дальне-Зеленецкой. 
Данные $1973$~года приведены по работе \cite{Agarova_et_al_1976}

\paragraph{\textit{Fabricia sabella}}
В течение исследованного периода средние значения плотности поселений {\it Fabricia sabella} находились в диапазоне от 5500~(32\%) до 169~тыс.~(10\%)~экз./м$^2$ в сообществе пескожилов и от 95~тыс.~(30\%) до 190~тыс.~(18\%)~~экз./м$^2$ для сообщества трубкостроителей (приложение~\ref{app:DZ_soobshestvo_dynamic}, рис.~\ref{ris:DZ_N_dynamic_dominants}). 
Однако изменения численности \textit{F.~sabella} были статистически недостоверны в обоих исследованных сообществах (табл.~\ref{tab:DZ_trends_Kruskal-Wallis}). 
\begin{table}[p]
\caption{Изменения плотности поселений массовых видов в исследованных сообществах на литорали г.~Дальне-Зеленецкая}
\label{tab:DZ_trends_Kruskal-Wallis}
\begin{tabularx}{\textwidth}{l|XXl|XXl}
\hline
сообщество: & \multicolumn{3}{c|}{трубкостроителей}&\multicolumn{3}{c|}{пескожилов}\\ \hline
                   & W               & p      &    & W         & p         &     \\ \hline
{\it Fabricia sabella}   & 7,5             & 0,11   &    & 6,2       & 0,18      &     \\
{\it Pygospio elegans}   & 15,2            & 0,0095 & ** & 2,2       & 0,82      &     \\
{\it Capitella capitata} & 16,5            & 0,0055 & ** & 20,8      & 0,0008    & *** \\
{\it Arenicola marina}   & 3,5             & 0,48   &    & 32,5      & 1.544e-06 & *** \\
{\it Oligochaeta}        & 9,3             & 0,0054 & ** & 5         & 0,28      &  	\\ \hline  
\end{tabularx}

{\footnotesize Примечание: W~--- значение критерия Краскела-Уоллеса, p~--- доверительная вероятность.}
\end{table}

По-видимому, это связано со значительным варьированием численности червей в отдельных пробах. 
Современная численность Fabricia sabella в сообществе пескожилов сравнима с обилием данного вида в 1973 году, но достоверно уменьшилась в сообществе трубкостроителей (табл.~\ref{tab:DZ_donimamnts_1973_2000}).
\begin{table}[p]
\caption{Сравнение плотностей поселения массовых видов на литорали Дальнего Пляжа г.~Дальне-Зеленецкой в 1973 и 2000-х годах.}
\label{tab:DZ_donimamnts_1973_2000}
\begin{tabularx}{\textwidth}{XX|X|XXX|X}
\hline
		 &		    & $1973$ год	      & \multicolumn {3}{c|}{2002-2007} &	различие    \\
вид              & сообщество       & $M$               & $Me$         & $2,5$\% $Q$ & $97,5$\% $Q$ & ($p < 0,05$) \\ \hline
{\it Fabricia sabella} & труб\-ко\-стро\-ите\-лей & 240000          & 99470      & 8538    & 217327   & есть                      \\
{\it Fabricia sabella} & пес\-ко\-жи\-лов       & 40000           & 19845      & 368     & 223808   & нет                       \\
{\it Pygospio elegans} & пес\-ко\-жи\-лов       & 20100           & 21805      & 838     & 101834   & нет                       \\
{\it Arenicola marina} & труб\-ко\-стро\-ите\-лей & 4               & 8          & 0       & 405,8    & нет                       \\
Oligochaeta      & пес\-ко\-жи\-лов       & 25000           & 24990      & 451     & 92793    & нет                       \\ \hline
\end{tabularx}

{\footnotesize Примечание: $M$~--- средняя плотность поселения, экз./м$^2$, $Me$~--- медианная плотность поселения, экз./м$^2$ $Q$~--- квантили распределения}
\end{table}

\paragraph{\textit{Pygospio elegans}} 
Средняя численность многощетинковых червей {\it Pygospio elegans} в разные годы была оценена в 27~(23\%)-36~(41\%)~тыс.~экз./м$^2$ в сообществе пескожилов и от 1800~(15\%)~экз./м$^2$ до 18~(36\%)~тыс.~экз./м$^2$ в сообществе трубкостроителей (приложение~\ref{app:DZ_soobshestvo_dynamic}, рис.~\ref{ris:DZ_N_dynamic_dominants}). 
В сообществе пескожилов обилие данного вида оставалось стабильным в течение всего периода наблюдений (табл.~\ref{tab:DZ_trends_Kruskal-Wallis}) и не отличалось от такового в 1973 году (табл.~\ref{tab:DZ_donimamnts_1973_2000}). 
В сообществе трубкостроителей наблюдались его достоверные колебания (табл.~\ref{tab:DZ_trends_Kruskal-Wallis}). 
В 2003 году численность данных червей была минимальна, после чего происходило ее плавное увеличение и к 2007 году она становилась сравнима со значением численности, отмеченным для 1973 года (приложение~\ref{app:DZ_soobshestvo_dynamic}, рис.~\ref{ris:DZ_N_dynamic_dominants}). 

\paragraph{\textit{Capitella capitata}}
Средняя численность {\it Capitella capitata} достоверно изменялась в течение исследованного периода (табл.~\ref{tab:DZ_trends_Kruskal-Wallis}). 
Максимальное обилие для обоих сообществ было отмечено в 2002 году (600~(62\%) в сообществе пескожилов и 1800~(71\%)~экз./м$^2$ в сообществе трубкостроителей), после чего до 2006 года были колебания, и в 2006 -- 2007 наметилось некоторое его увеличение (приложение~\ref{app:DZ_soobshestvo_dynamic}, рис.~\ref{ris:DZ_N_dynamic_dominants}). 
Численности, указанные для данного вида в 1973 году на порядок превышают  максимальные значения, полученные нами в исследованный период.

\paragraph{\textit{Arenicola marina}}
Средняя численность пескожилов в сообществе трубкостроителей была стабильно низкой (табл.~\ref{tab:DZ_trends_Kruskal-Wallis}) и не превышала 10~экз./м$^2$. 
В 1973 году численность не отличалась от современного уровня (табл.~\ref{tab:DZ_donimamnts_1973_2000}). 
В сообществе пескожилов численность титульного вида демонстрировала достоверные колебания (табл.~\ref{tab:DZ_trends_Kruskal-Wallis}), но во все годы была более 40~экз./м$^2$ (приложение~\ref{app:DZ_soobshestvo_dynamic}, рис.~\ref{ris:DZ_N_dynamic_dominants}). 
Максимальное обилие Arenicola marina здесь было отмечено в 2002 году (84~(14\%)~экз./м$^2$). 
Подобные численности были отмечены и в 1973 году (приложение~\ref{app:DZ_soobshestvo_dynamic}, рис.~\ref{ris:DZ_N_dynamic_dominants}).

\paragraph{\textit{Oligochaeta}}
Средняя численность Oligochaeta в течение исследованного периода находилась в диапазоне от 3~(46\%) до 38~(43\%)~тыс.~экз./м$^2$ в сообществе пескожилов и от 32~(54\%) до 106~тыс.~экз./м$^2$ в сообществе трубкостроителей (приложение~\ref{app:DZ_soobshestvo_dynamic}, рис.~\ref{ris:DZ_N_dynamic_dominants}). 
Обилие малощетинковых червей было стабильно в сообществе пескожилов и сравнимо со значениями обилия в 1973 году (табл.~\ref{tab:DZ_donimamnts_1973_2000}), но достоверно изменялось в сообществе трубкостроителей (табл.~\ref{tab:DZ_trends_Kruskal-Wallis}) за счет значительного его варьирования между отдельными пробами.

\medskip

Нами были проведены детальные исследования поселений инфаунных двустворчатых моллюсков, обитающих совместно с {\it Macoma balthica}: {\it Cerastoderma edule} и {\it Mya arenaria}. 
В связи с низкой встречаемостью моллюсков в пробах и низкой численностью мы проводили описание поселения в пределах всего Пляжа, без разделения на зоны.

Поскольку по методике сборов в 1970х годах пробы для учета моллюсков промывали на сите с диаметром ячеи $5$~мм, для сравнения мы корректировали наши данные с учетом размерной структуры и отдельно приводим численность особей крупнее $5$~мм.

\paragraph{\textit{Cerastoderma edule}}
Средняя численность {\it Cerastoderma edule} не превышала 25 экз./м2 (рис.~\ref{ris:DZ_N_dynamic_bivalve}). 
%
	\begin{figure}[p]
	
	\begin{minipage}[b]{.46\linewidth}
	\begin{center}
	{\footnotesize \textit{Cerastoderma edule}}
		\includegraphics[width=55mm]{../after_Deryuginskie/2_disser/cockle_N_dynamic_all1.pdf}
	\end{center}
	\end{minipage}
	%
	\hfil %Это пружинка отодвигающая рисунки друг от друга
	%
	\begin{minipage}[b]{.46\linewidth}
	\begin{center}
	{\footnotesize \textit{Mya arenaria}}\\
		\includegraphics[width=55mm]{../after_Deryuginskie/2_disser/Mya_N_dynamic_all1.pdf}
	\end{center}
	\end{minipage}
\caption{Динамика обилия инфаунных моллюсков, обитающих совместно с {\it Macoma balthica}, на литорали Дальнего пляжа губы Дальне-Зеленецкой}
\label{ris:DZ_N_dynamic_bivalve}

{\footnotesize Примечание: N~--- плотность поселения, экз./м$^2$: белые столбцы~--- всех особей, серые столбцы~--- особей крупнее $5$~мм.}
\end{figure}
%
За все годы наблюдения плотность поселения сердцевидки достоверно изменялась (тест Краскел-Уоллиса: $W=16,2, p=0,01$), и было отмечено $2$ локальных максимума: в $2002$ и в $2007$~годах. 
Минимальное обилие было отмечено в $2005$~году. 
Количество моллюсков с длиной раковины более $5$~мм в период низкой численности (2004-2006 гг.) было сравнимо с плотностью поселения в 1973 году.

В разные годы минимальный размер особей Cerastoderma edule в пробах колебался от $2$ до $7$~мм (рис.~\ref{ris:DZ_size_str_bivalve}). 
%
	\begin{figure}[p]
		\includegraphics[height=.9\textheight]{../after_Deryuginskie/2_disser/size_structure_Cerastoderma_Mya_2mmclass1.pdf}
\caption{Размерная структура инфаунных моллюсков, обитающих совместно с {\it Macoma balthica}, на литорали Дальнего пляжа губы Дальне-Зеленецкой}
\label{ris:DZ_size_str_bivalve}

{\footnotesize Примечание: L,~мм~--- длина раковины, по оси ординат указана доля особей с соответствующей длиной раковины в сборах}
\end{figure}
%
Для данного региона это соответствует возрасту $1-2$ года (\cite{Genelt-Yanovskiy_et_al_2010}). 
Таким образом, пополнение молодью происходит не ежегодно. 
Максимальный размер особей в разные годы колебался от $32$ до $45$~мм. 
Годы снижения численности (2003 -- 2005) характеризовались практически полным отсутствием мелких особей. 
В то же время, максимальной плотности поселения сердцевидок ($2007$ год) совпал максимальным обилием некрупных особей ($7 - 21$~мм). 
Таким образом, характер динамики определяется массовостью пополнения поселения молодью.

\paragraph{\textit{Mya arenaria}}
Плотность поселения {\it Mya arenaria} за исследованные годы не превышала 25~экз./м$^2$ (рис.~\ref{ris:DZ_N_dynamic_bivalve}). 
Отмеченные колебания были статистически достоверны (тест Краскел-Уоллиса: $W=38,4, p<0,0001$).  
После относительно стабильного периода 2002 -- 2003 годов произошло резкое уменьшение плотности поселения мий в 2004 году, после чего она медленно увеличивается в период до 2006 года. 
В 2007 -- 2008 годах происходит резкое увеличение плотности поселения мий до максимальных значений за весь период наблюдений.
По характеру динамики размерной структуры можно предположить, что в 2002 -- 2003 году мы наблюдаем одну генерацию моллюсков, которая практически полностью элиминируется к 2004 году (рис~\ref{ris:DZ_size_str_bivalve}). 
В 2006-2008 году мы наблюдаем следующую генерацию, по-видимому, 2004 или 2005 года оседания.

\afterpage{\clearpage}
 
\afterpage{\clearpage}

%%микрораспределение
%% для компиляции в lualatex!!
%\documentclass[12pt, a4paper]{article}
\documentclass[12pt, a4paper]{disser}
\usepackage[english,russian]{babel}
\usepackage[warn]{mathtext}
%\usepackage[T2A]{fontenc}
%\usepackage[utf8]{inputenc}

\usepackage{xecyr} % Продукт Вашего покорного слуги ;)

%\setmainfont{DejaVu Serif}
\setmainfont{Liberation Serif}

\usepackage{color}
\usepackage{amssymb,amsmath}
\usepackage{graphicx}
\usepackage{multicol}

\textheight=24cm           % высота текста
\textwidth=16cm            % ширина текста
\oddsidemargin=0pt         % отступ от левого края
\topmargin=-1.5cm          % отступ от верхнего края
\parindent=24pt            % абзацный отступ
\parskip=0pt               % интервал между абзацами
\tolerance=2000            % терпимость к "жидким" строкам
\flushbottom               % выравнивание высоты страниц
%\def\baselinestretch{1.5} % печать с большим интервалом

%\title{}
%\author{\copyright~~С.А.~Назарова \thanks{e-mail:~sophia.nazarova@gmail.com}}
%\date{}


\begin{document}
	
	\begin{figure}[h]

	\begin{minipage}[b]{.46\linewidth}
%Фигурка в первом ряду слева размер отведенный под весь этот объект \textendash 0.46 от ширины строки
%Параметр [b] означает, что выравнивание этих министраниц будет по нижнему краю
	\begin{center}
	{\small N~{\it Macoma balthica}}
		\includegraphics[width=65mm]{../Barenc_Sea/distribution_Moran/Pala_moran_N_Macoma_balthica_.pdf}
	\end{center}
	\end{minipage}
%
	\hfil %Это пружинка отодвигающая рисунки друг от друга
%
	\begin{minipage}[b]{.46\linewidth}
%Следующий рисунок - первый ряд справа %DUNGEON S_4 \ AB
	\begin{center}
	{\small B~{\it Macoma balthica}}
		\includegraphics[width=65mm]{../Barenc_Sea/distribution_Moran/Pala_moran_B_Macoma_balthica_.pdf}
	\end{center}
	\end{minipage}

	
	\begin{minipage}[b]{.46\linewidth}
	%Фигурка в первом ряду слева размер отведенный под весь этот объект \textendash 0.46 от ширины строки
	%Параметр [b] означает, что выравнивание этих министраниц будет по нижнему краю
	\begin{center}
	{\small N~{\it Cerastoderma edule}}
		\includegraphics[width=65mm]{../Barenc_Sea/distribution_Moran/Pala_moran_N_Cerastoderma_edule_.pdf}
	\end{center}
	\end{minipage}
	%
	\hfil %Это пружинка отодвигающая рисунки друг от друга
	%
	\begin{minipage}[b]{.46\linewidth}
%Следующий рисунок - первый ряд справа %DUNGEON S_4 \ AB
	\begin{center}
	{\small B~{\it Cerastoderma edule}}
		\includegraphics[width=65mm]{../Barenc_Sea/distribution_Moran/Pala_moran_B_Cerastoderma_edule_.pdf}
	\end{center}
	\end{minipage}

%\smallskip



%\smallskip

	\begin{minipage}[b]{.46\linewidth}
%Фигурка в первом ряду слева размер отведенный под весь этот объект \textendash 0.46 от ширины строки
%Параметр [b] означает, что выравнивание этих министраниц будет по нижнему краю
	\begin{center}
	{\small N~{\it Priapulus caudatus}}
		\includegraphics[width=65mm]{../Barenc_Sea/distribution_Moran/Pala_moran_N_Priapulus_caudatus_.pdf}
	\end{center}
	\end{minipage}
%
	\hfil %Это пружинка отодвигающая рисунки друг от друга
%
	\begin{minipage}[b]{.46\linewidth}
%Следующий рисунок - первый ряд справа %DUNGEON S_4 \ AB
	\begin{center}
	{\small B~{\it Priapulus caudatus}}
		\includegraphics[width=65mm]{../Barenc_Sea/distribution_Moran/Pala_moran_B_Priapulus_caudatus_.pdf}
	\end{center}
	\end{minipage}

%\smallskip


	\caption{Микрораспределение макробентоса на литорали Пала-губы}
	\label{ris:moransI_Pala}
	\end{figure}



	\begin{figure}[h]

	\begin{minipage}[b]{.46\linewidth}
%Фигурка в первом ряду слева размер отведенный под весь этот объект \textendash 0.46 от ширины строки
%Параметр [b] означает, что выравнивание этих министраниц будет по нижнему краю
	\begin{center}
	{\small N~{\it Crangon crangon}}
		\includegraphics[width=65mm]{../Barenc_Sea/distribution_Moran/Pala_moran_N_Crangon_crangon_.pdf}
	\end{center}
	\end{minipage}
%
%	\hfil %Это пружинка отодвигающая рисунки друг от друга
%
%	\begin{minipage}[b]{.46\linewidth}
%Следующий рисунок - первый ряд справа %DUNGEON S_4 \ AB
%	\begin{center}
%		\includegraphics[width=65mm]{../Barenc_Sea/distribution_Moran/}
%	{\small B~{\it Crangon crangon}}
%	\end{center}
%	\end{minipage}

	
%	\begin{minipage}[b]{.46\linewidth}
	%Фигурка в первом ряду слева размер отведенный под весь этот объект \textendash 0.46 от ширины строки
	%Параметр [b] означает, что выравнивание этих министраниц будет по нижнему краю
%	\begin{center}
%	\includegraphics[width=65mm]{../Barenc_Sea/distribution_Moran/Pala_moran_N_Cerastoderma_edule_.pdf}

%	\end{center}
%	\end{minipage}
	%
%	\hfil %Это пружинка отодвигающая рисунки друг от друга
	%
%	\begin{minipage}[b]{.46\linewidth}
%Следующий рисунок - первый ряд справа %DUNGEON S_4 \ AB
%	\begin{center}
%		\includegraphics[width=65mm]{../Barenc_Sea/distribution_Moran/Pala_moran_B_Cerastoderma_edule_.pdf}
%	\end{center}
%	\end{minipage}

%\smallskip



%\smallskip

%	\begin{minipage}[b]{.46\linewidth}
%Фигурка в первом ряду слева размер отведенный под весь этот объект \textendash 0.46 от ширины строки
%Параметр [b] означает, что выравнивание этих министраниц будет по нижнему краю
%	\begin{center}
%		\includegraphics[width=65mm]{../Barenc_Sea/distribution_Moran/Pala_moran_N_Priapulus_caudatus_.pdf}
%	\end{center}
%	\end{minipage}
%
%	\hfil %Это пружинка отодвигающая рисунки друг от друга
%
%	\begin{minipage}[b]{.46\linewidth}
%Следующий рисунок - первый ряд справа %DUNGEON S_4 \ AB
%	\begin{center}
%		\includegraphics[width=65mm]{../Barenc_Sea/distribution_Moran/Pala_moran_B_Priapulus_caudatus_.pdf}
%	\end{center}
%	\end{minipage}

%\smallskip



%	\caption{Микрораспределение макробентоса на литорали Пала-губы (Продолжение)}
%	\label{ris:moransI_Pala}
	\begin{center}
	Рис. \ref{ris:moransI_Pala} (продолжение). Микрораспределение макробентоса на литорали Пала-губы.
	\end{center}

	\end{figure}



	\begin{figure}[h]

	\begin{minipage}[b]{.46\linewidth}
%Фигурка в первом ряду слева размер отведенный под весь этот объект \textendash 0.46 от ширины строки
%Параметр [b] означает, что выравнивание этих министраниц будет по нижнему краю
	\begin{center}
	{\small N~{\it Macoma balthica}}
		\includegraphics[width=65mm]{../Barenc_Sea/distribution_Moran/Plyazh07_moran_N_Macoma_balthica_.pdf}
	\end{center}
	\end{minipage}
%
	\hfil %Это пружинка отодвигающая рисунки друг от друга
%
	\begin{minipage}[b]{.46\linewidth}
%Следующий рисунок - первый ряд справа %DUNGEON S_4 \ AB
	\begin{center}
	{\small B~{\it Macoma balthica}}
		\includegraphics[width=65mm]{../Barenc_Sea/distribution_Moran/Plyazh07_moran_B_Macoma_balthica_.pdf}
	\end{center}
	\end{minipage}

	
	\begin{minipage}[b]{.46\linewidth}
	%Фигурка в первом ряду слева размер отведенный под весь этот объект \textendash 0.46 от ширины строки
	%Параметр [b] означает, что выравнивание этих министраниц будет по нижнему краю
	\begin{center}
	{\small N~{\it Cerastoderma edule}}
		\includegraphics[width=65mm]{../Barenc_Sea/distribution_Moran/Plyazh07_moran_N_Cerastoderma_edule_.pdf}

	\end{center}
	\end{minipage}
	%
	\hfil %Это пружинка отодвигающая рисунки друг от друга
	%
	\begin{minipage}[b]{.46\linewidth}
%Следующий рисунок - первый ряд справа %DUNGEON S_4 \ AB
	\begin{center}
	{\small B~{\it Cerastoderma edule}}
		\includegraphics[width=65mm]{../Barenc_Sea/distribution_Moran/Plyazh07_moran_B_Cerastoderma_edule_.pdf}
	\end{center}
	\end{minipage}

%\smallskip



%\smallskip

	\begin{minipage}[b]{.46\linewidth}
%Фигурка в первом ряду слева размер отведенный под весь этот объект \textendash 0.46 от ширины строки
%Параметр [b] означает, что выравнивание этих министраниц будет по нижнему краю
	\begin{center}
	{\small N~{\it Mya arenaria}}
		\includegraphics[width=65mm]{../Barenc_Sea/distribution_Moran/Plyazh07_moran_N_Mya_arenaria_.pdf}
	\end{center}
	\end{minipage}
%
	\hfil %Это пружинка отодвигающая рисунки друг от друга
%
	\begin{minipage}[b]{.46\linewidth}
%Следующий рисунок - первый ряд справа %DUNGEON S_4 \ AB
	\begin{center}
	{\small B~{\it Mya arenaria}}
		\includegraphics[width=65mm]{../Barenc_Sea/distribution_Moran/Plyazh07_moran_B_Mya_arenaria_.pdf}
	\end{center}
	\end{minipage}

%\smallskip


	\caption{Микрораспределение макробентоса на литорали Дальнего пляжа г.~Дальнезеленецкая в 2007 году.}
	\label{ris:moransI_Plyazh07_1}
	\end{figure}



	\begin{figure}[h]

	\begin{minipage}[b]{.46\linewidth}
%Фигурка в первом ряду слева размер отведенный под весь этот объект \textendash 0.46 от ширины строки
%Параметр [b] означает, что выравнивание этих министраниц будет по нижнему краю
	\begin{center}
	{\small N~{\it Mytilus edulis}}
		\includegraphics[width=65mm]{../Barenc_Sea/distribution_Moran/Plyazh07_moran_N_Mytilus_edulis_.pdf}
	\end{center}
	\end{minipage}
%
	\hfil %Это пружинка отодвигающая рисунки друг от друга
%
	\begin{minipage}[b]{.46\linewidth}
%Следующий рисунок - первый ряд справа %DUNGEON S_4 \ AB
	\begin{center}
	{\small B~{\it Mytilus edulis}}
		\includegraphics[width=65mm]{../Barenc_Sea/distribution_Moran/Plyazh07_moran_B_Mytilus_edulis_.pdf}
	\end{center}
	\end{minipage}

	
	\begin{minipage}[b]{.46\linewidth}
	%Фигурка в первом ряду слева размер отведенный под весь этот объект \textendash 0.46 от ширины строки
	%Параметр [b] означает, что выравнивание этих министраниц будет по нижнему краю
	\begin{center}
	{\small N~{\it Pseudolibrotus littoralis}}
	\includegraphics[width=65mm]{../Barenc_Sea/distribution_Moran/Plyazh07_moran_N_Pseudolibrotus_littoralis_.pdf}

	\end{center}
	\end{minipage}
	%
	\hfil %Это пружинка отодвигающая рисунки друг от друга
	%
	\begin{minipage}[b]{.46\linewidth}
%Следующий рисунок - первый ряд справа %DUNGEON S_4 \ AB
	\begin{center}
	{\small B~{\it Pseudolibrotus littoralis}}
		\includegraphics[width=65mm]{../Barenc_Sea/distribution_Moran/Plyazh07_moran_B_Pseudolibrotus_littoralis_.pdf}
	\end{center}
	\end{minipage}

%\smallskip



%\smallskip

	\begin{minipage}[b]{.46\linewidth}
%Фигурка в первом ряду слева размер отведенный под весь этот объект \textendash 0.46 от ширины строки
%Параметр [b] означает, что выравнивание этих министраниц будет по нижнему краю
	\begin{center}
	{\small N~{\it Gammarus sp.}}
		\includegraphics[width=65mm]{../Barenc_Sea/distribution_Moran/Plyazh07_moran_N_Gammarus_sp_.pdf}
	\end{center}
	\end{minipage}
%
%	\hfil %Это пружинка отодвигающая рисунки друг от друга
%
%	\begin{minipage}[b]{.46\linewidth}
%Следующий рисунок - первый ряд справа %DUNGEON S_4 \ AB
%	\begin{center}
%		\includegraphics[width=65mm]{../Barenc_Sea/distribution_Moran/Plyazh07_moran_B_Mya_arenaria__.pdf}
%	\end{center}
%	\end{minipage}

%\smallskip


%	\caption{Микрораспределение макробентоса на литорали Дальнего пляжа г.~Дальнезеленецкая в 2007 году (продолжение).}
%	\label{ris:moransI_Plyazh07_1}
%	\end{figure}
	\begin{center}
	Рис. \ref{ris:moransI_Plyazh07_1} (продолжение). Микрораспределение макробентоса на литорали Дальнего пляжа г.~Дальнезеленецкая в 2007 году.
	\end{center}

\end{figure}



	\begin{figure}[h]

	\begin{minipage}[b]{.46\linewidth}
%Фигурка в первом ряду слева размер отведенный под весь этот объект \textendash 0.46 от ширины строки
%Параметр [b] означает, что выравнивание этих министраниц будет по нижнему краю
	\begin{center}
	{\small N~{\it Macoma balthica}}
		\includegraphics[width=65mm]{../Barenc_Sea/distribution_Moran/Plyazh081_moran_N_Macoma_balthica_.pdf}
	\end{center}
	\end{minipage}
%
	\hfil %Это пружинка отодвигающая рисунки друг от друга
%
	\begin{minipage}[b]{.46\linewidth}
%Следующий рисунок - первый ряд справа %DUNGEON S_4 \ AB
	\begin{center}
	{\small B~{\it Macoma balthica}}
		\includegraphics[width=65mm]{../Barenc_Sea/distribution_Moran/Plyazh081_moran_B_Macoma_balthica_.pdf}
	\end{center}
	\end{minipage}

	
	\begin{minipage}[b]{.46\linewidth}
	%Фигурка в первом ряду слева размер отведенный под весь этот объект \textendash 0.46 от ширины строки
	%Параметр [b] означает, что выравнивание этих министраниц будет по нижнему краю
	\begin{center}
	{\small N~{\it Cerastoderma edule}}
		\includegraphics[width=65mm]{../Barenc_Sea/distribution_Moran/Plyazh081_moran_B_Cerastoderma_edule_.pdf}

	\end{center}
	\end{minipage}
	%
	\hfil %Это пружинка отодвигающая рисунки друг от друга
	%
	\begin{minipage}[b]{.46\linewidth}
%Следующий рисунок - первый ряд справа %DUNGEON S_4 \ AB
	\begin{center}
	{\small B~{\it Cerastoderma edule}}
		\includegraphics[width=65mm]{../Barenc_Sea/distribution_Moran/Plyazh081_moran_B_Cerastoderma_edule_.pdf}
	\end{center}
	\end{minipage}

%\smallskip



%\smallskip

	\begin{minipage}[b]{.46\linewidth}
%Фигурка в первом ряду слева размер отведенный под весь этот объект \textendash 0.46 от ширины строки
%Параметр [b] означает, что выравнивание этих министраниц будет по нижнему краю
	\begin{center}
	{\small N~{\it Mya arenaria}}
		\includegraphics[width=65mm]{../Barenc_Sea/distribution_Moran/Plyazh081_moran_N_Mya_arenaria_.pdf}
	\end{center}
	\end{minipage}
%
	\hfil %Это пружинка отодвигающая рисунки друг от друга
%
	\begin{minipage}[b]{.46\linewidth}
%Следующий рисунок - первый ряд справа %DUNGEON S_4 \ AB
	\begin{center}
	{\small B~{\it Mya arenaria}}
		\includegraphics[width=65mm]{../Barenc_Sea/distribution_Moran/Plyazh081_moran_B_Mya_arenaria_.pdf}
	\end{center}
	\end{minipage}

%\smallskip


	\caption{Микрораспределение макробентоса на 1 участке на литорали Дальнего пляжа г.~Дальнезеленецкая в 2008 году.}
	\label{ris:moransI_Plyazh081_1}
	\end{figure}



	\begin{figure}[h]

	\begin{minipage}[b]{.46\linewidth}
%Фигурка в первом ряду слева размер отведенный под весь этот объект \textendash 0.46 от ширины строки
%Параметр [b] означает, что выравнивание этих министраниц будет по нижнему краю
	\begin{center}
	{\small N~{\it Mytilus edulis}}
		\includegraphics[width=65mm]{../Barenc_Sea/distribution_Moran/Plyazh081_moran_N_Mytilus_edulis_.pdf}
	\end{center}
	\end{minipage}
%
	\hfil %Это пружинка отодвигающая рисунки друг от друга
%
	\begin{minipage}[b]{.46\linewidth}
%Следующий рисунок - первый ряд справа %DUNGEON S_4 \ AB
	\begin{center}
	{\small B~{\it Mytilus edulis}}
		\includegraphics[width=65mm]{../Barenc_Sea/distribution_Moran/Plyazh081_moran_B_Mytilus_edulis_.pdf}
	\end{center}
	\end{minipage}

	
	\begin{minipage}[b]{.46\linewidth}
	%Фигурка в первом ряду слева размер отведенный под весь этот объект \textendash 0.46 от ширины строки
	%Параметр [b] означает, что выравнивание этих министраниц будет по нижнему краю
	\begin{center}
	{\small N~{\it Pseudolibrotus littoralis}}
	\includegraphics[width=65mm]{../Barenc_Sea/distribution_Moran/Plyazh081_moran_N_Pseudolibrotus_littoralis_.pdf}

	\end{center}
	\end{minipage}
	%
	\hfil %Это пружинка отодвигающая рисунки друг от друга
	%
	\begin{minipage}[b]{.46\linewidth}
%Следующий рисунок - первый ряд справа %DUNGEON S_4 \ AB
	\begin{center}
	{\small B~{\it Pseudolibrotus littoralis}}
		\includegraphics[width=65mm]{../Barenc_Sea/distribution_Moran/Plyazh081_moran_B_Pseudolibrotus_littoralis_.pdf}
	\end{center}
	\end{minipage}

%\smallskip



%\smallskip

	\begin{minipage}[b]{.46\linewidth}
%Фигурка в первом ряду слева размер отведенный под весь этот объект \textendash 0.46 от ширины строки
%Параметр [b] означает, что выравнивание этих министраниц будет по нижнему краю
	\begin{center}
	{\small N~{\it Gammarus sp.}}
		\includegraphics[width=65mm]{../Barenc_Sea/distribution_Moran/Plyazh081_moran_N_Gammarus_sp_.pdf}
	\end{center}
	\end{minipage}
%
	\hfil %Это пружинка отодвигающая рисунки друг от друга
%
	\begin{minipage}[b]{.46\linewidth}
%Следующий рисунок - первый ряд справа %DUNGEON S_4 \ AB
	\begin{center}
	{\small B~{\it Gammarus sp.}}
		\includegraphics[width=65mm]{../Barenc_Sea/distribution_Moran/Plyazh081_moran_B_Gammarus_sp_.pdf}
	\end{center}
	\end{minipage}

%\smallskip


%	\caption{Микрораспределение макробентоса на литорали Дальнего пляжа г.~Дальнезеленецкая в 2007 году (продолжение).}
%	\label{ris:moransI_Plyazh081_1}
%	\end{figure}
	\begin{center}
	Рис. \ref{ris:moransI_Plyazh081_1} (продолжение). Микрораспределение макробентоса на 1 участке литорали Дальнего пляжа г.~Дальнезеленецкая в 2008 году.
	\end{center}

\end{figure}


	\begin{figure}[h]

	\begin{minipage}[b]{.46\linewidth}
%Фигурка в первом ряду слева размер отведенный под весь этот объект \textendash 0.46 от ширины строки
%Параметр [b] означает, что выравнивание этих министраниц будет по нижнему краю
	\begin{center}
	{\small N~{\it Priapulus caudatus}}
		\includegraphics[width=65mm]{../Barenc_Sea/distribution_Moran/Plyazh081_moran_N_Priapulus_caudatus_.pdf}
	\end{center}
	\end{minipage}
%
	\hfil %Это пружинка отодвигающая рисунки друг от друга
%
	\begin{minipage}[b]{.46\linewidth}
%Следующий рисунок - первый ряд справа %DUNGEON S_4 \ AB
	\begin{center}
	{\small B~{\it Priapulus caudatus}}
		\includegraphics[width=65mm]{../Barenc_Sea/distribution_Moran/Plyazh081_moran_B_Priapulus_caudatus_.pdf}
	\end{center}
	\end{minipage}

	
%	\begin{minipage}[b]{.46\linewidth}
	%Фигурка в первом ряду слева размер отведенный под весь этот объект \textendash 0.46 от ширины строки
	%Параметр [b] означает, что выравнивание этих министраниц будет по нижнему краю
%	\begin{center}
%	{\small N~{\it Pseudolibrotus littoralis}}
%	\includegraphics[width=65mm]{../Barenc_Sea/distribution_Moran/Plyazh081_moran_N_Pseudolibrotus_littoralis_.pdf}
%
%	\end{center}
%	\end{minipage}
	%
%	\hfil %Это пружинка отодвигающая рисунки друг от друга
	%
%	\begin{minipage}[b]{.46\linewidth}
%Следующий рисунок - первый ряд справа %DUNGEON S_4 \ AB
%	\begin{center}
%	{\small B~{\it Pseudolibrotus littoralis}}
%		\includegraphics[width=65mm]{../Barenc_Sea/distribution_Moran/Plyazh081_moran_B_Pseudolibrotus_littoralis_.pdf}
%	\end{center}
%	\end{minipage}

%\smallskip



%\smallskip

%	\begin{minipage}[b]{.46\linewidth}
%Фигурка в первом ряду слева размер отведенный под весь этот объект \textendash 0.46 от ширины строки
%Параметр [b] означает, что выравнивание этих министраниц будет по нижнему краю
%	\begin{center}
%	{\small N~{\it Gammarus sp.}}
%		\includegraphics[width=65mm]{../Barenc_Sea/distribution_Moran/Plyazh081_moran_N_Gammarus_sp_.pdf}
%	\end{center}
%	\end{minipage}
%
%	\hfil %Это пружинка отодвигающая рисунки друг от друга
%
%	\begin{minipage}[b]{.46\linewidth}
%Следующий рисунок - первый ряд справа %DUNGEON S_4 \ AB
%	\begin{center}
%	{\small B~{\it Gammarus sp.}}
%		\includegraphics[width=65mm]{../Barenc_Sea/distribution_Moran/Plyazh081_moran_B_Gammarus_sp_.pdf}
%	\end{center}
%	\end{minipage}

%\smallskip


%	\caption{Микрораспределение макробентоса на литорали Дальнего пляжа г.~Дальнезеленецкая в 2007 году (продолжение).}
%	\label{ris:moransI_Plyazh081_1}
%	\end{figure}
	\begin{center}
	Рис. \ref{ris:moransI_Plyazh081_1} (продолжение). Микрораспределение макробентоса на 1 участке литорали Дальнего пляжа г.~Дальнезеленецкая в 2008 году.
	\end{center}
\end{figure}


	\begin{figure}[h]

	\begin{minipage}[b]{.46\linewidth}
%Фигурка в первом ряду слева размер отведенный под весь этот объект \textendash 0.46 от ширины строки
%Параметр [b] означает, что выравнивание этих министраниц будет по нижнему краю
	\begin{center}
	{\small N~{\it Macoma balthica}}
		\includegraphics[width=65mm]{../Barenc_Sea/distribution_Moran/Plyazh082_moran_N_Macoma_balthica_.pdf}
	\end{center}
	\end{minipage}
%
	\hfil %Это пружинка отодвигающая рисунки друг от друга
%
	\begin{minipage}[b]{.46\linewidth}
%Следующий рисунок - первый ряд справа %DUNGEON S_4 \ AB
	\begin{center}
	{\small B~{\it Macoma balthica}}
		\includegraphics[width=65mm]{../Barenc_Sea/distribution_Moran/Plyazh082_moran_B_Macoma_balthica_.pdf}
	\end{center}
	\end{minipage}

	
	\begin{minipage}[b]{.46\linewidth}
	%Фигурка в первом ряду слева размер отведенный под весь этот объект \textendash 0.46 от ширины строки
	%Параметр [b] означает, что выравнивание этих министраниц будет по нижнему краю
	\begin{center}
	{\small N~{\it Cerastoderma edule}}
		\includegraphics[width=65mm]{../Barenc_Sea/distribution_Moran/Plyazh082_moran_B_Cerastoderma_edule_.pdf}

	\end{center}
	\end{minipage}
	%
	\hfil %Это пружинка отодвигающая рисунки друг от друга
	%
	\begin{minipage}[b]{.46\linewidth}
%Следующий рисунок - первый ряд справа %DUNGEON S_4 \ AB
	\begin{center}
	{\small B~{\it Cerastoderma edule}}
		\includegraphics[width=65mm]{../Barenc_Sea/distribution_Moran/Plyazh082_moran_B_Cerastoderma_edule_.pdf}
	\end{center}
	\end{minipage}

%\smallskip



%\smallskip

	\begin{minipage}[b]{.46\linewidth}
%Фигурка в первом ряду слева размер отведенный под весь этот объект \textendash 0.46 от ширины строки
%Параметр [b] означает, что выравнивание этих министраниц будет по нижнему краю
	\begin{center}
	{\small N~{\it Mya arenaria}}
		\includegraphics[width=65mm]{../Barenc_Sea/distribution_Moran/Plyazh082_moran_N_Mya_arenaria_.pdf}
	\end{center}
	\end{minipage}
%
	\hfil %Это пружинка отодвигающая рисунки друг от друга
%
	\begin{minipage}[b]{.46\linewidth}
%Следующий рисунок - первый ряд справа %DUNGEON S_4 \ AB
	\begin{center}
	{\small B~{\it Mya arenaria}}
		\includegraphics[width=65mm]{../Barenc_Sea/distribution_Moran/Plyazh082_moran_B_Mya_arenaria_.pdf}
	\end{center}
	\end{minipage}

%\smallskip


	\caption{Микрораспределение макробентоса на 2 участке на литорали Дальнего пляжа г.~Дальнезеленецкая в 2008 году.}
	\label{ris:moransI_Plyazh082_1}
	\end{figure}



	\begin{figure}[h]

	\begin{minipage}[b]{.46\linewidth}
%Фигурка в первом ряду слева размер отведенный под весь этот объект \textendash 0.46 от ширины строки
%Параметр [b] означает, что выравнивание этих министраниц будет по нижнему краю
	\begin{center}
	{\small N~{\it Mytilus edulis}}
		\includegraphics[width=65mm]{../Barenc_Sea/distribution_Moran/Plyazh082_moran_N_Mytilus_edulis_.pdf}
	\end{center}
	\end{minipage}
%
	\hfil %Это пружинка отодвигающая рисунки друг от друга
%
	\begin{minipage}[b]{.46\linewidth}
%Следующий рисунок - первый ряд справа %DUNGEON S_4 \ AB
	\begin{center}
	{\small B~{\it Mytilus edulis}}
		\includegraphics[width=65mm]{../Barenc_Sea/distribution_Moran/Plyazh082_moran_B_Mytilus_edulis_.pdf}
	\end{center}
	\end{minipage}

	
	\begin{minipage}[b]{.46\linewidth}
	%Фигурка в первом ряду слева размер отведенный под весь этот объект \textendash 0.46 от ширины строки
	%Параметр [b] означает, что выравнивание этих министраниц будет по нижнему краю
	\begin{center}
	{\small N~{\it Pseudolibrotus littoralis}}
	\includegraphics[width=65mm]{../Barenc_Sea/distribution_Moran/Plyazh082_moran_N_Pseudolibrotus_littoralis_.pdf}

	\end{center}
	\end{minipage}
	%
	\hfil %Это пружинка отодвигающая рисунки друг от друга
	%
	\begin{minipage}[b]{.46\linewidth}
%Следующий рисунок - первый ряд справа %DUNGEON S_4 \ AB
	\begin{center}
	{\small B~{\it Pseudolibrotus littoralis}}
		\includegraphics[width=65mm]{../Barenc_Sea/distribution_Moran/Plyazh082_moran_B_Pseudolibrotus_littoralis_.pdf}
	\end{center}
	\end{minipage}

%\smallskip



%\smallskip

%	\begin{minipage}[b]{.46\linewidth}
%Фигурка в первом ряду слева размер отведенный под весь этот объект \textendash 0.46 от ширины строки
%Параметр [b] означает, что выравнивание этих министраниц будет по нижнему краю
%	\begin{center}
%	{\small N~{\it Gammarus sp.}}
%		\includegraphics[width=65mm]{../Barenc_Sea/distribution_Moran/Plyazh082_moran_N_Gammarus_sp_.pdf}
%	\end{center}
%	\end{minipage}
%
%	\hfil %Это пружинка отодвигающая рисунки друг от друга
%
%	\begin{minipage}[b]{.46\linewidth}
%Следующий рисунок - первый ряд справа %DUNGEON S_4 \ AB
%	\begin{center}
%	{\small B~{\it Gammarus sp.}}
%		\includegraphics[width=65mm]{../Barenc_Sea/distribution_Moran/Plyazh082_moran_B_Gammarus_sp_.pdf}
%	\end{center}
%	\end{minipage}

%\smallskip


%	\caption{Микрораспределение макробентоса на литорали Дальнего пляжа г.~Дальнезеленецкая в 2007 году (продолжение).}
%	\label{ris:moransI_Plyazh082_1}
%	\end{figure}
	\begin{center}
	Рис. \ref{ris:moransI_Plyazh082_1} (продолжение). Микрораспределение макробентоса на 2 участке литорали Дальнего пляжа г.~Дальнезеленецкая в 2008 году.
	\end{center}

\end{figure}



	\begin{figure}[h]

	\begin{minipage}[b]{.46\linewidth}
%Фигурка в первом ряду слева размер отведенный под весь этот объект \textendash 0.46 от ширины строки
%Параметр [b] означает, что выравнивание этих министраниц будет по нижнему краю
	\begin{center}
	{\small N~{\it Macoma balthica}}
		\includegraphics[width=65mm]{../Barenc_Sea/distribution_Moran/Plyazh0812_moran_N_Macoma_balthica_.pdf}
	\end{center}
	\end{minipage}
%
	\hfil %Это пружинка отодвигающая рисунки друг от друга
%
	\begin{minipage}[b]{.46\linewidth}
%Следующий рисунок - первый ряд справа %DUNGEON S_4 \ AB
	\begin{center}
	{\small B~{\it Macoma balthica}}
		\includegraphics[width=65mm]{../Barenc_Sea/distribution_Moran/Plyazh0812_moran_B_Macoma_balthica_.pdf}
	\end{center}
	\end{minipage}

	
	\begin{minipage}[b]{.46\linewidth}
	%Фигурка в первом ряду слева размер отведенный под весь этот объект \textendash 0.46 от ширины строки
	%Параметр [b] означает, что выравнивание этих министраниц будет по нижнему краю
	\begin{center}
	{\small N~{\it Cerastoderma edule}}
		\includegraphics[width=65mm]{../Barenc_Sea/distribution_Moran/Plyazh0812_moran_B_Cerastoderma_edule_.pdf}

	\end{center}
	\end{minipage}
	%
	\hfil %Это пружинка отодвигающая рисунки друг от друга
	%
	\begin{minipage}[b]{.46\linewidth}
%Следующий рисунок - первый ряд справа %DUNGEON S_4 \ AB
	\begin{center}
	{\small B~{\it Cerastoderma edule}}
		\includegraphics[width=65mm]{../Barenc_Sea/distribution_Moran/Plyazh0812_moran_B_Cerastoderma_edule_.pdf}
	\end{center}
	\end{minipage}

%\smallskip



%\smallskip

	\begin{minipage}[b]{.46\linewidth}
%Фигурка в первом ряду слева размер отведенный под весь этот объект \textendash 0.46 от ширины строки
%Параметр [b] означает, что выравнивание этих министраниц будет по нижнему краю
	\begin{center}
	{\small N~{\it Mya arenaria}}
		\includegraphics[width=65mm]{../Barenc_Sea/distribution_Moran/Plyazh0812_moran_N_Mya_arenaria_.pdf}
	\end{center}
	\end{minipage}
%
	\hfil %Это пружинка отодвигающая рисунки друг от друга
%
	\begin{minipage}[b]{.46\linewidth}
%Следующий рисунок - первый ряд справа %DUNGEON S_4 \ AB
	\begin{center}
	{\small B~{\it Mya arenaria}}
		\includegraphics[width=65mm]{../Barenc_Sea/distribution_Moran/Plyazh0812_moran_B_Mya_arenaria_.pdf}
	\end{center}
	\end{minipage}

%\smallskip


	\caption{Микрораспределение макробентоса на объединенном участке на литорали Дальнего пляжа г.~Дальнезеленецкая в 2008 году.}
	\label{ris:moransI_Plyazh0812_1}
	\end{figure}



	\begin{figure}[h]

	\begin{minipage}[b]{.46\linewidth}
%Фигурка в первом ряду слева размер отведенный под весь этот объект \textendash 0.46 от ширины строки
%Параметр [b] означает, что выравнивание этих министраниц будет по нижнему краю
	\begin{center}
	{\small N~{\it Mytilus edulis}}
		\includegraphics[width=65mm]{../Barenc_Sea/distribution_Moran/Plyazh0812_moran_N_Mytilus_edulis_.pdf}
	\end{center}
	\end{minipage}
%
	\hfil %Это пружинка отодвигающая рисунки друг от друга
%
	\begin{minipage}[b]{.46\linewidth}
%Следующий рисунок - первый ряд справа %DUNGEON S_4 \ AB
	\begin{center}
	{\small B~{\it Mytilus edulis}}
		\includegraphics[width=65mm]{../Barenc_Sea/distribution_Moran/Plyazh0812_moran_B_Mytilus_edulis_.pdf}
	\end{center}
	\end{minipage}

	
	\begin{minipage}[b]{.46\linewidth}
	%Фигурка в первом ряду слева размер отведенный под весь этот объект \textendash 0.46 от ширины строки
	%Параметр [b] означает, что выравнивание этих министраниц будет по нижнему краю
	\begin{center}
	{\small N~{\it Pseudolibrotus littoralis}}
	\includegraphics[width=65mm]{../Barenc_Sea/distribution_Moran/Plyazh0812_moran_N_Pseudolibrotus_littoralis_.pdf}

	\end{center}
	\end{minipage}
	%
	\hfil %Это пружинка отодвигающая рисунки друг от друга
	%
	\begin{minipage}[b]{.46\linewidth}
%Следующий рисунок - первый ряд справа %DUNGEON S_4 \ AB
	\begin{center}
	{\small B~{\it Pseudolibrotus littoralis}}
		\includegraphics[width=65mm]{../Barenc_Sea/distribution_Moran/Plyazh0812_moran_B_Pseudolibrotus_littoralis_.pdf}
	\end{center}
	\end{minipage}

%\smallskip



%\smallskip

	\begin{minipage}[b]{.46\linewidth}
%Фигурка в первом ряду слева размер отведенный под весь этот объект \textendash 0.46 от ширины строки
%Параметр [b] означает, что выравнивание этих министраниц будет по нижнему краю
	\begin{center}
	{\small N~{\it Gammarus sp.}}
		\includegraphics[width=65mm]{../Barenc_Sea/distribution_Moran/Plyazh0812_moran_N_Gammarus_sp_.pdf}
	\end{center}
	\end{minipage}
%
	\hfil %Это пружинка отодвигающая рисунки друг от друга
%
%	\begin{minipage}[b]{.46\linewidth}
%Следующий рисунок - первый ряд справа %DUNGEON S_4 \ AB
	\begin{center}
	{\small B~{\it Gammarus sp.}}
		\includegraphics[width=65mm]{../Barenc_Sea/distribution_Moran/Plyazh0812_moran_B_Gammarus_sp_.pdf}
	\end{center}
	\end{minipage}

%\smallskip


%	\caption{Микрораспределение макробентоса на литорали Дальнего пляжа г.~Дальнезеленецкая в 2007 году (продолжение).}
%	\label{ris:moransI_Plyazh0812_1}
%	\end{figure}
	\begin{center}
	Рис. \ref{ris:moransI_Plyazh081_1} (продолжение). Микрораспределение макробентоса на объединенном участке литорали Дальнего пляжа г.~Дальнезеленецкая в 2008 году.
	\end{center}

\end{figure}


	\begin{figure}[h]

	\begin{minipage}[b]{.46\linewidth}
%Фигурка в первом ряду слева размер отведенный под весь этот объект \textendash 0.46 от ширины строки
%Параметр [b] означает, что выравнивание этих министраниц будет по нижнему краю
	\begin{center}
	{\small N~{\it Priapulus caudatus}}
		\includegraphics[width=65mm]{../Barenc_Sea/distribution_Moran/Plyazh0812_moran_N_Priapulus_caudatus_.pdf}
	\end{center}
	\end{minipage}
%
	\hfil %Это пружинка отодвигающая рисунки друг от друга
%
	\begin{minipage}[b]{.46\linewidth}
%Следующий рисунок - первый ряд справа %DUNGEON S_4 \ AB
	\begin{center}
	{\small B~{\it Priapulus caudatus}}
		\includegraphics[width=65mm]{../Barenc_Sea/distribution_Moran/Plyazh0812_moran_B_Priapulus_caudatus_.pdf}
	\end{center}
	\end{minipage}

	
%	\begin{minipage}[b]{.46\linewidth}
	%Фигурка в первом ряду слева размер отведенный под весь этот объект \textendash 0.46 от ширины строки
	%Параметр [b] означает, что выравнивание этих министраниц будет по нижнему краю
%	\begin{center}
%	{\small N~{\it Pseudolibrotus littoralis}}
%	\includegraphics[width=65mm]{../Barenc_Sea/distribution_Moran/Plyazh0812_moran_N_Pseudolibrotus_littoralis_.pdf}
%
%	\end{center}
%	\end{minipage}
	%
%	\hfil %Это пружинка отодвигающая рисунки друг от друга
	%
%	\begin{minipage}[b]{.46\linewidth}
%Следующий рисунок - первый ряд справа %DUNGEON S_4 \ AB
%	\begin{center}
%	{\small B~{\it Pseudolibrotus littoralis}}
%		\includegraphics[width=65mm]{../Barenc_Sea/distribution_Moran/Plyazh0812_moran_B_Pseudolibrotus_littoralis_.pdf}
%	\end{center}
%	\end{minipage}

%\smallskip



%\smallskip

%	\begin{minipage}[b]{.46\linewidth}
%Фигурка в первом ряду слева размер отведенный под весь этот объект \textendash 0.46 от ширины строки
%Параметр [b] означает, что выравнивание этих министраниц будет по нижнему краю
%	\begin{center}
%	{\small N~{\it Gammarus sp.}}
%		\includegraphics[width=65mm]{../Barenc_Sea/distribution_Moran/Plyazh0812_moran_N_Gammarus_sp_.pdf}
%	\end{center}
%	\end{minipage}
%
%	\hfil %Это пружинка отодвигающая рисунки друг от друга
%
%	\begin{minipage}[b]{.46\linewidth}
%Следующий рисунок - первый ряд справа %DUNGEON S_4 \ AB
%	\begin{center}
%	{\small B~{\it Gammarus sp.}}
%		\includegraphics[width=65mm]{../Barenc_Sea/distribution_Moran/Plyazh0812_moran_B_Gammarus_sp_.pdf}
%	\end{center}
%	\end{minipage}

%\smallskip


%	\caption{Микрораспределение макробентоса на литорали Дальнего пляжа г.~Дальнезеленецкая в 2007 году (продолжение).}
%	\label{ris:moransI_Plyazh0812_1}
%	\end{figure}
	\begin{center}
	Рис. \ref{ris:moransI_Plyazh0812_1} (продолжение). Микрораспределение макробентоса на объединенном участке литорали Дальнего пляжа г.~Дальнезеленецкая в 2008 году.
	\end{center}
  \end{figure}


	\begin{figure}[h]

	\begin{minipage}[b]{.46\linewidth}
%Фигурка в первом ряду слева размер отведенный под весь этот объект \textendash 0.46 от ширины строки
%Параметр [b] означает, что выравнивание этих министраниц будет по нижнему краю
	\begin{center}
	{\small N~{\it Macoma balthica}}
		\includegraphics[width=65mm]{../Barenc_Sea/distribution_Moran/Yarnyshnaya07_moran_N_Macoma_balthica_.pdf}
	\end{center}
	\end{minipage}
%
	\hfil %Это пружинка отодвигающая рисунки друг от друга
%
	\begin{minipage}[b]{.46\linewidth}
%Следующий рисунок - первый ряд справа %DUNGEON S_4 \ AB
	\begin{center}
	{\small B~{\it Macoma balthica}}
		\includegraphics[width=65mm]{../Barenc_Sea/distribution_Moran/Yarnyshnaya07_moran_B_Macoma_balthica_.pdf}
	\end{center}
	\end{minipage}

	
	\begin{minipage}[b]{.46\linewidth}
	%Фигурка в первом ряду слева размер отведенный под весь этот объект \textendash 0.46 от ширины строки
	%Параметр [b] означает, что выравнивание этих министраниц будет по нижнему краю
	\begin{center}
	{\small N~{\it Cerastoderma edule}}
		\includegraphics[width=65mm]{../Barenc_Sea/distribution_Moran/Yarnyshnaya07_moran_B_Cerastoderma_edule_.pdf}

	\end{center}
	\end{minipage}
	%
	\hfil %Это пружинка отодвигающая рисунки друг от друга
	%
	\begin{minipage}[b]{.46\linewidth}
%Следующий рисунок - первый ряд справа %DUNGEON S_4 \ AB
	\begin{center}
	{\small B~{\it Cerastoderma edule}}
		\includegraphics[width=65mm]{../Barenc_Sea/distribution_Moran/Yarnyshnaya07_moran_B_Cerastoderma_edule_.pdf}
	\end{center}
	\end{minipage}

%\smallskip



%\smallskip

	\begin{minipage}[b]{.46\linewidth}
%Фигурка в первом ряду слева размер отведенный под весь этот объект \textendash 0.46 от ширины строки
%Параметр [b] означает, что выравнивание этих министраниц будет по нижнему краю
	\begin{center}
	{\small N~{\it Mya arenaria}}
		\includegraphics[width=65mm]{../Barenc_Sea/distribution_Moran/Yarnyshnaya07_moran_N_Mya_arenaria_.pdf}
	\end{center}
	\end{minipage}
%
	\hfil %Это пружинка отодвигающая рисунки друг от друга
%
	\begin{minipage}[b]{.46\linewidth}
%Следующий рисунок - первый ряд справа %DUNGEON S_4 \ AB
	\begin{center}
	{\small B~{\it Mya arenaria}}
		\includegraphics[width=65mm]{../Barenc_Sea/distribution_Moran/Yarnyshnaya07_moran_B_Mya_arenaria_.pdf}
	\end{center}
	\end{minipage}

%\smallskip


	\caption{Микрораспределение макробентоса на литорали  г.~Ярнышная в 2007 году.}
	\label{ris:moransI_Yarn07_1}
	\end{figure}



	\begin{figure}[h]

	\begin{minipage}[b]{.46\linewidth}
%Фигурка в первом ряду слева размер отведенный под весь этот объект \textendash 0.46 от ширины строки
%Параметр [b] означает, что выравнивание этих министраниц будет по нижнему краю
	\begin{center}
	{\small N~{\it Mytilus edulis}}
		\includegraphics[width=65mm]{../Barenc_Sea/distribution_Moran/Yarnyshnaya07_moran_N_Mytilus_edulis_.pdf}
	\end{center}
	\end{minipage}
%
	\hfil %Это пружинка отодвигающая рисунки друг от друга
%
	\begin{minipage}[b]{.46\linewidth}
%Следующий рисунок - первый ряд справа %DUNGEON S_4 \ AB
	\begin{center}
	{\small B~{\it Mytilus edulis}}
		\includegraphics[width=65mm]{../Barenc_Sea/distribution_Moran/Yarnyshnaya07_moran_B_Mytilus_edulis_.pdf}
	\end{center}
	\end{minipage}

	
%	\begin{minipage}[b]{.46\linewidth}
	%Фигурка в первом ряду слева размер отведенный под весь этот объект \textendash 0.46 от ширины строки
	%Параметр [b] означает, что выравнивание этих министраниц будет по нижнему краю
%	\begin{center}
%	{\small N~{\it Pseudolibrotus littoralis}}
%	\includegraphics[width=65mm]{../Barenc_Sea/distribution_Moran/Yarnyshnaya07_moran_N_Pseudolibrotus_littoralis_.pdf}

%	\end{center}
%	\end{minipage}
	%
%	\hfil %Это пружинка отодвигающая рисунки друг от друга
%	%
%	\begin{minipage}[b]{.46\linewidth}
%Следующий рисунок - первый ряд справа %DUNGEON S_4 \ AB
%	\begin{center}
%	{\small B~{\it Pseudolibrotus littoralis}}
%		\includegraphics[width=65mm]{../Barenc_Sea/distribution_Moran/Yarnyshnaya07_moran_B_Pseudolibrotus_littoralis_.pdf}
%	\end{center}
%	\end{minipage}

%\smallskip



%\smallskip

%	\begin{minipage}[b]{.46\linewidth}
%Фигурка в первом ряду слева размер отведенный под весь этот объект \textendash 0.46 от ширины строки
%Параметр [b] означает, что выравнивание этих министраниц будет по нижнему краю
%	\begin{center}
%	{\small N~{\it Gammarus sp.}}
%		\includegraphics[width=65mm]{../Barenc_Sea/distribution_Moran/Yarnyshnaya07_moran_N_Gammarus_sp_.pdf}
%	\end{center}
%	\end{minipage}
%
%	\hfil %Это пружинка отодвигающая рисунки друг от друга
%
%	\begin{minipage}[b]{.46\linewidth}
%Следующий рисунок - первый ряд справа %DUNGEON S_4 \ AB
%	\begin{center}
%	{\small B~{\it Gammarus sp.}}
%		\includegraphics[width=65mm]{../Barenc_Sea/distribution_Moran/Yarnyshnaya07_moran_B_Gammarus_sp_.pdf}
%	\end{center}
%	\end{minipage}

%\smallskip


%	\caption{Микрораспределение макробентоса на литорали Дальнего пляжа г.~Дальнезеленецкая в 2007 году (продолжение).}
%	\label{ris:moransI_Yarnyshnaya07_1}
%	\end{figure}
	\begin{center}
	Рис. \ref{ris:moransI_Yarn07_1} (продолжение). Микрораспределение макробентоса на литорали г.~Ярнышная в 2007 году.
	\end{center}
  \end{figure}

\end{document}






%\afterpage{\clearpage}

%%популяционная структура

	\section{Численность {\it Macoma balthica}}
	\subsection{Белое море}
Данные по обилию маком в Кандалакшском заливе Белого моря получены для $10$ участков, всего 140 пространственно-временных точек оценки.
Средняя численность {\it M.~balthica} была представлена в диапазоне от $10$ (о.~Горелый) до $8500$~экз./м$^2$(Западная Ряшкова салма) (табл. \ref{tab:mean_N_White}).
	\begin{footnotesize}
	\begin{longtable}{|p{2cm}|p{3cm}|p{1cm}|p{2cm}|p{1.5cm}|p{1cm}|*{3}{c|}}
	\caption{Средняя численность {\it Macoma balthica} на различных участках Белого моря}\label{tab:mean_N_White}\\
	\hline
	Район & Участок & год & ма\-ре\-ографи\-ческий уровень & число повторностей & площадь учета & $N$, экз./м$^2$ & $S_x$  & $D, \%$ 
	\\ \hline \endfirsthead
	\hline
	\multicolumn{9}{|c|}{продолжение таблицы \ref{tab:mean_N_White}} \\ \hline
	Район & Участок & год & ма\-ре\-ографи\-ческий уровень & число повторностей & площадь учета & $N$, экз./м$^2$ & $S_x$  & $D, \%$ 
	\\ \hline \endhead
	\hline 
	\multicolumn{9}{|c|}{продолжение таблицы \ref{tab:mean_N_White} на следующей странице}
	\\ \hline \endfoot
	 \endlastfoot
	г. Чупа & б. Клющиха & 2006 & СГЛ & 10 & 1/20 & 444 & 53,7 & 12
		\\ \cline{3-9}
		 &  & 2006 & НГЛ & 10 & 1/20 & 362 & 26,4 & 7
		\\ \cline{3-9}
		 &  & 2006 & ВСЛ & 10 & 1/20 & 1136 & 55,4 & 5
		\\ \cline{2-9}
		 & Сухая салма & 2006 & СГЛ & 10 и & 2/20 & 1165 & 169,3 & 15
		\\ \cline{3-9}
		 &  & 2006 & НГЛ & 5 & 1/20 & 1132 & 82,6 & 7
		\\ \cline{3-9}
		 &  & 2006 & НГЛ, пояс зостеры & 5 & 1/20 & 992 & 174,4 & 18
		\\ \cline{2-9}
		 & б. Лисья & 2006 & СГЛ & 10 & 1/20 & 1346 & 209,8 & 16
		\\ \cline{3-9}
		 &  & 2006 & НГЛ & 10 & 1/20 & 2832 & 277,8 & 10
		\\ \cline{3-9}
		 &  & 2006 & ВСЛ & 10 & 1/20 & 1006 & 159,8 & 16
		\\ \cline{2-9}
		 & пр. Подпахта & 2006 & СГЛ & 10 & 1/20 & 688 & 145,2 & 21
		\\ \cline{3-9}
		 &  & 2006 & НГЛ & 10 & 1/20 & 372 & 57,9 & 16
		\\ \hline
	Лувеньга & материковая литораль, Лувеньга & 1992 & верхний пляж & 7 & 1/30 & 94 & 35,5 & 38
		\\ \cline{4-9}
		 &  & 1992 & пояс фукоидов & 5 & 1/30 & 114 & 55,6 & 49
		\\ \cline{4-9}
		 &  &  & пояс зостеры & 5 & 1/30 & 222 & 103,3 & 47
		\\ \cline{4-9}
		 &  &  & нижний пляж & 3 & 1/30 & 560 & 457,1 & 82
		\\ \cline{3-9}
		 &  & 1993 & верхний пляж & 4 & 1/30 & 413 & 127,5 & 31
		\\ \cline{4-9}
		 &  &  & пояс фукоидов & 5 & 1/30 & 336 & 120,9 & 36
		\\ \cline{4-9}
		 &  &  & пояс зостеры & 6 & 1/30 & 405 & 80,0 & 20
		\\ \cline{4-9}
		 &  & & нижний пляж & 5 & 1/30 & 354 & 77,3 & 22
		\\ \cline{3-9}
		 &  & 1994 & верхний пляж & 5 & 1/30 & 462 & 179,1 & 39
		\\ \cline{4-9}
		 &  &  & пояс фукоидов & 6 & 1/30 & 745 & 220,6 & 30
		\\ \cline{4-9}
		 &  &  & пояс зостеры & 6 & 1/30 & 765 & 112,7 & 15
		\\ \cline{4-9}
		 &  &  & нижний пляж & 3 & 1/30 & 930 & 170,6 & 18
		\\ \cline{3-9}
		 &  & 1995 & верхний пляж & 4 & 1/30 & 908 & 222,3 & 24
		\\ \cline{4-9}
		 &  &  & пояс фукоидов & 5 & 1/30 & 1134 & 269,7 & 24
		\\ \cline{4-9}
		 &  &  & пояс зостеры & 5 & 1/30 & 660 & 117,7 & 18
		\\ \cline{4-9}
		 &  &  & нижний пляж & 6 & 1/30 & 685 & 154,8 & 23
		\\ \cline{3-9}
		 &  & 1996 & верхний пляж & 4 & 1/30 & 698 & 257,0 & 37
		\\ \cline{4-9}
		 &  &  & пояс фукоидов & 6 & 1/30 & 770 & 214,9 & 28
		\\ \cline{4-9}
		 &  &  & пояс зостеры & 4 & 1/30 & 645 & 71,9 & 11
		\\ \cline{4-9}
		 &  &  & нижний пляж & 6 & 1/30 & 870 & 68,8 & 8
		\\ \cline{3-9}
		 &  & 1997 & верхний пляж & 3 & 1/30 & 620 & 130,0 & 21
		\\ \cline{4-9}
		 &  &  & пояс фукоидов & 6 & 1/30 & 720 & 265,6 & 37
		\\ \cline{4-9}
		 &  &  & пояс зостеры & 5 & 1/30 & 702 & 70,7 & 10
		\\ \cline{4-9}
		 &  &  & нижний пляж & 6 & 1/30 & 880 & 97,0 & 11
		\\ \cline{3-9}
		 &  & 1998 & верхний пляж & 4 & 1/30 & 2130 & 623,9 & 29
		\\ \cline{4-9}
		 &  &  & пояс фукоидов & 6 & 1/30 & 2750 & 820,0 & 30
		\\ \cline{4-9}
		 &  &  & пояс зостеры & 5 & 1/30 & 2424 & 437,1 & 18
		\\ \cline{4-9}
		 &  &  & нижний пляж & 5 & 1/30 & 1182 & 239,0 & 20
		\\ \cline{3-9}
		 &  & 1999 & верхний пляж & 3 & 1/30 & 7240 & 5833,7 & 81
		\\ \cline{4-9}
		 &  &  & пояс фукоидов & 6 & 1/30 & 3895 & 1354,6 & 35
		\\ \cline{4-9}
		 &  &  & пояс зостеры & 6 & 1/30 & 2405 & 498,8 & 21
		\\ \cline{4-9}
		 &  &  & нижний пляж & 5 & 1/30 & 2328 & 623,8 & 27
		\\ \cline{3-9}
		 &  & 2000 & верхний пляж & 2 & 1/30 & 2640 & 870,0 & 33
		\\ \cline{4-9}
		 &  &  & пояс фукоидов & 4 & 1/30 & 2760 & 373,1 & 14
		\\ \cline{4-9}
		 &  & & пояс зостеры & 5 & 1/30 & 2562 & 721,0 & 28
		\\ \cline{4-9}
		 &  &  & нижний пляж & 4 & 1/30 & 2018 & 394,3 & 20
		\\ \cline{3-9}
		 &  & 2002 & верхний пляж & 3 & 1/30 & 1360 & 401,5 & 30
		\\ \cline{4-9}
		 &  &  & пояс фукоидов & 3 & 1/30 & 3250 & 337,8 & 10
		\\ \cline{4-9}
		 &  &  & пояс зостеры & 4 & 1/30 & 2498 & 952,6 & 38
		\\ \cline{4-9}
		 &  &  & нижний пляж & 2 & 1/30 & 810 & 240,0 & 30
		\\ \cline{3-9}
		 &  & 2004 & верхний пляж & 3 & 1/30 & 2800 & 1066,6 & 38
		\\ \cline{4-9}
		 &  &  & пояс фукоидов & 4 & 1/30 & 3090 & 889,0 & 29
		\\ \cline{4-9}
		 &  &  & пояс зостеры & 5 & 1/30 & 1818 & 302,6 & 17
		\\ \cline{2-9}
		 & о. Горелый & 1992 & ВГЛ & 7 & 1/30 & 73 & 23,7 & 32
		\\ \cline{4-9}
		 &  &  & СГЛ & 5 & 1/30 & 108 & 9,7 & 9
		\\ \cline{4-9}
		 &  &  & НГЛ & 2 & 1/30 & 50 & 20,0 & 40
		\\ \cline{4-9}
		 &  &  & ноль глубин & 3 & 1/30 & 13 & 3,3 & 25
		\\ \cline{3-9}
		 &  & 1993 & ВГЛ & 3 & 1/30 & 143 & 29,1 & 20
		\\ \cline{4-9}
		 &  &  & СГЛ & 3 & 1/30 & 480 & 11,5 & 2
		\\ \cline{4-9}
		 &  &  & НГЛ & 4 & 1/30 & 183 & 34,5 & 19
		\\ \cline{4-9}
		 &  &  & ноль глубин & 3 & 1/30 & 97 & 43,7 & 45
		\\ \cline{3-9}
		 &  & 2004 & ВГЛ & 3 & 1/30 & 2620 & 219,3 & 8
		\\ \cline{4-9}
		 &  &  & СГЛ & 3 & 1/30 & 1700 & 208,8 & 12
		\\ \cline{4-9}
		 &  &  & НГЛ & 3 & 1/30 & 1040 & 176,9 & 17
		\\ \cline{4-9}
		 &  &  & ноль глубин & 3 & 1/30 & 1540 & 60,8 & 4
		\\ \cline{3-9}
		 &  & 2006 & ВГЛ & 3 & 1/30 & 2200 & 353,4 & 16
		\\ \cline{4-9}
		 &  &  & СГЛ & 3 & 1/30 & 1910 & 342,2 & 18
		\\ \cline{4-9}
		 &  &  & НГЛ & 3 & 1/30 & 650 & 87,2 & 13
		\\ \cline{4-9}
		 &  &  & ноль глубин & 3 & 1/30 & 760 & 160,9 & 21
		\\ \cline{3-9}
		 &  & 2007 & ВГЛ & 3 & 1/30 & 1940 & 341,8 & 18
		\\ \cline{4-9}
		 &  &  & СГЛ & 3 & 1/30 & 1990 & 449,8 & 23
		\\ \cline{4-9}
		 &  &  & НГЛ & 3 & 1/30 & 540 & 195,2 & 36
		\\ \cline{4-9}
		 &  &  & ноль глубин & 3 & 1/30 & 660 & 45,8 & 7
		\\ \cline{3-9}
		 &  & 2008 & ВГЛ & 3 & 1/30 & 1100 & 98,5 & 9
		\\ \cline{4-9}
		 &  &  & СГЛ & 3 & 1/30 & 2740 & 125,3 & 5
		\\ \cline{4-9}
		 &  &  & НГЛ & 3 & 1/30 & 1030 & 404,5 & 39
		\\ \cline{4-9}
		 &  &  & ноль глубин & 3 & 1/30 & 740 & 147,3 & 20
		\\ \cline{3-9}
		 &  & 2011 & ВГЛ & 3 & 1/30 & 2000 & 926,0 & 46
		\\ \cline{4-9}
		 &  &  & СГЛ & 3 & 1/30 & 1210 & 216,6 & 18
		\\ \cline{4-9}
		 &  &  & НГЛ & 3 & 1/30 & 1590 & 199,7 & 13
		\\ \cline{4-9}
		 &  &  & ноль глубин & 3 & 1/30 & 1100 & 208,8 & 19
		\\ \cline{2-9}
	 & Эстуарий р.~Лувень\-ги & 1992 & НГЛ & 6 & 1/30 & 55 & 14,8 & 27
		\\ \cline{3-9}
		 &  & 1993 & НГЛ & 6 & 1/30 & 202 & 31,3 & 16
		\\ \cline{3-9}
		 &  & 1994 & НГЛ & 3 и & 3/30 & 777 & 129,9 & 17
		\\ \cline{3-9}
		 &  & 1995 & НГЛ & 3 и & 3/30 & 473 & 44,8 & 9
		\\ \cline{3-9}
		 &  & 1996 & НГЛ & 3 и & 3/30 & 337 & 29,1 & 9
		\\ \cline{3-9}
		 &  & 1997 & НГЛ & 3 и & 3/30 & 213 & 14,5 & 7
		\\ \cline{3-9}
		 &  & 1998 & НГЛ & 3 и & 3/30 & 750 & 15,3 & 2
		\\ \cline{3-9}
		 &  & 1999 & НГЛ & 3 и & 3/30 & 2073 & 633,3 & 31
		\\ \cline{3-9}
		 &  & 2000 & НГЛ & 3 и & 3/30 & 1913 & 86,5 & 5
		\\ \cline{3-9}
		 &  & 2001 & НГЛ & 3 и & 3/30 & 2607 & 139,6 & 5
		\\ \cline{3-9}
		 &  & 2002 & НГЛ & 3 и & 3/30 & 1917 & 209,0 & 11
		\\ \cline{3-9}
		 &  & 2003 & НГЛ & 3 и & 3/30 & 2220 & 235,4 & 11
		\\ \cline{3-9}
		 &  & 2004 & НГЛ & 3 и & 3/30 & 3330 & 315,0 & 9
		\\ \cline{3-9}
		 &  & 2005 & НГЛ & 3 и & 3/30 & 1623 & 161,8 & 10
		\\ \cline{3-9}
		 &  & 2006 & НГЛ & 3 и & 3/30 & 993 & 131,3 & 13
		\\ \cline{3-9}
		 &  & 2007 & НГЛ & 9 & 1/30 & 2547 & 341,8 & 13
		\\ \cline{3-9}
		 &  & 2008 & НГЛ & 3 и & 3/30 & 1683 & 343,5 & 20
		\\ \cline{3-9}
		 &  & 2009 & НГЛ & 3 и & 3/30 & 1860 & 146,4 & 8
		\\ \cline{3-9}
		 &  & 2010 & НГЛ & 3 и & 3/30 & 2057 & 231,5 & 11
		\\ \cline{3-9}
		 &  & 2011 & НГЛ & 9 & 1/30 & 1637 & 60,2 & 4
		\\ \cline{3-9}
		 &  & 2012 & НГЛ & 3 и & 3/30 & 1170 & 23,1 & 2
		\\ \hline
	Северный архипелаг & Западная Ряшкова салма & 1994 & СГЛ & 2 и & 3/30 & 450 & 100,0 & 22
		\\ \cline{3-9}
		 &  & 1995 & СГЛ & 2 и & 3/30 & 490 & 10,0 & 2
		\\ \cline{3-9}
		 &  & 1996 & СГЛ & 2 и & 3/30 & 260 & 130,0 & 50
		\\ \cline{3-9}
		 &  & 1997 & СГЛ & 2 и & 3/30 & 220 & 90,0 & 41
		\\ \cline{3-9}
		 &  & 1998 & СГЛ & 2 и & 3/30 & 755 & 185,0 & 25
		\\ \cline{3-9}
		 &  & 1999 & СГЛ & 2 и & 3/30 & 8530 & 800,0 & 9
		\\ \cline{3-9}
		 &  & 2000 & СГЛ & 2 и & 3/30 & 2910 & 440,0 & 15
		\\ \cline{3-9}
		 &  & 2001 & СГЛ & 2 и & 3/30 & 2515 & 295,0 & 12
		\\ \cline{3-9}
		 &  & 2002 & СГЛ & 2 и & 3/30 & 2690 & 570,0 & 21
		\\ \cline{3-9}
		 &  & 2003 & СГЛ & 2 и & 3/30 & 1930 & 300,0 & 16
		\\ \cline{3-9}
		 &  & 2004 & СГЛ & 2 и & 3/30 & 2355 & 55,0 & 2
		\\ \cline{3-9}
		 &  & 2005 & СГЛ & 2 и & 3/30 & 1825 & 115,0 & 6
		\\ \cline{3-9}
		 &  & 2006 & СГЛ & 2 и & 3/30 & 795 & 165,0 & 21
		\\ \cline{3-9}
		 &  & 2007 & СГЛ & 2 и & 3/30 & 1055 & 185,0 & 18
		\\ \cline{3-9}
		 &  & 2008 & СГЛ & 2 и & 3/30 & 1840 & 460,0 & 25
		\\ \cline{3-9}
		 &  & 2009 & СГЛ & 2 и & 3/30 & 1745 & 65,0 & 4
		\\ \cline{3-9}
		 &  & 2010 & СГЛ & 2 и & 3/30 & 1680 & 460,0 & 27
		\\ \cline{3-9}
		 &  & 2011 & СГЛ & 2 и & 3/30 & 1455 & 535,0 & 37
		\\ \cline{3-9}
		 &  & 2012 & СГЛ & 2 и & 3/30 & 910 & 340,0 & 37
		\\ \cline{2-9}
	 & Южная губа о. Ряшкова & 2001 & ноль глубин & 9 & 1/30 & 1257 & 174,8 & 14
		\\ \cline{3-9}
		 &  & 2002 & ноль глубин & 16 & 1/30 & 1196 & 212,5 & 18
		\\ \cline{3-9}
		 &  & 2003 & ноль глубин & 15 & 1/30 & 1758 & 333,3 & 19
		\\ \cline{3-9}
		 &  & 2004 & ноль глубин & 13 & 1/30 & 1913 & 576,0 & 30
		\\ \cline{3-9}
		 &  & 2005 & ноль глубин & 15 & 1/30 & 860 & 178,0 & 21
		\\ \cline{3-9}
		 &  & 2006 & ноль глубин & 12 & 1/30 & 843 & 203,9 & 24
		\\ \cline{3-9}
		 &  & 2007 & ноль глубин & 15 & 1/30 & 1412 & 387,8 & 27
		\\ \cline{3-9}
		 &  & 2008 & ноль глубин & 10 & 1/30 & 1434 & 333,4 & 23
		\\ \cline{3-9}
		 &  & 2009 & ноль глубин & 15 & 1/30 & 1122 & 198,5 & 18
		\\ \cline{3-9}
		 &  & 2010 & ноль глубин & 15 & 1/30 & 682 & 106,5 & 16
		\\ \cline{3-9}
		 &  & 2011 & ноль глубин & 15 & 1/30 & 364 & 151,5 & 42
		\\ \cline{3-9}
		 &  & 2012 & ноль глубин & 15 & 1/30 & 142 & 39,1 & 28
		\\ \cline{2-9}
	 & о. Ломнишный & 2007 & ноль глубин & 10 & 1/30 & 501 & 88,7 & 18
		\\ \cline{3-9}
		 &  & 2008 & ноль глубин & 5 & 1/30 & 1530 & 295,0 & 19
		\\ \cline{3-9}
		 &  & 2009 & ноль глубин & 10 & 1/30 & 813 & 241,1 & 30
	\\ \cline{3-9}
	 &  & 2010 & ноль глубин & 10 & 1/30 & 540 & 168,1 & 31
	\\ \cline{3-9}
	 &  & 2011 & ноль глубин & 10 & 1/30 & 378 & 118,4 & 31
	\\ \cline{3-9}
	 &  & 2012 & ноль глубин & 10 & 1/30 & 513 & 90,9 & 18
	\\ \hline
	\multicolumn{9}{p{16cm}}{Примечания: градации мареографического уровня: ВГЛ --- верхний горизонт литорали, СГЛ --- средний горизонт литорали, НГЛ --- нижний горидонт литорали, ВСЛ --- верхняя сублитораль. 

	$N$, экз./м$^2$ --- средняя численность {\it M.~balthica}. 
	$S_x$ --- ошибка среднего.
	 $D, \%$ ---  точность учета.

	В обозначении числа повторностей индекс ''и'' означает интегральную пробу, в этом случае в графе площадь учета указано сколько проб какой площади объединялись в одну.}
	\end{longtable}
	\end{footnotesize}
%
Однако экстремально высокие численности --- более $2800$~экз./м$^2$ --- встречаются единично, всего $8$ наблюдений из $140$ (рис. \ref{ris:Nmean_hist}).
%
	\begin{figure}[ht]
		\includegraphics[height=.3\textheight]{../All_N/Nmean_hist_White.pdf}
		\includegraphics[height=.3\textheight]{../All_N/Nmean_hist_Barents.pdf}
	\caption{Частота встречаемости поселений с различным обилием {\it Macoma balthica}}
	{\footnotesize Примечание: по оси $X$ --- средняя численность {\it Macoma balthica}, экз./м$^2$ (шаг --- $100$~экз./м$^2$), по оси $Y$ 	--- частота встречаемости}
	\label{ris:Nmean_hist}
	\end{figure}
%
Наиболее часто встречаются поселения со средней численностью $700-800$~экз./м$^2$.
Отдельные районы Кандалакшского залива Белого моря не отличались по средней численности маком ($Kruskal-Wallis\ \chi^2 = 5,6$, $p = 0,2$). 
При сравнении средних обилий маком на разных участках в пределах одного горизонта не показало достоверных отличий (табл.~\ref{tab:Nmean_Kruskal_mareography_White}).
%
	\begin{table}[ht]
	\caption{Сравнение среднего обилия {\it M.~balthica} в пределах одного мареографического уровня в Белом море}
	\label{tab:Nmean_Kruskal_mareography_White}
	\begin{tabular}{|*{4}{p{0.2\textwidth}|}} \hline
	ма\-ре\-ографи\-ческий уровень & $Kruskal-Wallis\ \chi^2$ & $df$ & $p$ \\
	\hline
	СГЛ & $2,7$ & $5$ & $0,7$ \\
	\hline
	НГЛ & $5,8$ & $4$ & $0,2$ \\
	\hline
	ноль глубин & $0,16$ & $1$ & $0,7$ \\
	\hline
	ВСЛ & $1$ & $1$ & $0,3$ \\
	\hline
	\end{tabular}

	{\footnotesize Примечания: градации мареографического уровня: ВГЛ --- верхний горизонт литорали, СГЛ --- средний горизонт литорали, НГЛ --- нижний горидонт литорали, ВСЛ --- верхняя сублитораль}
	\end{table}
%
    Сравнение средних численностей на разных горизонтах в пределах одного участка показало различные результаты (табл.~\ref{tab:N2_area_mareography_Kruskal_White}). 
%
	\begin{table}[ht]
	\caption{Сравнение обилия {\it M.~balthica} в поселених на разном мареографическом уровне в Белом море}
	\label{tab:N2_area_mareography_Kruskal_White}
        \begin{tabular}{|p{0.25\textwidth}|*{4}{p{0.15\textwidth}|}} \hline
    участок & $Kruskal-Wallis\ \chi^2$ & $df$ & $p$ & \\
	\hline
    Клющиха & $19,7$ & $2$ & $5,2 \times 10^{-05}$ & ***\\
    \hline
    Клющиха (только литораль) & $1,1$ & $1$ & $0,31$ & \\
    \hline
    Сухая & $0,0057$ & $1$ & $0,94$ & \\
    \hline
    Лисья & $17,5$ & $2$ & $0,00016$ & ***\\
    \hline
    Лисья (только литораль) & $11,06$ & $1$ & $0,00088$ & ***\\
    \hline
    Подпахта  & $2,3$ & $1$ & $0,13$ & \\
    \hline
    Горелый & $10,2$ & $3$ & $0,01658$ & ** \\
    \hline
    материк, Лувеньга & $2,4$ & $3$ & $0,50$ &  \\
    \hline
	\end{tabular}
    {\footnotesize Примечание: достоверность различий *** --- $p<0,001$; ** --- $p<0,05$; * --- $p<0,1$.}
	\end{table}
%
Для участков в Сухой салме, проливе Подпахта, материковой литорали в Лувеньге варьирование численности между пробами перекрывало варьирование между горизонтами литорали.
При этом для участков в бухтах Клющиха и Лисья и на о.~Горелом Лувеньгских шхер  было показано достоверное влияние мареографического уровня на обилие маком. 
Интересно отметить, что в бухте Клющиха численность маком на нижнем и среднем горизонтах литорали не отличается ($403$~($7$)~экз./м$^2$), но в сублиторали она значительно выше ($1136$~($5$)~экз./м$^2$).
В бухте Лисья ситуация отличается, обилие маком на нижнем горизонте достоверно выше ($2832$~($10$)~экз./м$^2$), чем в среднем и в сублиторали ($1346$~($16$) и $1006$~($16$)~экз./м$^2$, соответсвенно). 



	\subsection{Баренцево море}

В Баренцевом море данные по обилию маком были получены для $12$ участков Мурманского побережья.
Минимальная средняя численность составляла $30$~экз./м$^2$ (г.~Дальнезеленецкая), что сравнимо с показателями для Белого моря. 
Максимальная средняя численность была значительно меньше, чем беломорская --- $3350$~экз./м$^2$ (Абрам-мыс) (табл.~\ref{tab:mean_N_Barents}). 
	\begin{footnotesize}
	\begin{longtable}{|p{2cm}|p{3cm}|p{1cm}|p{2cm}|p{1.5cm}|p{1cm}|*{3}{c|}}
	\caption{Средняя численность {\it Macoma balthica} на различных участках Баренцева моря}\label{tab:mean_N_Barents}\\
	\hline
	Район & Участок & год & ма\-ре\-ографи\-ческий уровень & число повторностей & площадь учета & $N$, экз./м$^2$ & $S_x$  & $D, \%$ 
	\\ \hline \endfirsthead
	\hline
	\multicolumn{9}{|c|}{продолжение таблицы \ref{tab:mean_N_Barents}} \\ \hline
	Район & Участок & год & ма\-ре\-ографи\-ческий уровень & число повторностей & площадь учета & $N$, экз./м$^2$ & $S_x$  & $D, \%$ 
	\\ \hline \endhead
	\hline 
	\multicolumn{9}{|c|}{продолжение таблицы \ref{tab:mean_N_Barents} на следующей странице}
	\\ \hline \endfoot
	\endlastfoot
	Западный Мурман & Ура-губа & 2005 & СГЛ & 3 & 1/30 & 1267 & 288,8 & 23
		\\ \cline{2-9}
		 & Печенга & 2005 & СГЛ & 3 & 1/30 & 767 & 218,6 & 29
		\\ \hline
	Кольский Залив & Северное Нагорное & 2005 & СГЛ & 2 & 1/30 & 390 & 90,0 & 23
		\\ \cline{2-9}
		 & Абрам-мыс & 2005 & СГЛ & 2 & 1/30 & 3350 & 520,0 & 16
		\\ \cline{3-9}
		 &  & 2008 & СГЛ & 5 & 1/20 & 540 & 208,5 & 39
		\\ \cline{4-9}
		 &  &  & НГЛ & 5 & 1/20 & 1804 & 78,6 & 4
		\\ \cline{2-9}
		 & Ретинское & 2005 & СГЛ & 2 & 1/30 & 660 & 300,0 & 45
		\\ \cline{2-9}
		 & Пала-губа & 2007 & СГЛ & 16 & 1/30 & 936 & 76,4 & 8
		\\ \cline{3-9}
		 &  & 2007 осень & НГЛ & 36 & 1/30 & 790 & 61,7 & 8
		\\ \cline{3-9}
		 &  & 2008 зима & НГЛ & 11 & 1/20 & 864 & 154,4 & 18
		\\ \cline{3-9}
		 &  & 2008 & НГЛ & 10 & 1/30 & 1644 & 192,5 & 12
		\\ \hline
	Восточный Мурман & Гаврилово & 2008 & СГЛ & 5 & 1/30 & 138 & 20,3 & 15
		\\ \cline{4-9}
		 &  & 2008 & НГЛ & 5 & 1/30 & 24 & 11,2 & 47
		\\ \cline{2-9}
		 & Ярнышная & 2007 & СГЛ & 36 & 1/30 & 70 & 9,6 & 14
		\\ \cline{3-9}
		 &  & 2008 & ВГЛ & 5 & 1/30 & 414 & 47,8 & 12
		\\ \cline{4-9}
		 &  & & НГЛ & 5 & 1/30 & 387 & 109,1 & 28
		\\ \cline{2-9}
		 & Дальнезеленецкая & 2002 & СГЛ & 43 & 1/30 & 52 & 7,0 & 13
		\\ \cline{3-9}
		 &  & 2003 & СГЛ & 48 & 1/30 & 34 & 6,6 & 20
		\\ \cline{3-9}
		 &  & 2004 & СГЛ & 44 & 1/30 & 32 & 5,3 & 16
		\\ \cline{3-9}
		 &  & 2005 & СГЛ & 30 & 1/30 & 30 & 4,5 & 15
		\\ \cline{3-9}
		 &  & 2006 & СГЛ & 28 & 1/30 & 39 & 6,0 & 16
		\\ \cline{3-9}
		 &  & 2007 & СГЛ & 33 & 1/30 & 72 & 6,6 & 9
		\\ \cline{3-9}
		 &  & 2008 & СГЛ & 72 & 1/30 & 72 & 5,5 & 8
		\\ \cline{4-9}
		 &  &  & ВГЛ & 10 & 1/30 & 30 & 8,9 & 30
		\\ \cline{4-9}
		 &  &  & НГЛ & 5 & 1/30 & 42 & 7,3 & 17
		\\ \cline{2-9}
		 & Шельпино & 2008 & СГЛ & 5 & 1/30 & 54 & 11,2 & 21
		\\ \cline{4-9}
		 &  &  & ВГЛ & 5 & 1/30 & 36 & 17,5 & 49
		\\ \cline{2-9}
		 & Порчниха & 2007 & СГЛ & 32 & 1/30 & 87 & 10,8 & 12
		\\ \cline{3-9}
		 &  & 2008 & СГЛ & 5 & 1/30 & 60 & 13,4 & 22
		\\ \cline{2-9}
		 & Ивановская & 2008 & ВСЛ & 5 & 1/20 & 1208 & 72,8 & 6
		\\ \hline
	\multicolumn{9}{p{16cm}}{Примечания: градации мареографического уровня: ВГЛ --- верхний горизонт литорали, СГЛ --- средний 	горизонт литорали, НГЛ --- нижний горидонт литорали, ВСЛ --- верхняя сублитораль. 

	$N$, экз./м$^2$ --- средняя численность {\it M.~balthica}. 
	$S_x$ --- ошибка среднего.
	 $D, \%$ ---  точность учета.
	
	В обозначении числа повторностей индекс ''и'' означает интегральную пробу, в этом случае в графе площадь учета указано сколько проб какой площади 	объединялись в одну.}
	\end{longtable}
	\end{footnotesize}
%
Среди иследованных, наиболее часто встречались поселения со средним обилием менее $100$~экз./м$^2$ (рис.~\ref{ris:N_region_Barents}).

Важно отметить, что для Мурманского побережья Баренцева моря показаны различия между отдельными районами: Западным, Восточным Мурманом и Кольским заливом (\ref{} \textcolor{red}{Гурьянова и ко?}). 
Это подтверждается нашими данными (рис.~\ref{ris:N_region_Barents}) по размаху варьирования среднего обилия в пределах районов ($Kruskal-Wallis\ \chi^2 = 17,6$, $p = 0,00015$).
%
	\begin{figure}[h]
		\includegraphics{../All_N/Nmean_region_Barents1.pdf}
	\caption{Варьирование среднего обилия {\it Macoma balthica} в разных районах Мурманского побережья Баренцева моря}
	{\footnotesize Примечание: По оси абсцисс --- численность {\it M.~balthica}, ~экз./м$^2$.

	На графике: жирная горизонтальная линия --- медиана, границы ''ящика'' --- 1 и 3 квартили, ''усы'' --- $1,5$ интерквартильного расстояния, точки - значения выпадающие за $1,5$ интерквартильных расстояния}
	\label{ris:N_region_Barents}
	\end{figure}
%
На литорали Восточного Мурмана численность {\it M.~balthica} в основном не превышала $100$~экз./м$^2$. 
Единственное исключение ---\ литораль губы Ярнышная, где численность маком достигала $410$~($12$)~экз./м$^2$. 
Между тем, на единственном участке, где были учеты в сублиторали, в губе Ивановской, численность на порядок выше, чем ее значения на литорали Восточного мурмана, и составляет $1200$~экз./м$^2$. 
В Кольском заливе минимальные значения обилия были отмечены на литорали в районе Северного Нагорного ($390$~($23$)~экз./м$^2$). 
Максимальных значений численности как для региона, так и для всей исследованной части Мурманского побережья, достигали поселения маком на учатске в районе Абрам-мысса ($3350$~($16$)~экз./м$^2$). 
На Западном Мурмане обилие флуктуировало вокруг $1000$~экз./м$^2$.  

При сравнении численности маком на различных мареографических уровнях различия между горизонтами литорали были показаны для губ Гаврилово и Ярнышная (табл.~\ref{tab:N2_area_mareography_Kruskal_Barents}).
В Гаврилово средняя численность {\it M.~balthica} в среднем горизонте литорали превышала аналогичные значения для нижнего горизонта на порядок ($138$~($15$) и $24$~($47$)~экз./м$^2$, соответственно).
В губе Ярнышная численность маком в верхнем и нижнем горизонтах не различалась ($414$~($12$) и $360$~($43$)~экз./м$^2$, соответсвенно), в то время как в среднем горизонте литорали она была значительно ниже ($70$~($14$)~экз./м$^2$).  
%
	\begin{table}[ht]
	\caption{Сравнение обилия {\it Macoma balthica} в поселених на разном мареографическом уровне в Баренцевом море}
	\label{tab:N2_area_mareography_Kruskal_Barents}
        \begin{tabular}{|p{0.25\textwidth}|*{4}{p{0.15\textwidth}|}} \hline
    участок & $Kruskal-Wallis\ \chi^2$ & $df$ & $p$ & \\
    \hline
    Абрам-мыс &  $1,5$ & $1$ & $0,224$ & \\
    \hline
    Пала-губа & $0,4$ & $1$ & $0,54$ & \\
    \hline
    Гаврилово & $6,9$ & $1$ & $0,0084$ & *** \\
    \hline
    Ярнышная & $19,4$ &  $2$ &  $6,09 \times 10^{-5}$ & *** \\
    \hline
    Дальнезеленецкая & $1,6$ & $2$ & $0,45$ & \\
    \hline
    Шельпино & $0,7$ & $1$ & $0,39$ & \\
    \hline
	\end{tabular}
    {\footnotesize Примечание: достоверность различий *** --- $p<0,001$; ** --- $p<0,05$; * --- $p<0,1$.}
	\end{table}
%

    \subsection{Влияние состава грунта на численность {\it Macoma balthica}}
Нет сомнений, что основной параметр, определяющий обилие маком ---\ это доступные пищевые   ресурсы.   
Косвенным   показателем   наличия   пищевых   ресурсов   служит гранулометрический состав грунта и общее содержание органических веществ. 
Поэтому по полученым для участков на Баренцевом море данным мы провели корреляционный анализ связи среднего обилия маком на участке с характеристиками  грунта.  
В   результате  оказалось,   что   соотношение   песчаных  фракций   различного   размера влияет   на   обилие  {\it M.~balthica}  (табл.~\ref{tab:grunt_N_correlation_Barents}).  
%
	\begin{table}[ht]
	\caption{Сравнение обилия {\it Macoma balthica} в поселених на разном мареографическом уровне в Баренцевом море}
    \label{tab:grunt_N_correlation_Barents}
     \begin{tabular}{|*{4}{p{0.15\textwidth}|}} \hline
    фракция & $R_s$ & $p-value$ & \\
    \hline
    $>10$~мм & $-0,2$ &  $0,36$ & \\
    \hline
    $10 - 5$~мм & $-0,01$ & $0,98$ & \\
    \hline
    $5 - 3$~мм & $0,07$ & $0,87$ & \\
    \hline
    $3 - 1$~мм & $0,12$ & $0,78$ & \\
    \hline
    $1 - 0,5$~мм & $-0,74$ & $0,04$ & ** \\
    \hline
    $0,5 - 0,25$~мм & $-0,67$  & $0,07$ & * \\
    \hline
    $0,25 - 0,1$~мм & $0,71$ & $0,04$ & ** \\
    \hline
    $<0,1$~мм & $0,6$ &  $0,12$ & \\
    \hline
    доля органических веществ & $0,36$ & $0,38$ & \\
    \hline
	\end{tabular}
    
    {\footnotesize Примечание: $R_s$ --- корреляция Спирмена. \\
    достоверность различий *** --- $p<0,001$; ** --- $p<0,05$; * --- $p<0,1$.}
	\end{table}
%
При   этом  наблюдается   достоверная   отрицательная корреляция численности маком с долей крупного  песка и положительная — с долей мелкого.


%% для компиляции в lualatex!!
\documentclass[12pt, a4paper]{article}
\usepackage[english,russian]{babel}
\usepackage[warn]{mathtext}
%\usepackage[T2A]{fontenc}
%\usepackage[utf8]{inputenc}

\usepackage{xecyr} % Продукт Вашего покорного слуги ;)

\setmainfont{DejaVu Serif}

\usepackage{color}
\usepackage{amssymb,amsmath}
\usepackage{graphicx}
\usepackage{multicol}

\textheight=24cm           % высота текста
\textwidth=16cm            % ширина текста
\oddsidemargin=0pt         % отступ от левого края
\topmargin=-1.5cm          % отступ от верхнего края
\parindent=24pt            % абзацный отступ
\parskip=0pt               % интервал между абзацами
\tolerance=2000            % терпимость к "жидким" строкам
\flushbottom               % выравнивание высоты страниц
%\def\baselinestretch{1.5} % печать с большим интервалом

%\title{}
%\author{\copyright~~С.А.~Назарова \thanks{e-mail:~sophia.nazarova@gmail.com}}
%\date{}

\begin{document}



%\maketitle

%Эстуарий Лувеньги
\begin{figure}[h]

\begin{multicols}{3}
\hfill
\includegraphics[width=60mm]{../White_Sea/Estuatiy_Luvenga/sizestr_1992_.pdf}
\hfill
\includegraphics[width=60mm]{../White_Sea/Estuatiy_Luvenga/sizestr_1996_.pdf}
\hfill
\includegraphics[width=60mm]{../White_Sea/Estuatiy_Luvenga/sizestr_2000_.pdf}
\end{multicols}

%\smallskip


\begin{multicols}{3}
\hfill
\includegraphics[width=60mm]{../White_Sea/Estuatiy_Luvenga/sizestr_1993_.pdf}
\hfill
\includegraphics[width=60mm]{../White_Sea/Estuatiy_Luvenga/sizestr_1997_.pdf}
\hfill
\includegraphics[width=60mm]{../White_Sea/Estuatiy_Luvenga/sizestr_2001_.pdf}
\end{multicols}

%\smallskip

\begin{multicols}{3}
\hfill
\includegraphics[width=60mm]{../White_Sea/Estuatiy_Luvenga/sizestr_1994_.pdf}
\hfill
\includegraphics[width=60mm]{../White_Sea/Estuatiy_Luvenga/sizestr_1998_.pdf}
\hfill
\includegraphics[width=60mm]{../White_Sea/Estuatiy_Luvenga/sizestr_2002_.pdf}
\end{multicols}

%\smallskip

\begin{multicols}{3}
\hfill
\includegraphics[width=60mm]{../White_Sea/Estuatiy_Luvenga/sizestr_1995_.pdf}
\hfill
\includegraphics[width=60mm]{../White_Sea/Estuatiy_Luvenga/sizestr_1999_.pdf}
\hfill
\includegraphics[width=60mm]{../White_Sea/Estuatiy_Luvenga/sizestr_2003_.pdf}
\end{multicols}


\caption{Размерная структура {\it Macoma balthica} в СГЛ эстуария р. Лувеньги}
\label{ris:size_str_estuary_Luv}
\end{figure}


\begin{figure}[h]

\begin{multicols}{3}
\hfill
\includegraphics[width=60mm]{../White_Sea/Estuatiy_Luvenga/sizestr_2004_.pdf}
\hfill
\includegraphics[width=60mm]{../White_Sea/Estuatiy_Luvenga/sizestr_2008_.pdf}
\hfill
\includegraphics[width=60mm]{../White_Sea/Estuatiy_Luvenga/sizestr_2012_.pdf}
\end{multicols}

%\smallskip


\begin{multicols}{3}
\hfill
\includegraphics[width=60mm]{../White_Sea/Estuatiy_Luvenga/sizestr_2005_.pdf}
\hfill
\includegraphics[width=60mm]{../White_Sea/Estuatiy_Luvenga/sizestr_2009_.pdf}
%\hfill
%\includegraphics[width=60mm]{../White_Sea/Estuatiy_Luvenga/sizestr_2002_.pdf}
\end{multicols}

%\smallskip

\begin{multicols}{3}
\hfill
\includegraphics[width=60mm]{../White_Sea/Estuatiy_Luvenga/sizestr_2006_.pdf}
\hfill
\includegraphics[width=60mm]{../White_Sea/Estuatiy_Luvenga/sizestr_2010_.pdf}
%\hfill
%\includegraphics[width=60mm]{../White_Sea/Estuatiy_Luvenga/sizestr_2003_.pdf}
\end{multicols}

%\smallskip

\begin{multicols}{3}
\hfill
\includegraphics[width=60mm]{../White_Sea/Estuatiy_Luvenga/sizestr_2007_.pdf}
\hfill
\includegraphics[width=60mm]{../White_Sea/Estuatiy_Luvenga/sizestr_2011_.pdf}
%\hfill
%\includegraphics[width=60mm]{../White_Sea/Estuatiy_Luvenga/sizestr_2004_.pdf}
\end{multicols}


%\caption{Размерная структура {\it Macoma balthica} в СГЛ эстуария р. Лувеньги}
%\label{ris:size_str_estuaty_Luv}
\begin{center}
Рис. \ref{ris:size_str_estuary_Luv} (продолжение). Размерная структура {\it Macoma balthica} в СГЛ эстуария р. Лувеньги

\end{center}
\end{figure}

%Горелый Лувеньга
\begin{figure}[h]

\begin{multicols}{3}
\hfill
\includegraphics[width=60mm]{../White_Sea/Luvenga_Goreliy/high_1992_.pdf}
\hfill
\includegraphics[width=60mm]{../White_Sea/Luvenga_Goreliy/high_1996_.pdf}
\hfill
\includegraphics[width=60mm]{../White_Sea/Luvenga_Goreliy/high_2000_.pdf}
\end{multicols}

%\smallskip


\begin{multicols}{3}
\hfill
\includegraphics[width=60mm]{../White_Sea/Luvenga_Goreliy/high_1993_.pdf}
\hfill
\includegraphics[width=60mm]{../White_Sea/Luvenga_Goreliy/high_1997_.pdf}
\hfill
\includegraphics[width=60mm]{../White_Sea/Luvenga_Goreliy/high_2001_.pdf}
\end{multicols}

%\smallskip

\begin{multicols}{3}
\hfill
\includegraphics[width=60mm]{../White_Sea/Luvenga_Goreliy/high_1994_.pdf}
\hfill
\includegraphics[width=60mm]{../White_Sea/Luvenga_Goreliy/high_1998_.pdf}
\hfill
\includegraphics[width=60mm]{../White_Sea/Luvenga_Goreliy/high_2002_.pdf}
\end{multicols}

%\smallskip

\begin{multicols}{3}
\hfill
\includegraphics[width=60mm]{../White_Sea/Luvenga_Goreliy/high_1995_.pdf}
\hfill
\includegraphics[width=60mm]{../White_Sea/Luvenga_Goreliy/high_1999_.pdf}
\hfill
\includegraphics[width=60mm]{../White_Sea/Luvenga_Goreliy/high_2003_.pdf}
\end{multicols}


\caption{Размерная структура {\it Macoma balthica} в ВГЛ о. Горелого}
\label{ris:size_str_Goreliy_high}
\end{figure}


\begin{figure}[h]

\begin{multicols}{3}
\hfill
\includegraphics[width=60mm]{../White_Sea/Luvenga_Goreliy/high_2004_.pdf}
\hfill
\includegraphics[width=60mm]{../White_Sea/Luvenga_Goreliy/high_2008_.pdf}
\hfill
\includegraphics[width=60mm]{../White_Sea/Luvenga_Goreliy/high_2012_.pdf}
\end{multicols}

%\smallskip


\begin{multicols}{3}
\hfill
\includegraphics[width=60mm]{../White_Sea/Luvenga_Goreliy/high_2005_.pdf}
\hfill
\includegraphics[width=60mm]{../White_Sea/Luvenga_Goreliy/high_2009_.pdf}
%\hfill
%\includegraphics[width=60mm]{../White_Sea/Luvenga_Goreliy/high_2002_.pdf}
\end{multicols}

%\smallskip

\begin{multicols}{3}
\hfill
\includegraphics[width=60mm]{../White_Sea/Luvenga_Goreliy/high_2006_.pdf}
\hfill
\includegraphics[width=60mm]{../White_Sea/Luvenga_Goreliy/high_2010_.pdf}
%\hfill
%\includegraphics[width=60mm]{../White_Sea/Luvenga_Goreliy/high_2003_.pdf}
\end{multicols}

%\smallskip

\begin{multicols}{3}
\hfill
\includegraphics[width=60mm]{../White_Sea/Luvenga_Goreliy/high_2007_.pdf}
\hfill
\includegraphics[width=60mm]{../White_Sea/Luvenga_Goreliy/high_2011_.pdf}
%\hfill
%\includegraphics[width=60mm]{../White_Sea/Luvenga_Goreliy/high_2004_.pdf}
\end{multicols}


%\caption{Размерная структура {\it Macoma balthica} в СГЛ эстуария р. Лувеньги}
%\label{ris:size_str_estuaty_Luv}
\begin{center}
Рис. \ref{ris:size_str_Goreliy_high} (продолжение). Размерная структура {\it Macoma balthica} в ВГЛ о. Горелого

\end{center}
\end{figure}

% Горелый СГЛ
\begin{figure}[h]

\begin{multicols}{3}
\hfill
\includegraphics[width=60mm]{../White_Sea/Luvenga_Goreliy/middle_1992_.pdf}
\hfill
\includegraphics[width=60mm]{../White_Sea/Luvenga_Goreliy/middle_1996_.pdf}
\hfill
\includegraphics[width=60mm]{../White_Sea/Luvenga_Goreliy/middle_2000_.pdf}
\end{multicols}

%\smallskip


\begin{multicols}{3}
\hfill
\includegraphics[width=60mm]{../White_Sea/Luvenga_Goreliy/middle_1993_.pdf}
\hfill
\includegraphics[width=60mm]{../White_Sea/Luvenga_Goreliy/middle_1997_.pdf}
\hfill
\includegraphics[width=60mm]{../White_Sea/Luvenga_Goreliy/middle_2001_.pdf}
\end{multicols}

%\smallskip

\begin{multicols}{3}
\hfill
\includegraphics[width=60mm]{../White_Sea/Luvenga_Goreliy/middle_1994_.pdf}
\hfill
\includegraphics[width=60mm]{../White_Sea/Luvenga_Goreliy/middle_1998_.pdf}
\hfill
\includegraphics[width=60mm]{../White_Sea/Luvenga_Goreliy/middle_2002_.pdf}
\end{multicols}

%\smallskip

\begin{multicols}{3}
\hfill
\includegraphics[width=60mm]{../White_Sea/Luvenga_Goreliy/middle_1995_.pdf}
\hfill
\includegraphics[width=60mm]{../White_Sea/Luvenga_Goreliy/middle_1999_.pdf}
\hfill
\includegraphics[width=60mm]{../White_Sea/Luvenga_Goreliy/middle_2003_.pdf}
\end{multicols}


\caption{Размерная структура {\it Macoma balthica} в СГЛ о. Горелого}
\label{ris:size_str_Goreliy_mid}
\end{figure}


\begin{figure}[h]

\begin{multicols}{3}
\hfill
\includegraphics[width=60mm]{../White_Sea/Luvenga_Goreliy/middle_2004_.pdf}
\hfill
\includegraphics[width=60mm]{../White_Sea/Luvenga_Goreliy/middle_2008_.pdf}
\hfill
\includegraphics[width=60mm]{../White_Sea/Luvenga_Goreliy/middle_2012_.pdf}
\end{multicols}

%\smallskip


\begin{multicols}{3}
\hfill
\includegraphics[width=60mm]{../White_Sea/Luvenga_Goreliy/middle_2005_.pdf}
\hfill
\includegraphics[width=60mm]{../White_Sea/Luvenga_Goreliy/middle_2009_.pdf}
%\hfill
%\includegraphics[width=60mm]{../White_Sea/Luvenga_Goreliy/middle_2002_.pdf}
\end{multicols}

%\smallskip

\begin{multicols}{3}
\hfill
\includegraphics[width=60mm]{../White_Sea/Luvenga_Goreliy/middle_2006_.pdf}
\hfill
\includegraphics[width=60mm]{../White_Sea/Luvenga_Goreliy/middle_2010_.pdf}
%\hfill
%\includegraphics[width=60mm]{../White_Sea/Luvenga_Goreliy/middle_2003_.pdf}
\end{multicols}

%\smallskip

\begin{multicols}{3}
\hfill
\includegraphics[width=60mm]{../White_Sea/Luvenga_Goreliy/middle_2007_.pdf}
\hfill
\includegraphics[width=60mm]{../White_Sea/Luvenga_Goreliy/middle_2011_.pdf}
%\hfill
%\includegraphics[width=60mm]{../White_Sea/Luvenga_Goreliy/middle_2004_.pdf}
\end{multicols}


%\caption{Размерная структура {\it Macoma balthica} в СГЛ эстуария р. Лувеньги}
%\label{ris:size_str_estuaty_Luv}
\begin{center}
Рис. \ref{ris:size_str_Goreliy_mid} (продолжение). Размерная структура {\it Macoma balthica} в СГЛ о. Горелого

\end{center}
\end{figure}

% Горелый midlow


\begin{figure}[h]

\begin{multicols}{3}
\hfill
\includegraphics[width=60mm]{../White_Sea/Luvenga_Goreliy/midlow_1992_.pdf}
\hfill
\includegraphics[width=60mm]{../White_Sea/Luvenga_Goreliy/midlow_1996_.pdf}
\hfill
\includegraphics[width=60mm]{../White_Sea/Luvenga_Goreliy/midlow_2000_.pdf}
\end{multicols}

%\smallskip


\begin{multicols}{3}
\hfill
\includegraphics[width=60mm]{../White_Sea/Luvenga_Goreliy/midlow_1993_.pdf}
\hfill
\includegraphics[width=60mm]{../White_Sea/Luvenga_Goreliy/midlow_1997_.pdf}
\hfill
\includegraphics[width=60mm]{../White_Sea/Luvenga_Goreliy/midlow_2001_.pdf}
\end{multicols}

%\smallskip

\begin{multicols}{3}
\hfill
\includegraphics[width=60mm]{../White_Sea/Luvenga_Goreliy/midlow_1994_.pdf}
\hfill
\includegraphics[width=60mm]{../White_Sea/Luvenga_Goreliy/midlow_1998_.pdf}
\hfill
\includegraphics[width=60mm]{../White_Sea/Luvenga_Goreliy/midlow_2002_.pdf}
\end{multicols}

%\smallskip

\begin{multicols}{3}
\hfill
\includegraphics[width=60mm]{../White_Sea/Luvenga_Goreliy/midlow_1995_.pdf}
\hfill
\includegraphics[width=60mm]{../White_Sea/Luvenga_Goreliy/midlow_1999_.pdf}
\hfill
\includegraphics[width=60mm]{../White_Sea/Luvenga_Goreliy/midlow_2003_.pdf}
\end{multicols}


\caption{Размерная структура {\it Macoma balthica} в поясе фукоидов о. Горелого}
\label{ris:size_str_Goreliy_midlow}
\end{figure}


\begin{figure}[h]

\begin{multicols}{3}
\hfill
\includegraphics[width=60mm]{../White_Sea/Luvenga_Goreliy/midlow_2004_.pdf}
\hfill
\includegraphics[width=60mm]{../White_Sea/Luvenga_Goreliy/midlow_2008_.pdf}
\hfill
\includegraphics[width=60mm]{../White_Sea/Luvenga_Goreliy/midlow_2012_.pdf}
\end{multicols}

%\smallskip


\begin{multicols}{3}
\hfill
\includegraphics[width=60mm]{../White_Sea/Luvenga_Goreliy/midlow_2005_.pdf}
\hfill
\includegraphics[width=60mm]{../White_Sea/Luvenga_Goreliy/midlow_2009_.pdf}
%\hfill
%\includegraphics[width=60mm]{../White_Sea/Luvenga_Goreliy/midlow_2002_.pdf}
\end{multicols}

%\smallskip

\begin{multicols}{3}
\hfill
\includegraphics[width=60mm]{../White_Sea/Luvenga_Goreliy/midlow_2006_.pdf}
\hfill
\includegraphics[width=60mm]{../White_Sea/Luvenga_Goreliy/midlow_2010_.pdf}
%\hfill
%\includegraphics[width=60mm]{../White_Sea/Luvenga_Goreliy/midlow_2003_.pdf}
\end{multicols}

%\smallskip

\begin{multicols}{3}
\hfill
\includegraphics[width=60mm]{../White_Sea/Luvenga_Goreliy/midlow_2007_.pdf}
\hfill
\includegraphics[width=60mm]{../White_Sea/Luvenga_Goreliy/midlow_2011_.pdf}
%\hfill
%\includegraphics[width=60mm]{../White_Sea/Luvenga_Goreliy/midlow_2004_.pdf}
\end{multicols}


%\caption{Размерная структура {\it Macoma balthica} в СГЛ эстуария р. Лувеньги}
%\label{ris:size_str_estuaty_Luv}
\begin{center}
Рис. \ref{ris:size_str_Goreliy_midlow} (продолжение). Размерная структура {\it Macoma balthica} в поясе фукоидов о. Горелого

\end{center}
\end{figure}


% Горелый НГЛ
\begin{figure}[h]

\begin{multicols}{3}
\hfill
\includegraphics[width=60mm]{../White_Sea/Luvenga_Goreliy/low_1992_.pdf}
\hfill
\includegraphics[width=60mm]{../White_Sea/Luvenga_Goreliy/low_1996_.pdf}
\hfill
\includegraphics[width=60mm]{../White_Sea/Luvenga_Goreliy/low_2000_.pdf}
\end{multicols}

%\smallskip


\begin{multicols}{3}
\hfill
\includegraphics[width=60mm]{../White_Sea/Luvenga_Goreliy/low_1993_.pdf}
\hfill
\includegraphics[width=60mm]{../White_Sea/Luvenga_Goreliy/low_1997_.pdf}
\hfill
\includegraphics[width=60mm]{../White_Sea/Luvenga_Goreliy/low_2001_.pdf}
\end{multicols}

%\smallskip

\begin{multicols}{3}
\hfill
\includegraphics[width=60mm]{../White_Sea/Luvenga_Goreliy/low_1994_.pdf}
\hfill
\includegraphics[width=60mm]{../White_Sea/Luvenga_Goreliy/low_1998_.pdf}
\hfill
\includegraphics[width=60mm]{../White_Sea/Luvenga_Goreliy/low_2002_.pdf}
\end{multicols}

%\smallskip

\begin{multicols}{3}
\hfill
\includegraphics[width=60mm]{../White_Sea/Luvenga_Goreliy/low_1995_.pdf}
\hfill
\includegraphics[width=60mm]{../White_Sea/Luvenga_Goreliy/low_1999_.pdf}
\hfill
\includegraphics[width=60mm]{../White_Sea/Luvenga_Goreliy/low_2003_.pdf}
\end{multicols}


\caption{Размерная структура {\it Macoma balthica} в НГЛ о. Горелого}
\label{ris:size_str_Goreliy_low}
\end{figure}


\begin{figure}[h]

\begin{multicols}{3}
\hfill
\includegraphics[width=60mm]{../White_Sea/Luvenga_Goreliy/low_2004_.pdf}
\hfill
\includegraphics[width=60mm]{../White_Sea/Luvenga_Goreliy/low_2008_.pdf}
\hfill
\includegraphics[width=60mm]{../White_Sea/Luvenga_Goreliy/low_2012_.pdf}
\end{multicols}

%\smallskip


\begin{multicols}{3}
\hfill
\includegraphics[width=60mm]{../White_Sea/Luvenga_Goreliy/low_2005_.pdf}
\hfill
\includegraphics[width=60mm]{../White_Sea/Luvenga_Goreliy/low_2009_.pdf}
%\hfill
%\includegraphics[width=60mm]{../White_Sea/Luvenga_Goreliy/low_2002_.pdf}
\end{multicols}

%\smallskip

\begin{multicols}{3}
\hfill
\includegraphics[width=60mm]{../White_Sea/Luvenga_Goreliy/low_2006_.pdf}
\hfill
\includegraphics[width=60mm]{../White_Sea/Luvenga_Goreliy/low_2010_.pdf}
%\hfill
%\includegraphics[width=60mm]{../White_Sea/Luvenga_Goreliy/low_2003_.pdf}
\end{multicols}

%\smallskip

\begin{multicols}{3}
\hfill
\includegraphics[width=60mm]{../White_Sea/Luvenga_Goreliy/low_2007_.pdf}
\hfill
\includegraphics[width=60mm]{../White_Sea/Luvenga_Goreliy/low_2011_.pdf}
%\hfill
%\includegraphics[width=60mm]{../White_Sea/Luvenga_Goreliy/low_2004_.pdf}
\end{multicols}


%\caption{Размерная структура {\it Macoma balthica} в СГЛ эстуария р. Лувеньги}
%\label{ris:size_str_estuaty_Luv}
\begin{center}
Рис. \ref{ris:size_str_Goreliy_low} (продолжение). Размерная структура {\it Macoma balthica} в НГЛ
 о. Горелого

\end{center}
\end{figure}


%II разрез high

\begin{figure}[h]

\begin{multicols}{3}
\hfill
\includegraphics[width=60mm]{../White_Sea/Luvenga_II_razrez/high_beatch_1992_.pdf}
\hfill
\includegraphics[width=60mm]{../White_Sea/Luvenga_II_razrez/high_beatch_1996_.pdf}
\hfill
\includegraphics[width=60mm]{../White_Sea/Luvenga_II_razrez/high_beatch_2000_.pdf}
\end{multicols}

%\smallskip


\begin{multicols}{3}
\hfill
\includegraphics[width=60mm]{../White_Sea/Luvenga_II_razrez/high_beatch_1993_.pdf}
\hfill
\includegraphics[width=60mm]{../White_Sea/Luvenga_II_razrez/high_beatch_1997_.pdf}
\hfill
\includegraphics[width=60mm]{../White_Sea/Luvenga_II_razrez/high_beatch_2002_.pdf}
\end{multicols}

%\smallskip

\begin{multicols}{3}
\hfill
\includegraphics[width=60mm]{../White_Sea/Luvenga_II_razrez/high_beatch_1994_.pdf}
\hfill
\includegraphics[width=60mm]{../White_Sea/Luvenga_II_razrez/high_beatch_1998_.pdf}
\hfill
\includegraphics[width=60mm]{../White_Sea/Luvenga_II_razrez/high_beatch_2004_.pdf}
\end{multicols}

%\smallskip

\begin{multicols}{3}
\hfill
\includegraphics[width=60mm]{../White_Sea/Luvenga_II_razrez/high_beatch_1995_.pdf}
\hfill
\includegraphics[width=60mm]{../White_Sea/Luvenga_II_razrez/high_beatch_1999_.pdf}

\end{multicols}


\caption{Размерная структура {\it Macoma balthica} на верхнем пляже материковой литорали в районе пос.~ Лувеньга}
\label{ris:size_str_2razrez_high}
\end{figure}

% 2razrez fucus
\begin{figure}[h]

\begin{multicols}{3}
\hfill
\includegraphics[width=60mm]{../White_Sea/Luvenga_II_razrez/fucus_zone_1992_.pdf}
\hfill
\includegraphics[width=60mm]{../White_Sea/Luvenga_II_razrez/fucus_zone_1996_.pdf}
\hfill
\includegraphics[width=60mm]{../White_Sea/Luvenga_II_razrez/fucus_zone_2000_.pdf}
\end{multicols}

%\smallskip


\begin{multicols}{3}
\hfill
\includegraphics[width=60mm]{../White_Sea/Luvenga_II_razrez/fucus_zone_1993_.pdf}
\hfill
\includegraphics[width=60mm]{../White_Sea/Luvenga_II_razrez/fucus_zone_1997_.pdf}
\hfill
\includegraphics[width=60mm]{../White_Sea/Luvenga_II_razrez/fucus_zone_2002_.pdf}
\end{multicols}

%\smallskip

\begin{multicols}{3}
\hfill
\includegraphics[width=60mm]{../White_Sea/Luvenga_II_razrez/fucus_zone_1994_.pdf}
\hfill
\includegraphics[width=60mm]{../White_Sea/Luvenga_II_razrez/fucus_zone_1998_.pdf}
\hfill
\includegraphics[width=60mm]{../White_Sea/Luvenga_II_razrez/fucus_zone_2004_.pdf}
\end{multicols}

%\smallskip

\begin{multicols}{3}
\hfill
\includegraphics[width=60mm]{../White_Sea/Luvenga_II_razrez/fucus_zone_1995_.pdf}
\hfill
\includegraphics[width=60mm]{../White_Sea/Luvenga_II_razrez/fucus_zone_1999_.pdf}

\end{multicols}


\caption{Размерная структура {\it Macoma balthica} в поясе фукоидов материковой литорали в районе пос. Лувеньга}
\label{ris:size_str_2razrez_fucus}
\end{figure}

% 2razrez zostera
\begin{figure}[h]

\begin{multicols}{3}
\hfill
\includegraphics[width=60mm]{../White_Sea/Luvenga_II_razrez/zostera_zone_1992_.pdf}
\hfill
\includegraphics[width=60mm]{../White_Sea/Luvenga_II_razrez/zostera_zone_1996_.pdf}
\hfill
\includegraphics[width=60mm]{../White_Sea/Luvenga_II_razrez/zostera_zone_2000_.pdf}
\end{multicols}

%\smallskip


\begin{multicols}{3}
\hfill
\includegraphics[width=60mm]{../White_Sea/Luvenga_II_razrez/zostera_zone_1993_.pdf}
\hfill
\includegraphics[width=60mm]{../White_Sea/Luvenga_II_razrez/zostera_zone_1997_.pdf}
\hfill
\includegraphics[width=60mm]{../White_Sea/Luvenga_II_razrez/zostera_zone_2002_.pdf}
\end{multicols}

%\smallskip

\begin{multicols}{3}
\hfill
\includegraphics[width=60mm]{../White_Sea/Luvenga_II_razrez/zostera_zone_1994_.pdf}
\hfill
\includegraphics[width=60mm]{../White_Sea/Luvenga_II_razrez/zostera_zone_1998_.pdf}
\hfill
\includegraphics[width=60mm]{../White_Sea/Luvenga_II_razrez/zostera_zone_2004_.pdf}
\end{multicols}

%\smallskip

\begin{multicols}{3}
\hfill
\includegraphics[width=60mm]{../White_Sea/Luvenga_II_razrez/zostera_zone_1995_.pdf}
\hfill
\includegraphics[width=60mm]{../White_Sea/Luvenga_II_razrez/zostera_zone_1999_.pdf}

\end{multicols}


\caption{Размерная структура {\it Macoma balthica} в поясе взморника {\it Zostera marina} материковой литорали в районе пос. Лувеньга}
\label{ris:size_str_2razrez_zostera}
\end{figure}

% 2razrez low
\begin{figure}[h]

\begin{multicols}{3}
\hfill
\includegraphics[width=60mm]{../White_Sea/Luvenga_II_razrez/low_beatch_1992_.pdf}
\hfill
\includegraphics[width=60mm]{../White_Sea/Luvenga_II_razrez/low_beatch_1996_.pdf}
\hfill
\includegraphics[width=60mm]{../White_Sea/Luvenga_II_razrez/low_beatch_2000_.pdf}
\end{multicols}

%\smallskip


\begin{multicols}{3}
\hfill
\includegraphics[width=60mm]{../White_Sea/Luvenga_II_razrez/low_beatch_1993_.pdf}
\hfill
\includegraphics[width=60mm]{../White_Sea/Luvenga_II_razrez/low_beatch_1997_.pdf}
\hfill
\includegraphics[width=60mm]{../White_Sea/Luvenga_II_razrez/low_beatch_2002_.pdf}
\end{multicols}

%\smallskip

\begin{multicols}{3}
\hfill
\includegraphics[width=60mm]{../White_Sea/Luvenga_II_razrez/low_beatch_1994_.pdf}
\hfill
\includegraphics[width=60mm]{../White_Sea/Luvenga_II_razrez/low_beatch_1998_.pdf}

\end{multicols}

%\smallskip

\begin{multicols}{3}
\hfill
\includegraphics[width=60mm]{../White_Sea/Luvenga_II_razrez/low_beatch_1995_.pdf}
\hfill
\includegraphics[width=60mm]{../White_Sea/Luvenga_II_razrez/low_beatch_1999_.pdf}

\end{multicols}


\caption{Размерная структура {\it Macoma balthica} на нижнем пляже материковой литорали в районе пос. Лувеньга}
\label{ris:size_str_2razrez_low}
\end{figure}



%ЮГ Ряшкова
\begin{figure}[h]

\begin{multicols}{3}
\hfill
\includegraphics[width=60mm]{../White_Sea/Ryashkov_YuG/YuG_2001_.pdf}
\hfill
\includegraphics[width=60mm]{../White_Sea/Ryashkov_YuG/YuG_2005_.pdf}
\hfill
\includegraphics[width=60mm]{../White_Sea/Ryashkov_YuG/YuG_2009_.pdf}
\end{multicols}

%\smallskip


\begin{multicols}{3}
\hfill
\includegraphics[width=60mm]{../White_Sea/Ryashkov_YuG/YuG_2002_.pdf}
\hfill
\includegraphics[width=60mm]{../White_Sea/Ryashkov_YuG/YuG_2006_.pdf}
\hfill
\includegraphics[width=60mm]{../White_Sea/Ryashkov_YuG/YuG_2010_.pdf}
\end{multicols}

%\smallskip

\begin{multicols}{3}
\hfill
\includegraphics[width=60mm]{../White_Sea/Ryashkov_YuG/YuG_2003_.pdf}
\hfill
\includegraphics[width=60mm]{../White_Sea/Ryashkov_YuG/YuG_2007_.pdf}
\hfill
\includegraphics[width=60mm]{../White_Sea/Ryashkov_YuG/YuG_2011_.pdf}
\end{multicols}

%\smallskip

\begin{multicols}{3}
\hfill
\includegraphics[width=60mm]{../White_Sea/Ryashkov_YuG/YuG_2004_.pdf}
\hfill
\includegraphics[width=60mm]{../White_Sea/Ryashkov_YuG/YuG_2008_.pdf}
\hfill
\includegraphics[width=60mm]{../White_Sea/Ryashkov_YuG/YuG_2012_.pdf}
\end{multicols}


\caption{Размерная структура {\it Macoma balthica} у нуля глубин в Южной губе о. Ряшкова}
\label{ris:size_str_YuG}
\end{figure}


%Западная Ряшкова Салма
\begin{figure}[h]

\begin{multicols}{3}
\hfill
\includegraphics[width=60mm]{../White_Sea/Ryashkov_ZRS/zrs_1994_.pdf}
\hfill
\includegraphics[width=60mm]{../White_Sea/Ryashkov_ZRS/zrs_1998_.pdf}
\hfill
\includegraphics[width=60mm]{../White_Sea/Ryashkov_ZRS/zrs_2002_.pdf}
\end{multicols}

%\smallskip


\begin{multicols}{3}
\hfill
\includegraphics[width=60mm]{../White_Sea/Ryashkov_ZRS/zrs_1995_.pdf}
\hfill
\includegraphics[width=60mm]{../White_Sea/Ryashkov_ZRS/zrs_1999_.pdf}
\hfill
\includegraphics[width=60mm]{../White_Sea/Ryashkov_ZRS/zrs_2003_.pdf}
\end{multicols}

%\smallskip

\begin{multicols}{3}
\hfill
\includegraphics[width=60mm]{../White_Sea/Ryashkov_ZRS/zrs_1996_.pdf}
\hfill
\includegraphics[width=60mm]{../White_Sea/Ryashkov_ZRS/zrs_2000_.pdf}
\hfill
\includegraphics[width=60mm]{../White_Sea/Ryashkov_ZRS/zrs_2004_.pdf}
\end{multicols}

%\smallskip

\begin{multicols}{3}
\hfill
\includegraphics[width=60mm]{../White_Sea/Ryashkov_ZRS/zrs_1997_.pdf}
\hfill
\includegraphics[width=60mm]{../White_Sea/Ryashkov_ZRS/zrs_2001_.pdf}
\hfill
\includegraphics[width=60mm]{../White_Sea/Ryashkov_ZRS/zrs_2005_.pdf}
\end{multicols}


\caption{Размерная структура {\it Macoma balthica} в СГЛ Западной Ряшковой салмы}
\label{ris:size_str_ZRS}
\end{figure}



\begin{figure}[h]

\begin{multicols}{3}
\hfill
\includegraphics[width=60mm]{../White_Sea/Ryashkov_ZRS/zrs_2006_.pdf}
\hfill
\includegraphics[width=60mm]{../White_Sea/Ryashkov_ZRS/zrs_2010_.pdf}
\hfill

\end{multicols}

%\smallskip


\begin{multicols}{3}
\hfill
\includegraphics[width=60mm]{../White_Sea/Ryashkov_ZRS/zrs_2007_.pdf}
\hfill
\includegraphics[width=60mm]{../White_Sea/Ryashkov_ZRS/zrs_2011_.pdf}
\hfill

\end{multicols}

%\smallskip

\begin{multicols}{3}
\hfill
\includegraphics[width=60mm]{../White_Sea/Ryashkov_ZRS/zrs_2008_.pdf}
\hfill
\includegraphics[width=60mm]{../White_Sea/Ryashkov_ZRS/zrs_2012_.pdf}
\hfill

\end{multicols}

%\smallskip

\begin{multicols}{3}
\hfill
\includegraphics[width=60mm]{../White_Sea/Ryashkov_ZRS/zrs_2009_.pdf}
\hfill
%\includegraphics[width=60mm]{../White_Sea/Ryashkov_ZRS/zrs_2012_.pdf}
%\hfill

\end{multicols}


%\caption{Размерная структура {\it Macoma balthica} в СГЛ Западной Ряшковой салмы}
%\label{ris:size_str_ZRS}
\begin{center}
Рис.\ref{ris:size_str_ZRS} (продолжение). Размерная структура {\it Macoma balthica} в СГЛ Западной Ряшковой салмы
\end{center}
\end{figure}

%Ломнишный

\begin{figure}[h]

\begin{multicols}{2}
\hfill
\includegraphics[width=65mm]{../White_Sea/Lomnishniy/Lomnishniy_2007_.pdf}
\hfill
\includegraphics[width=65mm]{../White_Sea/Lomnishniy/Lomnishniy_2010_.pdf}
\end{multicols}

%\smallskip


\begin{multicols}{2}
\hfill
\includegraphics[width=65mm]{../White_Sea/Lomnishniy/Lomnishniy_2008_.pdf}
\hfill
\includegraphics[width=65mm]{../White_Sea/Lomnishniy/Lomnishniy_2011_.pdf}
\end{multicols}

%\smallskip

\begin{multicols}{2}
\hfill
\includegraphics[width=65mm]{../White_Sea/Lomnishniy/Lomnishniy_2009_.pdf}
\hfill
\includegraphics[width=65mm]{../White_Sea/Lomnishniy/Lomnishniy_2012_.pdf}
\end{multicols}

%\smallskip


\caption{Размерная структура {\it Macoma balthica} у нуля глубин литорали о.Ломнишный}
\label{ris:size_str_Lomnishniy}
\end{figure}


\end{document}

\afterpage{\clearpage}

	\chapter{Размерная структура поселений {\it Macoma~balthica}}

		\section{Белое море}

	\subsection*{Эстуарий реки Лувеньги}
На данном участке размерную структуру поселения маком в среднем горизонте литорали (СГЛ) отслеживали на протяжении $20$~лет ($1992 - 2012$).
За все время наблюдения максимальная длина особи, отмеченная в поселении составляла $18$~мм.

Характер размерно-частотного распределения особей неоднократно менялся на протяжении периода наблюдений (приложение~\ref{app:White_sizestr_hist}, рис.~\ref{ris:size_str_estuary_Luv}).
С $1993$ до $1997$ года в размерной структуре поселения выделялось три модальных класса, причем за все $5$~лет один из них попадал на особей до $4$~мм , второй на $7 - 9$ мм и третий --- это особи длиной более $10$~мм. 
В $1998$ году размерная структура поселения стала мономодальной, так как практически не осталось крупных особей, но
появилось много моллюсков длиной $1-2$~мм. 
В дальнейшем до $2002$ года оставалось мономодальное распределение особей по размерам, и происходило смещение модального класса --- в $2002$ году это были особи размером $6-7$ мм. 

В $2003$ году можно было выделить два пика: моллюски длиной $1-2$ мм и $7-9$~мм, то есть размерная структура поселения вновь стала бимодальной. 
В дальнейшем до $2012$~года размерная структура маком в данном поселении остается бимодальной. 
Первый модальный класс сохраняется --- особи длиной $1-2$~мм, а второй модальный класс варьирует, его составляют в разные годы особи длиной от $9$ до $12$~мм.
Количественное соотношение особей двух модальных классов менялось. Чаще ($2004$, $2007 - 2010$~ года) преобладали мелкие моллюски, но в отдельные годы ($2006$, $2012$) доля крупных была выше, либо представительство крупных и мелких доминирующих классов было сравнимым ($2005$, $2011$~годы).



	\subsection*{Остров Горелый}

На данном участке размерную структуру поселения маком отслеживали на протяжении $20$~лет ($1992 - 2012$) в пределах трех горизонтов литорали и у нуля глубин.
За все время наблюдения максимальная длина особи, отмеченная в поселении составляла $20$~мм.

В верхнем горизонте литорали (ВГЛ) размерная структура поселения до $1997$ года (приложение~\ref{app:White_sizestr_hist}, рис.~\ref{ris:size_str_Goreliy_high}) представляла собой бимодальное распределение с модальными классами $2-5$~мм и $7-13$~мм.
В $1998$ году появилось значительное количество особей длиной $1-4$~мм. 
В дальнейшем можно было наблюдать смещение по оси размеров данного модального класса. 
В $2001$ году в поселении вновь сформировалась бимодальная размерная структура (модальные классы $1-3$ и $5-6$ мм), и в дальнейшем такое распределение сохранялось до $2007$ года.
В $2008 - 2009$ годах распределение было мономодальное с модальным классом $1-2$~мм.
Интересно отметить, что с $2002$ по $2009$ год доминирующим размерным классом в поселении были особи длиной $1-2$~мм.
В $2011-2012$ году восстановилась бимодальная размерная структура с модальными классами $1-4$ и $9-11$~мм.

%\bigskip
В среднем горизонте литорали (СГЛ) до $1996$ года в этой зоне выделялась бимодальная размерная структура (приложение~\ref{app:White_sizestr_hist}, рис.~\ref{ris:size_str_Goreliy_mid}) (модальные классы --- моллюски длиной $1-4$~мм и $6-13$~мм). 
В $1997$ году распределение было практически равномерное при общей низкой численности. 
В $1998$ году появилось значительное количество моллюсков длиной до $1$ мм. 
После чего наблюдалось смещение модального класса до $2003$ года. 
До $2001$ года размерная структура поселения оставалась мономодальной, но в $2002-03$ годах появился еще один модальный класс -- моллюски длиной до $2$ мм. 
Таким образом, после $2002$ года  в поселении вновь восстановилась бимодальная размерная структура, которая сохраняется вплоть до $2007$ года.
В $2008$ году распределение особей по размерам становится мономодальным за счет элиминирования особей крупных размеров. 
В $2011-2012$ году восстанавливается бимодальное распределение.


%\bigskip
В нижнем горизонте литорали (НГЛ) в $1992$ году в связи с малой численностью моллюсков сложно говорить о характерной размерной структуре поселения (приложение~\ref{app:White_sizestr_hist}, рис.~\ref{ris:size_str_Goreliy_midlow}).
В $1993$ году фактически можно выделить только один пик ($2-3$~мм), хотя и было очень незначительное повышение при длине $9-10$~мм. 
Но с $1994$ по $1996$ год было представлено бимодальное распределение с модальными классами $1-3$ мм и $9-11$~мм.
В $1997$ году численность моллюсков значительно снизилась, и распределение по размерам было практически равномерное. 
В $1998-1999$ году в значительных количествах появились особи длиной $2-3$~мм и можно было наблюдать смещение модального класса по оси размеров вплоть до $2003$ года, когда его значение становится $5-6$~мм. 
Кроме того, с $2002$ года можно было выделить еще один модальный класс -- особи длиной $1-2$~мм, то есть размерная структура поселения вновь стала бимодальной, каковой и оставалась до конца периода наблюдений.


%\bigskip
У нуля глубин в $1992$ году моллюсков практически не было (приложение~\ref{app:White_sizestr_hist}, рис.~\ref{ris:size_str_Goreliy_low}), но в $1993$ году можно говорить о бимодальной размерной структуре поселения, которая сохранялась до $1997$ года. 
В $1998-1999$ году произошло элиминирование крупных особей на фоне появления значительного количества особей длиной $1-2$~мм. 
В $2001-2003$ годах в поселении восстановилась бимодальная структура и в $2003$ году модальные классы образовывали особи длиной до $1$ мм и $8.1-9.0$ мм. 
С $2003$ до $2007$ года преобладали особи длиной $9-12$~мм, а с $2008$ появляется второй модальный пик --- особи размером $1-3$~мм.


		\subsection*{Материковая литораль в районе поселка Лувеньга}
На данном участке размерную структуру поселения маком отслеживали на протяжении $10$~лет ($1992 - 2004$) в пределах четырех биотопов.
За все время наблюдения максимальная длина особи, отмеченная в поселении составляла $24$~мм.

В зоне верхнего пляжа размерная структура поселения (приложение~\ref{app:White_sizestr_hist}, рис.~\ref{ris:size_str_2razrez_high}) в $1993$ году была мономодальная, но с $1994$ по $1997$ годы стала бимодальной с модальными классами $2-5$ и $6-10$~мм.
В $1998$ году появилось значительное число особей размером менее $1$ мм, после чего до $2002$ года прослеживалось смещение модального класса. 
В $2002$ году в поселении восстановилась бимодальная структура (модальные классы -- $1-2$ мм и $5-6$ мм).

%\bigskip
В поясе фукоидов размерная структура поселения (приложение~\ref{app:White_sizestr_hist}, рис.~\ref{ris:size_str_2razrez_fucus}) в $1992-1997$ году характеризовалась наличием двух модальных классов: $1-6$ и $7-12$~мм. 
С $1998$ по $2000$ года размерная структура поселения была мономодальной, причем все $3$ года пик формировали особи длиной $1-2$~мм. 
В $2002$ году вновь выделялось два модальных класса: $1-2$ и $7-8$~мм.

%\bigskip
В поясе зостеры до $1998$ года в размерной структуре поселения пояса зостеры выделялись незначительные пики и можно говорить о равномерном распределении моллюсков(приложение~\ref{app:White_sizestr_hist}, рис.~\ref{ris:size_str_2razrez_zostera}). 
После $1998$ года она стала мономодальной, причем пик формировали моллюски длиной $1-2$ мм.


%\bigskip
В зоне нижнего пляжа до $1999$ года размерная структура поселения была полимодальная, хотя эти пики нельзя было четко выделить (приложение~\ref{app:White_sizestr_hist}, рис.~\ref{ris:size_str_2razrez_low}). 
В $1999-2000$ годах практически не осталось крупных особей, но появилось значительное число моллюсков размером $1-2$~мм. 

		\subsection*{Южная губа о.~Ряшкова}

На данном участке наблюдения проводили с $2001$ года, размерную структуру поселения у нуля глубин отслеживали в течение $12$ лет.
Максимальный размер маком в данном поселении составил $23$~мм в $2003$ году, однако в другие годы максимальный размер не превышал $16$~мм.

В Южной губе на протяжении всего периода наблюдений размерная структура (приложение~\ref{app:White_sizestr_hist}, рис.~\ref{ris:size_str_YuG}) поселения была мономодальной с преобладанием особей длиной $1-3$~мм. 



		\subsection*{Западная Ряшкова салма}

На литорали о.~Ряшкова в Западной Ряшковой салме наблюдения проводили с $1994$ по $2012$ год ($18$~лет). Наблюдения проводили в среднем горизонте литорали.
Максимальный размер моллюсков, отмеченный в поселении составил $20$~мм.

На данном участке  до $1998$ года размерная структура была полимодальной(приложение~\ref{app:White_sizestr_hist}, рис.~\ref{ris:size_str_ZRS}). 
В $1999$ году крупные особи в основном элиминировали, и размерная структура стала мономодальной с доминированием моллюсков длиной $1-2$~мм.
В дальнейшем $2001$ года до конца наблюдений размерная структура была бимодальной с модальными классами $1-3$ и $9-11$~мм.


		\subsection*{о.~Ломнишный}

На литорали острова Ломнишный наблюдения проводили с $2007$ года в течение $6$~лет у нуля глубин. 
Максимальный размер особи, отмеченный в поселении составлял $17$~мм.

Размерная структура на данном участке в течение всего периода наблюдений была мономодальной (приложение~\ref{app:White_sizestr_hist}, рис.~\ref{ris:size_str_Lomnishniy}).
В основном доминировали особи длиной $1-3$~мм, за исключением $2009-2010$ годов, когда доминировали особи длиной $5$ и $7$~мм, соответственно.


\bigskip
Таким образом, наиболее распространенный вариант динамики размерной структуры в поселениях {\it M.~balthica} в Белом море это чередование бимодальной и мономодальной размерных структур.
Мономодальная структура обычно формируется на фоне практически полной элиминации крупных особей при пополнении поселения новой генерацией маком.
В дальнейшем, если новое пополнение происходит быстрее, чем предыдущая генерация элиминирует, то формируется бимодальная размерная структура.

Среди 6 мониторинговых участков в Кандалакшском заливе Белого моря для двух из них --- в Южной губе острова Ряшков и на о. Ломнишный --- динамика размерной структуры принципиально отличалась, и мы ежегодно видим мономодальное распределение особей по размерам с доминированием мелких особей.

\afterpage{\clearpage}

		\section{Баренцево море}

		\subsection{Губы Кольского залива}

На участке Абрам-мыс (рис.~\ref{ris:Barents_sizestr}) были представлены особи длиной от $2$ до $16$~мм. 
В среднем горизонте литорали характер распределения был мономодальный с преобладанием моллюсков длиной $10-13$~мм. 
В нижнем горизонте литорали к аналогичному пику (особи длиной $12-14$~мм) добавляется второй — моллюски длиной $2-3$~мм.

На участке в Пала-губе (рис.~\ref{ris:Barents_sizestr}) также в среднем горизонте распределение особей по размерам было мономодальным, а на нижнем --- бимодальным. 
Однако при этом наблюдалась обратная ситуация: в среднем горизонте литорали доминировали особи самой мелкой размерной группы --- $3-4$~мм, в то время как  в нижнем горизонте кроме таких особей  хорошо представлена размерная группа $10-12$~мм.

		\subsection{Губы побережья Восточного Мурмана}

В губе Гаврилово (прил.~\ref{app:Barents_sizestr_hist}, рис.~\ref{ris:Barents_sizestr}) распределение особей практически равномерное. 
В среднем горизонте литорали несколько преобладают особи длиной $15-20$~мм. 
В нижнем горизонте литорали представлены лишь единичные особи различных возрастов.	

Во всех горизонтах губы Ярнышной (прил.~\ref{app:Barents_sizestr_hist}, рис.~\ref{ris:Barents_sizestr}) доминировали особи длиной $4-6$~мм. 
На всех участках можно отметить присутствие относительно крупных моллюсков (особи длиной более $14$~мм), однако их представленность на порядок варьирует в разных горизонтах.

В губе Шельпино (прил.~\ref{app:Barents_sizestr_hist}, рис.~\ref{ris:Barents_sizestr})  представлены единичные особи длиной от $6$ до $16$~мм. 
В среднем горизонте литорали некоторое превышение формируют особи длиной $15$~мм, однако и они остаются немногочисленны.

В губе Порчниха (прил.~\ref{app:Barents_sizestr_hist}, рис.~\ref{ris:Barents_sizestr}) были представлены особи длиной от $4$ до $21$~мм. 
Распределение особей по размерам было полимодальным. 
Выделяется по крайней мере три моды: $4-7$~мм, $9-12$~мм и $18-20$~мм. 
Несущественное превышение численности  отмечено для особей длиной $13-15$~мм.

В губе Ивановская (прил.~\ref{app:Barents_sizestr_hist}, рис.~\ref{ris:Barents_sizestr}) были обнаружены макомы длиной от $2$ до $13$~мм. 
Количество особей в каждой размерной группе колебалось от $20$ до $30$ экземпляров, лишь моллюсков длиной $2$~мм было отмечено около $50$. 
Распределение особей по размерам было практически равномерным при некотором превышении доли особей длиной $2$ и $10$~мм. 

		\subsection{Дальний пляж губы Дальне-Зеленецкой (Восточный Мурман)}
На данном участке ни в один год в пробах не было отмечено особей {\it M.~balthica} с длиной раковины менее $2$~мм (прил.~\ref{app:Barents_sizestr_hist}, рис.~\ref{ris:size_str_DZ}). 
Максимальный размер моллюсков в разные годы колебался от $18$ до $20$~мм. 

С $2002$ до $2004$ года размерная структура маком в данном поселении была полимодальной. 
Можно говорить по крайней мере о трех модальных группах.
Доминировали все эти годы особи размером $8 - 14$~мм.
В $2005$ году размерная структура фактически мономодальная с преобладанием крупных особей длиной больше $12$~мм, и встречаются единичные моллюски размером $3 - 4$~мм.
В $2006$ году добавляется вторая модальная группа~--- особи длиной $3 - 5$~мм.
После $2007$ года восстанавливается полимодальное распределение особей по размерам.


\bigskip

Таким образом, на исследованных участках был представлены все возможные варианты размерной структуры: мономодальное (участки: Абрам-мыс СГЛ, Пала-губа СГЛ, губа Гаврилово СГЛ), бимодальное (участки: Абрам-мыс НГЛ, Пала-губа НГЛ, губа Ярнышная, губа Дальне-Зеленецкая СГЛ, губа Порчниха СГЛ) и практически равномерное (участки: губа Гаврилово НГЛ, губа Дальне-Зеленецкая ВГЛ и НГЛ, губа Шельпино ВГЛ и СГЛ, губа Ивановская ВСЛ) распределение особей по размерам. 

Мономодальное распределение особей по размерам наблюдается либо при доминировании мелких особей длиной $3-5$~мм, либо при доминировании крупных --- $12-18$~мм.
При бимодальном распределении обычно первую моду формировали мелкие макомы длиной $2-5$~мм, а вторую --- моллюски длиной более $10$~мм.

\afterpage{\clearpage}

\afterpage{\clearpage}

%%рост в разных условиях
    \section{Линейный рост {\it Macoma balthica}}
Рост особей рассматривается как отклик особей на совокупность условий обитания. 
Анализ роста проводили по усредненным возрастным рядам. 
Для их получения по каждому описанию   были   построены   треугольные   матрицы   (табл.~\ref{tab:Abram_sgl_growth_matrix}~--~\ref{tab:Porchnikha_sgl_growth_matrix},   Приложение~\ref{app:growth_matrix}),   полностью описывающие рост особей в поселении.

В   первую   очередь   анализ   был   проведен   по  усредненным   возрастным   рядам, построенным как взвешенная оценка (с учетом числа особей) характера роста всех генераций по   результатам   измерений   размеров  моллюсков  в   периоды   зимней   остановки   роста.  
Такая кумулятивная   характеристика   должна   в   наибольшей   мере   отражать   особенности   условий роста маком в каждом местообитании.
Наиболее длинный возрастной ряд удалось получить для среднего горизонта литорали губы Гаврилово --- $15$~лет при длине $17,9$~мм (табл.~\ref{tab:Gavrilovo_sgl_growth_matrix},   Приложение~\ref{app:growth_matrix}). 
Однако максимальный размер особей был отмечен в верхнем горизонте литорали губы Ярнышная --- $20,1$~мм при возрасте $13$~лет (табл. \ref{tab:Yarnyshnaya_sgl_growth_matrix},   Приложение~\ref{app:growth_matrix})). 

Полученные   возрастные   ряды   были   аппроксимированы   с   помощью   уравнения Берталанфи   (рис.~\ref{ris:Barents_growth_gorizonts_all}).
    \begin{figure}
        \includegraphics[width=\textwidth]{../Barenc_Sea/growth_from_MSc/Rost_gorizonts_all.jpg}
    \caption{Разнообразие моделей линейного роста, описывающих взвешенные характеристики возрастных рядов генераций в изученных поселениях маком}
    \label{ris:Barents_growth_gorizonts_all}
    \end{figure}

Быстрее   всего   росли   макомы   в   среднем   горизонте   литорали   губы Порчниха, достигая длины $19,4$~мм за $9$~лет и в среднем горизонте литорали губы Ярнышная --- $16,7$~мм за $8$~лет. 
Остальные кривые не распадаются на очевидные группы, и некоторые пересекают   друг   друга.   
Поэтому   была   использована   формальная   процедура   сравнения полученных   кривых   роста   с   учетом   разброса   эмпирических   данных   относительно регрессионной модели (рис. \ref{ris:dendrogramma_linear_all_gorizonts}).
    \begin{figure}
        \includegraphics[width=\textwidth]{../Barenc_Sea/growth_from_MSc/dendrogramma_sravnenie_rosta_linear_all_gorizonts.jpg}
    \caption{Классификация поселений маком по моделям линейного роста, описывающих взвешенные характеристики возрастных рядов генераций}
    \label{ris:dendrogramma_linear_all_gorizonts}
    \end{figure}

В   ходе   классификации   было   выделено   три   кластера.   
В   первый   вошли   следующие описания (уровень различий внутри кластера менее $0,87$): Абрам-мыс, Пала-губа НГЛ, губа Гаврилово   СГЛ,   губа   Ярнышная   НГЛ,   Шельпино   ВГЛ.   
Второй   кластер   (уровень   различий внутри кластера менее $0,76$) составили участки Пала-губа СГЛ, губа Гаврилово НГЛ, губа Дальнезеленецкая,   губа   Ярнышная   ВГЛ,   Шельпино   СГЛ.   
В   последний   кластер   (уровень различий внутри кластера менее $0,38$) вошли участки губа Ярнышная СГЛ и губа Порчниха СГЛ. 
На участках Абрам-мыс и губа Дальнезеленецкая характер роста был одинаковый на всех горизонтах литорали. 

Однако в распределении остальных описаний нет географической приуроченности. 
Как и ожидалось, поселения  из средних горизонтов литорали губы Ярнышной и губы Порчниха   выделились   в   отдельный   кластер.   
Низкий   уровень   различий   ($0,38$)   говорит   о большом   разбросе   наблюдаемых   значений   относительно   модели   роста.   
Это   могло   бы свидетельствовать   об   относительно   грубом   описании   соответствующих   возрастных   рядов, хотя   значительный объем  выборки ($76$  и $65$  особей, соответственно)  позволяет  говорить  о значительном варьировании роста маком в пределах каждого участка.

Интересно,   что   при   незначительном   расхождении   кривых   роста,   уровень   различий между первым и вторым кластером оказался очень высоким ($2,52$). 
Не было отмечено явного разделения участков по мареографическому уровню, хотя во второй кластер попало больше описаний   с   более   высоких   горизонтов   литорали.   Максимальное   различие   было   между кластерами $2$ и $3$ ($2,76$).

По   итогам   классификации   было   выделено   три   группы   маком,   отличающиеся   по характеру роста (рис.~\ref{ris:Barents_clusters_gorizonts_all}).
    \begin{figure}
        \includegraphics[width=\textwidth]{../Barenc_Sea/growth_from_MSc/rost_clusters_all.jpg}
    \caption{Модели роста, передающие  принципиальные свойства вариации характера линейного роста маком в изученных местообитаниях}
    \label{ris:Barents_clusters_gorizonts_all}
    \end{figure}
Первая группа — особи с наименьшей скоростью роста достигали длины $16,4$ мм за $14$ лет, обитавшие на относительно более низком уровне осушки. 
Макомы с промежуточной   скоростью   роста   вырастали   за   $13$   лет   $до   19,3$   мм.   
Особи   с   максимальной скоростью роста за $9$ лет достигали длины $18$ мм.

Таким   образом,   не   удалось   выделить   ни   географической,   ни   мареографической приуроченности особей с одинаковой скоростью роста. 
Возможно, это связано с тем, что во взвешенных оценках возрастных рядов могут сильнее проявиться черты нехарактерных, но сильно   представленных   в   поселении   сегодня   генераций,   и,   следовательно,   в   каждом возрастном   ряду   получается   разная   представленность   межгодовой   составляющей   условий 
роста маком. 

Для   того,   чтобы   снять   эти   влияния,   следующий   анализ   проводили   с   купированием исходных  данных  до  объединения   нескольких  описаний возрастных рядов  только  старших (>8+) генераций (рис.~\ref{ris:Barents_growth_gorizonts_8year}).
    \begin{figure}
        \includegraphics[width=\textwidth]{../Barenc_Sea/growth_from_MSc/Rost_8+_gorizonts.jpg}
        \caption{Разнообразие моделей линейного роста, описывающих усредненные возрастные ряды генераций маком старше 8 лет} 
    \label{ris:Barents_growth_gorizonts_8year}
    \end{figure}
Полученная   картина   аналогична   полученной   по   интегральным   описаниям:   быстрее всего росли макомы в среднем горизонте литорали губы Порчниха и в среднем горизонте литорали губы Ярнышная, в то время как остальные кривые не распадаются на очевидные группы, и некоторые пересекают друг друга. 
Однако при сравнении полученных кривых роста с учетом разброса эмпирических данных относительно регрессионной модели было выделено 4 кластера (рис.~\ref{ris:dendrogramma_linear_clusters_8year}).
    \begin{figure}
        \includegraphics[width=\textwidth]{../Barenc_Sea/growth_from_MSc/dendrogramma_sravnenie_rosta_linear_8year_gorizonts.jpg}
        \caption{Классификация поселений маком по моделям линейного роста, описывающих усредненные возрастные ряды генераций маком старше 8 лет}
    \label{ris:dendrogramma_linear_clusters_8year}
    \end{figure}

В первый кластер (уровень различий внутри кластера менее $0,86$) вошли следующие описания:  Абрам-мыс, Пала-губа НГЛ, губа  Гаврилово  СГЛ, губа Ярнышная НГЛ. 
Второй кластер (уровень различий внутри кластера менее $0,57$) составили участки Пала-губа СГЛ, губа Гаврилово НГЛ, губа Дальнезеленецкая, губа Ярнышная ВГЛ, Шельпино СГЛ. 
В третий кластер (уровень различий внутри кластера менее $0,61$) вошли участки губа Ярнышная СГЛ и губа Порчниха СГЛ. 
В отдельный кластер попал участок губа Шельпино ВГЛ (минимальное различие $2,53$ --- с кластером $1$). 
Таким   образом,   единственное   качественное   изменение   относительно   результатов, полученных при сравнении усредненных кривых роста --- это выделение верхнего горизонта литорали губы Шельпино в отдельный кластер. 
Однако, коэффициенты различия значительно изменились. 
В два раза увеличилось различие между описаниями внутри кластера $3$, различие внутри кластера $2$ уменьшились. 
Максимальное различие было отмечено между кластерами два и три ($5,1$).

По итогам классификации было выделено четыре группы маком, отличающиеся по характеру роста (рис. \ref{ris:Barents_clusters_gorizonts_8year}). 
    \begin{figure}
        \includegraphics[width=\textwidth]{../Barenc_Sea/growth_from_MSc/Rost_8+_clusters.jpg}
        \caption{Модели роста, передающие  принципиальные свойства вариации характера  линейного роста маком старше 8 лет в изученных местообитаниях}
    \label{ris:Barents_clusters_gorizonts_8year}
    \end{figure}
Особи с минимальной скоростью роста ($14$~мм за $12$~лет) обитали в верхнем горизонте литорали губы Шельпино.
Среди групп с промежуточной скоростью роста более   низкой   скоростью   роста   ($16,4$~мм   за   $14$~лет)   обладали   моллюски,   обитавшие   на относительно более низком уровне осушки. 
Особи с максимальной скоростью роста за $9$ лет достигали длины $18$ мм.

Использование   интегральных   моделей   роста   маком   вполне   отвечает   задаче сравнительного анализа их поселений. 
Однако скорость роста моллюсков зависит не только от внешних,   общих   для   всего   поселения,   факторов,   но   и   от   локальных   микроусловий.      
Материалы   настоящей   работы   не   позволяют   нам   провести   анализ   вариации индивидуальных   особенностей   роста   маком   как   отклика   на   условия   их   роста.   
Для   этого нужны специальные экспериментальные исследования. 
Однако можно попытаться выделить групповые   эффекты.   
Речь   идет   о   снижении   уровня   рассматриваемой   биосистемы   до возрастной группы. 

В таблицах приложения~\ref{app:growth_matrix} приведены усредненные для каждой возрастной группы результаты измерений  расстояния  от верхушки раковины до каждой метки  зимней остановки роста.
Используем их для анализа характера вариации средних величин годового прироста. 
Величины годового прироста варьировали от $0,05$ до $3,58$ мм (табл.~\ref{tab:godovoy_prirost_min_max}).
\begin{table}[h]
    \caption{Размах варьирования годового прироста {\it Macoma balthica} в зависимости от участка, горизонта     литорали и начального размера особи}
    \label{tab:godovoy_prirost_min_max}
    \begin{tabularx}{\textwidth}{|lX|XX|XX|XX|XX|}
        \hline
        Участок        &         & \multicolumn{8}{c|}{начальный размер}                                             \\ \cline{3-10}
                              &                 & \multicolumn{2}{c|}{$< 3$ мм} & \multicolumn{2}{c|}{$3-6$ мм} & \multicolumn{2}{c|}{$6-9$ мм} &\multicolumn{2}{c|}{$> 9$ мм}\\ \hline
     \multicolumn{2}{|r|}{годовой прирост}&  мин              & макс   & мин    & макс              & мин  & макс & мин  & макс \\ \hline
        Абрам-мыс      & сгл             & 0,69             & 1,68   & 0,69   & 1,31              & 0,73 & 1,57 & 1,00 & 1,23 \\
                       &  нгл            & 0,90            & 1,77             & 0,88   & 1,48   & 0,80              & 1,73 & 0,67 & 1,50 \\ \hline
        Пала-губа      & сгл             & 0,77             & 2,15   & 1,20   & 2,90              & 1,05 & 1,68 & 1,40 & 1,40 \\
                       & нгл            & 1,01            & 1,43             & 1,01   & 1,86   & 0,83              & 1,73 & 0,85 & 0,85 \\ \hline
        губа Гаврилово & сгл             & 0,70             & 2,10   & 0,93   & 2,40              & 0,80 & 2,10 & 0,70 & 1,75 \\
                       &  нгл            & 0,60            & 2,30             & 1,00   & 2,20   & 0,80              & 2,10 & 0,60 & 1,90 \\ \hline
        губа Ярнышная  & сгл             & 1,08             & 3,30   & 1,80   & 3,58              & 2,60 & 2,75 & 1,22 & 2,52 \\
                       &  нгл            & 0,80            & 1,60             & 0,80   & 1,50   & 0,95              & 1,56 & 0,05 & 1,72 \\   \hline
    \end{tabularx}
\end{table}

В качестве переменных воздействия в контексте данной работы логично обратиться к  таким   причинам   вариации   скорости   маком   как   география   положения   местообитаний, мареография положения станций наблюдений. 
Кроме того, нельзя не учесть очевидную связь величины годового прироста маком с их возрастом. 

В   проведенном   выше   сравнительном   анализе   интегральных   кривых   роста   мы выравнивали эмпирические возрастные ряды с помощью линейной модификации уравнения роста Берталанфи. При этом очевидным образом снижается объективность представлений о межгодовых различиях годовых приростов особей в возрастных группах. 
Попробуем отойти от возраста как от условия, организующего скорость роста маком, и в качестве одного из предикторов   величины   годового   прироста   возьмём   начальный   (к   данному   годовому интервалу) средний размер особей возрастной группы.  
Такой анализ логично провести с помощью дисперсионного анализа. 

%из статьи
На первом этапе анализа (факторы <<горизонт литорали>>, <<начальный средний размер особей в возрастной группе>>) установлено (табл.~\ref{tab:prirost_ANOVA_tidal}), что каждая из назначенных причин вариации достоверно определяет величину годового прироста. 
\begin{table}[h]
    \caption{Структура вариансы средних величин годового прироста {\it M. balthica} в возрастных группах в градиентах величины начального среднего размера особей в возрастной группе и мареографического уровня положения станций наблюдения}
    \label{tab:prirost_ANOVA_tidal}
    \begin{center}
    \begin{tabular}{|l|rrrrr|}
    \hline
    Источник вариации & $SS$   & $\nu$   & $M_S$   & $F$     & $\alpha$     \\ \hline
    A                 & 4,74   & 3   & 1,58 & 4,2   & 0,006 \\
    B                 & 11,98  & 2   & 5,99 & 15,92 & 0     \\
    AB                & 2,75   & 6   & 0,46 & 1,22  & 0,295 \\
    W                 & 193,82 & 515 & 0,38 &       &       \\ \hline
\end{tabular}
\end{center}

    \footnotesize{Источники вариации: А --- величины начального среднего размера особей в возрастной группе (4 градации размерных классов),\\ 
    В --- мареографический уровень положения станций наблюдения (три градации)\\
    W --- внутригрупповая вариация.\\
    $SS$ --- общий квадрат, $\nu$ --- степень свободы, $M_S$ --- средний квадрат (варианса), $F$ --- значение статистики Фишера, $\alpha$ --- уровень значимости критерия.}
\end{table}
Весьма примечательно, что при этом наибольшая доля вариации величин годового прироста определяется не начальным размером маком ($SS = 4,74$), а мареографическим уровнем положения станции ($SS = 11,98$).
При анализе структуры вариансы исходного комплекса в градиентах начального среднего размера особей в возрастной группе и географии местообитаний выяснилось, что достоверное влияние на величину среднего годового прироста маком оказывают также оба фактора (табл.~\ref{tab:prirost_ANOVA_geography}).
\begin{table}[h]
    \caption{Структура вариансы средних величин годового прироста {\it M. balthica} в возрастных группах в градиентах величины начального среднего размера особей в возрастной группе и географического положения участка наблюдений}
    \label{tab:prirost_ANOVA_geography}
    \begin{center}
    \begin{tabular}{|l|rrrrr|}
        \hline
    Источник вариации & $SS$   & $\nu$   & $M_S$   & $F$     & $\alpha$     \\ \hline
        A                 & 8,23   & 2   & 4,12 & 13,14 & 0,000003 \\
        C                 & 14,44  & 5   & 2,89 & 9,22  & 0        \\
        AC                & 14,16  & 17  & 0,83 & 2,66  & 0,000351 \\
        W                 & 156,62 & 500 & 0,31 &       &         \\ \hline
    \end{tabular}
\end{center}

    \footnotesize{Источники вариации: А --- величины начального среднего размера особей в возрастной группе (4 градации размерных классов),\\ 
        C --- географическое положение участка наблюдений (шесть градаций))\\
    W --- внутригрупповая вариация.\\
    $SS$ --- общий квадрат, $\nu$ --- степень свободы, $M_S$ --- средний квадрат (варианса), $F$ --- значение статистики Фишера, $\alpha$ --- уровень значимости критерия.}
\end{table}
Причем и в этом случае наибольшая доля вариации обусловлена не начальным размером раковины, а фактором <<участок>> ($SS = 14,44$).
Общим для проведенных вариантов двухфакторного дисперсионного анализа оказалось, что в обоих случаях внутригрупповая вариация на порядок превышает факторную составляющую. 
Это говорит о том, что основной причиной вариации величины годового прироста маком в изученных акваториях является крайняя степень разнокачественности особей в местообитаниях. 
%Такая разнокачественность очевидна уже по величине размаха вариации длины раковины маком в возрастных группах (см. табл. 1).
В качестве рабочей гипотезы можно предположить, что в краевой части ареала резкой дифференциации особей {\it M. balthica} по скорости роста могут способствовать любые проявления микрорельефной гетеротопности локальных местообитаний.
Полученные положительные итоги дисперсионного анализа интересно визуализировать для выявления характера мареографического и географического трендов в изменении величины годового прироста маком. 
Для этого представим итоги двухфакторных дисперсионных анализов в виде соответствующих поверхностей отклика.
Весьма показательно, что величины годового прироста маком по мере роста начального среднего размера особей в возрастных группах меняются куполообразно (рис.~\ref{ris:prirost_otklik}). 
	\begin{figure}[h]
		\begin{minipage}[b]{.5\linewidth}
		%Фигурка в первом ряду слева размер отведенный под весь этот объект -- 0.46 от ширины строки
		%Параметр [b] означает, что выравнивание этих министраниц будет по нижнему краю
			\begin{center}
				{\small А}
				\includegraphics[width=70mm]{../Barenc_Sea/growth_from_MSc/prirost_otklik_mareography.jpg}
			\end{center}
		\end{minipage}
	\hfil %Это пружинка отодвигающая рисунки друг от друга
		\begin{minipage}[b]{.5\linewidth}
			\begin{center}
				{\small B}
				\includegraphics[width=70mm]{../Barenc_Sea/growth_from_MSc/prirost_otklik_geography.jpg}
			\end{center}
		\end{minipage}
	\caption{Характер изменений средней величины годового прироста особей {\it Macoma balthica} возрастной группы в зависимости от начальной средней длины их раковин, мареографического уровня обитания и условного смещения участка по побережью Мурмана на восток}
\footnotesize{Примечания: Участки: 1 --- Абрам-мыс, 2 --- Пала-губа, 3 --- Гаврилово, 4 --- Ярнышная, 5 --- Дальнезеленецкая, 6 --- Шельпино, 7 --- Порчниха\\
ВГЛ --- верхний горизонт литорали, СГЛ --- средний горизонт литорали, НГЛ --- нижний горизонт литорали}
	\label{ris:prirost:otklik}
	\end{figure}
Во всех исследованных поселениях максимальный прирост наблюдается у особей размерного класса $6 - 9$~мм. 
Таким образом, в изученных поселениях максимальную скорость роста следует ожидать у маком среднего возраста (размера). 
Совершенно неожиданным для нас было явление макcимальной скорости роста маком не в нижнем, а в среднем горизонте осушной зоны (см. рис.~\ref{ris:prirost:otklik}, А). 
По-видимому, в условиях Мурмана фактор осушки начинает оказывать заметное влияние на скорость роста маком только в верхнем горизонте литорали. 
Причины снижения скорости роста маком в условиях нижнего горизонта литорали на данном этапе исследований нам не ясны.

\afterpage{\clearpage}

%%динамика популяций
% для компиляции в lualatex!!
\documentclass[12pt, a4paper]{article}
\usepackage[english,russian]{babel}
\usepackage[warn]{mathtext}
%\usepackage[T2A]{fontenc}
%\usepackage[utf8]{inputenc}

\usepackage{xecyr} % Продукт Вашего покорного слуги ;)

\setmainfont{DejaVu Serif}

\usepackage{color}
\usepackage{amssymb,amsmath}
\usepackage{graphicx}
\usepackage{multicol}

\textheight=24cm           % высота текста
\textwidth=16cm            % ширина текста
\oddsidemargin=0pt         % отступ от левого края
\topmargin=-1.5cm          % отступ от верхнего края
\parindent=24pt            % абзацный отступ
\parskip=0pt               % интервал между абзацами
\tolerance=2000            % терпимость к "жидким" строкам
\flushbottom               % выравнивание высоты страниц
%\def\baselinestretch{1.5} % печать с большим интервалом

%\title{}
%\author{\copyright~~С.А.~Назарова \thanks{e-mail:~sophia.nazarova@gmail.com}}
%\date{}

\begin{document}
\begin{figure}[h]

\begin{multicols}{2}
\hfill
\includegraphics[width=65mm]{../White_Sea/Luvenga_Goreliy/N_dynamic.pdf}
\hfill
\includegraphics[width=65mm]{../White_Sea//Luvenga_II_razrez/N_dynamic.pdf}
\end{multicols}

%\smallskip


\begin{multicols}{2}
\hfill
\includegraphics[width=65mm]{../White_Sea/Estuatiy_Luvenga/N_dynamic.pdf}
\hfill
\includegraphics[width=65mm]{../White_Sea/Ryashkov_ZRS/N_dynamic.pdf}
\end{multicols}

%\smallskip

\begin{multicols}{2}
\hfill
\includegraphics[width=65mm]{../White_Sea/Ryashkov_YuG/N_dynamic.pdf}
\hfill
\includegraphics[width=65mm]{../White_Sea/Lomnishniy/N_dynamic.pdf}
\end{multicols}

%\smallskip


\caption{Динамика плотности поселений {\it Macoma balthica} в вершине Кандалакшского залива}
\label{ris:dynamic_Kandalaksha_all}
\end{figure}


\begin{figure}[h]

\begin{multicols}{2}
\hfill
\includegraphics[width=65mm]{../White_Sea/Luvenga_Goreliy/N2_dynamic.pdf}
\hfill
\includegraphics[width=65mm]{../White_Sea//Luvenga_II_razrez/N2_dynamic.pdf}
\end{multicols}

%\smallskip


\begin{multicols}{2}
\hfill
\includegraphics[width=65mm]{../White_Sea/Estuatiy_Luvenga/N2_dynamic.pdf}
\hfill
\includegraphics[width=65mm]{../White_Sea/Ryashkov_ZRS/N2_dynamic.pdf}
\end{multicols}

%\smallskip

\begin{multicols}{2}
\hfill
\includegraphics[width=65mm]{../White_Sea/Ryashkov_YuG/N2_dynamic.pdf}
\hfill
\includegraphics[width=65mm]{../White_Sea/Lomnishniy/N2_dynamic.pdf}
\end{multicols}

%\smallskip


\caption{Динамика численности {\it Macoma balthica} с длиной раковины более $1$~мм в поселениях вершины Кандалакшского залива}
\label{ris:dynamic_Kandalaksha_all}
\end{figure}

\end{document}

%%
	\subsection{Анализ динамики численности {\it Macoma balthica} в Кандалакшском заливе Белого моря}
При изучении динамики численности можно анализировать несколько компонентов.
Первый компонент --- наличие или отсутсвие тренда как направленноого изменения численности.
При убирании тренда остается компонент динамики, для которого двумя крайими случаями будет: стабильная численность, которая поддерживается за счет плотностнозависимых процессов как систем обраной связи и неконтролируемый рост численности популяции по экспоненте.

Мы проанализировали динамику численности {\it M.~balthica} на каждом участке на наличие тренда при помощи теста Мантеля (табл.~\ref{tab:Mantel_N2_trend}).
	\begin{table}[ht]
	\caption{Выявление трендов в динамике численности {\it Macoma balthica} на различных участках Белого моря.}
	\label{tab:Mantel_N2_trend}
        \begin{tabular}{|p{0.25\textwidth}|*{2}{p{0.2\textwidth}|p{0.25\textwidth}|}} \hline
	Участок & $Mantel$ & $p$ & наличие тренда
	\\ \hline
	Эстуарий р. Лувеньга & 0,3168 & 0,003 & есть
	\\ \hline
	о. Горелый & 0,0269 & 0,368 & нет
	\\ \hline
	материковая литораль (Лувеньга) & 0,6103 & 0,001 & есть
	\\ \hline
	Южная губа о. Ряшков & 0,3687 & 0,015 & есть
	\\ \hline
	Запдная Ряшкова салма & 0,0108 & 0,404 & нет
	\\ \hline
	Ломнишный & -0,0999 & 0,47 & нет
	\\ \hline
	г. Медвежья & 0,0154 & 0,385 & нет
	\\ \hline
	г. Сельдяная & 0,2524 & 0,003 & есть
	\\ \hline
	\end{tabular}
	%    {\footnotesize Примечание: достоверность различий *** \textemdash $p<0,001$; ** \textemdash $p<0,05$; * \textemdash $p<0,1$.}
	\end{table}

Было показано наличие тренда на 4 участках: эстуарий р.~Лувеньга, материковая литораль в районе пос. Лувеньга, Южная губа о.~Ряшкова, г. Сельдяная.
Для удаления тренда из исходных значений были вычтены предсказанные значения из регрессионной модели $N = a + b*T$, где $N$ --- численность, экз./м$^2$, $T$ --- годы.
По детрендированному ряду были рассчитаны частные автокорреляции ($PRCF$ - partial rate correlation function).  
Коррелограммы представлены на рисунке \ref{ris:perm_PRCF_Kandalaksha_N2_detrend}.
	\begin{figure}[ht]
	
	\begin{minipage}[b]{.46\linewidth}
	%Фигурка в первом ряду слева размер отведенный под весь этот объект \textendash 0.46 от ширины строки
	%Параметр [b] означает, что выравнивание этих министраниц будет по нижнему краю
	\begin{center}
	{\footnotesize Эстуарий р.~Лувеньги}
		\includegraphics[width=65mm]{../White_Sea/dynamic_N_N1/perm_PRCF_Estuary_detrend.pdf}

	\end{center}
	\end{minipage}
		\hfil %Это пружинка отодвигающая рисунки друг от друга
	\begin{minipage}[b]{.46\linewidth}
%Следующий рисунок - первый ряд справа %DUNGEON S_4 \ AB
	\begin{center}
	{\footnotesize о.~Горелый}
		\includegraphics[width=65mm]{../White_Sea/dynamic_N_N1/perm_PRCF_Goreliy_all_detrend.pdf}
	\end{center}
	\end{minipage}

	\begin{minipage}[b]{.46\linewidth}
%Фигурка в первом ряду слева размер отведенный под весь этот объект \textendash 0.46 от ширины строки
%Параметр [b] означает, что выравнивание этих министраниц будет по нижнему краю
	\begin{center}
	{\footnotesize материковая литораль (Лувеньга)}
	\includegraphics[width=65mm]{../White_Sea/dynamic_N_N1/perm_PRCF_razrez2_all_detrend.pdf}
	\end{center}
	\end{minipage}
		\hfil %Это пружинка отодвигающая рисунки друг от друга
	\begin{minipage}[b]{.46\linewidth}
%Следующий рисунок - первый ряд справа %DUNGEON S_4 \ AB
	\begin{center}
	{\footnotesize о.~Ломнишный}
	\includegraphics[width=65mm]{../White_Sea/dynamic_N_N1/perm_PRCF_Lomnishniy_detrend.pdf}
	\end{center}
	\end{minipage}

	\begin{minipage}[b]{.46\linewidth}
%Фигурка в первом ряду слева размер отведенный под весь этот объект \textendash 0.46 от ширины строки
%Параметр [b] означает, что выравнивание этих министраниц будет по нижнему краю
	\begin{center}
	{\footnotesize Южная губа о.~Ряшкова}
	\includegraphics[width=65mm]{../White_Sea/dynamic_N_N1/perm_PRCF_YuG_detrend.pdf}
	\end{center}
	\end{minipage}
		\hfil %Это пружинка отодвигающая рисунки друг от друга
	\begin{minipage}[b]{.46\linewidth}
	\begin{center}	
	{\footnotesize Западная Ряшкова салма}
	\includegraphics[width=65mm]{../White_Sea/dynamic_N_N1/perm_PRCF_ZRS_detrend.pdf}
	\end{center}
	\end{minipage}
	\caption{Частные корреляции численности {\it Macoma balthica} (без учета особей длиной менее 1 мм) в Кандалакшском заливе. Детрендированные данные. Оценка достоверности пермутационным методом.}
	\label{ris:perm_PRCF_Kandalaksha_N2_detrend}	
	\end{figure}

	\begin{figure}[ht]
%\smallskip

	\begin{minipage}[b]{.46\linewidth}
%Фигурка в первом ряду слева размер отведенный под весь этот объект \textendash 0.46 от ширины строки
%Параметр [b] означает, что выравнивание этих министраниц будет по нижнему краю
	\begin{center}
	{\tiny Медвежья}
	\includegraphics[width=65mm]{../White_Sea/dynamic_N_N1/perm_PRCF_Medvezhya_detrend.pdf}
	\end{center}
	\end{minipage}
%
	\hfil %Это пружинка отодвигающая рисунки друг от друга
%
	\begin{minipage}[b]{.46\linewidth}
	\begin{center}
	{\tiny Сельдяная}
	\includegraphics[width=65mm]{../White_Sea/dynamic_N_N1/perm_PRCF_Seldyanaya_detrend.pdf}
	\end{center}
	\end{minipage}

%\smallskip
%	\caption{Динамика плотности поселений {\it Macoma balthica} в вершине Кандалакшского залива}
%	\label{ris:dynamic_Kandalaksha_all}
\begin{center}
Рисунок \ref{ris:perm_PRCF_Kandalaksha_N2_detrend}, продолжение. Частные автокорреляции численности {\it Macoma balthica} (без учета особей длиной менее 1 мм) в Кандалакшском заливе. Детреднированные данные. Оценка достоверности пермутационным методом.
\end{center}
	\end{figure}
Для большинства временных рядов значение максимального значения достигает $PRCF$ с лагом 1, что характерно для динамики в отсутствие тренда. 
Достоверность частных автокорреляций оценивалась пермутационным методом.
Для участков в Южной губе о.~Ряшкова и на материковой литорали в Лувеньге были показаны достоверные значений $PRCF[2]$, причем в Южной губе $PRCF[2] > PRCF[1]$. 
Это показывает наличие в поселении плотностнозависимых процессов второго порядка.
Предположительно, это может быть воздействие хищников.
Мы надеемся проверить эту гипотезу в ходе дальнейших наблюдений.
Биологическая интерпретация $PRCF$ с большим лагом на настоящий момент представляется нам сомнительной.

		\subsection{Синхронность динамики численности {\it Macoma balthica} в Кандалакшском заливе Белого моря}
Для изучения синхронности колебаний численности маком мы использовали тест Мантеля.
Для включения большего количества рядов в анализ, он был проведен по двум наборам данных.
Первый набор данных включал участки, где при отборе проб промывка была на сите с диаметром ячеи $0,5$~мм. 
Сюда вошли участки в эстуарии р.~Лувеньги, на материковой литорали в районе Лувеньги, на о.~Горелый, в Западной Ряшковой салме и в губах Медвежья и Сельдяная.
Результаты корреляционного анализа представлены в таблице \ref{tab:Mantel_dynamic_N}.
	\begin{table}[ht]
	\caption{Синхронность динамики численности {\it Macoma balthica}.}
	\label{tab:Mantel_dynamic_N}
        \begin{tabular}{|p{0.2\textwidth}|*{8}{p{0.08\textwidth}|}} \hline
	$Mantel r \setminus p_{perm}$ & [1] & [2] & [3] & [4] & [5] & [6] & [7] & [8]
    \\ \hline
	[1] эстуарий р.~Лувеньги & & \cellcolor{yellow}{$0,002$} & $0,989$ & \cellcolor{yellow}{$0,009$} & \cellcolor{yellow}{$0,001$} & $0,264$ & \cellcolor{yellow}{$0,018$} & $0,441$
	\\ \hline
	[2] о.~Горелый & \cellcolor{yellow}{$0,929$} & & $0,393$ & \cellcolor{yellow}{$0,014$} & \cellcolor{yellow}{$0,001$} & $0,388$ & $0,992$ & $0,089$
	\\ \hline
	[3] о.~Ломнишный & $-0,439$ & $-0,067$ & & $0,208$ & $NA$ & $0,79$ & $0,082$ & $0,399$
	\\ \hline
 	[4] г.~Медвежья & \cellcolor{yellow}{$0,821$} & \cellcolor{yellow}{$0,86$} & $-0,028$ & & \cellcolor{yellow}{$0,001$} & $0,184$ & $0,932$ & $0,441$
	\\ \hline
	[5] материковая литораль (Лувеньга) & \cellcolor{yellow}{$0,781$} & \cellcolor{yellow}{$0,784$} & $NA$ & \cellcolor{yellow}{$0,704$} & & \cellcolor{yellow}{$0,044$} & $NA$ & $0,123$
	\\ \hline
	[6] г.~Сельдяная & $0,089$ & $-0,009$ & $-0,303$ & $0,087$ & \cellcolor{yellow}{$0,364$} & & $0,763$ & $0,818$
	\\ \hline
	[7] Южная губа о.~Ряшкова & \cellcolor{yellow}{$0,427$} & $-0,309$ & $0,333$ & $-0,213$ & $NA$ & $-0,127$ & & $0,585$
	\\ \hline
	[7] Западная Ряшкова салма & $-0,045$ & $0,057$ & $0$ & $-0,05$ & $0,284$ & $-0,141$ & $-0,038$
	\\ \hline
	\end{tabular}
	   {\footnotesize Примечание: Нижняя половина таблицы --- значение теста Мантеля, верхняя половина --- уровень значимости, определенный пермутационным методом. \\
Желтым выделены значения с уровнем значимости $< 0,1$. \\
$NA$ --- ряды не пересекаются во времени.}
	\end{table}
Три участка в районе Лувеньгских шхер (эстуарий р.~Лувеньги, о.~Горелый, материковая литораль) демонстрировали синхронную динамику поселений.
С данными участками была синхронна динамика поселения маком в г.~Медвежья. 
Низкая, хотя и достоверная корреляция была показана между динамикой на материковой литорали в районе Лувеньги и в г.~Сельдяной ($0,36$) и между эстуарием р.~Лувеньги и Южной губой о.~Ряшкова ($0,43$).


Второй набор данных включал участки, где при отборе проб промывку проводили на сите с диаметром ячеи $1$~мм.
Также сюда вошли те участки из предыдущего набора данных, где была известна размерная структура моллюсков --- из общей численности были вычтены численность особей длиной менее $1$~мм для возмодности сравнения.
Всего в данный анализ вошло 8 рядов данных: эстуарий р.~Лувеньги, материковая литораль в районе Лувеньги, о.~Горелый, Западная Ряшкова салма, Южная губа о.~ Ряшкова, о.~Ломнишный, б.~Клющиха и Сухая салма (табл.~\ref{tab:Mantel_dynamic_N2}).
	\begin{table}[ht]
	\caption{Синхронность динамики численности {\it Macoma balthica}.}
	\label{tab:Mantel_dynamic_N2}
        \begin{tabular}{|p{0.2\textwidth}|*{8}{p{0.08\textwidth}|}} \hline
	$Mantel r \setminus p_{perm}$ & [1] & [2] & [3] & [4] & [5] & [6] & [7] & [8]
	\\ \hline
	[1] эстуарий р.~Лувеньги & & $0,082$ & $0,646$ & $0,995$ & \cellcolor{yellow}{$0,029$} & $0,482$ & \cellcolor{yellow}{$0,013$} & $0,19$
	\\ \hline
	[2] о.~Горелый & $0,176$ &  & $0,067$ & $0,73$ & \cellcolor{yellow}{$0,001$} & $0,261$ & $0,986$ & \cellcolor{yellow}{$0,001$}
	\\ \hline
	[3] б.~Клющиха & $-0,046$ & $0,52$ &  & $0,673$ & \cellcolor{yellow}{$0,034$} & $0,213$ & $0,062$ & $0,065$
	\\ \hline
	[4] о.~Ломнишный & $-0,451$ & $-0,181$ & $-0,22$ &  & $NA$ & $1$ & $0,088$ & $0,341$
	\\ \hline
	[5] материковая литораль (Лувеньга) & \cellcolor{yellow}{$0,32$} & \cellcolor{yellow}{$0,862$} & \cellcolor{yellow}{$0,577$} & $NA$ &  & $0,117$ & $NA$ & \cellcolor{yellow}{$0,006$}
	\\ \hline
	[6]Сухая салма & $-0,019$ & $0,067$ & $0,085$ & $-1$ & $0,443$ &  & $0,688$ & $0,314$
	\\ \hline
	[7] Южная губа о.~ Ряшкова & \cellcolor{yellow}{$0,419$} & $-0,332$ & $0,434$ & $0,333$ & $NA$ & $-0,243$ &  & $0,605$
	\\ \hline
	[8] Западная Ряшкова салма & $0,114$ & \cellcolor{yellow}{$0,86$} & $0,72$ & $0,093$ & \cellcolor{yellow}{$0,755$} & $0,088$ & $-0,048$ & 
	\\ \hline
	\end{tabular}
	   {\footnotesize Примечание: Нижняя половина таблицы --- значение теста Мантеля, верхняя половина --- уровень значимости, определенный пермутационным методом. \\
Желтым выделены значения с уровнем значимости $< 0,05$. \\
$NA$ --- ряды не пересекаются во времени.}
	\end{table}
Интересно отметить, что при редукции данных до численности особей длиной более $1$~мм картина меняется.
Без изменения остается синхронность динамик поселений маком на материковой литорали в Лувеньге c о.~Горелый и эстуарием р.~Лувеньги.
Такжесохранияется синхронность динамик численности в поселениях в эстуарии р.~Лувеньга и Южной губе о.~Ряшкова.
В то же время поселение в Западной Ряшковой салме, который в предыдущем анализе показывало асинхронность по сравнению с остальными участками, в данном случае демонстрирует синхронность с поселениями на о.~Горелый и материковой литорали в Лувеньге.
Также показана синхронность динамик поселений на материковой литорали в Лувеньге и в бухте Клющиха.

Мы использовали значение теста Мантеля как меру сходства рядов данных для тестирования гипотезы, что на более близкорасположенных участках динамика численности {\it Macoma balthica} более сходна.
Для этого по координатам участков была рассчитана матрица расстояний между участками (табл.~\ref{tab:distance_area_km}).
	\begin{table}[ht]
	\caption{Расстояние между исследованными участками литорали.}
	\label{tab:distance_area_km}
        \begin{tabular}{|p{0.3\textwidth}|*{10}{p{0.04\textwidth}|}} \hline
	 & [1] & [2] & [3] & [4] & [5] & [6] & [7] & [8] & [9] & [10]
	\\ \hline
	[1] материковая литораль (Лувеньга) & 0,0 &  &  &  &  &  &  &  &  & 
	\\ \cline{1-3}
	[2] о.~Горелый & 1,5 & 0,0 &  &  &  &  &  &  &  &  
	\\ \cline{1-4}
	[3]эстуарий р.~Лувеньги & 1,0 & 1,0 & 0,0 &  &  &  &  &  &  &  
	\\ \cline{1-5}
	[4] Южная губа о.~Ряшкова & 11,7 & 10,7 & 11,7 & 0,0 &  &  &  &  &  & 
	\\ \cline{1-6}
	[5] о.~Ломнишный & 13,5 & 12,9 & 13,8 & 3,7 & 0,0 &  &  &  &  &  
	\\ \cline{1-7}
	[6] Западная Ряшкова салма & 11,9 & 10,8 & 11,8 & 1,7 & 5,3 & 0,0 &  &  &  &  
	\\ \cline{1-8}
	[7] г.~Сельдяная & 93,6 & 94,0 & 94,5 & 87,8 & 84,1 & 89,3 & 0,0 &  &  &  
	\\ \cline{1-9}
	[8] г.~Медвежья & 91,9 & 92,4 & 92,8 & 86,1 & 82,4 & 87,6 & 1,7 & 0,0 &  &  
	\\ \cline{1-10}
	[9] Сухая салма & 97,1 & 97,5 & 97,9 & 91,2 & 87,6 & 92,7 & 3,5 & 5,1 & 0,0 &  
	\\ \hline
	[10] б.~Клющиха & 100,1 & 100,6 & 101,0 & 94,8 & 91,1 & 96,3 & 8,1 & 9,7 & 5,8 & 0,0
	\\ \hline
	\end{tabular}
	   {\footnotesize Примечание:Расстояние дано в километрах.}
	\end{table}

Для обоих наборов данных тест Мантеля показал отсутсвие зависимости сходства динамики численности маком от расстояния ( $Mantel r = --0,058 (p_{perm} = 0,746)$ и $Mantel r = -0,105 (p_{perm} = 0,638)$ для первого и второго набора данных, соответственно).


 %вставили его в dymanic.tex
\afterpage{\clearpage}

%пополнение
	\chapter{Количественные характеристики формирования спата в поселениях {\it Macoma balthica}  на литорали губы Чупа (Белое море)}
Для получения прямой информации о формировании спата в $2006$~году были проведены ограниченные наблюдения за поселениями в губе Чупа.
Было обследовано 2 участка на о.~Кереть: в Сухой салме и в бухте Клющиха, и 2 материковых участка: в бухте Лисья и в проливе Подпахта.

Обилие {\it Macoma balthica} на исследованных участках варьировало в значительных пределах. 
Так, плотность поселений на разных участках составляла от $228$ до $1230$~экз./м$^2$, а биомасса от $1,1$ до $6,2$~г/м$^2$ (табл.~\ref{tab:NMacoma_recruitment}). 
\begin{table}[p]
\caption{Характеристики обилия взрослых {\it Macoma balthica} и спата на участках в губе Чупа в 2006 году}
\label{tab:NMacoma_recruitment}
\begin{center}
\begin{tabular}{|l|cc|c|}
\hline
Участок         & $N_{ad}$  & $B_{ad}$   & $N_{juv}$ \\ \hline
Сухая салма     & 1230 (17) & 6,2 (19) & 4980 (13)  \\  
Бухта Лисья     & 1200 (17) & 1,9 (18) & 4040 (21)  \\ 
бухта Клющиха   & 476 (19)  & 1,1 (24) & 4240 (10)  \\  
пролив Подпахта & 228 (30)  & 1,8 (64) & 10060 (15) \\ \hline
\end{tabular}
\end{center}

\footnotesize{Примечание: $N_{ad}$ --- средняя плотность поселения взрослых маком в поселении,~экз./м$^2$; 
$B_{ad}$ --- средняя биомасса взрослых маком в поселении,~г/м$^2$; 
$N_{juv}$ --- средняя плотность поселения спата маком в поселении,~экз./м$^2$. 
В скобках приведена точность учета $d$ в процентах.}
\end{table}

Плотность поселения взрослых особей {\it M.~balthica} на участке в Сухой салме составляла $1230 \pm 207$~экз./м$^2$, а биомасса "--- $6,2 \pm 1,17$~г/м$^2$. 
На участке были представлены моллюски с раковиной длиной от $1,1$ до $15,7$~мм. 
Размерная структура в Сухой салме характеризовалась бимодальностью с модальными классами $1,1 - 2,0$~мм и $6,1 - 8,0$~мм (рис.~\ref{ris:Chupa_spat_sizestr}). 
	\begin{figure}[p]
	\begin{minipage}[b]{.46\linewidth}
	\begin{center}
	Взрослые особи
	\end{center}
	\end{minipage}
	%
	\hfil %Это пружинка отодвигающая рисунки друг от друга
	\begin{minipage}[b]{.46\linewidth}
	\begin{center}
	Спат (генерация 2006 года)
	\end{center}
	\end{minipage}
%
	\begin{minipage}[b]{\linewidth}
	\begin{center}
		Сухая салма
	\end{center}
	\end{minipage}
%
	\begin{minipage}[b]{.46\linewidth}
	%Фигурка в первом ряду слева размер отведенный под весь этот объект \textendash 0.46 от ширины строки
	%Параметр [b] означает, что выравнивание этих министраниц будет по нижнему краю
	\begin{center}
		\includegraphics[width=0.21\textheight]{../White_Sea/spat/adult_str_Suhaya_1.pdf}
	\end{center}
	\end{minipage}
	%
	\hfil %Это пружинка отодвигающая рисунки друг от друга
	\begin{minipage}[b]{.46\linewidth}
%Следующий рисунок - первый ряд справа %DUNGEON S_4 \ AB
	\begin{center}
		\includegraphics[width=0.21\textheight]{../White_Sea/spat/spat_str_Suhaya_1.pdf}
	\end{center}
	\end{minipage}
%\smallskip
%
	\begin{minipage}[b]{\linewidth}
	\begin{center}
		бухта Лисья
	\end{center}
	\end{minipage}
%
	\begin{minipage}[b]{.46\linewidth}
%Фигурка в первом ряду слева размер отведенный под весь этот объект \textendash 0.46 от ширины строки
%Параметр [b] означает, что выравнивание этих министраниц будет по нижнему краю
	\begin{center}
		\includegraphics[width=0.215\textheight]{../White_Sea/spat/adult_str_Lisya_1.pdf}
	\end{center}
	\end{minipage}
%
	\hfil %Это пружинка отодвигающая рисунки друг от друга
%
	\begin{minipage}[b]{.46\linewidth}
%Следующий рисунок - первый ряд справа %DUNGEON S_4 \ AB
	\begin{center}	
		\includegraphics[width=0.215\textheight]{../White_Sea/spat/spat_str_Lisya_1.pdf}
	\end{center}
	\end{minipage}
%\smallskip
%
	\begin{minipage}[b]{\linewidth}
	\begin{center}
		бухта Клющиха
	\end{center}
	\end{minipage}
%
	\begin{minipage}[b]{.49\linewidth}
%Фигурка в первом ряду слева размер отведенный под весь этот объект \textendash 0.46 от ширины строки
%Параметр [b] означает, что выравнивание этих министраниц будет по нижнему краю
	\begin{center}
		\includegraphics[width=0.215\textheight]{../White_Sea/spat/adult_str_Klushiha_1.pdf}
	\end{center}
	\end{minipage}
%
	\hfil %Это пружинка отодвигающая рисунки друг от друга
%
	\begin{minipage}[b]{.49\linewidth}
%Следующий рисунок - первый ряд справа %DUNGEON S_4 \ AB
	\begin{center}	
		\includegraphics[width=0.215\textheight]{../White_Sea/spat/spat_str_Klushuha_1.pdf}
	\end{center}
	\end{minipage}
%\smallskip
%
	\begin{minipage}[b]{\linewidth}
	\begin{center}
		пролив Подпахта
	\end{center}
	\end{minipage}
%
	\begin{minipage}[b]{.49\linewidth}
%Фигурка в первом ряду слева размер отведенный под весь этот объект \textendash 0.46 от ширины строки
%Параметр [b] означает, что выравнивание этих министраниц будет по нижнему краю
	\begin{center}
		\includegraphics[width=0.215\textheight]{../White_Sea/spat/adult_str_Podpahta_1.pdf}
	\end{center}
	\end{minipage}
%
	\hfil %Это пружинка отодвигающая рисунки друг от друга
%
	\begin{minipage}[b]{.49\linewidth}
%Следующий рисунок - первый ряд справа %DUNGEON S_4 \ AB
	\begin{center}	
		\includegraphics[width=0.215\textheight]{../White_Sea/spat/spat_str_Podpahta_1.pdf}
	\end{center}
	\end{minipage}
		\caption{Размерная структура поселений {\it Macoma balthica} на участках в губе Чупа в 2006 году и спата, осевшего в данных поселениях}
		\label{ris:Chupa_spat_sizestr}
\footnotesize{Примечание: по оси абсцисс~--- длина раковины, мм; по оси ординат~--- плотность поселения, экз./м$^2$. Планки погрешностей соответствуют ошибкам средних}
	\end{figure}
Плотность поселения спата составляла $4980 \pm 618$~экз./м$^2$. 
Размерная структура спата на данном участке была мономодальная с максимумом при длине раковины $0,6$~мм (рис.~\ref{ris:Chupa_spat_sizestr}).

Плотность поселения взрослых моллюсков в Лисьей бухте составляла $1200 \pm 199$~экз./м$^2$, а биомасса "--- $1,9 \pm 0,76$~г/м$^2$. 
На участке были представлены моллюски с раковиной длиной от $1,0$ до $14,3$ мм. 
Размерная структура в Лисьей бухте характеризовалась бимодальностью с модальными классами $1,1 - 3,0$~мм и $8,1 - 10,0$~мм (рис.~\ref{ris:Chupa_spat_sizestr}). 
Плотность поселения спата составляла $4040 \pm 832$~экз./м$^2$ (рис. 5). 
Размерная структура спата на данном участке была мономодальная с максимумом при длине раковины $0,5$~мм (рис.~\ref{ris:Chupa_spat_sizestr}).

Плотность поселения взрослых маком на участке в бухте Клющиха составляла $476 \pm 291$~экз./м$^2$, а биомасса "--- $1,1 \pm 0,27$~г/м$^2$. 
На участке были представлены моллюски с раковиной длиной от $1,3$ до $11,5$~мм. 
Размерная структура в бухте Клющиха характеризовалась бимодальностью с модальными классами $1,1 - 2,0$~мм и $6,1 - 8,0$~мм (рис.~\ref{ris:Chupa_spat_sizestr}). 
Плотность поселения спата составляла $4240 \pm 441$~экз./м$^2$. 
Размерная структура спата на данном участке была мономодальная с максимумом при длине раковины $0,75$~мм (рис.~\ref{ris:Chupa_spat_sizestr}).

Плотность поселения {\it M.~balthica} в проливе Подпахта составляла $228 \pm 69$~экз./м$^2$, а биомасса "--- $1,9 \pm 1,21$~г/м$^2$. 
На участке были представлены моллюски с раковиной длиной от $1,1$ до $13,5$~мм. 
Размерная структура на участке в проливе Подпахта характеризовалась бимодальностью с модальными классами $1,1 - 2,0$~мм и $9,1 - 10,0$~мм (рис.~\ref{ris:Chupa_spat_sizestr}). 
Плотность поселения спата составляла $10060 \pm 1493$~экз./м$^2$. 
Размерная структура спата на данном участке была мономодальная с максимумом при длине раковины $0,5$~мм (рис.~\ref{ris:Chupa_spat_sizestr}).

Для выявления связи плотности поселения спата с обилием (плотностью поселения и биомассой) взрослых маком был рассчитан ранговый коэффициент корреляции Спирмена (табл.~\ref{spat_abult_correlation}). 
\begin{table}[p]
\caption{Корреляция плотности поселения спата \textit{M.~balthica} с  обилием взрослых маком в поселениях}
\label{spat_abult_correlation}
\begin{center}
\begin{tabular}{|l|lll|}
\hline
     & $r_S$    & $t_{N-2}$   & $p$    \\ \hline
$N_{ad}$ & -0,46 & -2,209 & 0,04 \\
$B_{ad}$ & -0,05 & -0,214 & 0,83\\ \hline
\end{tabular}
\end{center}

\footnotesize{Примечание: $N_{ad}$ --- средняя плотность поселения взрослых маком в поселении; 
$B_{ad}$ --- средняя биомасса взрослых маком в поселении; 
$r_S$ --- значение рангового коэффициента корреляции Спирмена; 
$t_{N-2}$ --- критерий Стьюдента;   
$p$ --- уровень значимости нулевой гипотезы.}
\end{table}
До\-сто\-вер\-ная корреляция ($r_S = -0,46$) была показана между плотностью поселения спата и средней плотностью поселения взрослых маком в поселении, в то время как корреляция количества спата со средней биомассой взрослых особей оказалась недостоверной.

Также был рассчитан ранговый коэффициент корреляции Спирмена для обилия спата и средней плотности поселения отдельных размерных групп взрослых маком. 
Для этого были выделены размерные группы с шагом $3$~мм (рис.~\ref{ris:spearman_size},~А).
	\begin{figure}[p]
	\begin{minipage}[b]{.46\linewidth}
	\begin{center}
	А
	\end{center}
	\end{minipage}
	%
	\hfil %Это пружинка отодвигающая рисунки друг от друга
	\begin{minipage}[b]{.46\linewidth}
	\begin{center}
	Б
	\end{center}
	\end{minipage}
	\begin{minipage}[b]{.46\linewidth}
%Фигурка в первом ряду слева размер отведенный под весь этот объект \textendash 0.46 от ширины строки
%Параметр [b] означает, что выравнивание этих министраниц будет по нижнему краю
	\begin{center}
		\includegraphics[width=\textwidth]{../White_Sea/spat/spearman_spat_3mm_1.pdf}
	\end{center}
	\end{minipage}
%
	\hfil %Это пружинка отодвигающая рисунки друг от друга
	\begin{minipage}[b]{.46\linewidth}
%Фигурка в первом ряду слева размер отведенный под весь этот объект \textendash 0.46 от ширины строки
%Параметр [b] означает, что выравнивание этих министраниц будет по нижнему краю
	\begin{center}
		\includegraphics[width=\textwidth]{../White_Sea/spat/spearman_spat_2mm_1.pdf}
	\end{center}
	\end{minipage}
	\caption{Изменение силы и характера корреляции плотности поселений спата с плотностью поселений взрослых особей, с учетом размерной характеристики последних}
	\label{ris:spearman_size}
	
	\footnotesize{Примечание: $r_S$ – значение рангового коэффициента корелляции Спирмена;
 $L_{ad}$ – длина взрослых особей,~мм. \\
Зеленые точки --- достоверные коэффициенты при $p \le 0,05$}
	\end{figure}
Достоверный отрицательный коэффициент корреляции ($-0,46 - -0,57$) был показан для маком длиной до $12$~мм, при этом максимальная корреляция ($-0,57$) достигалась дважды: для групп $1-3$~мм и $9-12$~мм. 
Достоверная положительная корреляция ($r_S=0,55$) была показана между обилием спата маком и плотностью поселения взрослых особей длиной $12-15$~мм.

Однако при расчете аналогичного показателя при разделении взрослых особей на классы с шагом $2$~мм, если первая группа (особи длиной менее $12$~мм) также достоверно коррелирует с плотностью поселения спата, то группа $12-14$~мм, хотя и положительно коррелирует, но эта связь уже не достоверна (рис.~\ref{ris:spearman_size},~Б).

Поскольку объем выборки небольшой, то мощность корреляционного анализа невелика. 
Поэтому для оценки влияния плотности поселения взрослых маком на размеры пополнения был проведен дисперсионный анализ и оценена сила влияния факторов (табл.~\ref{tab:ANOVA_site_Nad_spat}).
Поскольку невозможно изолировать влияние условий на локальном участке и анализировать влияние только плотности поселения взрослых маком, то была выбрана иерархическая схема дисперсионного анализа, в которой фактор <<участок>> был вложен в фактор <<плотность поселения крупных маком>>.
\begin{table}[p]
\caption{Анализ структуры вариансы (иерархический дисперсионный анализ) показателей плотности поселения спата маком в градиентах плотности взрослых маком в поселениях и местоположения участка}
\label{tab:ANOVA_site_Nad_spat}
\begin{center}
\begin{tabular}{|l|lll|ll|ll|ll|}
\hline
                & $SS$        & $df$ & $MS$       & $F$      & $p$        & $\nu^2$      & $m_{\nu^2}$       & $F_{\nu^2}$            & $F_{cr}$  \\ \hline
  site($N_{ad}$) & 86890000  & 2  & 43445000 & 9,9326 & 0,0016 & 0,45 & 0,068 & 6,63 & 3,63 \\
$N_{ad}$         & 34848000  & 1  & 34848000 & 7,9671 & 0,0123 & 0,18 & 0,051 & 3,55 & 4,49 \\
error       & 69984000  & 16 & 4374000  &        &          &              &              &              &      \\ \hline
\end{tabular}
\end{center}

\footnotesize{Примечание: Источник вариации: $N_{ad}$ --- фактор <<плотность поселения взрослых особей>>, 
site ($N_{ad}$) --- фактор <<участок>> (вложен в фактор $N_{ad}$),
error ---  внутригрупповая вариация. \\
$SS$ --- девиата, 
$df$ --- число степеней свободы, 
$MS$ --- варианса, 
$F$ --- значение критерия Фишера, 
$p$  --- уровень значимости,
$\nu^2$ --- сила влияния фактора,
$m_{\nu^2}$ --- ошибка силы влияния,
$F_{\nu^2}$ – значение критерия Фишера для силы влияния.}
\end{table}
По результатам дисперсионного анализа как плотность поселения взрослых особей, так и уникальный набор условий каждого участка достоверно влияют на количество маком, осевших в поселении, причем вариабельность от участка к участку выше, чем вариабельность, обусловленная высокой или низкой плотностью поселения взрослых особей в поселении. 
%Однако достоверно оценить силу влияния возможно только для фактора <<участок>>.

Также исследованные участки отличались по суммарному обилию макрозообентоса (табл.~\ref{tab:NB_fauna_spat}). 
	\begin{table}[p]
	\caption{Характеристики общего обилия макрозообентоса  на участках в губе Чупа в 2006 году}
	\label{tab:NB_fauna_spat}
	\begin{center}
		\begin{tabular}{|l|c|c|}
		\hline
		                & $N_f$, экз./м$^2$ (d, \%) & $B_f$ г/м$^2$ (d, \%)      \\ \hline 
		Сухая салма     & 9381 (12,7) & 141,7 (12,3) \\
		Лисья губа      & 42544 (11,2) & 151,3  (11,3) \\
		бухта Клющиха   & 1344 (19,1) & 37,8 (34,2) \\
		пролив Подпахта & 7169 (28,4) & 46,6 (19,4) \\ \hline
		\end{tabular}
	\end{center}

\footnotesize{Примечание: $N_f$ --- средняя суммарная плотность поселений макробентоса,~экз./м$^2$; $B_f$ --- средняя суммарная биомасса макробентоса в поселении,~г/м$^2$. В скобках приведена точность учета (в \%)}
	\end{table}
Наименьшее обилие макрозообентоса было отмечено на участке в бухте Клющиха ($N = 1344 \pm 256,2$~экз./м$^2$; $B = 37,8 \pm 12,9$~г/м$^2$). 
Б\'{o}льшие плотности поселения были отмечены в Сухой Салме ($N = 9381 \pm 2678$~экз./м$^2$) и проливе Подпахта ($N = 7169 \pm 4545$~экз./м$^2$), но различия между этими участками недостоверное. 
Однако по биомассе макрозообентоса участок в Сухой Салме на порядок отличается от пролива Подпахта ($B = 147,1 \pm 17,3$~г/м$^2$ и $46,6 \pm 9,0$~г/м$^2$, соответственно). 
Максимальное обилие макробентоса отмечено на участке в бухте Лисьей, где плотность поселения ($42544 \pm 4753,4$) достоверно отличается от всех других участков, а биомасса достоверно больше, чем в проливе Подпахта и бухте Клющиха, но не отличается от аналогичного показателя в Сухой Салме.

Для выявления связи плотности поселения и биомассы макрозообентоса с плотностью поселения спата {\it M.~balthica} был рассчитан ранговый коэффициент корреляции Спирмена (табл.~\ref{spat_fauna_correlation}). 
\begin{table}[p]
\caption{Корреляция плотности поселения спата M. balthica с обилием макробентоса в поселениях}
\label{spat_fauna_correlation}
\begin{center}
\begin{tabular}{|l|lll|}
\hline
     & $r_S$    & $t_{N-2}$   & $p$    \\ \hline
$N_{fauna}$  & -0,16 & -0,68 & 0,50 \\
$B_{fauna}$  & -0,16 & -0,68 & 0,50\\
\hline
\end{tabular}
\end{center}

\footnotesize{Примечание: $N_{fauna}$ --- средняя плотность поселения взрослых маком в поселении; 
$B_{fauna}$ --- средняя биомасса взрослых маком в поселении; 
$r_S$ --- значение рангового коэффициента корреляции Спирмена; 
$t_{N-2}$ --- критерий Стьюдента;   
$p$ --- уровень значимости нулевой гипотезы.}
\end{table}
Достоверной корреляции между плотностью поселения спата макомы с суммарными плотностью поселения и биомассой макрозообентоса обнаружено не было.


\bigskip
Таким образом, оседание спата широко варьирует в пределах локальных акваторий.
Причем уникальное сочетание условий, характерных для каждого поселения, то есть локальный участок, оказывает значительное влияние на плотность поселения спата.
В то же время удалось показать влияние плотности поселения крупных маком на величину оседания.


\afterpage{\clearpage}

		\chapter{Динамика пополнения поселений {\it Macoma balthica} в Белом море}
При изучении динамики поселений бентосных организмов с планктонной личинкой важную роль играет пополнение поселений молодью. 
Оседание {\it M.~balthica} в Белом море происходит с июля по сентябрь (\cite{Semenova_1980, Maximovich_1985}), поэтому данные, собранные в июле, не описывают величину оседания в текущем году. 
Однако мы можем оценить пополнение предыдущего года по обилию особей возрастом 1+. 
Для Северного моря показано, что в пополнении поселений молодью выживаемость спата в первую зиму не менее важна, чем непосредственно количество осевших особей (\cite{Beukema_et_al_1998, Strasser_Gunter_2001}). Подобные данные известны и для Белого моря (\cite{Maximovich_Gerasimova_2004}). Поэтому, на наш взгляд, оценка пополнения поселения как численности особей, переживших первую зиму, более информативна.




	\paragraph{Размер моллюсков {\it M.~balthica} в возрасте 1 года}

Поскольку в мониторинговых исследованиях в вершине Кандалакшского залива фиксировалась только длина раковины без определения возраста, то в $2012 - 2013$ году были проведены  измерения длин колец зимней остановки роста у особей длиной менее $3$~мм (рис. \ref{ris:vozrast_menee_3mm}, A). 
Данные получены для участков на о.~Горелый, в эстуарии р.~Лувеньги и в Западной Ряшковой салме. 
Распределение измереных особей по возрастам представлено на рис. \ref{ris:vozrast_menee_3mm}, B.
	\begin{figure}[p]
		\includegraphics{../White_Sea/growth_young/hist_obili_po_godam1.pdf}
	\caption{Распределение моллюсков {\it M.~balthica} длиной менее $3$~см по размеру (А) и возрасту (В)}
	\label{ris:vozrast_menee_3mm}
	{\footnotesize Примечание: N, экз. \textemdash количество особей, L, мм \textemdash длина раковины}
	\end{figure}

Особи возрастом 1+ с различных горизонтов литорали острова Горелый не различаются по размеру ($Kruskal-Wallis\ \chi^2 = 3,12, p = 0,37$), поэтому в дальнейшем мы рассматриваем их как одну выборку (рис. \ref{ris:Goreliy_length1+_gorizonty}).
	\begin{figure}[p]
		\includegraphics{../White_Sea/growth_young/boxplot_Goreliy_length_1+_tidal.pdf}
	\caption{Размеры  годовалых моллюсков {\it M.~balthica} на разных горизонтах литорали о. Горелый}
	\label{ris:Goreliy_length1+_gorizonty}
	{\footnotesize Примечание: L, мм \textemdash длина раковины. <<Ящик>> на графике соответствует 1 и 3 квартилю, жирная горизонтальная линия \textemdash 		медиана, <<усы>> \textemdash $1,5$ межквартильных размаха}
	\end{figure}

По результатам теста Краскел-Уоллиса годовалые моллюски с разных участков различались по длине ($Kruskal-Wallis\ \chi^2 = 17,6, p = 0,00015$) (рис.~\ref{ris:length_1+_uchastki}, поэтому было проведено попарное сравнение участков (табл.~\ref{tab:Tukey_1+_uchastki}). 
Размер годовалых особей не различался на участках, расположенных в районе Лувеньгиских шхер (о.~Горелый и эстуарий р.~Лувеньги), и отличался от особей из Западной Ряшковой салмы.
	\begin{figure}[p]
		\includegraphics{../White_Sea/growth_young/boxplot_length_1age_area1.pdf}
	\caption{Размеры  годовалых моллюсков {\it M.~balthica} на разных участках литорали}
	\label{ris:length_1+_uchastki}
	{\footnotesize Примечание: L, мм \textemdash длина раковины. <<Ящик>> на графике соответствует 1 и 3 квартилю, жирная горизонтальная линия \textemdash 		медиана, <<усы>> \textemdash $1,5$ межквартильных размаха}
	\end{figure}
	
	\begin{table}[p]
	\caption{Результаты множественного сравнения длины годовалых {\it Macoma balthica} на различных участках методом Тьюки (Tukey's ‘Honest Significant Difference’).}
	\label{tab:Tukey_1+_uchastki}
	\begin{tabular}{|*{4}{p{0.2\textwidth}|}} \hline
	участки & различия средних & p-value & достоверность различий\\
	\hline
	о.~Горелый \textemdash\ эстуарий р.~Лувеньги & $0,053$ & $0,2$ & \\
	\hline
	о.~Горелый \textemdash\ Западная Ряшкова салма & $0,11$ & $0,005$ & ** \\
	\hline
	эстуарий р.~Лувеньги \textemdash\ Западная Ряшкова салма & $0,17$ & $0.00002$ & ***\\
	\hline
	\end{tabular}
	
	{\footnotesize Примечание: достоверность различий *** \textemdash $p<0,001$; ** \textemdash $p<0,05$; * \textemdash $p<0,1$.}
	\end{table}

Для определения границ размерно-возрастных классов {\it Macoma balthica} возрастом $0+$, $1+$ и $2+$ были рассчитаны средние размеры особей каждого возраста (табл~\ref{tab:mean_length_ages}).
	\begin{table}[p]	
\caption{Средний размер {\it Macoma balthica}в возрасте до 2 лет на различных участках.}
	\label{tab:mean_length_ages}
	\begin{tabular}{|l|*{3}{p{0.2\textwidth}|}} \hline
	возраст & $0+$ & $1+$ & $2+$\\
	\hline
	о.~Горелый & $1,0 \pm 0,001$ & $1,4 \pm 0,002$ & $2,2 \pm 0,008$ \\ 
	\hline
	эстуарий р.~Лувеньги & $1,0 \pm 0,004$ & $1,4 \pm 0,002$ & $2,2 \pm 0,02$ \\
	\hline
	Западная Ряшкова салма & $1,1 \pm 0,04$ & $1,5 \pm 0,003$ & $2,3 \pm 0,02$ \\ 
	\hline
	\end{tabular}
	
	{\footnotesize Примечание: В ячейках указано среднее арифметическое с ошибкой.}
	\end{table}
Пограничный размер между двумя когортами рассчитывали как середину между средними размерами особей в когорте. 
Таким образом, в дальнейшем для участков, расположенных в акватории Лувеньгских шхер, маком длиной менее $1,2$~мм рассматривали как спат, а длиной от $1,2$ до $1,8$~мм \textemdash\ как особей возрастом 1+.
Для участков на о.~Ряшков пограничные значения составили $1,3$ и $1,9$,~мм соответственно.
Для участка на о.Ломнишном мы использовали данные, полученные для о.~Ряшкова, как географически наиболее близкой акватории.

\afterpage{\clearpage}




\par\bigskip

Таким образом были получены данные по динамике плотность поселения однолетних маком на 6 мониторинговых участках (рис.~\ref{ris:dynamic_1year_Kandalaksha}).
	\begin{figure}[p]
	
	\begin{minipage}[b]{.49\linewidth}
	%Фигурка в первом ряду слева размер отведенный под весь этот объект \textendash 0.46 от ширины строки
	%Параметр [b] означает, что выравнивание этих министраниц будет по нижнему краю
	\begin{center}
	\includegraphics[width=\linewidth]{../White_Sea/Estuatiy_Luvenga/Estuary_N_oneyear1.pdf}

	\end{center}
	\end{minipage}
	%
	\hfil %Это пружинка отодвигающая рисунки друг от друга
	%
	\begin{minipage}[b]{.49\linewidth}
%Следующий рисунок - первый ряд справа %DUNGEON S_4 \ AB
	\begin{center}
		\includegraphics[width=\linewidth]{../White_Sea/Luvenga_Goreliy/Goreliy_N_oneyear1.pdf}
	\end{center}
	\end{minipage}

%\smallskip


	\begin{minipage}[b]{.49\linewidth}
%Фигурка в первом ряду слева размер отведенный под весь этот объект \textendash 0.46 от ширины строки
%Параметр [b] означает, что выравнивание этих министраниц будет по нижнему краю
	\begin{center}
		\includegraphics[width=\linewidth]{../White_Sea/Luvenga_II_razrez/2razrez_N_oneyear1.pdf}
	\end{center}
	\end{minipage}
%
	\hfil %Это пружинка отодвигающая рисунки друг от друга
%
	\begin{minipage}[b]{.49\linewidth}
%Следующий рисунок - первый ряд справа %DUNGEON S_4 \ AB
	\begin{center}
		\includegraphics[width=\linewidth]{../White_Sea/Ryashkov_ZRS/ZRS_N_oneyear1.pdf}
	\end{center}
	\end{minipage}

%\smallskip

	\begin{minipage}[b]{.49\linewidth}
%Фигурка в первом ряду слева размер отведенный под весь этот объект \textendash 0.46 от ширины строки
%Параметр [b] означает, что выравнивание этих министраниц будет по нижнему краю
	\begin{center}
		\includegraphics[width=\linewidth]{../White_Sea/Ryashkov_YuG/YuG_N_oneyear1.pdf}
	\end{center}
	\end{minipage}
%
	\hfil %Это пружинка отодвигающая рисунки друг от друга
%
	\begin{minipage}[b]{.49\linewidth}
%Следующий рисунок - первый ряд справа %DUNGEON S_4 \ AB
	\begin{center}
		\includegraphics[width=\linewidth]{../White_Sea/Lomnishniy/Lomnishniy_N_oneyear1.pdf}
	\end{center}
	\end{minipage}

%\smallskip
	\caption{Динамика плотность поселения однолетних особей {\it Macoma balthica} в вершине Кандалакшского залива}
	\label{ris:dynamic_1year_Kandalaksha}
	\end{figure}
Плотность поселения однолетних особей значительно варьирует год от года. 
В некоторые годы макомы возрастом $1+$ отсутствуют в поселениях.
Максимальные плотности поселений однолетних особей варьировали между участками от $600$ на Ломнишном до $5500$~экз./м$^2$ на верхнем горизонте материковой литорали в Лувеньге.

Важно отметить, что именно флуктуации плотности поселения однолетних особей во-многом определяют изменения общего обилия маком (рис.~\ref{ris:N1year_vs_Nall}).
    \begin{figure}[p]
        \includegraphics[width=\textwidth]{../White_Sea/oneyear_all_Kandalaksha_all/N1y_vs_N2_1.pdf}
    \caption{Соотношение общей плотности поселения {\it Macoma balthica} и плотности поселения особей возрастом 1+}
    \label{ris:N1year_vs_Nall}

	\footnotesize{Примечание: N1+~--- плотность поселения маком возрастом 1 год, экз./м$^2$. N --- общая плотность поселения маком, экз./м$^2$.}
    \end{figure}
Корреляция между данными показателями показывает сильную связь ($Spearman\ \rho = 0,83, p < 0,0001$).

Для проверки связи размера пополнения с плотностью поселения половозрелых особей в поселении мы использовали плотность поселения маком крупнее $8$~мм, поскольку в Белом море показано (\cite{Semenova_1980, Maximovich_1985}), что ключевым фактором для возможности половозрелости является именно размер (рис.~\ref{ris:N1year_vs_N8mm}).
    \begin{figure}[p]
        \includegraphics[width=\textwidth]{../White_Sea/oneyear_all_Kandalaksha_all/N8mm_vs_N1y_1.pdf}
    \caption{Связь плотности поселения однолетних особей {\it Macoma balthica} и количества половозрелых особей в год их оседания}
    \label{ris:N1year_vs_N8mm}

\footnotesize{Примечание: N8mm~--- плотность поселения маком с длиной раковины больше $8$~мм в год оседания, экз./м$^2$. N1+~--- плотность поселения однолетних маком через год после года оседания, экз./м$^2$.}
    \end{figure}
Коэффициент корреляции Спирмена между указанными параметрами был достоверный, но низкий ($Spearman\ \rho = 0,39, p < 0,0001$). 

Если при размножении формируется общий личиночный пул, а в дальнейшем на выживаемость влияют зимние условия, можно предположить, что географически близкие поселения должны пополнятся синхронно.
Мы проверили гипотезу о синхронности пополнения поселений при помощи корреляции Мантеля (табл.~\ref{tab:Mantel_dynamic_N1y}).
	\begin{table}[p]
	\caption{Синхронность динамики пополнения поселений {\it Macoma balthica}}
	\label{tab:Mantel_dynamic_N1y}
        \begin{tabularx}{\textwidth}{|p{0.2\textwidth}|*{6}{X|}} 
	\hline
	$Mantel \ r \setminus p_{perm}$ & [1] & [2] & [3] & [4] & [5] & [6] \\ \hline
	[1] Эстуарий р.~Лувеньги             &         & $0,13$    & $0,793$      & $0,118$   & \cellcolor{yellow}{$0,001$} & $0,176$ \\ \hline
	[2] о.~Горелый             & $0,089$   &         & $0,413$      & \cellcolor{yellow}{$0,009$}   & \cellcolor{yellow}{$0,004$} & \cellcolor{yellow}{$0,001$} \\ \hline
	[3] о.~Ломнишный          & $-0,226$  & $-0,003$  &            & NA      & $0,189$ & $0,128$ \\ \hline
	[4] материк (Лувеньга)             & $0,388$   & \cellcolor{yellow}{$0,955$}   & NA         &         & NA    & \cellcolor{yellow}{$0,02$}  \\ \hline
	[5] Южная губа, о.~Ряшков                 & \cellcolor{yellow}{$0,793$}   & \cellcolor{yellow}{$0,515$}   & $0,212$      & NA      &       & $0,12$  \\ \hline
	[6] Западная Ряшкова салма                 & $0,029$   & \cellcolor{yellow}{$0,986$}   & $0,914$      & \cellcolor{yellow}{$0,965$}   & $0,276$ &       \\ \hline 
	\end{tabularx}
	   {\footnotesize Примечание: нижняя треугольная матрица~--- значение теста Мантеля, верхняя треугольная матрица~--- уровень значимости, определенный пермутационным методом. \\
	Выделены значения с уровнем значимости $< 0,05$. \\
	$NA$ --- ряды не пересекаются во времени.}
	\end{table}
Синхронность в пополнении была показана для ряда участков.
В поселении на о.~Горелом успешные пополнения происходили синхронно с поселениями на материковой литорали в Лувеньге и двумя участками литорали на о.~Ряшкове (Южная губа и Западная Ряшкова салма).
Также элементы синхронности были отмечены для поселений на о.~Ряшкове с участком в эстуарии р.~Лувеньги, и участка в Западной Ряшковой Салме с Лувеньгой.

Также было проведено сравнение степени синхронности динамики пополнения поселений (в качестве меры синхронности использовали значение коэффициента корреляции Мантеля) и расстояния между участками.
Не было показано достоверной связи между указанными параметрами ($Mantel\ r = -0,08, p = 0,67$ ).


\afterpage{\clearpage}

%%обсуждение
		\section{Обсуждение результатов.}

%сравнение Белого и Баренцева морей по условиям. Про типичность наших
	\subsection{Физико-географическая характеристика Белого и Баренцева морей}
Белое и Баренцево моря~--- арктические моря, однако литоральная фауна во многом сформирована бореальными видами (\cite{Zenkevich_1963}).
Условия обитания гидробионтов в них значительно отличаются в связи с географическим положением и особенностями гидрологии.
Рассмотрим их подробнее.

	\subsubsection{Белое море}

Белое море глубоко врезается в материк, и с этим связывают континентальность климата: лето относительно теплое, зима продолжительная и суровая. 
Зимой температура воздуха может опускаться до $-20 - -30^{\circ}C$, а летом подниматься до $+30^{\circ}C$, хотя обычно не превышает $15-20^{\circ}C$. 
В северных районах Белого моря температура воздуха в среднем ниже, чем в южных (\cite{Babkov_Golikov_1984}). 
Для губы Чупа минимальная температура воздуха наблюдается в январе (в среднем $-11^{\circ}C$), а максимальная в июле (в среднем $+14,7^{\circ}C$) (\cite{Babkov_1982}). 

Летом в вершинных частях заливов и на мелководье вода может прогреваться до $20 - 24^{\circ}C$. 
Зимой температура воды отрицательная, порядка $-1,5^{\circ}C$ (\cite{Babkov_Golikov_1984}).
Кандалакшский залив является наиболее прогреваемым участком. 
В западной его части среднегодовая температура воды составляет $4^{\circ}C$ (при разбросе от $3,2$ до $5,1^{\circ}C$), а амплитуда межсезонных колебаний составляет в среднем $14,8^{\circ}C$ (от $13,0$ до $16,5^{\circ}C$) (\cite{Kuznecov_1960}). 
В губе Чупа среднегодовая температура всей толщи воды составляет менее $2^{\circ}C$. 
Поскольку литораль находится в зоне влияния поверхностной водной массы, то зимой обитатели подвергаются воздействию отрицательных температур ($-1,5^{\circ}C$), в то время как летом вода на литорали прогревается до $+19,3^{\circ}C$ (\cite{Babkov_1982}). 

Другим важным для гидробионтов фактором является соленость воды. 
В Белом море среднегодовая соленость поверхностных вод составляет $23-25$\permil. 
По данным А.И.Бабкова и А.Н.Голикова (\cite*{Babkov_Golikov_1984}) в районе Кандалакши соленость может изменяться от $7$ до $26$\permil. 
Такие колебания связаны с обширным материковым стоком, частично с осадками и, в первую очередь, с весенним таянием льдов (\cite{Naumov_Fedyakov_1993}).
Вода в губе Чупа значительно распреснена, в первую очередь за счет стока рек Пулонга и Кереть, но также за счет ручьев. 
В верхнем $10$ метровом слое, то есть в слое, омывающем литораль, отмечены сезонные колебания солености более $10$\permil\ (от $15$ до $26$\permil), при этом максимальная соленость достигается в ноябре, а минимальная~--- в апреле (\cite{Babkov_1982}). 

В зимнее время для Белого моря характерен ледовый покров. 
При подвижках припая возможно истирание выступающих над поверхностью структур, в том числе живых организмов. 
Кроме того, возможен перенос организмов, вмерзших в лед или находящихся на примерзших водорослях.
 Время ледостава в разных районах Белого моря отличается. 
В губах Кандалакшского залива лед появляется в первой половине сентября и держится до второй половины мая. 
В губе Чупа формирование льда начинается в устьях рек и ручьев, а также в небольших закрытых губах, где на формирование льда мало оказывает влияние ветрового волнения. 
Неподвижный лед обычно формируется в первой половине декабря. 
Продолжительность ледостава в среднем составляет $5$~месяцев, но в суровые годы может доходить до $7$~месяцев (\cite{Babkov_Golikov_1984}). 

Исследованные нами участки были расположены в основном вершине Кандалакшского залива, кроме того, мы располагаем данными о поселениях маком в губе Чупа. 
Представленные в исследовании участки были достаточно разнообразны в географическом и абиотическом плане.
Представлены поселения, расположенные как на материковой литорали (бухта Лисья, пролив Подпахта, Лувеньга), так и на островах (два участка на о.~Кереть, два участка на о.~Ряшков, о.~Ломнишный, о.~Горелый Лувеньгских шхер). 
Два участка (эстуарий р.~Лувеньги, Сухая салма) расположены в области влияния эстуариев рек (Лувеньга и Кереть, соответственно)  и характеризуются пониженной соленостью по сравнению с остальными.
Разнообразна и степень прибойности: от прибойной литорали в б.~Клющиха до затишных губ (участки в Сухой салме, в Южной губе о.~Ряшкова, на о.~Горелом).

Таким образом, участки биотопически разнородны и относительно полно характеризуют разнообразие илисто-песчаных литоралей в Кандалакшском заливе.

		\paragraph{Динамика температур}
Для Кандалакшского залива доступны данные о среднемесячной температуре воздуха в Кандалакше (\cite{KGZ_letopis, rp5_Kandalaksha}) и данные по температуре воды на декадной станции в губе Чупа (\cite{Berger_et_al_2003}).  
Динамика среднегодовых температур в Кандалакшском заливе показана на рисунке \ref{ris:White_temp_year_dynamic}.
	\begin{figure}[p]
    \includegraphics[width=\textwidth]{../temperatures_water_air/White_temp_air_water_dynamic1.pdf}
    \caption{Динамика среднегодовых температур воды и воздуха в Кандалакшском заливе Белого моря}
    \label{ris:White_temp_year_dynamic}
	\end{figure}

Среднегодовая температура воздуха и температура воды достоверно скоррелированы (корреляция Спирмена для температуры поверхности воды: $\rho = 0,3, p = 0,0035$, для температуры верхнего 50-метрового слоя: $\rho = 0,7, p = 0,0008$).


Использование среднегодовых значений температуры скрывает сезонное варьирование, которое может быть принципиально важно для поселений маком (например: \cite{Beukema_et_al_1998, Beukema_Dekker_2003, Beukema_et_al_2009}). 
Корреляция среднесезонных температур воздуха и поверхности воды выше, чем среднегодовых значений (корреляция Спирмена для температуры поверхности воды: $\rho = 0,92, p < 0,0001$ (рис.~\ref{ris:White_temp_water_vs_air_seasons}). 
	\begin{figure}[p]
    \includegraphics[width=\textwidth]{../temperatures_water_air/temp_air_water1.pdf}
    \caption{Соответсвие среднесезонных температур воды и воздуха в Кандалакшском заливе Белого моря}
    \label{ris:White_temp_water_vs_air_seasons}
	\end{figure}
Динамика средней температуры воды в разные сезоны представлена на рисунке~\ref{ris:White_temp_seasons_dynamic}.
	\begin{figure}[p]
    \includegraphics[width=\textwidth]{../White_Sea/temperature_Kartesh/t_mean_season_year1.pdf}
    \caption{Динамика среднесезонной температуры воды в губе Чупа(Кандалакшский залив Белого моря) (\cite{Berger_et_al_2003})}

{\footnotesize Примечание: t, C~--- температура поверхности воды: синий~--- зимняя, зеленый~--- веснняя, красный~--- летняя, оранжевый~--- осенняя, черный~--- среднегодовая. }
    \label{ris:White_temp_seasons_dynamic}
	\end{figure}

Очевидно, что локальные условия могут значительно варьировать в зависимости, например, от закрытости акватории.
Однако для оценки глобальных климатических воздействий мы считаем возможным использовать данные по Чупе для сравнения температурного режима в разные годы для всех участков в Кандалакшском заливе.

Динамика среднегодовых температур в Белом море характеризуется значительными флуктуациями.
При рассмотрении сезонных данных оказывается, что средневесенняя температура наиболее вариабельна, в то время как среднезимняя до $1999$ года была относительно стабильна, а в дальнейшем также значительно варьировала из года в год (рис.~\ref{ris:White_temp_seasons_dynamic}).

\afterpage{\clearpage}

	\subsubsection{Баренцево море}

Баренцево море~--- окраинное море, характерной особенностью гидрологического режима которого является наличие двух  водных масс~--- арктической (полярные воды, большую часть года покрытых плавучими льдами) и субарктической (субполярных вод, свободных от плавучих льдов) (\cite{Adrov_1992}). 

%Мурманским побережьем или Мурманом называют береговую линию Северного ледовитого океана от мыса Святой нос на востоке до реки Ворьемы на западе. 
%Данный район разделяют на несколько областей: Западный Мурман~--- от реки Ворьемы до острова Кильдин или до Кольского залива, и Восточный Мурман~--- далее на восток до мыса Святой нос (\cite{Derugin_1915}).

Постоянный подток теплых атлантических вод препятствует образованию льда вдоль Мурманского побережья, и он встречается главным образом во внутренних частях губ и заливов.
Несколько большее количество льда образуется ежегодно в юго-восточном районе Мурмана, в то время как по Западному Мурману, как правило, не образуется сплошного припая. 
В основном, исключая некоторые опресненные закрытые бухты и заливы, влияние морского льда на распределение животных невелико, гораздо большее значение зимой играет сильное промораживание литорали во время отлива (\cite{Propp_1971}).

Приливы на Мурмане являются правильными полусуточными и образуются единой атлантической приливной волной. 
Далее она распространяется вдоль Мурмана на восток до Новой Земли. 
Высота приливной волны составляет $3$ метра. 

В среднем, соленость вод у Мурманского побережья составляет $33,2 - 33,6$\permil. 
Только весной во время сезонного увеличения берегового стока наблюдается краткое распреснение поверхностных слоев до $28 - 30$\permil, однако толщина опресненного слоя не превышает $2 - 3$~м.

Кольский залив~--- самый крупный из заливов Мурманского побережья Баренцева моря, лежит на границе Восточного и Западного Мурмана.
Географически в Кольском заливе выделяется три части, называемые коленами залива. 

Первое, северное или нижнее колено простирается от входа в Кольский залив до линии, соединяющей устье губы Средней и мыс Лас. 
Эта часть залива наиболее глубоководная (более $400$~м). 
Береговая линия северного колена Кольского залива чрезвычайно изрезана, и  здесь находятся самые крупные губы (\cite{Derugin_1915}), в том числе Пала-губа, ставшая объектом наших наблюдений .


Среднее колено (глубины до $200$~м) изогнуто в направлении к северо-западу и простирается на юг до мысов Пинагория и Мишукова. 
Второй участок наблюдений был расположен в районе границы северного и среднего колена Кольского залива (Ретинское).

Южная или верхняя часть наиболее мелкая (глубина около $50$~м), имеет направление с севера на юг, как и нижняя. 
В кут Кольского залива впадает две крупные реки~--- Тулома и Кола, и одна более мелкая~--- Лавна (\cite{Derugin_1915}).  
В районе самого узкого участка Кольского залива (Абрам-мыс) был расположен третий участок исследования в данном районе.
Последний участок, исследованный в Кольском заливе был расположен на западном берегу залива в черте города Мурманск (Северное Нагорное) в $3$~км от устья реку Туломы.

Воды Кольского залива неоднородны по своим свойствам. 
Это связано с несколькими причинами: большая протяженность залива, наличие глубоко вдающихся в побережье губ, влияние стока рек и ручьев. 
Гидрологическое лето начинается в поверхностных слоях воды в начале июля и продолжается до конца августа. 
Летом вода прогревается до $+8 - +18^{\circ}C$ в различных частях залива.

В  северном колене залива летом поверхностный слой значительно распреснен и соленость может достигать $8$\permil, причем толщина распресненного слоя может достигать $3-4$~метров. 
Глубже соленость не опускается ниже $30$\permil и у дна достигает $34$\permil. 
Зимой соленость поверхностного слоя также составляет $30 - 34$\permil. 

В южном колене в районе Абрам-мыса колебания солености на поверхности еще более заметны. 
Здесь сказывается не только сезонность стока, но и значительное влияние оказывает приливно-отливные течения. 
Летом во время прилива поверхностный слой толщиной до 3 метров обладает соленостью от $2$ до $16$\permil, в то время как на глубине $3$~метра соленость колеблется в пределах от $28$ до $31$\permil. 
В отлив мощность опресненного слоя увеличивается до $8$~метров, а поверхностная вода становится практически пресной (\cite{Derugin_1915}).

Таким образом, исследованные нами участки в Кольском заливе расположены в контрастных по географическим условиям его частях и позволяют относительно полно судить о данной акватории.

Фауна литораль Западного Мурмана наиболее богата по сравнению с остальным Мурманским побережьем. 
Традиционно, это связывают с более высокой среднегодовой температурой (температура воздуха в губах Западного Мурмана может быть на $0,4^{\circ}C$ выше по сравнению с Восточным Мурманом) и соленостью (выше $31$\permil\ в поверхностном слое) и закрытости губ Западного Мурмана от основной акватории моря (\cite{Guryanova_et_al_1930}). 
К сожалению, данный регион оказался для нас малодоступен при исследованиях, и мы располагаем лишь данными об обилии маком в губах Ура и Печенга.
Однако данные губы расположены в разных частях Западного Мурмана, что позволяет нам делать предвательные выводы о данном регионе.

Береговая линия Восточного Мурмана менее изрезана, чем Западного Мурмана. 
Побережье большинства небольших заливов и губ не защищено от прибойного воздействия (\cite{Guryanova_Ushakov_1929}).
Таким образом, Восточный Мурман на большем его протяжении не является благоприятным для развития литоральных инфаунных сообществ, однако существуют глубоко вдающиеся в побережье бухты, в которых обнаруживается меньшее волновое воздействие. 
Именно на литорали таких губ и заливов и формируются наиболее богатые инфаунные сообщества данного региона, включающие {\it M.~balthica}.

Наши исследования охватывают Восточный Мурман на значительном его протяжении: $6$~участков от губы Гаврилово до губы Ивановская (длина береговой линии более $150$ километров).
Обследованные бухты варьируют по длине, степени изолированности и наличию в них ручьев и небольших рек, влияющих на локальное опреснение.


География исследований охватывает в том числе Дальний пляж губы Дальне-Зеленецкой~--- исторически наиболее обследованной бухты на Мурмане.
Губа Дальне-Зеленецкая включает в себя две бухты~--- бухта Оскара и бухта, в кутовой части которой располагается литоральная отмель Дальнего Пляжа. 
Важной характеристикой губы является изолированность ее от интенсивного волнового воздействия за счет наличия островов на входе в губу.
	
При максимальных отливах протяженность литорали Дальнего пляжа с северо-запада на юго-восток составляет около $460$~м, а с юго-запада на северо-восток -- около $400$~м. 
	
В южной части отмели располагается дельта небольшого Зеленецкого ручья, вызывающего незначительное опреснение. 
Так, грунтовая вода, взятая у самого ручья, имеет соленость $32,9$\permil, а взятая на два метра в стороне от ручья~--- $34,07$\permil (\cite{Prigorovskiy_1948}). 
Гидрологический режим характеризуется тем, что в бухту заходят воды из более глубоких и холодных слоев открытого моря, что вызывает уменьшение температуры и повышение солености (\cite{Voronkov_et_al_1948}).

Волновая активность в губе не превышает $1,5 - 2$ балла (\cite{Alexeev_1976}). 
Наиболее сильному волновому воздействию подвержена южная и юго-восточная части отмели, где на галечно-валунном пляже располагается зона штормовых выбросов.
Придонная скорость течений, вызванных приливной волной, составляет $0,8$~м/сек. при глубине 0,3-0,5 метров и 0,06 м/сек. при глубине более $2$~метров.

Для песчаных отмелей характерна только одна граница~--- уровень высачивания, который делит пляж на две части, отличающиеся по условиям увлажненности донного осадка во время отлива (\cite{Streltsov_Agarova_1978}). 
Обширный, располагающийся ниже уровня высачивания и увлажненный во время отлива <<ватт>> простирается от отметок 1,25 до 2,1 м. над нулем глубин, сменяясь выше уровня высачивания узким $30$-метровым пляжем, где вода, занимавшая во время прилива интерстициальное пространство, вместе с грунтовыми водами вытекает на поверхность донного осадка. 
В западной части пляжа, самые верхние горизонты заняты валунной грядой (\cite{Agarova_et_al_1976}). 

Грунты отмели однообразны почти на всем ее протяжении. 
Мощность верхнего слоя ничтожна, и составляет $5 - 8$~см (\cite{Prigorovskiy_1948}). 
Для отмели процессы размыва преобладают над накоплением. 
Даже в зоне относительно высокой аккумуляции, в <<языках>> дельты ручья, мощность голоценовых отложений составляет всего $15 - 30$~см.

Максимальная концентрация песков (более $90$\% по массе) отмечена в юго-восточной оконечности у подножья террасы, сложенной древними морскими песками. Еще одной особенностью пляжа является повышенное содержание алевропелитов (\cite{Pavlova_1976}). 
Их локализация на пляже обусловлена эрозивной волноприбойной деятельностью, доминирующей при среднем уровне малой воды (\cite{Alexeev_1976}).

Органическое вещество представлено гумусовыми соединениями и битумоидами местного и континентального происхождения (\cite{Gurevich_Yakovleva_1976}).
Наши мониторинговые работы в губе Дальне-Зеленецкая продолжают череду количественных гидробиологических исследований данного района (\cite{Prigorovskiy_1948, Matveeva_et_al_1955, Streltsov_et_al_1974, Agarova_et_al_1976, Zhukov_1984, Strelkov_et_al_2001}).


Таким образом, выбранные участки достаточно разнообразны по своей географической приуроченности и связанных с ней абиотических градиентов (температура и соленость).

		\paragraph{Динамика температур}
Для Баренцева моря доступны данные по динамике температур на разрезе Кольский меридиан (\cite{pinro}). 
Наиболее адекватными данными для оценки динамики литоральных температурных условий представляются данные о средней температуре в верхнем 50-метровом слое воды на прибрежных станциях (рис.~\ref{ris:Barents_temp_dynamic}).
	\begin{figure}[p]
    \includegraphics[width=\textwidth]{../Barenc_Sea/temperature/t_air_mean_season_year1.pdf}
    \caption{Динамика температуры воды верхнего 50-метрового слоя на разрезе Кольский меридиан(станции 1-3) (\cite{pinro})}

{\footnotesize Примечание: t, C~--- температура поверхности воды: синий~--- зимняя, зеленый~--- веснняя, красный~--- летняя, оранжевый~--- осенняя, черный~--- среднегодовая. }
    \label{ris:Barents_temp_dynamic}
	\end{figure}

В Баренцевом море за иследованное время ($2002 - 2008$) можно говорить об относительно более теплом периоде~--- с $2004$ по $2007$ год.
При этом данное потепление охватывало все сезоны (рис.~\ref{ris:Barents_temp_dynamic}).
Если рассматривать среднезимние температуры, то относительно теплый период захватывает также $2008$ год.


\vspace{2cm}

Таким образом, условия обитания маком в Белом и Баренцевом море различаются по многим параметрам.
Температурный режим прибрежной части Кандалакшского залива Белого характеризует более значительные сезонные колебания(рис.~\ref{ris:temp_White_Barents}).
	\begin{figure}[p]
    \includegraphics[width=\textwidth]{../temperatures_water_air/temp_White_Barents1.pdf}
    \caption{Соотношение среднесезонных температур в верхнем 50-метровом слоях воды в Белом и Баренцевом морях}

{\footnotesize Примечание: t, C~--- температура поверхности воды: синий~--- зимняя, зеленый~--- веснняя, красный~--- летняя, оранжевый~--- осенняя}
    \label{ris:temp_White_Barents}
	\end{figure}
В пределах каждого сезона межгодовые изменения в Белом море также выше, чем в Баренцевом.
Кроме того, различается сезонность хода температур. 
В Белом море лето является наиболее теплым сезоном, а зима~--- наиболее холодным.
Для Баренцева моря гидрологическая сезонность сдвинута относительно календарной: самый теплый сезон это осень, а самый холодный~--- весна.

Данные о солености в маштабах крупных акваторий не очень показательны для бентосных огранизмов, поскольку локальные условия, например, наличие берегового стока в данном месте, значительно меняют данный показатель. 
Cоленостная толерантность взрослых особей {\it M.~balthica} достаточно высока (\cite{Naumov_2006}), однако соленость может играть роль на начальных стадиях развития.
В целом, исследованный район в Баренцевом море характеризуется соленостью близкой к океанической.
Характерно, что все поселения на Западном и Восточном Мурмане расположены в губах, в которые впадают небольшие реки или ручьи, то есть находятся в распресненных условиях.
Соленость в Кольском заливе ниже океанической за счет впадения в кут залива крупных рек Колы и Туломы, и таким образом участки, расположенные вне губ с локальным стоком, тоже находились в распресненных условиях.
Тем не менее, невозможно утверждать, что распределение маком на Мурмане находится под влиянием солености, так как невозможно изолировать несколько важных абиотических фактора: соленость, характер грунта и степень прибойности/закрытости акватории, поскольку для Мурманского побережья характерно наличие берегового стока в закрытых губах (\cite{Guryanova_Ushakov_1929, Guryanova_et_al_1930}).

Белое море в целом характеризуется пониженной соленостью и её среднегодовое значение не превышает 25\permil.
В данной акватории нельзя говорить о приуроченности поселений маком в локальному береговому стоку, и среди исследованных участков были как участки, находящиеся под влиянием рек и ручьев, так и вне зоны влияния оных.

		\paragraph{Топические условия}
\textit{M.~balthica}~--- обитатель мягких грунтов.
По данным А.~Д.~Наумова, в Белом море $35$\% находок относятся к биотопам с илистыми грунтами и $46$\%~--- с песчаными. 
Исследованные нами участки представляли собой песчаные отмели с различной примесью ила, то есть относились к типичным местообитаниям маком.
Интересно, что в Белом море максимальные биомассы \textit{M.~balthica} отмечены на мелко-гравийном грунте (\cite{Naumov_2006}), что хорошо прослеживается на наших данных: биомасса на литорали в Западной Ряшковой салме выше, чем на других участках (табл.~\ref{tab:mean_NB_White}, приложение~\ref{app:NB_table}).

\afterpage{\clearpage}

%%%%%%%%%%%%%%%%%%%%%%%%%%%%%%%%%
%макома как типичный компонент литоральных сообществ.? Гурьянова и Ко - в Баренцевом. Белое - ээээ.
%		\subsection{{\it Macoma balthica} как массовый элемент в сообществах литорали северных морей}
		\subsection{{\it Macoma balthica} как типичный компонент литорали Белого и Баренцева морей}

Моллюски {\it M.~balthica}~--- амфибореальный вид. 
По Американскому побережью Атлантики вид распространен на север до Лабрадора.
В Европеской части ареала {\it M.~balthica} заходит в арктические моря, и встречается в Норвежском, Баренцевом, Белом и Карском морях.
Наиболее северной точкой считается Шпицберген (\cite{Zacepin_Filatova_1968}).

В Баренцевом море макомы вместе с другими представителями бореальной и бореально-арктической фауны заселяют пляжи осушной зоны и верхней сублиторали. 
По данным Е.~Ф.~Гурьяновой, И.~Г.~Закса и П.~В.~Ушакова (\cite{Guryanova_et_al_1928, Guryanova_Ushakov_1929, Guryanova_et_al_1930}, макома повсеместно встречается на мягких грунтах в бухтах Кольского залива и Мурманского побережья. 
Все отмеченные нами виды характерны для литорали Кольского залива и Восточного Мурмана (\cite{Derugin_1915, Guryanova_Ushakov_1929}).

В Белом море \textit{M.~balthica} входит в литоральный комплекс двустворчатых моллюсков и отмечена во всех заливах (\cite{Naumov_2006}).
По данным различных исследователей (\cite{Babkov_Golikov_1984, Naumov_Fedyakov_1993}) для среднего и нижнего горизонта литорали с мягкими грунтами характерно формирование сообществ с доминированием \textit{M.~balthica}. 
Все встреченные нами виды являются характерным окружением для \textit{M.~balthica} (например, \cite{Chertoprud_et_al_2004, Naumov_2006, Gerasimova_et_al_2010, Derevenschikov_Kravets_2010, Stolyarov_2010}).

Таким образом, состав макробентоса в изученных местообитаниях позволяет говорить, что  мы имели дело с типичными для исследованных акваторий биосистемами. 


\afterpage{\clearpage}

%%%%%%%%%%%%%%%%%%%%%%%%%%%%%%%%%
%Обилие макомы - соотношение наших данных и литературных по нашим регионам. обилие макомы в разных частях ареала. не работает гипотеза что в центре много.
		\subsection{Обилие {\it Macoma balthica} в европейской части ареала}
Полученные для Белого и Баренцева данные хорошо согласуются с литературными данными об этих регионах.
Так, по нашим данным, на литорали Кольского залива численность {\it M.~balthica} составляла около $1000$~экз./м$^2$, что хорошо соотносится с результатами, полученными ранее для других областей данной акватории. 
Л.~Басова, обладая данными по большему количеству участков, приводит средние показатели плотности поселения маком $802 \pm 273$~экз./м$^2$ при максимальной численности $2900$~экз./м$^2$ (\cite{Basova_2004}).
На Восточном Мурмане численность {\it M.~balthica} в основном не превышала $100$~экз./м$^2$, лишь на одном участке достигая $500$~экз./м$^2$. 
И.~Я.~Агарова с соавторами (\cite{Agarova_et_al_1976}) даёт оценку численности {\it M.~balthica} крупнее $5$~мм для разных сообществ Дальнего пляжа губы Дальне-Зеленецкой в 1973 году от $12$ до $42$~экз./м$^2$ (рис.~\ref{ris:dynamic_Zelency_Agarova}). 
	\begin{figure}[p]
%		\includegraphics{../Barenc_Sea/Dalnezeleneckaya/N_dynamic_with_Agarova.pdf}
		\includegraphics{../after_Deryuginskie/Macoma_N_dynamic_all1.pdf}
	\caption{Динамика плотности поселений {\it Macoma balthica} на литорали Дальнего пляжа г.~Дальнезеленецкой (Баренцево море)}
{\footnotesize Примечание: по оси $X$~--- годы наблюдений, по оси $Y$~--- средняя плотность поселения,~экз./м$^2$. \\
Светлые столбцы~--- общая численность, темные столбцы~--- численность моллюсков крупнее $5$~мм. Данные $1973$ года взяты из статьи \cite{Agarova_et_al_1976}}
	\label{ris:dynamic_Zelency_Agarova}
	\end{figure}
Плотность поселения {\it Macoma balthica} на Дальнем пляже в $1973$ году была сравнима с таковой в $2002-2006$ годах (табл.~\ref{tab:DZ_N_1973_sravnenie}).
	\begin{table}[p]
	\begin{tabularx}{\textwidth}{|*{4}{X|}} \hline
	годы сравнения & $W$ & $p-value$ & достоверность различий \\ 
	\hline
	$1973 - 2002$ & $31,5$ & $0,08$ & *\\
	\hline
	$1973 - 2003$ & $80,5$ & $0,86$ & \\
	\hline
	$1973 - 2004:2006$ &  $214$ & $0,44$ & \\
	\hline
	$1973 - 2007:2008$ & $22$ & $0,0048$ & ** \\
	\hline
	\end{tabularx}
	{\footnotesize Примечание: $W$ - значение критерия Вилкоксона, достоверность различий ***~--- $p<0,001$; **~--- $p<0,05$; *~--- $p<0,1$.}
	\caption{Сравнение численности {\it Macoma balthica} на Дальнем пляже губы Дальнезеленецкой в $1973$ году и $2002-2008$.}
	\label{tab:DZ_N_1973_sravnenie}
	\end{table}


Для Белого моря максимальные численности по нашим данным сравнимы с приводимыми в работе А.Д.~Наумова (\cite{Naumov_2006}) максимальными значениями для Белого моря ($4581$~экз./м$^2$ в Оленьей салме в куту Кандалакшского залива). 
Размах варьирования численности маком по данным других мониторинговых программ в Кандалакшском заливе Белого моря аналогичен нашим наблюдениям~--- от нескольких десятков особей до $1-3$ тысяч особей на квадратный метр (\cite{Semenova_1974, Maximovich_et_al_1991, Varfolomeeva_Naumov_2013}). 


%Согласно классическим представлениям {\it M.~balthica} описывается как амфибореальный вид. 
%В Атлантическом океане данный вид встречаются по всему европейскому побережью до Франции на юге. 
%По американскому~--- от штата Джорджия на юге до моря Баффина и западного побережья Гренландии на севере. 
%В Тихом океане {\it M.~balthica} встречается до залива Посьет в Японском море по азиатскому берегу, и до Сан-Диего~--- по американскому. 
%Также данный вид заходит в моря Северного Ледовитого океана:  Норвежское, Баренцево, Белое, Карское, Чукотское и Бофорта. 
%В западном секторе Арктики, самые восточные находки вида~--- из Байдарацкой губы Карского моря, а самые северные~--- со Шпицбергена (\cite{Semenova_1974, Kafanov_1985, Maximovich_1985, Meehan_1985, Naumov_2006, Meehan_et_al_1989, Hummel_et_al_1997_stress}).

%Современные молекулярные исследования показывают неоднородность вида {\it M.~balthica} sensu lato в пределах ареала, причем можно говорить о нескольких уровнях гетерогенности. 
%Рассматривая самый верхний уровень внутривидовой генетической структуры, в настоящее время предлагают выделять атлантический ({\it M.~balthica rubra} и тихоокеанский {\it M.~balthica balthica} подвиды.
%Однако  в морях, связанных с  Атлантикой, существуют очаги распространения тихоокеанской формы. 
%Так, в Балтийском, Белом и Баренцевом морях обитает тихоокеанская форма {\it M.~balthica balthica}. В Баренцевом море она распространена на восток до Варангер-фьорда (\cite{Vainola_2003}). 

Для сравнения наших данных по Белому и Баренцеву морям с данными по обилию маком в других частях европейской части ареала была собрана опубликованная информация о среднем обилии особей {\it M.~balthica} в различных акваториях (прил.~\ref{app:NB_areal}). 
Из анализа исключали данные об обилии сеголетков, и учитывали только информацию об обилии особей старше $1$~года.
Полученные данные визуализировали на карте (рис.~\ref{ris:macrodistribution}).
	\begin{figure}[p]
    \includegraphics[width=\textwidth]{../macrodistribution/Nmean_ru1.pdf}
    \caption{Численность {\it Macoma balthica} в европейской части ареала (прил.~\ref{app:NB_areal})}

{\footnotesize Примечание: Площадь кругов пропорциональна средней численности (N) моллюсков,~экз./м$^2$}
    \label{ris:macrodistribution}
	\end{figure}
Численность {\it M.~balthica} на Западном Мурмане и в Кольском заливе была сравнима с численностями моллюсков в Белом море, Балтийском море и северной части Норвежского моря (\cite{Semenova_1974, Aschan_1988, Maximovich_et_al_1991, Bonsdorff_et_al_1995, Bostrom_Bonsdorff_2000, Oug_2001, Laine_et_al_2003, Khaitov_et_al_2007, Varfolomeeva_Naumov_2013}).
Численности маком, сходные по величине с отмеченными на Восточном Мурмане, характерны для Норвежского и Северного морей (включая Ваттово море) (\cite{Brady_1943, Sneli_1968, Stromgren_et_al_1973, Beukema_1976, Jensen_Jensen_1985, Jensen_et_al_1985, Madsen_Jensen_1987, Beukema_1979, Zwarts_Wanink_1993, Reise_et_al_1994}) (рис.~\ref{ris:macrodistribution}).

Численность в сублиторали Восточного Мурмана (Ивановская губа) была выше, чем численность моллюсков на литорали (рис.~\ref{ris:N_area_Barents}).
В верхней сублиторали Печерского моря (восточная часть Баренцева моря, \cite{Denisenko_et_al_2003}) численность маком была в два раза ниже, чем отмеченная нами, однако также была значительно выше обилия данного вида на литорали Восточного Мурмана (рис.~\ref{app:NB_areal}).
Более высокие численности маком в верхней сублиторали относительно литорали отмечены для некоторых участков в Белом море (\cite{Semenova_1974}), хотя чаще отмечается обратный эффект (\cite{Semenova_1974, Maximovich_et_al_1991}).

При описании распределения обилия видов в ареале часто используют т.~н. гипотезу об обилии в центре <<abundant centre hypothesis>>, постулирующую, что максимальное обилие вида характерно для центральной части ареала, но снижается по направлению к границам ареала (\cite{Sagarin_et_al_2006}).
Корреляция между географический широтой и средним обилием маком оказалась слабой, но достоверной (коэффициент Спирмена: $r_{s} = 0,38$, $p = 0,003$).
Слабость данной связи определяется большим размахом варьирования численности моллюсков не только в пределах одного региона, но и для одного поселения в разные периоды времени (рис.~\ref{ris:lat_vs_Nmean}). 
	\begin{figure}[p]
    \includegraphics[width=\textwidth]{../macrodistribution/lat_vs_Nmean1.pdf}
    \caption{Изменение численности {\it Macoma balthica} с географической широтой}

{\footnotesize Примечание: N~--- средняя численность {\it M.~balthica},~экз./м$^2$}
    \label{ris:lat_vs_Nmean}
	\end{figure}
Возможно, более показательно рассматривать максимальные средние значения, поскольку они показывают, какого максимального значения может достигать обилие в данном регионе.
По данным, представленным на рисунке \ref{ris:lat_vs_Nmean}, видно, что максимальная средняя численность маком монотонно увеличивается с широтой.
Таким образом, распределение вида {\it M.~balthica} в европейской части ареала может быть описано как увеличивающееся к северу (<<ramped north>>) (\cite{Sagarin_Gaines_2002}).

Максимальные средние численности маком в пределах европейской части ареала отмечены для Белого и Балтийского морей (рис.~\ref{app:NB_areal}).
Интересно, что оба этих водоема характеризуются пониженной соленостью (\cite{Dobrovolskiy_Zalogin_1982}).
Возможно, в условиях пониженной солености конкуренция оказывается ниже, за счет исчезновения более стеногалинных видов, и макома может достигать большего обилия.
Также на обилие может влиять доступность пищевых ресурсов. 
Такой эффект известен при сравнении условий обитания в отдельных поселениях.
Л.~Басовой для Кольского залива была показана достоверная положительная корреляция между численностью {\it M.~balthica} и содержанием органических веществ в грунте (\cite{Basova_2004}).   
Мы не обнаружили подобной закономерности, в то же время, по нашим данным, численность маком достоверно коррелировала с долей песчаный фракций. 
Была показана прямая связь с мелким песком и обратная~--- с крупным (табл.~\ref{tab:grunt_N_correlation_Barents}).
Обычно предполагается, что предпочтение особями более мелкодисперсных грунтов связано с более высокой концентрацией органических веществ в таком грунте. 
Хотя часто концентрация органических веществ положительно коррелирует с долей мелкого песка и алевро-пеллита (\cite{Bubnova_1972, Basova_2004}), для исследованных частков на статистическом уровне этого не показано, хотя и наблюдается тенденция к этому. 
Показано (\cite{Olafsson_1989}), что на песчаном грунте {\it M.~balthica} начинают питаться не как собирающие детритофаги, а как фильтраторы. 
Таким образом, основную роль начинают играть не органические вещества в осадках, растворенные в воде. 
В таком случае наличие в Кольском заливе поселков и городов, в которых есть бытовые стоки, может объяснять более высокое обилие маком именно в данной акватории.

%Написать что-то про продуктивность Белого и Балтийского моря. Возможно сравить их с тем же Северным. Понять только по какому параметру. Прямая органика. Планктон???

\afterpage{\clearpage}

%%%%%%%%%%%%%%%%%%%%%%%%%%%%%%%%%
%структура поселений маком.
	\subsection{Особенности структуры поселений {\it Macoma balthica}}
Для \textit{M.~balthica} описано бимодальное и мономодальное распределение особей (\cite{Segerstrale_1969, Maximovich_et_al_1991, Nikolaeva_1997, Nikolaeva_1998}). 
При массовом оседании личинок  \textit{M.~balthica}, в зависимости от выживаемости сеголеток, возможно два варианта развития поселения. 
Если выживаемость хорошая, то можно наблюдать ежегодное смещение модального класса по оси размеров. 
При новом оседании личинок до полного отмирания особей первой генерации формируется бимодальное распределение. 
Другой описанный вариант~--- к следующему сезону сеголетки практически исчезают, и происходит новое оседание личинок. 
При повторении этой схемы наблюдается мономодальное распределение с доминированием по численности самых мелких особей (сеголеток) при практически полном отсутствии крупных особей. Естественно, при плохой выживаемости и отсутствии значительного оседания личинок поселение достаточно быстро отмирает (\cite{Maximovich_et_al_1991}).

В исследованных нами поселениях размерная структура \textit{M.~balthica} значительно варьирует, однако при достаточно высокой численности моллюсков мы наблюдаем две наиболее характерные ситуации: мономодальное распределение особей по размерам чаще всего с преобладанием молодых особей, и бимодальное распределение.

Рассматривая динамику размерной структуры, можно говорить о  двух ситуациях, которые мы наблюдали в исследованных поселниях.
Наиболее распространена ситуация, в которой наблюдается смена типа структуры со временем. 
Сначала в поселении наблюдается мономодальная структура с преобладанием относительно молодых, и со временем мы можем наблюдать смещение модального класса по оси размеров. 
Через несколько лет происходит следующее успешное пополнение поселения молодью и формируется бимодальное распределение.
Со временем происходит элиминация старших особей и, в зависимости от периода через который происходит следующее успешное пополнение поселения молодью, мы либо продолжаем наблюдать бимодальное распределение, либо оно вновь становится мономодальным.
Такой тип динамики отмечен нами для всех поселений в районе Лувеньгских шхер, Западной Ряшковой салмы (прил.~\ref{app:White_sizestr_hist}) и для Дальнего пляжа губы Дальнезеленецкая (прил.~\ref{app:Barents_sizestr_hist}).
Подобная картина была ранее описана для Сухой салмы в губе Чупа Белого моря (\cite{Maximovich_et_al_1991}).
В Балтийском море описан аналогичный тип динамики (\cite{Segerstrale_1969}).

Другой вариант динамики размерной структуры, по-видимому, менее распространен.
Он выглядит как ежегодное повторение мономодальной размерной структуры в течение нескольких лет.
Такой вариант был описан для поселений \textit{M.~balthica} в Южной губе о.~Ряшкова и на о.~Ломнишном (прил.~\ref{app:White_sizestr_hist}).
Интересно отметить, что оба поселения находились под влиянием хищной улитки \textit{Amauropsis islandica} (\cite{Aristov_Granovich_2011}).
Однако для того чтобы аргументированно говорить о влиянии хищников, необходимы отдельные исследования.
Сходный тип динамики был описан для бухты Клющиха в губе Чупа Белого моря (\cite{Maximovich_et_al_1991, Gerasimova_Maximovich_2013}.
Все участки, на которых описан подобный тип развития поселения, сходны по топическим условиям~--- песчаный пляж с минимальным заилением.
Это подтверждает предположение, высказанное ранее (\cite{Gerasimova_Maximovich_2013}), что возможность формирования такого типа динамики может быть связана с расхождением по типу питания у молодых и взрослых маком.
Для Балтийского моря было показано, что на илисто-песчаном грунте и взрослые, и молодые моллюски питаются как собирающие детритофаги, в то время как на песчаом грунте, в условиях активной гидродинамики, где молодь питаются как собирающие детритофаги, а взрослые~--- как фильтраторы (\cite{Olafsson_1989}). 
Аналогичное различие в пищевом поведении было показано и для Белого моря (\cite{Gerasimova_1988}).

\afterpage{\clearpage}

%%%%%%%%%%%%%%%%%%%%%%%%%%%%%%%%%
		\subsection{Скорость роста {\it Macoma balthica} как отражение условий обитания}
% скорость роста как показатель "условий жизни"

Рост рассматривается как комплексный отклик организма на совокупность условий в локальном местообитании. 
Однако не менее интересной представляется попытка разложить всю совокупность условий на отдельные факторы, влияющие на ростовые характеристики. 

Одним из главных, определяющих рост факторов, является температура (\cite{Gilbert_1973, De_Wilde_1975, Bachelet_1980}). 
При повышении температуры происходит увеличение скорости метаболических процессов, в том числе темпов роста моллюсков в толерантных пределах.  
Для {\it M.~balthica} показано, что оптимальные условия роста~--- температура $0 - 10^{\circ} C$, а когда температура превышает $15^{\circ} C$ рост прекращается (\cite{De_Wilde_1975}). 
Ограничение роста при высоких температурах было отмечено и другими авторами, хотя на южной границе ареала (по-видимому, за счет физиологической адаптации) рост происходил и при более высоких температурах (\cite{Bachelet_1980}).

Другим фактором, влияющим на процесс роста, является обилие пищи. 
Наблюдается достоверная связь между содержанием хлорофилла A на поверхности грунта, концентрацией фитопланктона и скоростью роста особей {\it M.~balthica} (\cite{Beukema_et_al_1977, Kube_et_al_1996}). 
С обилием пищи тесно связано влияние на рост моллюсков гранулометрического состава грунта и содержание в нем органических веществ. 
Чем меньше диаметр частиц грунта, тем больше площадь их поверхности и тем больше на них бактерий, соответственно более мелкодисперсный грунт оказывается для маком <<питательнее>>. 
Показано, что скорость роста особей на песчаном грунте ниже, чем на илистом (\cite{Wenne_Klusek_1985, Maximovich_et_al_1992}). 
Выявлена достоверная связь скорости роста моллюсков с долей мелкой фракции грунта и содержанием в нем органических веществ (\cite{Kube_et_al_1996}).

Соленость также оказывает влияние на рост моллюсков, хотя данные о характере этого влияния различны. 
Некоторые авторы отрицают влияние солености на скорость роста (\cite{Bachelet_1980}), другие авторы утверждают, что скорость роста и размеры моллюсков имеют тенденцию уменьшаться с уменьшением солености (\cite{Segerstrale_1960, Kube_et_al_1996}). 

Литературные данные о скорости роста моллюсков на различном мареографическом уровне противоречивы. 
Башле (\cite{Bachelet_1980}) обнаружил, что в эстуарии р.~Жиронда (южной границе ареала макомы в Европе) скорость роста моллюсков на верхней литорали значительно выше, чем на нижней. 
На верхней литорали моллюски достигают большего размера и дольше живут. 
Обратная связь найдена Грином (\cite{Green_1973}) и Харвеем и Винсентом (\cite{Harvey_Vincent_1990}) для канадских популяций {\it M.~balthica}. 
В качестве причины  таких различий авторы предполагают большее время питания на нижней литорали и негативное влияние высоких температур, ограничивающих рост, на верхней. 
Бьёкема и соавторы (\cite{Beukema_et_al_1977}) показали, что наибольшие скорости роста имеют моллюски со средней литорали, поскольку на верхней литорали скорость роста ограничивается временем питания, а на нижней~--- количеством пищи. 
В Белом море при сравнении темпов роста моллюсков из литоральных и сублиторального поселений, максимальный темп роста обнаружен в сублиторали. 
Однако различий в скорости роста между горизонтами литорали отчемено не было (\cite{Maximovich_et_al_1992}).
В Гданьском заливе скорость роста возрастала с глубиной~--- более высокая скорость роста обнаружена у моллюсков на глубине $35 - 75$~м, по сравнению с особями из мелководной ($5 - 6$~м) части залива (\cite{Wenne_Klusek_1985}). 
Обратная ситуация наблюдается в других частях Балтийского моря~--- минимальную скорость роста имеют моллюски с глубины $35$~м, максимальную с глубины $3$~м (\cite{Segerstrale_1960}). 

Таким образом, по-видимому сама по себе глубина обитания не оказывает влияние на темпы роста моллюсков. 
Кроме того, значительная подвижность маком затрудняет интерпретацию результатов. 
Скорость роста моллюсков определяются в первую очередь температурой и обилием пищи, а возникающая в ряде случаев зависимость от глубины может появляться за счет комбинирования этих параметров.

% MSc discussion
Поскольку время питания зависит от осушки, для Баренцева моря было проведено сравнение ростовых характеристик по горизонтам литорали. Однако выделяющиеся группы не были связаны с мареографическим уровнем.
Межгодовые различия в условиях обитания (например, масштабные температурные и соленостные колебания, характерные для Баренцева моря (Терещенко, 1997) могут вносить значительный шум в наблюдаемую картину сравнений темпов роста. 
Для того, чтобы снять их влияние, необходимо проанализировать рост особей из одной или максимально близких генераций. 
Однако, при анализе особей старше 8 лет наблюдаемая картина не отличалась от сравнения тотальных выборок.

Для ряда видов Bivalvia отмечалось определяющее влияние стартовых (ко второму сезону роста) средних размеров моллюсков на темп их роста впоследствии (в течение всего жизненного цикла). 
Так, это было показано для {\it Macoma incongrua}  в Японском море (\cite{Maximovich_Lysenko_1986}), {\it Mytilus trossulus septentrionalis} в Чаунской губе Восточно-Сибирского моря (\cite{Gagaev_et_al_1994}) и {\it Mytilus edulis} в Кандалакшском заливе (\cite{Maximovich_et_al_1993}). 
Для {\it M.~balthica} аналогичная зависимость было показана на поселениях в заливе Сан-Франциско (\cite{Cloern_Nichols_1978}). 

По нашим данным, стартовый размер особи оказывал достоверное влияние на годовой прирост, однако с увеличением стартового размера годовой прирост изменялся немонотонно — максимум приходился на стартовый размер $6-9$~мм. 
Таким образом, можно говорить об $S$-образном характере роста  {\it M.~balthica}, что характерно для живых организмов.
Более высокие значения годового прироста на нижнем горизонте литорали скорее всего связаны с условиями питания: при меньшей осушке время питания увеличивается.
Поскольку географический градиент запад-восток оказался связан с увелчением размера частиц грунта, возможно, что именно гранулометрический состав грунта влияет на годовой прирост. 

% про методы сравнения роста - сравнение коэффициентов Берталанфи


% сравнение кривых роста по Максимовичу


%скорость роста в разных частях ареала. Букма-Меган - врисовать наши точки в их картинку...
\afterpage{\clearpage}

\par\bigskip

Изучение широтных измерений характера роста {\it M.~balthica} интересовали многих исследователей (\cite{Gilbert_1973, Bachelet_1980, Beukema_Meehan_1985, Wenne_Klusek_1985, Hummel_et_al_1998}).
Для сравнения использовали различные параметры: среднюю скорость роста роста моллюсков (отношение максимальной длины к возрасту особей), коэффициент $k$ уравнения Берталанфи, параметр $\omega$ ( произведение коэффициентов $L_{\infty}$ и $k$ из уравнения роста Берталанфи), годовой прирост.

Бьёкма и Меган (\cite{Beukema_Meehan_1985}) показали, что ростовые характеристики {\it M.~balthica} имеют выраженный широтный градиент.
В качестве параметра сравнения в этой работе был использован параметр $\omega$, который считается более адекватным для задач сравнения ростовых характеристик, чем сравнение параметров уравнения Берталанффи напрямую (\cite{Appeldoorn_1983}). 
Не смотря на широкое варьирование данного параметра, наблюдается уменьшение скорости роста в более северных популяциях маком.
В данной работе данные по российской части ареала {\it M.~balthica} ограничены работой Н.~Л.~Семёновой (\cite*{Semenova_1970}).

Хюммель с соавторами (\cite{Hummel_et_al_1998}) расширили географию исследования роста маком в северных морях, проанализировав годовой прирост моллюсков из Норвежского, Печорского, Баренцева и Карского морей.
Было показано, что группировки, генетически различные по результатам аллозимного анализа, отличались по величинам годового прироста. 
Макомы в популяциях с южной границы ареала росли медленнее, чем в центральной части ареала, а размах варьирования прироста в Белом море был сравним с таковым в европейских популяциях.
Печорские макомы, значительно отличающиеся генетически, также характеризовались более низкими годовыми приростами, однако дотигали при этом наибольших размеров.

В рамках анализа полученных нами данных по росту маком в Баренцевом море, мы провели анализ широтных изменений параметра $\omega$ с использованием доступных литературных источников, добавив работы по российской части Балтийского моря и данные по Белому морю (рис.~\ref{ris:omega_vs_lat}).
	\begin{figure}[p]
	\begin{center}	
		\includegraphics[width=\textwidth]{../Growth_sravnenie/long_vs_omega_ru.pdf}
	\end{center}
		\caption{Широтное изменение ростовых характеристик {\it M.~balthica} в европейской части ареала}

	\footnotesize{Примечание: $\omega = L_{\infty} \times k$, где $L_{\infty}$ и $k$~--- коэффициенты уравнения роста Берталанффи.\\ 
	Источники см. в приложении \ref{app:growth_omega}}
		\label{ris:omega_vs_lat}
	\end{figure}
Наши данные подтверждают гипотезу о снижении скорости роста в северных частях ареала маком (корреляция Спирмена: $r_{s} = -0,60$, $p < 0,0001$).

Однако, в Балтийском море присутствуют поселения со скоростью роста, сравнимыми с характеристиками для арктических морей~--- Белого и Баренцева (рис.~\ref{ris:omega_vs_lat}). 
По-видимому, это связано с влиянием низкой солености на скорость роста (\cite{Segerstrale_1960, Kube_et_al_1996}).
Данные по Балтийскому морю наиболее разнородны: параметр $\omega$ варьирует от $1,7$ до $8,6$ (приложение~\ref{app:growth_omega}), при этом даже оценки для одного района, данные разными исследователями, могут значительно отличаться.


Для учета варьирования реальных ростовых характеристик мы сравнили имеющиеся в литературе данные и полученные нами данные с учетом разброса эмпирических данных относительно регрессионной модели.
Всего было использовано $33$ описания с $23$ географических точек на Европейском побережье Северной Атлантики (приложение~\ref{app:growth_sources}).
Мы использовали данные о первых $6$ годах роста особей, для унификации длины сравниваемых рядов.
Было выделено $6$ групп моллюсков, различающихся по ростовым характеристикам (рис.~\ref{ris:growth_cluster_literature}).
	\begin{figure}[p]
	\begin{center}	
		\includegraphics[width=\textwidth]{../Growth_sravnenie/Europe_clusters_usrednenie.pdf}
	\end{center}
		\caption{Классификация поселений маком на Европейском побережье в Северной Атлантике по моделям линейного роста}


	\footnotesize{Примечание: Дендрограмма сходства 33 рядов, аппроксимированных уравнением Берталанффи. 
Способ объединения рядов в кластеры~--- усреднение значений переменной $Y$, соответствующих одному значению $X$.
Мера сходства~--- $F/F_{kp}$ (уровень значимости $\alpha = 0,05$)

Обозначения поселений указаны в приложении~\ref{app:growth_sources} \\
Цвета: Красный~--- Баренцево море, синий~--- Белое море, голубой~--- Балтийское море, зеленый~--- Северное море, оранжевый~--- Бискайский залив}
		\label{ris:growth_cluster_literature}
	\end{figure}

Максимальная скорость роста была отмечена для группы $6$ (рис.~\ref{ris:growth_model_europe})~--- поселение в Северном море (\cite{Vogel_1959}). 
	\begin{figure}[p]
	\begin{center}	
		\includegraphics[width=\textwidth]{../Growth_sravnenie/Europe_growth_groups1.pdf}
	\end{center}
		\caption{Модели роста, передающие принципиальные свойства вариации характера линейного роста маком в Европейской части ареала}

	\footnotesize{Примечание: L, мм~--- длина раковины. Номера групп в легенде соответствуют рис.~\ref{ris:growth_cluster_literature})}
		\label{ris:growth_model_europe}
	\end{figure}
Группа $4$, в которую вошло большинство изученных нами поселений в Баренцевом море, характеризуется минимальной скоростью роста.
Также в эту группу вошла часть Беломорских поселений (\cite{Semenova_1970}) и одно поселение в Балтийском море (\cite{Bergh_1974}).
Часть исследованных поселений в Баренцевом море отличалась более высокой скоростью роста, и попала в группы $3$ (<<Беломорский>> кластер) и $1$ (Беломорские, Балтийские и Бренцевоморские поселения).
Интересно отметить, что более южные поселения (входщие в состав групп $2$ и $5$~--- <<Балтийский>> кластер), в Бискайском заливе (\cite{Bachelet_1980}), характеризуются более низкой скоростью роста , чем в центральной части ареала (рис.~\ref{ris:growth_model_europe}).
Данный результат хорошо согласуется с <<гипотезой об обилии в центре>> (<<abindant-centre hypothesis>>, \cite{Sagarin_et_al_2006}) и ранее проведенными исследованиями (\cite{Beukema_Meehan_1985, Hummel_et_al_1998}).

\afterpage{\clearpage}

%%%%%%%%%%%%%%%%%%%%%%%%%%%%%%%%%
%динамика поселений
\subsection{Долговременные тренды в поселениях \textit{Macoma~balthica}}


	\subsection{Анализ динамики численности {\it Macoma balthica} в Кандалакшском заливе Белого моря}
При изучении динамики численности можно анализировать несколько компонентов.
Первый компонент --- наличие или отсутсвие тренда как направленноого изменения численности.
При убирании тренда остается компонент динамики, для которого двумя крайими случаями будет: стабильная численность, которая поддерживается за счет плотностнозависимых процессов как систем обраной связи и неконтролируемый рост численности популяции по экспоненте.

Мы проанализировали динамику численности {\it M.~balthica} на каждом участке на наличие тренда при помощи теста Мантеля (табл.~\ref{tab:Mantel_N2_trend}).
	\begin{table}[ht]
	\caption{Выявление трендов в динамике численности {\it Macoma balthica} на различных участках Белого моря.}
	\label{tab:Mantel_N2_trend}
        \begin{tabular}{|p{0.25\textwidth}|*{2}{p{0.2\textwidth}|p{0.25\textwidth}|}} \hline
	Участок & $Mantel$ & $p$ & наличие тренда
	\\ \hline
	Эстуарий р. Лувеньга & 0,3168 & 0,003 & есть
	\\ \hline
	о. Горелый & 0,0269 & 0,368 & нет
	\\ \hline
	материковая литораль (Лувеньга) & 0,6103 & 0,001 & есть
	\\ \hline
	Южная губа о. Ряшков & 0,3687 & 0,015 & есть
	\\ \hline
	Запдная Ряшкова салма & 0,0108 & 0,404 & нет
	\\ \hline
	Ломнишный & -0,0999 & 0,47 & нет
	\\ \hline
	г. Медвежья & 0,0154 & 0,385 & нет
	\\ \hline
	г. Сельдяная & 0,2524 & 0,003 & есть
	\\ \hline
	\end{tabular}
	%    {\footnotesize Примечание: достоверность различий *** \textemdash $p<0,001$; ** \textemdash $p<0,05$; * \textemdash $p<0,1$.}
	\end{table}

Было показано наличие тренда на 4 участках: эстуарий р.~Лувеньга, материковая литораль в районе пос. Лувеньга, Южная губа о.~Ряшкова, г. Сельдяная.
Для удаления тренда из исходных значений были вычтены предсказанные значения из регрессионной модели $N = a + b*T$, где $N$ --- численность, экз./м$^2$, $T$ --- годы.
По детрендированному ряду были рассчитаны частные автокорреляции ($PRCF$ - partial rate correlation function).  
Коррелограммы представлены на рисунке \ref{ris:perm_PRCF_Kandalaksha_N2_detrend}.
	\begin{figure}[ht]
	
	\begin{minipage}[b]{.46\linewidth}
	%Фигурка в первом ряду слева размер отведенный под весь этот объект \textendash 0.46 от ширины строки
	%Параметр [b] означает, что выравнивание этих министраниц будет по нижнему краю
	\begin{center}
	{\footnotesize Эстуарий р.~Лувеньги}
		\includegraphics[width=65mm]{../White_Sea/dynamic_N_N1/perm_PRCF_Estuary_detrend.pdf}

	\end{center}
	\end{minipage}
		\hfil %Это пружинка отодвигающая рисунки друг от друга
	\begin{minipage}[b]{.46\linewidth}
%Следующий рисунок - первый ряд справа %DUNGEON S_4 \ AB
	\begin{center}
	{\footnotesize о.~Горелый}
		\includegraphics[width=65mm]{../White_Sea/dynamic_N_N1/perm_PRCF_Goreliy_all_detrend.pdf}
	\end{center}
	\end{minipage}

	\begin{minipage}[b]{.46\linewidth}
%Фигурка в первом ряду слева размер отведенный под весь этот объект \textendash 0.46 от ширины строки
%Параметр [b] означает, что выравнивание этих министраниц будет по нижнему краю
	\begin{center}
	{\footnotesize материковая литораль (Лувеньга)}
	\includegraphics[width=65mm]{../White_Sea/dynamic_N_N1/perm_PRCF_razrez2_all_detrend.pdf}
	\end{center}
	\end{minipage}
		\hfil %Это пружинка отодвигающая рисунки друг от друга
	\begin{minipage}[b]{.46\linewidth}
%Следующий рисунок - первый ряд справа %DUNGEON S_4 \ AB
	\begin{center}
	{\footnotesize о.~Ломнишный}
	\includegraphics[width=65mm]{../White_Sea/dynamic_N_N1/perm_PRCF_Lomnishniy_detrend.pdf}
	\end{center}
	\end{minipage}

	\begin{minipage}[b]{.46\linewidth}
%Фигурка в первом ряду слева размер отведенный под весь этот объект \textendash 0.46 от ширины строки
%Параметр [b] означает, что выравнивание этих министраниц будет по нижнему краю
	\begin{center}
	{\footnotesize Южная губа о.~Ряшкова}
	\includegraphics[width=65mm]{../White_Sea/dynamic_N_N1/perm_PRCF_YuG_detrend.pdf}
	\end{center}
	\end{minipage}
		\hfil %Это пружинка отодвигающая рисунки друг от друга
	\begin{minipage}[b]{.46\linewidth}
	\begin{center}	
	{\footnotesize Западная Ряшкова салма}
	\includegraphics[width=65mm]{../White_Sea/dynamic_N_N1/perm_PRCF_ZRS_detrend.pdf}
	\end{center}
	\end{minipage}
	\caption{Частные корреляции численности {\it Macoma balthica} (без учета особей длиной менее 1 мм) в Кандалакшском заливе. Детрендированные данные. Оценка достоверности пермутационным методом.}
	\label{ris:perm_PRCF_Kandalaksha_N2_detrend}	
	\end{figure}

	\begin{figure}[ht]
%\smallskip

	\begin{minipage}[b]{.46\linewidth}
%Фигурка в первом ряду слева размер отведенный под весь этот объект \textendash 0.46 от ширины строки
%Параметр [b] означает, что выравнивание этих министраниц будет по нижнему краю
	\begin{center}
	{\tiny Медвежья}
	\includegraphics[width=65mm]{../White_Sea/dynamic_N_N1/perm_PRCF_Medvezhya_detrend.pdf}
	\end{center}
	\end{minipage}
%
	\hfil %Это пружинка отодвигающая рисунки друг от друга
%
	\begin{minipage}[b]{.46\linewidth}
	\begin{center}
	{\tiny Сельдяная}
	\includegraphics[width=65mm]{../White_Sea/dynamic_N_N1/perm_PRCF_Seldyanaya_detrend.pdf}
	\end{center}
	\end{minipage}

%\smallskip
%	\caption{Динамика плотности поселений {\it Macoma balthica} в вершине Кандалакшского залива}
%	\label{ris:dynamic_Kandalaksha_all}
\begin{center}
Рисунок \ref{ris:perm_PRCF_Kandalaksha_N2_detrend}, продолжение. Частные автокорреляции численности {\it Macoma balthica} (без учета особей длиной менее 1 мм) в Кандалакшском заливе. Детреднированные данные. Оценка достоверности пермутационным методом.
\end{center}
	\end{figure}
Для большинства временных рядов значение максимального значения достигает $PRCF$ с лагом 1, что характерно для динамики в отсутствие тренда. 
Достоверность частных автокорреляций оценивалась пермутационным методом.
Для участков в Южной губе о.~Ряшкова и на материковой литорали в Лувеньге были показаны достоверные значений $PRCF[2]$, причем в Южной губе $PRCF[2] > PRCF[1]$. 
Это показывает наличие в поселении плотностнозависимых процессов второго порядка.
Предположительно, это может быть воздействие хищников.
Мы надеемся проверить эту гипотезу в ходе дальнейших наблюдений.
Биологическая интерпретация $PRCF$ с большим лагом на настоящий момент представляется нам сомнительной.

		\subsection{Синхронность динамики численности {\it Macoma balthica} в Кандалакшском заливе Белого моря}
Для изучения синхронности колебаний численности маком мы использовали тест Мантеля.
Для включения большего количества рядов в анализ, он был проведен по двум наборам данных.
Первый набор данных включал участки, где при отборе проб промывка была на сите с диаметром ячеи $0,5$~мм. 
Сюда вошли участки в эстуарии р.~Лувеньги, на материковой литорали в районе Лувеньги, на о.~Горелый, в Западной Ряшковой салме и в губах Медвежья и Сельдяная.
Результаты корреляционного анализа представлены в таблице \ref{tab:Mantel_dynamic_N}.
	\begin{table}[ht]
	\caption{Синхронность динамики численности {\it Macoma balthica}.}
	\label{tab:Mantel_dynamic_N}
        \begin{tabular}{|p{0.2\textwidth}|*{8}{p{0.08\textwidth}|}} \hline
	$Mantel r \setminus p_{perm}$ & [1] & [2] & [3] & [4] & [5] & [6] & [7] & [8]
    \\ \hline
	[1] эстуарий р.~Лувеньги & & \cellcolor{yellow}{$0,002$} & $0,989$ & \cellcolor{yellow}{$0,009$} & \cellcolor{yellow}{$0,001$} & $0,264$ & \cellcolor{yellow}{$0,018$} & $0,441$
	\\ \hline
	[2] о.~Горелый & \cellcolor{yellow}{$0,929$} & & $0,393$ & \cellcolor{yellow}{$0,014$} & \cellcolor{yellow}{$0,001$} & $0,388$ & $0,992$ & $0,089$
	\\ \hline
	[3] о.~Ломнишный & $-0,439$ & $-0,067$ & & $0,208$ & $NA$ & $0,79$ & $0,082$ & $0,399$
	\\ \hline
 	[4] г.~Медвежья & \cellcolor{yellow}{$0,821$} & \cellcolor{yellow}{$0,86$} & $-0,028$ & & \cellcolor{yellow}{$0,001$} & $0,184$ & $0,932$ & $0,441$
	\\ \hline
	[5] материковая литораль (Лувеньга) & \cellcolor{yellow}{$0,781$} & \cellcolor{yellow}{$0,784$} & $NA$ & \cellcolor{yellow}{$0,704$} & & \cellcolor{yellow}{$0,044$} & $NA$ & $0,123$
	\\ \hline
	[6] г.~Сельдяная & $0,089$ & $-0,009$ & $-0,303$ & $0,087$ & \cellcolor{yellow}{$0,364$} & & $0,763$ & $0,818$
	\\ \hline
	[7] Южная губа о.~Ряшкова & \cellcolor{yellow}{$0,427$} & $-0,309$ & $0,333$ & $-0,213$ & $NA$ & $-0,127$ & & $0,585$
	\\ \hline
	[7] Западная Ряшкова салма & $-0,045$ & $0,057$ & $0$ & $-0,05$ & $0,284$ & $-0,141$ & $-0,038$
	\\ \hline
	\end{tabular}
	   {\footnotesize Примечание: Нижняя половина таблицы --- значение теста Мантеля, верхняя половина --- уровень значимости, определенный пермутационным методом. \\
Желтым выделены значения с уровнем значимости $< 0,1$. \\
$NA$ --- ряды не пересекаются во времени.}
	\end{table}
Три участка в районе Лувеньгских шхер (эстуарий р.~Лувеньги, о.~Горелый, материковая литораль) демонстрировали синхронную динамику поселений.
С данными участками была синхронна динамика поселения маком в г.~Медвежья. 
Низкая, хотя и достоверная корреляция была показана между динамикой на материковой литорали в районе Лувеньги и в г.~Сельдяной ($0,36$) и между эстуарием р.~Лувеньги и Южной губой о.~Ряшкова ($0,43$).


Второй набор данных включал участки, где при отборе проб промывку проводили на сите с диаметром ячеи $1$~мм.
Также сюда вошли те участки из предыдущего набора данных, где была известна размерная структура моллюсков --- из общей численности были вычтены численность особей длиной менее $1$~мм для возмодности сравнения.
Всего в данный анализ вошло 8 рядов данных: эстуарий р.~Лувеньги, материковая литораль в районе Лувеньги, о.~Горелый, Западная Ряшкова салма, Южная губа о.~ Ряшкова, о.~Ломнишный, б.~Клющиха и Сухая салма (табл.~\ref{tab:Mantel_dynamic_N2}).
	\begin{table}[ht]
	\caption{Синхронность динамики численности {\it Macoma balthica}.}
	\label{tab:Mantel_dynamic_N2}
        \begin{tabular}{|p{0.2\textwidth}|*{8}{p{0.08\textwidth}|}} \hline
	$Mantel r \setminus p_{perm}$ & [1] & [2] & [3] & [4] & [5] & [6] & [7] & [8]
	\\ \hline
	[1] эстуарий р.~Лувеньги & & $0,082$ & $0,646$ & $0,995$ & \cellcolor{yellow}{$0,029$} & $0,482$ & \cellcolor{yellow}{$0,013$} & $0,19$
	\\ \hline
	[2] о.~Горелый & $0,176$ &  & $0,067$ & $0,73$ & \cellcolor{yellow}{$0,001$} & $0,261$ & $0,986$ & \cellcolor{yellow}{$0,001$}
	\\ \hline
	[3] б.~Клющиха & $-0,046$ & $0,52$ &  & $0,673$ & \cellcolor{yellow}{$0,034$} & $0,213$ & $0,062$ & $0,065$
	\\ \hline
	[4] о.~Ломнишный & $-0,451$ & $-0,181$ & $-0,22$ &  & $NA$ & $1$ & $0,088$ & $0,341$
	\\ \hline
	[5] материковая литораль (Лувеньга) & \cellcolor{yellow}{$0,32$} & \cellcolor{yellow}{$0,862$} & \cellcolor{yellow}{$0,577$} & $NA$ &  & $0,117$ & $NA$ & \cellcolor{yellow}{$0,006$}
	\\ \hline
	[6]Сухая салма & $-0,019$ & $0,067$ & $0,085$ & $-1$ & $0,443$ &  & $0,688$ & $0,314$
	\\ \hline
	[7] Южная губа о.~ Ряшкова & \cellcolor{yellow}{$0,419$} & $-0,332$ & $0,434$ & $0,333$ & $NA$ & $-0,243$ &  & $0,605$
	\\ \hline
	[8] Западная Ряшкова салма & $0,114$ & \cellcolor{yellow}{$0,86$} & $0,72$ & $0,093$ & \cellcolor{yellow}{$0,755$} & $0,088$ & $-0,048$ & 
	\\ \hline
	\end{tabular}
	   {\footnotesize Примечание: Нижняя половина таблицы --- значение теста Мантеля, верхняя половина --- уровень значимости, определенный пермутационным методом. \\
Желтым выделены значения с уровнем значимости $< 0,05$. \\
$NA$ --- ряды не пересекаются во времени.}
	\end{table}
Интересно отметить, что при редукции данных до численности особей длиной более $1$~мм картина меняется.
Без изменения остается синхронность динамик поселений маком на материковой литорали в Лувеньге c о.~Горелый и эстуарием р.~Лувеньги.
Такжесохранияется синхронность динамик численности в поселениях в эстуарии р.~Лувеньга и Южной губе о.~Ряшкова.
В то же время поселение в Западной Ряшковой салме, который в предыдущем анализе показывало асинхронность по сравнению с остальными участками, в данном случае демонстрирует синхронность с поселениями на о.~Горелый и материковой литорали в Лувеньге.
Также показана синхронность динамик поселений на материковой литорали в Лувеньге и в бухте Клющиха.

Мы использовали значение теста Мантеля как меру сходства рядов данных для тестирования гипотезы, что на более близкорасположенных участках динамика численности {\it Macoma balthica} более сходна.
Для этого по координатам участков была рассчитана матрица расстояний между участками (табл.~\ref{tab:distance_area_km}).
	\begin{table}[ht]
	\caption{Расстояние между исследованными участками литорали.}
	\label{tab:distance_area_km}
        \begin{tabular}{|p{0.3\textwidth}|*{10}{p{0.04\textwidth}|}} \hline
	 & [1] & [2] & [3] & [4] & [5] & [6] & [7] & [8] & [9] & [10]
	\\ \hline
	[1] материковая литораль (Лувеньга) & 0,0 &  &  &  &  &  &  &  &  & 
	\\ \cline{1-3}
	[2] о.~Горелый & 1,5 & 0,0 &  &  &  &  &  &  &  &  
	\\ \cline{1-4}
	[3]эстуарий р.~Лувеньги & 1,0 & 1,0 & 0,0 &  &  &  &  &  &  &  
	\\ \cline{1-5}
	[4] Южная губа о.~Ряшкова & 11,7 & 10,7 & 11,7 & 0,0 &  &  &  &  &  & 
	\\ \cline{1-6}
	[5] о.~Ломнишный & 13,5 & 12,9 & 13,8 & 3,7 & 0,0 &  &  &  &  &  
	\\ \cline{1-7}
	[6] Западная Ряшкова салма & 11,9 & 10,8 & 11,8 & 1,7 & 5,3 & 0,0 &  &  &  &  
	\\ \cline{1-8}
	[7] г.~Сельдяная & 93,6 & 94,0 & 94,5 & 87,8 & 84,1 & 89,3 & 0,0 &  &  &  
	\\ \cline{1-9}
	[8] г.~Медвежья & 91,9 & 92,4 & 92,8 & 86,1 & 82,4 & 87,6 & 1,7 & 0,0 &  &  
	\\ \cline{1-10}
	[9] Сухая салма & 97,1 & 97,5 & 97,9 & 91,2 & 87,6 & 92,7 & 3,5 & 5,1 & 0,0 &  
	\\ \hline
	[10] б.~Клющиха & 100,1 & 100,6 & 101,0 & 94,8 & 91,1 & 96,3 & 8,1 & 9,7 & 5,8 & 0,0
	\\ \hline
	\end{tabular}
	   {\footnotesize Примечание:Расстояние дано в километрах.}
	\end{table}

Для обоих наборов данных тест Мантеля показал отсутсвие зависимости сходства динамики численности маком от расстояния ( $Mantel r = --0,058 (p_{perm} = 0,746)$ и $Mantel r = -0,105 (p_{perm} = 0,638)$ для первого и второго набора данных, соответственно).



\subsection{Влияние температуры на обилие \textit{Macoma~balthica}}
\textit{M.~balthica} --- вид, обладающий планктонной личинкой, при этом в условиях Белого моря от стадии велигера до метаморфоза и оседания проходит около месяца ($25 - 30$ суток) (\cite{Flyachinskaya_1999}). 
Известно, что общий личиночный пул формируется для достаточно крупных акваторий (\cite{Maximovich_Shilin_2012}). 
Поэтому расположенные на расстоянии около километра исследованные поселения, скорее всего, пополняются за счет общего личиночного пула, что влияет на синхронизацию динамики поселений. 
Однако данные по другим акваториям (\cite{Varfolomeeva_Naumov_2013}; А.В.~Герасимова, личное сообщение) показывают, что по крайней мере в $1998 - 1999$ году увеличение численности наблюдалось в разных районах Кандалакшского залива. 
Это дает основание предполагать влияние глобальных абиотических факторов, первым из которых может быть температура. 

Для проверки влияния температуры на динамику обилия \textit{M.~balthica} было проведено моделирование и использованием линейных моделей. 
Были использованы данные о температуре воздуха в Кандалакше. 
Полная модель включала в себя независимую переменную среднюю численность маком в данный год ($N_{t1}$) и независимые факторы: численность маком в предыдущий год ($N_{t}$), среднелетнюю температуру в предыдущий год ($T_{st}$) как отражение условий созревание гонад и формирования спата и среднезимнюю температуру в текущий год ($T_{wt1}$) как отражение критических условий первой зимы для сеголетков. 
Для выполнения условия о линейности зависимости, а также уменьшения воздействия влиятельных наблюдений в модели были использованы логарифмированные значения численности. 
В дальнейшем модель была редуцирована (полная и минимальная модели, ANOVA: $F = 0,43$; $p = 0,79$) и в минимальную модель в качестве факторов были включены $N_{t}$ и $T_{wt1}$. 
Характеристики полученной модели приведены в таблице~\ref{tab:model_koeff}. 
	\begin{table}[p]
	\caption{Характеристики модели зависимости обилия маком от их обилия в предыдущий год и зимней температуры.}
	\label{tab:model_koeff}
		\begin{tabularx}{\textwidth}{|X|X|X|X|X|}
			\hline
			факторы & Оценки  коэффициентов модели & Стандартная ошибка коэффициентов модели & $t$ & $P$ \\ \hline
			Свободный член & $1,96$ & $0,664$ & $2,96$ & $0,005$ \\ \hline
			$\ln(N_{t})$ & $0,60$ & $0,071$ & $8,44$ & $<0,0001$ \\ \hline
			$T_{wt1}$ & $-0,09$ & $0,036$ & $-2,50$ & $0,016$ \\ \hline
		\end{tabularx}
	\end{table}
Построенная модель удовлетворяла условиям применимости линейных моделей: отсутствия автокорреляций (критерий Дарбина-Уотсона: $1,71$; $p = 0,27$), нормальности распределения остатков (критерий Шапиро-Уилка: $W = 0,99$; $p = 0,86$) и гомогенности дисперсий (критерий Бройша-Пагана: $BP = 5,25$; $p-value = 0,15$). 
Таким образом, связь между обилием маком в текущий и в предыдущий год и зимней температурой описывается моделью вида: $\ln(N_{t1}) = 1,96 + 0,60 \times \ln(N_{t}) - 0,09 \times T_{wt1}$ ($F = 37,04$; $p < 0,0001$. $R^2 = 0,6$) (рис.~\ref{ris:model_temperature}).
	\begin{figure}[p]
		\includegraphics[height=0.4\textheight]{../article_Macoma_dynamic_White_sea/N_vs_temperature/lodNt_vs_logNt1_1.pdf}

		\includegraphics[height=0.4\textheight]{../article_Macoma_dynamic_White_sea/N_vs_temperature/Twt1_vs_logNt1_1.pdf}

	\caption{Зависимость численности  \textit{Macoma balthica} ($\ln(N_{t1})$) от численности в предыдущий год ($\ln(N_{t})$) и зимней температуры ($T_{wt1}$).}
	\label{ris:model_temperature}
	\footnotesize{Показаны линейная модель (синяя линия) и ее 95\% доверительный интервал (серая область).}
	\end{figure}

Полученные данные о влиянии зимней температуры противоречат нашей исходной гипотезе о том, что холодные зимы в Белом море критичны для маком. 
Результаты моделирования позволяют говорить о том, что обилие маком увеличивается после более холодных зим и уменьшается после относительно теплых (рис.~\ref{ris:model_temperature}). 
Данный результат хорошо согласуется с результатами полученными Бьёкема с соавторами (\cite{Beukema_et_al_1998, Beukema_et_al_2009}) для Ваттового моря. 

Для данной акватории было показано, что одним из ключевых факторов, влияющих на пополнение поселений \textit{M.~balthica}, является температура в зимний период. 
Пополнение после суровых зим было больше, чем после мягких. 
Было предложено два механизма, зависимые от зимней температуры: 
1. количество яиц маком, выметанных в апреле больше после холодной зимы, поскольку при низкой температуре меньше уровень обмена, а, значит, меньше потери веса за зиму, и больше энергии остается на продукцию. 
2. Биомасса \textit{Crangon crangon}, одного из важных хищников для маком, была значительно выше после холодных зим. 
При проверке обеих гипотез, было показано, что второй механизм влияет значительно сильнее (\cite{Beukema_et_al_1998, Beukema_Dekker_2014, Dekker_Beukema_2014}). 

В настоящее время у нас нет прямых данных, позволяющих говорить о механизмах влияния температуры на \textit{M.~balthica} в Белом море. 
Проведение аналогий с Ваттовым морем затруднено, поскольку считается, что роль хищников снижается в более полярных сообществах (\cite{Pianka_1966, Freestone_et_al_2011}). 
Возможно, уменьшение обилия маком после теплых зим связано с тем, что при более теплых зимах ледостав менее стабилен, и литораль во время отлива оказывается напрямую подвержена воздействию отрицательных температур воздуха, в то время как в холодные зимы стабильный ледовый покров создает изолирующий слой, и колебания температуры подо льдом оказываются значительно ниже (\cite{Kuznecov_1960}).

\afterpage{\clearpage}


\afterpage{\clearpage}
%%%%%%%%%%%%%%%%%%%%%%%%%
	\subsection{Особенности пополнения поселений \textit{Macoma balthica} в Белом и Баренцевом морях}
Большинство исследований, посвященных проблеме пополнения, выполнено в одном из районов Северного моря~--- так называемом Ваттовом море.	
Изначально было показано, что одним из ключевых факторов, влияющих на пополнение поселений Macoma balthica, является температура в зимний период, которая воздействует не напрямую, а через влияние на обилие хищников (\cite{Beukema_et_al_1998, Beukema_Dekker_2014, Dekker_Beukema_2014})

В ряде других работ также было показано влияние различных хищников на численность и распределение молоди маком. 
Так, для Ваттового моря именно обилием хищников объясняется формирование временных скоплений молоди маком на верхней литорали. 
При изучении факторов, обуславливающих такое распределение для \textit{M.~balthica} было показано, что обилие бентосных хищников больше на нижней литорали, и лишь молодь краба \textit{Carcinus maenas} в значительных количествах встречается на верхней литорали. 
В полевых и лабораторных экспериментах было показано, что присутствие хищников значительно снижает численность спата, в то время как влияния на крупных особей обнаружено не было. 
По-видимому, за первый год макомы выходят из-под контроля бентосными хищниками за счет увеличения размеров тела (\cite{Hiddink_et_al_2002_predation_epifauna}). 

Также при анализе динамики личинок различных беспозвоночных в планктоне было показано, что после суровых зим численность личинок краба \textit{Carcinus maenas} значительно снижалась, и они появлялись в планктоне на $6-8$ недель позже, чем после мягких зим. 
По-видимому, с этим временным несоответствием связано большее пополнение поселений маком после суровых зим (\cite{Strasser_Gunter_2001}).

В более поздних исследованиях на Ваттовом море было показано, что влияние суровых зим на пополнение \textit{M.~balthica} не столь широкомасштабно, как предполагалось ранее, и, по-видимому, существуют другие факторы, определяющие более локальные вариации в пополнении поселений (\cite{Strasser_et_al_2003, Flatch_2003}).
Пресс хищников не объяснил эти различия, изменения сообществ и поступления биогенных элементов не объяснили картину, поскольку действовали лишь на отдельных участках. 
Наиболее вероятным фактором, по мнению данных авторов, является топографическая разница между двумя акваториями, где располагались исследованные участки. 
Предполагается, что в зависимости от закрытости акватории островами, и преобладающего направления ветров, будет идти более или менее эффективный перенос личинок и биссусный дрифт, а, значит, и пополнение поселения (\cite{Strasser_et_al_2003}).

Для другого участка Ваттового моря было показано, что комбинация эффектов высокого пресса хищников вместе с высоким обилием взрослой макрофауны обуславливает 95 процентное снижение количества спата теллинид (\textit{M.~balthica} и \textit{Tellina tenuis}) после мягких зим (\cite{Flatch_2003}). 

Хотя влияние на пополнение поселения молодью плотности взрослых особей того же вида представляется достаточно логичным механизмом, существуют лишь отдельные работы, посвященные внутривидовым взаимодействиям у \textit{M.~balthica}. 
Так, в ряде работ показано, что плотность молоди не зависит от обилия взрослых маком (\cite{Olafsson_1989, Vincent_et_al_1989, Beukema_et_al_2001, Richards_et_al_2002}). 

 Также было показано, что влияние плотности взрослых маком на рост спата зависит от типа грунта. 
На илисто-песчаном грунте, где и взрослые, и молодые моллюски питаются как собирающие детритофаги, рост спата подавляется при увеличении плотности взрослых особей. 
На песке, где молодь питаются как собирающие детритофаги, а взрослые~--- как фильтраторы, влияния на рост спата показано не было (\cite{Olafsson_1989}).

Для Белого моря существуют лишь несколько работ, посвященных отдельным аспектам пополнения поселений маком. 
Так, И.В.Бурковским с соавторами показано, что макомы оседают вне плотных поселений взрослых (\cite{Burkovskiy_et_al_1998}). 
Также показано, что важную роль в динамике численности личинок и спата влияет принос личинок с соседних акваторий. 
В течение лета формируется сначала бимодальная размерная структура спата, с двумя пиками личинок в планктоне, которая к концу августа сливается в мономодальную (\cite{Zubakha_et_al_2000}). 
Показана высокая смертность особей на всех этапах пополнения поселения. Так, смертность пелагических личинок оценивают в $36,4$\% за сезон, а смертность спата~--- $59$\% за сезон (\cite{Burkovskiy_et_al_1998}).

\par\bigskip
%обсуждение
По нашим данным численность спата была одного порядка на трех участках ($4-5$~тыс.~экз./м$^2$), но в проливе Подпахта была выше на порядок (более $10$~тыс.~экз./м$^2$). 
Интересно отметить, что высокая плотность спата была отмечена именно в Подпахте, т.е. на участке с минимальной численностью взрослых особей. 

Учитывая имеющиеся оценки смертности спата нам показалось интересным посмотреть соотношение спата этого года и сеголетков. 
Сеголетками (возраст $1+$) считали особей длиной $1,1-2,0$~мм в соответствии с работой Н.В.Максимовича с соавторами, выполненной в исследованной акватории (\cite{Maximovich_et_al_1992}). 
Смертность спата маком за сезон оценивается в $59.1$\% (\cite{Burkovskiy_et_al_1998}). Полученные расчетные величины представлены в табл.~\ref{tab:spat_rasschet} и \ref{tab:N1_rasschet}.

\begin{table}[p]
\caption{Предположительное пополнение исследованных поселений \textit{Macoma balthica} в $2005$ году, рассчитанные на основе оценки смертности спата.}
\label{tab:spat_rasschet}
\begin{center}
\begin{tabular}{|l|c|c|}
\hline
                & $2005$ (расчет) & $2006$ \\ 
возраст         & $N_{sp}$      & $N_{1+}$  \\ \hline
Сухая салма     & $473$           & $194$  \\ \hline
бухта Лисья     & $415$           & $170$  \\ \hline
пролив Подпахта & $166$           & $68$   \\ \hline
бухта Клющиха   & $351$           & $144$  \\ \hline
\end{tabular}
\end{center}
	\footnotesize{Примечание: $N_{sp}$.~--- численность спата, экз./м$^2$, $N_{1+}$~--- численность сеголетков, экз./м$^2$.}

\end{table}


\begin{table}[p]
\caption{Предположительная эффективность пополнения исследованных поселений Macoma balthica к 2007 году, рассчитанные на основе оценки смертности спата.}
\label{tab:N1_rasschet}
\begin{center}
\begin{tabular}{|l|c|c|}
\hline
                & $2006$  & $2007$ (расчет) \\
возраст         & $N_{sp}$      & $N_{1+}$  \\ \hline
Сухая салма     & $4980$  & $2042$          \\ \hline
бухта Лисья     & $4040$  & $1656$          \\ \hline
пролив Подпахта & $4240$  & $1738$          \\ \hline
бухта Клющиха   & $10060$ & $4125$         \\ \hline
\end{tabular}
\end{center}
	\footnotesize{Примечание: $N_{sp}$.~--- численность спата, экз./м$^2$, $N_{1+}$~--- численность сеголетков, экз./м$^2$.}
\end{table}

Таким образом, расчетные величины обилия спата в $2005$ году на порядок отличаются от величин, показанных для $2006$ года. 
Возможно, это связано с значительными межгодовыми различиями в пополнении поселений, показанных для других участков (рис.~\ref{ris:dynamic_1year_Kandalaksha}). 
Также это может быть связано с тем, что приведенная оценка сделана для смертности за сезон, и смертность за последующую зиму может значительно занижать нашу оценку пополнения в $2005$ году. 

Размерная структура спата на всех исследованных участках характеризуется мономодальностью. 
Подобные данные были получены М.А. Зубахой с соавторами (2000), однако в данной работе было показано, что мономодальное распределение спата формируется в конце лета. 
Изначально при оседании формируется бимодальная размерная структура, связанная с двумя пиками численности личинок в планктоне, и затем за счет различной скорости роста личинок два пика постепенно сливаются (\cite{Zubakha_et_al_2000}).
На исследованных участках максимальный размер плантиграды имели на участке в бухте Клющиха ($0,4 - 1,5$~мм с модой $0,75$~мм), а минимальный в проливе Подпахта ($0,35 - 0,8$~мм с модой $0,5$~мм). 
Это хорошо согласуется с данными Е. Олафссона, который показал, что на песчаных грунтах нет подавления роста спата взрослыми особями, наблюдаемого на илисто-песчаном грунте (\cite{Olafsson_1989}). 
Участок в бухте Клющиха отличался отсутствием алевритов и пелитов (табл.~\ref{tab:grunt_granulometriya_White}), в то время как остальные характеризовались значительным заилением. 
В $1998$ году на участке Сухая салма к $25$~августа моду формировали особи длиной $0,55 - 0,75$~мм с небольшим преобладанием группы $0,65$~мм (\cite{Zubakha_et_al_2000}). 
По нашим данным к $20$~августа структура поселения была с выраженным пиком при длине спата $0,65$~мкм. 
Разброс размеров в $1998$ году был от $0,35$ до $0,95$~мм, а в $2006$ от $0,3$ до $0,85$~мм, то есть в $2006$ году особи были более мелкие, не смотря на более поздние сроки сбора материала. 

При анализе корреляции количества спата и обилием взрослых особей \textit{M.~balthica} было показано, что с биомассой достоверной корреляции нет, а есть отрицательная с численностью. 
Между тем, если предполагать трофическую или топическую конкуренцию, то следовало бы ожидать наличия именно отрицательной корреляции с биомассой, поскольку более крупные макомы имеют больший радиус облова и отбирают на себя больший поток энергии (\cite{Olafsson_1989, Zwarts_et_al_1994}). 
Тогда возникло предположение, что такая картина может объясняться взаимодействием спата с более мелкими, но более многочисленными макомами. 
Однако при анализе корреляции численности спата и количества маком различных размеров это показать не удалось.

Дисперсионный анализ показал, что численность спата сильно варьирует в зависимости от участка, и фактор участок определяет $45 \pm 6,8$\% вариации. 
Это может быть связано с сильной вариабельностью численности личинок в планктоне на различных участках (\cite{Maximovich_Shilin_2012}). 
Кроме того, поскольку в данном исследовании не проводилось наблюдение за динамикой оседания спата, то наблюдаемая картина является результирующей оседания и последующего перераспределения маком за счет биссусного дрифта (\cite{Armonies_Hellwig-Armonies_1992, Huxham_Richards_2003}).
Хотя для фактора численность взрослых маком сила влияния недостоверна, но поскольку анализ силы влияния фактора более слабый, чем дисперсионный анализ, то можно говорить о влиянии численности взрослых маком на количество спата. Но оценить влияние на имеющемся материале невозможно.

Интересные результаты получились при анализе влияния местообитания и численности взрослых маком на отдельные размерные группы маком. 
Для особей длиной более $5$~мм характер влияния факторов аналогичен таковому у спата, в то время как для особей длиной $1,1 - 5,0$~мм влияние фактора <<участок>> недостоверно, а влияние численности взрослых маком больше. 
Можно было бы предположить, что именно количество особей длиной $1,1 - 5,0$~мм определяет численность маком в поселении, однако корреляция между этими параметрами оказалась недостоверной.

Попытка выявить влияние на численность спата маком обилия макрозообентоса, что было показано в Ваттовом море (\cite{Flatch_2003}), не показала достоверной связи между данными показателями. 
Возможно, это связано с тем, что в Ваттовом море в исследованных поселениях обилие определялось количеством двустворчатого моллюска \textit{Cerastoderma edule} и пескожила \textit{Arenicola marina}, в то время как в исследованных нами поселениях не было столь крупных и активно изменяющих среду организмов. 

\par\bigskip
В данной работе мы оценивали успешность пополнения беломорских поселений \textit{M.~balthica} по численности годовалых особей.
Данный показатель варьировал в значительных пределах: от $0$ до $5,5$~тыс.~экз./м$^2$.
Таким образом, исследованные поселения демонстрируют характерную для Белого моря нерегулярность пополнения (\cite{Maximovich_et_al_1991, Maximovich_Gerasimova_2004, Gerasimova_Maximovich_2009}).

Считается, что пополнение локальных поселений массовых бентосных организмов с планктонной личинкой не зависит от количества половозрелых особей в нем, поскольку единый личиночный пул в планктоне формируется за счет всех половозрелых особей в гидрологически-замкнутой акватории (\cite{Maximovich_Shilin_2012}.
Мы попробовали на имеющихся материалах проверить данную гипотезу.
Поскольку для маком в Белом море показано (\cite{Semenova_1980, Maximovich_1985}), что ключевым фактором для возможности половозрелости является именно размер, а не возраст животного, и этот размер для макомы составляет 8 мм, мы оценивали корреляцию численности годовалых особей в поселении с численностью особей длиной более $8$~мм в предыдущий год (т.е.в год оседания).
Хотя была обнаружена низкая достоверная корреляция данных параметров, очевидно (рис.~\ref{ris:N1year_vs_N8mm}) что разброс данных величин достаточно высокий и влияние данного вактора невелико.
В пользу гипотезы о формировании общего личиночного пула на значительной акватории говорит и синхронность пополнения, наблюдаемая в ряде исследованных поселений (табл.~\ref{tab:Mantel_dynamic_N1y}).
Единственное поселение, для которого не показана синхроность с остальными --- на острове Ломнишном, наиболее удаленном поселении от остальных участков.
Однако прямой связи расстояния со степенью синхронности пополнения поселений обнаружено не было.

В Баренцевом море мы не проводили анализа пополнения, однако по данным размерной структуры можно сделать некоторые выводы.
На Дальнем пляже особи размером $2-3$~мм встречаются ежегодно, хотя бы в единичном количестве.
В данном районе такой размер характерен для маком возрастом $1+$ (\cite{Nazarova_et_al_2010}), таким образом, можно говорить о регулярном пополнении поселений молодью. 
Однако эффективность пополнения различается год от года. 
Наиболее успешные пополнения поселения молодью, по-видимому, происходили в $2005-2007$ годах, что и обусловило увеличение численности маком в $2006-2008$ годах на данном участке.
Таким образом, значительные межгодовые различия в эффективности пополнения поселений маком молодью характерны как для Белого, так и для Баренцева морей.



\afterpage{\clearpage}


%%    макома как массовый элемент в сообществах литорали северных морей
%%    об организации сообществ с участием Macoma balthica
%%    Структура популяции Mb как отражение характеристик экотопов
%%    скорость роста как показатель "условий жизни"
%%    о долговременных трендах в популяции Mb

	\chapter*{Заключение}		
\addcontentsline{toc}{chapter}{Заключение}	% Добавляем его в оглавление

Двустворчатый моллюск \textit{Macoma balthica} является типичным представителем литоральной фауны в Белом и Баренцевом морях. 
В Белом море данный вид формирует плотные скопления, причем поселения в Кандалакшском заливе характеризуются максимальной средней численностью в европейской части ареала вида.
В Баренцевом море \textit{M.~balthica} на литорали Западного Мурмана и Кольского залива также формирует плотные поселения, в то время как на литорали Восточного Мурмана численность данного вида редко превышает $100$~экз./м$^2$.

Динамика размерной структуры поселений {\it M.~balthica} в Белом и Баренцевом представлена двумя типами.
Более распространенный вариант: чередование бимодального и мономодального распределение особей по размерам. 
При этом первый пик формируют молодые особи (обычно длиной до 5~мм), а в случае бимодальной добавляется второй модальный класс из взрослых особей (в Белом море длиной 9 -- 12~мм, в Баренцевом 10 -- 17~мм). 
В Баренцевом море часто новое пополнение происходит до ухода старшей генерации и наблюдается три модальных группы. 
Такой тип динамики связан с различной успешностью ежегодного пополнения поселений молодью и, по-видимому, наличием внутривидовой конкуренции между взрослыми и молодыми особями.
В некоторых условиях формируется более редкий тип динамики с ежегодным повторением мономодальной размерной структуры. 
Возможно, это связано со специфическими условиями гидродинамики, в которых происходт разделение молодых и старых особей по способу питания и, таким образом, снижение внутривидовой конкуренции и возможность большего успеха ежегодного пополнения поселения молодью. 
Другое возможное объяснение --- формирование такого типа динамики в поселениях, находящихся под прессом хищников, которые уменьшают численность взрослых особей.

Макомы в Баренцевом море гетерогенны по скорости роста. 
Более высокая скорость роста была отмечена у особей {\it M.~balthica} обитающих в среднем горизонте литорали.
Также показано увеличение среднего годового прироста в более восточных поселениях на Мурманском побережьи.
Анализ скорости роста в европейской части ареала показывает снижение данного показателя в краевых популяциях, причем на севере это снижение более выражено.

Динамика численности поселений {\it M.~balthica} в Белом море характеризуется значительными колебаниями, связанными в первую очередь с численностью сеголетков. Изменения плотности поселений маком оказываются синхронными в пределах значительной акватории.
Численность маком оказывается выше после холодных зим, таким образом, по-видимому, основное влияние оказывают ледовые условия.
Предположительно, в более холодные зимы устойчивый ледовый покров формируется раньше и надежнее, поэтому выживаемость спата в зимний период выше, что фиксируется в наших наблюдениях, как более эффективное пополнение поселения, приводящее к увеличению общей численности {\it M.~balthica}.

Обнаружение в поселениях, обитающих в присутствии хищников, плотностно-зависимых процессов второго порядка позволяет говорить о том, что традиционно-предполагаемое минимальное влияние хищников на бентосные популяции в Арктических морях не соответствует действительности, и оказываемое на конкретное поселение воздейсвтие может быть значимо.

Численность спата на порядок варьирует в пределах незначительной акватории.
Основное влияние оказывает топология местности.
Также, по-видимому, оседание спата снижено в поселениях в высокой численностью взрослых особей {\it M.~balthica}, хотя масштабы этого явления и конкретные механизмы остаются неизвестными.




%%выводы
	\begin{enumerate}
%		\item Для Белого моря типичны поселения {\it Macoma balthica} с численностью  700 -- 800~экз./м$^2$ (при варьировании от 10 до 8500~экз./м$^2$). Варьирование обилия связано в первую очередь с численностью годовалых особей.
		\item В Кольском заливе Баренцева моря и Кандалакшском заливе  Белого моря значения биомассы (до 200 г/м$^2$) поселений {\it Macoma balthica} сопоставимы с аналогичным показателем в европейской части ареала, а плотность поселений нередко оказывается выше (до 8~тыс.~экз./м$^2$). Для литорали восточной части Мурманского побережья Баренцева моря типичны поселения {\it M.~balthica} с численностью менее 100 экз./м2 
		\item Плотность поселений спата {\it Macoma balthica} в Белом море может варьировать на порядок в пределах незначительной акватории, и достигать десятков тысяч экз./м$^2$.
		\item Беломорские и баренцевоморские поселения {\it M.~balthica} не различаются по средней скорости роста моллюсков, и отличаются по этому показателю минимальными характеристиками в пределах европейской части ареала вида. 
		\item Динамика размерной структуры поселений {\it Macoma balthica} в Белом и Баренцевом представлена двумя типами. \\
Наболее обычный вариант~--- чередование бимодального и мономодального распределений особей по размерам. При этом первый пик формируют молодые
особи (обычно длиной до 5 мм), а второй модальный класс состоит из взрослых особей (в Белом море длиной 9--12~мм, в Баренцевом море~--- 10--17~мм).
Как относительно редкое событие наблюдается мономодальная структура поселений с ежегодным преобладаем молоди.
		\item Динамика плотности поселений {\it Macoma balthica} в Кандалакшском заливе Белого моря демонстрирует элементы синхронности в поселениях, расположенных на расстоянии от 1 до 100~км, что происходит на фоне резкой межгодовой неравномерности пополнения поселений молодью.  

%		\item Для литорали восточной части Мурманского побережья Баренцева моря типичны поселения {\it Macoma balthica} с численностью  менее 100~экз./м$^2$, и эти поселения не достигают плотностей, которые показаны для поселений на литорали Западного Мурмана и в Кольском заливе. %(при варьировании от 30 до 3350~экз./м$^2$).
%		\item Среднее обилие {\it Macoma balthica} в поселениях Белого моря и Кольского залива Баренцева моря выше, чем в других частях ареала, а биомасса сравнима со значениями в центральной части ареала. 
%		\item Макомы в Баренцевом море гетерогенны по скорости роста: Максимальный годовой прирост отмечен у особей среднего размера (возраста) --- $6 - 9$~мм в среднем горизонте литорали. В пределах Восточного Мурмана средний годовой прирост особей {\it Macoma balthica} увеличивается в более восточных районах по сравнению с западными.
%		\item В пределах европейской части ареала особи {\it Macoma balthica} из поселений в Белом и Баренцевом морях характеризуются минимальными скоростями роста. При этом нет принципиальных различий в скорости роста беломорских и баренцевоморских маком.

%		\item Динамика численности годовалых особей {\it Macoma balthica} позволяет говорить о не ежегодном успехе пополнения их поселений в Белом море.
%		\item Динамика численности {\it Macoma balthica} в Кандалакшском заливе Белого моря демонстрирует элементы синхронности в поселениях, расположенных на расстоянии от 1 до 100~км. Кроме того, показано что численность маком оказывается выше в годы с холодными зимами.
%		\item Динамика размерной структуры поселений {\it Macoma balthica} в Белом и Баренцевом представлена двумя типами. \\
%Более распространенный вариант: чередование бимодального и мономодального распределение особей по размерам. При этом первый пик формируют молодые особи (обычно длиной до 5~мм), а в случае бимодального добавляется второй модальный класс из взрослых особей (в Белом море длиной 9 -- 12~мм, в Баренцевом 10 -- 17~мм). В Баренцевом море часто новое пополнение происходит до ухода старшей генерации и наблюдается три модальных группы. %Такой тип динамики связан с различной успешностью ежегодного пополнения поселений молодью и наличием внутривидовой конкуренции между взрослыми и молодыми особями.
%В некоторых условиях формируется более редкий тип динамики с ежегодным повторением мономодальной размерной структуры. %Возможно, это связано со специфическими условиями гидродинамики, в которых происходит разделение молодых и старых особей по способу питания и, таким образом, снижение внутривидовой конкуренции и возможность большего успеха ежегодного пополнения поселения молодью. Другое возможное объяснение --- формирование такого типа динамики в поселениях, находящихся под прессом хищников, которые уменьшают численность взрослых особей.
	\end{enumerate}





%печатаем список литературы
\newpage
\printbibliography[heading=bibintoc]
%\printbibliography

%приложение

\renewcommand{\thetable}{\Roman{table}}
\setcounter{table}{0}
\renewcommand{\thefigure}{\Roman{figure}}
\setcounter{figure}{0}
\appendix
\renewcommand{\thesection}{\Asbuk{section}}
\setcounter{section}{0}
\chapter*{Приложение}
\addcontentsline{toc}{chapter}{Приложение}	% Добавляем его в оглавление
%таблица про пробы и среднее обилие
\section{Характеристики пробоотбора и среднее обилие {\it Macoma balthica} на исследованных участках}
\label{app:NB_table}
	\begin{footnotesize}
    \begin{center}
	\begin{longtable}{|p{1.6cm}|p{2.3cm}|p{0.8cm}|p{1.8cm}|p{1.1cm}|p{1.1cm}|*{4}{p{1cm}|}}
	\caption{Среднее обилие {\it Macoma balthica} на различных участках Белого моря} \label{tab:mean_NB_White}\\
	\hline
	Район & Участок & год & ма\-ре\-ографи\-ческий уровень & число повторностей & пло\-щадь учета & $N$, экз./м$^2$ & $SEM_N$  & $B$, г/м$^2$ & $SEM_B$
	\\ \hline \endfirsthead
	\hline
	\multicolumn{10}{|c|}{продолжение таблицы \ref{tab:mean_NB_White}} \\ \hline
	Район & Участок & год & ма\-ре\-ографи\-ческий уровень & число повторностей & пло\-щадь учета & $N$, экз./м$^2$ & $SEM_N$  & $B$, г/м$^2$ & $SEM_B$ 
	\\ \hline \endhead
	\hline 
	\multicolumn{10}{|c|}{продолжение таблицы \ref{tab:mean_NB_White} на следующей странице}
	\\ \hline \endfoot
	 \endlastfoot
г. Чупа            & б. Клющиха                     & 2006 & СГЛ               & 10   & 1/20 & 444  & 53,7   & 1,1   & 0,27  \\
                   &                                & 2006 & НГЛ               & 10   & 1/20 & 362  & 26,4   & --    & --    \\
                   &                                & 2006 & ВСЛ               & 10   & 1/20 & 1136 & 55,4   & --    & --    \\ \cline{2-10}
                   & Сухая салма                    & 2006 & СГЛ               & 10 и & 2/20 & 1165 & 169,3  & 6,2   & 1,17  \\
                   &                                & 2006 & НГЛ               & 5    & 1/20 & 1132 & 82,6   & --    & --    \\
                   &                                & 2006 & НГЛ, зостера      & 5    & 1/20 & 992  & 174,4  & --    & --    \\\cline{2-10}
                   & б. Лисья                       & 2006 & СГЛ               & 10   & 1/20 & 1346 & 209,8  & 1,9   & 0,76  \\
                   &                                & 2006 & НГЛ               & 10   & 1/20 & 2832 & 277,8  & --    & --    \\
                   &                                & 2006 & ВСЛ               & 10   & 1/20 & 1006 & 159,8  & --    & --    \\\cline{2-10}
                   & пр. Подпахта                   & 2006 & СГЛ               & 10   & 1/20 & 688  & 145,2  & 1,9   & 1,21  \\
                   &                                & 2006 & НГЛ               & 10   & 1/20 & 372  & 57,9   & --    & --    \\ \hline
Лувеньга           & материковая литораль, Лувеньга & 1992 & верхний пляж      & 7    & 1/30 & 94   & 35,5   & 12,4  & 3,73  \\
                   &                                & 1992 & пояс фукоидов     & 5    & 1/30 & 114  & 55,6   & 23,9  & 10,73 \\
                   &                                & 1992 & пояс зостеры      & 5    & 1/30 & 222  & 103,3  & 22,5  & 10,95 \\
                   &                                & 1992 & нижний пляж       & 3    & 1/30 & 560  & 457,1  & 52,0  & 34,64 \\
                   &                                & 1993 & верхний пляж      & 4    & 1/30 & 413  & 127,5  & 11,5  & 4,56  \\
                   &                                & 1993 & пояс фукоидов     & 5    & 1/30 & 336  & 120,9  & 25,6  & 11,27 \\
                   &                                & 1993 & пояс зостеры      & 6    & 1/30 & 405  & 80     & 73,7  & 12,88 \\
                   &                                & 1993 & нижний пляж       & 5    & 1/30 & 354  & 77,3   & 50,5  & 15,95 \\
                   &                                & 1994 & верхний пляж      & 5    & 1/30 & 462  & 179,1  & 24,6  & 2,06  \\
                   &                                & 1994 & пояс фукоидов     & 6    & 1/30 & 745  & 220,6  & 66,9  & 16,81 \\
                   &                                & 1994 & пояс зостеры      & 6    & 1/30 & 765  & 112,7  & 108,9 & 24,64 \\
                   &                                & 1994 & нижний пляж       & 3    & 1/30 & 930  & 170,6  & 121,1 & 2,89  \\
                   &                                & 1995 & верхний пляж      & 4    & 1/30 & 908  & 222,3  & 68,8  & 9,20  \\
                   &                                & 1995 & пояс фукоидов     & 5    & 1/30 & 1134 & 269,7  & 83,0  & 19,32 \\
                   &                                & 1995 & пояс зостеры      & 5    & 1/30 & 660  & 117,7  & 61,5  & 9,75  \\
                   &                                & 1995 & нижний пляж       & 6    & 1/30 & 685  & 154,8  & 113,7 & 4,21  \\
                   &                                & 1996 & верхний пляж      & 4    & 1/30 & 698  & 257    & 62,2  & 20,58 \\
                   &                                & 1996 & пояс фукоидов     & 6    & 1/30 & 770  & 214,9  & 94,2  & 23,14 \\
                   &                                & 1996 & пояс зостеры      & 4    & 1/30 & 645  & 71,9   & 65,2  & 8,55  \\
                   &                                & 1996 & нижний пляж       & 6    & 1/30 & 870  & 68,8   & 153,0 & 19,42 \\
                   &                                & 1997 & верхний пляж      & 3    & 1/30 & 620  & 130    & 74,2  & 32,49 \\
                   &                                & 1997 & пояс фукоидов     & 6    & 1/30 & 720  & 265,6  & 88,4  & 22,91 \\
                   &                                & 1997 & пояс зостеры      & 5    & 1/30 & 702  & 70,7   & 96,7  & 18,36 \\
                   &                                & 1997 & нижний пляж       & 6    & 1/30 & 880  & 97     & 160,6 & 21,58 \\
                   &                                & 1998 & верхний пляж      & 4    & 1/30 & 2130 & 623,9  & 25,6  & 8,52  \\
                   &                                & 1998 & пояс фукоидов     & 6    & 1/30 & 2750 & 820    & 93,0  & 27,49 \\
                   &                                & 1998 & пояс зостеры      & 5    & 1/30 & 2424 & 437,1  & 136,8 & 22,56 \\
                   &                                & 1998 & нижний пляж       & 5    & 1/30 & 1182 & 239    & 174,8 & 17,02 \\
                   &                                & 1999 & верхний пляж      & 3    & 1/30 & 7240 & 5833,7 & 14,5  & 11,84 \\
                   &                                & 1999 & пояс фукоидов     & 6    & 1/30 & 3895 & 1354,6 & 88,8  & 29,72 \\
                   &                                & 1999 & пояс зостеры      & 6    & 1/30 & 2405 & 498,8  & 95,7  & 12,27 \\
                   &                                & 1999 & нижний пляж       & 5    & 1/30 & 2328 & 623,8  & 140,1 & 19,60 \\
                   &                                & 2000 & верхний пляж      & 2    & 1/30 & 2640 & 870    & 71,3  & 6,22  \\
                   &                                & 2000 & пояс фукоидов     & 4    & 1/30 & 2760 & 373,1  & 91,8  & 20,69 \\
                   &                                & 2000 & пояс зостеры      & 5    & 1/30 & 2562 & 721    & 117,7 & 11,30 \\
                   &                                & 2000 & нижний пляж       & 4    & 1/30 & 2018 & 394,3  & 133,6 & 30,76 \\
                   &                                & 2002 & верхний пляж      & 3    & 1/30 & 1360 & 401,5  & 63,3  & 12,48 \\
                   &                                & 2002 & пояс фукоидов     & 3    & 1/30 & 3250 & 337,8  & 150,0 & 36,88 \\
                   &                                & 2002 & пояс зостеры      & 4    & 1/30 & 2498 & 952,6  & 140,2 & 43,61 \\
                   &                                & 2002 & нижний пляж       & 2    & 1/30 & 810  & 240    & 76,7  & 27,47 \\
                   &                                & 2004 & верхний пляж      & 3    & 1/30 & 2800 & 1066,6 & 62,5  & 26,54 \\
                   &                                & 2004 & пояс фукоидов     & 4    & 1/30 & 3090 & 889    & 151,9 & 23,16 \\
                   &                                & 2004 & пояс зостеры      & 5    & 1/30 & 1818 & 302,6  & 117,0 & 10,28 \\\cline{2-10}
                   & о. Горелый                     & 1992 & ВГЛ               & 7    & 1/30 & 73   & 23,7   & 11,8  & 2,64  \\
                   &                                & 1992 & СГЛ               & 5    & 1/30 & 108  & 9,7    & 9,3   & 1,10  \\
                   &                                & 1992 & НГЛ               & 2    & 1/30 & 50   & 20     & 3,2   & 2,42  \\
                   &                                & 1992 & ноль глубин       & 3    & 1/30 & 13   & 3,3    & 1,3   & 0,58  \\
                   &                                & 1993 & ВГЛ               & 3    & 1/30 & 143  & 29,1   & 7,8   & 3,59  \\
                   &                                & 1993 & СГЛ               & 3    & 1/30 & 480  & 11,5   & 25,8  & 5,33  \\
                   &                                & 1993 & НГЛ               & 4    & 1/30 & 183  & 34,5   & 10,9  & 2,65  \\
                   &                                & 1993 & ноль глубин       & 3    & 1/30 & 97   & 43,7   & 9,8   & 5,04  \\
                   &                                & 2004 & ВГЛ               & 3    & 1/30 & 2620 & 219,3  & 70,4  & 11,71 \\
                   &                                & 2004 & СГЛ               & 3    & 1/30 & 1700 & 208,8  & 91,3  & 8,00  \\
                   &                                & 2004 & НГЛ               & 3    & 1/30 & 1040 & 176,9  & 85,5  & 3,09  \\
                   &                                & 2004 & ноль глубин       & 3    & 1/30 & 1540 & 60,8   & 177,9 & 16,77 \\
                   &                                & 2006 & ВГЛ               & 3    & 1/30 & 2200 & 353,4  & 86,7  & 23,82 \\
                   &                                & 2006 & СГЛ               & 3    & 1/30 & 1910 & 342,2  & 74,0  & 16,22 \\
                   &                                & 2006 & НГЛ               & 3    & 1/30 & 650  & 87,2   & 66,2  & 9,79  \\
                   &                                & 2006 & ноль глубин       & 3    & 1/30 & 760  & 160,9  & 88,2  & 18,32 \\
                   &                                & 2007 & ВГЛ               & 3    & 1/30 & 1940 & 341,8  & 61,0  & 6,55  \\
                   &                                & 2007 & СГЛ               & 3    & 1/30 & 1990 & 449,8  & 50,1  & 3,74  \\
                   &                                & 2007 & НГЛ               & 3    & 1/30 & 540  & 195,2  & 45,9  & 16,56 \\
                   &                                & 2007 & ноль глубин       & 3    & 1/30 & 660  & 45,8   & 85,9  & 4,57  \\
                   &                                & 2008 & ВГЛ               & 3    & 1/30 & 1100 & 98,5   & 50,2  & 6,27  \\
                   &                                & 2008 & СГЛ               & 3    & 1/30 & 2740 & 125,3  & 50,9  & 2,57  \\
                   &                                & 2008 & НГЛ               & 3    & 1/30 & 1030 & 404,5  & 45,6  & 15,77 \\
                   &                                & 2008 & ноль глубин       & 3    & 1/30 & 740  & 147,3  & 81,3  & 44,67 \\
                   &                                & 2011 & ВГЛ               & 3    & 1/30 & 2000 & 926    & 23,9  & 10,41 \\
                   &                                & 2011 & СГЛ               & 3    & 1/30 & 1210 & 216,6  & 54,6  & 21,70 \\
                   &                                & 2011 & НГЛ               & 3    & 1/30 & 1590 & 199,7  & 77,2  & 16,54 \\
                   &                                & 2011 & ноль глубин       & 3    & 1/30 & 1100 & 208,8  & 69,5  & 9,22  \\\cline{2-10}
                   & Эстуарий р.Лувеньги            & 1992 & НГЛ               & 6    & 1/30 & 55   & 14,8   & 13,7  & 3,33  \\
                   &                                & 1993 & НГЛ               & 6    & 1/30 & 202  & 31,3   & 12,2  & 2,98  \\
                   &                                & 1994 & НГЛ               & 3 и  & 3/30 & 777  & 129,9  & 73,7  & 13,23 \\
                   &                                & 1995 & НГЛ               & 3 и  & 3/30 & 473  & 44,8   & 47,7  & 7,62  \\
                   &                                & 1996 & НГЛ               & 3 и  & 3/30 & 337  & 29,1   & 45,1  & 5,10  \\
                   &                                & 1997 & НГЛ               & 3 и  & 3/30 & 213  & 14,5   & 38,1  & 8,15  \\
                   &                                & 1998 & НГЛ               & 3 и  & 3/30 & 750  & 15,3   & 54,6  & 5,50  \\
                   &                                & 1999 & НГЛ               & 3 и  & 3/30 & 2073 & 633,3  & 18,2  & 3,38  \\
                   &                                & 2000 & НГЛ               & 3 и  & 3/30 & 1913 & 86,5   & 54,1  & 4,83  \\
                   &                                & 2001 & НГЛ               & 3 и  & 3/30 & 2607 & 139,6  & 109,0 & 8,74  \\
                   &                                & 2002 & НГЛ               & 3 и  & 3/30 & 1917 & 209    & 90,6  & 11,61 \\
                   &                                & 2003 & НГЛ               & 3 и  & 3/30 & 2220 & 235,4  & 120,5 & 11,27 \\
                   &                                & 2004 & НГЛ               & 3 и  & 3/30 & 3330 & 315    & 141,4 & 7,73  \\
                   &                                & 2005 & НГЛ               & 3 и  & 3/30 & 1623 & 161,8  & 90,3  & 5,88  \\
                   &                                & 2006 & НГЛ               & 3 и  & 3/30 & 993  & 131,3  & 86,4  & 6,71  \\
                   &                                & 2007 & НГЛ               & 9    & 1/30 & 2547 & 341,8  & 111,0 & 13,18 \\
                   &                                & 2008 & НГЛ               & 3 и  & 3/30 & 1683 & 343,5  & 113,8 & 14,63 \\
                   &                                & 2009 & НГЛ               & 3 и  & 3/30 & 1860 & 146,4  & 95,1  & 26,69 \\
                   &                                & 2010 & НГЛ               & 3 и  & 3/30 & 2057 & 231,5  & 125,1 & 2,97  \\
                   &                                & 2011 & НГЛ               & 9    & 1/30 & 1637 & 60,2   & 159,5 & 8,50  \\
                   &                                & 2012 & НГЛ               & 3 и  & 3/30 & 1170 & 23,1   & 111,0 & 9,20  \\ \hline
Северный архипелаг & Западная Ряшкова салма         & 1994 & СГЛ               & 2 и  & 3/30 & 450  & 100    & 58,3  & 5,38  \\
                   &                                & 1995 & СГЛ               & 2 и  & 3/30 & 490  & 10     & 74,1  & 6,42  \\
                   &                                & 1996 & СГЛ               & 2 и  & 3/30 & 260  & 130    & 45,7  & 14,62 \\
                   &                                & 1997 & СГЛ               & 2 и  & 3/30 & 220  & 90     & 37,1  & 15,07 \\
                   &                                & 1998 & СГЛ               & 2 и  & 3/30 & 755  & 185    & 101,7 & 13,83 \\
                   &                                & 1999 & СГЛ               & 2 и  & 3/30 & 8530 & 800    & 134,4 & 59,88 \\
                   &                                & 2000 & СГЛ               & 2 и  & 3/30 & 2910 & 440    & 58,8  & 28,01 \\
                   &                                & 2001 & СГЛ               & 2 и  & 3/30 & 2515 & 295    & 130,5 & 29,17 \\
                   &                                & 2002 & СГЛ               & 2 и  & 3/30 & 2690 & 570    & 165,6 & 24,94 \\
                   &                                & 2003 & СГЛ               & 2 и  & 3/30 & 1930 & 300    & 139,2 & 25,66 \\
                   &                                & 2004 & СГЛ               & 2 и  & 3/30 & 2355 & 55     & 133,0 & 16,36 \\
                   &                                & 2005 & СГЛ               & 2 и  & 3/30 & 1825 & 115    & 137,4 & 2,63  \\
                   &                                & 2006 & СГЛ               & 2 и  & 3/30 & 795  & 165    & 75,1  & 16,79 \\
                   &                                & 2007 & СГЛ               & 2 и  & 3/30 & 1055 & 185    & 122,7 & 12,00 \\
                   &                                & 2008 & СГЛ               & 2 и  & 3/30 & 1840 & 460    & 122,5 & 53,38 \\
                   &                                & 2009 & СГЛ               & 2 и  & 3/30 & 1745 & 65     & 110,5 & 13,99 \\
                   &                                & 2010 & СГЛ               & 2 и  & 3/30 & 1680 & 460    & 154,5 & 30,87 \\
                   &                                & 2011 & СГЛ               & 2 и  & 3/30 & 1455 & 535    & 136,5 & 55,75 \\
                   &                                & 2012 & СГЛ               & 2 и  & 3/30 & 910  & 340    & 88,8  & 28,64 \\\cline{2-10}
                   & Южная губа о. Ряшкова          & 2001 & ноль глубин       & 9    & 1/30 & 1257 & 174,8  & 33,0  & 7,53  \\
                   &                                & 2002 & ноль глубин       & 16   & 1/30 & 1196 & 212,5  & 37,0  & 10,80 \\
                   &                                & 2003 & ноль глубин       & 15   & 1/30 & 1758 & 333,3  & 26,7  & 9,10  \\
                   &                                & 2004 & ноль глубин       & 13   & 1/30 & 1913 & 576    & 9,4   & 2,35  \\
                   &                                & 2005 & ноль глубин       & 15   & 1/30 & 860  & 178    & 7,3   & 1,38  \\
                   &                                & 2006 & ноль глубин       & 12   & 1/30 & 843  & 203,9  & 5,6   & 1,32  \\
                   &                                & 2007 & ноль глубин       & 15   & 1/30 & 1412 & 387,8  & 11,3  & 2,49  \\
                   &                                & 2008 & ноль глубин       & 10   & 1/30 & 1434 & 333,4  & 20,8  & 3,77  \\
                   &                                & 2009 & ноль глубин       & 15   & 1/30 & 1122 & 198,5  & 42,7  & 10,79 \\
                   &                                & 2010 & ноль глубин       & 15   & 1/30 & 682  & 106,5  & 30,4  & 5,42  \\
                   &                                & 2011 & ноль глубин       & 15   & 1/30 & 364  & 151,5  & 19,1  & 10,56 \\
                   &                                & 2012 & ноль глубин       & 15   & 1/30 & 142  & 39,1   & 1,9   & 1,36  \\\cline{2-10}
                   & о. Ломнишный                   & 2007 & ноль глубин       & 10   & 1/30 & 501  & 88,7   & 7,8   & 4,08  \\
                   &                                & 2008 & ноль глубин       & 5    & 1/30 & 1530 & 295    & 29,5  & 8,71  \\
                   &                                & 2009 & ноль глубин       & 10   & 1/30 & 813  & 241,1  & 41,3  & 13,29 \\
                   &                                & 2010 & ноль глубин       & 10   & 1/30 & 540  & 168,1  & 49,2  & 13,93 \\
                   &                                & 2011 & ноль глубин       & 10   & 1/30 & 378  & 118,4  & 13,8  & 7,78  \\
                   &                                & 2012 & ноль глубин       & 10   & 1/30 & 513  & 90,9   & 8,7   & 5,39 \\ \hline
	\multicolumn{10}{p{16cm}}{Примечания: градации мареографического уровня: ВГЛ --- верхний горизонт литорали, СГЛ --- средний горизонт литорали, НГЛ --- нижний горизонт литорали, ВСЛ --- верхняя сублитораль. 

	$N$, экз./м$^2$ --- средняя численность {\it M.~balthica},
	$SEM_N$ --- ошибка среднего для численности,
	$B$, г/м$^2$ --- средняя биомасса {\it M.~balthica},
	$SEM_B$ --- ошибка среднего для биомассы.

	В обозначении числа повторностей индекс ''и'' означает интегральную пробу, в этом случае в графе площадь учета указано сколько проб какой площади объединялись в одну. Прочерк в ячейке --- отсутствие данных.}
	\end{longtable}
\end{center}
	\end{footnotesize}


	\begin{footnotesize}
        \begin{center}
%	\begin{longtable}{|p{2cm}|p{3cm}|p{1cm}|p{2cm}|p{1.5cm}|p{1cm}|*{3}{c|}}
	\begin{longtable}{|p{1.6cm}|p{2.3cm}|p{1cm}|p{1.6cm}|p{1.1cm}|p{1.1cm}|*{4}{p{1cm}|}}
	\caption{Среднее обилие {\it Macoma balthica} на различных участках Баренцева моря}\label{tab:mean_NB_Barents}\\
	\hline
	Район & Участок & год & ма\-ре\-ографи\-ческий уровень & число повторностей & пло\-щадь учета & $N$, экз./м$^2$ & $SEM_N$  & $B$, г/м$^2$ & $SEM_B$
	\\ \hline \endfirsthead
	\hline
	\multicolumn{10}{|c|}{продолжение таблицы \ref{tab:mean_NB_Barents}} \\ \hline
	Район & Участок & год & ма\-ре\-ографи\-ческий уровень & число повторностей & пло\-щадь учета & $N$, экз./м$^2$ & $SEM_N$  & $B$, г/м$^2$ & $SEM_B$
	\\ \hline \endhead
	\hline 
	\multicolumn{10}{|c|}{продолжение таблицы \ref{tab:mean_NB_Barents} на следующей странице}
	\\ \hline \endfoot
	\endlastfoot
Западный Мурман  & Ура-губа          & 2005       & СГЛ & 3  & 1/30 & 1267 & 288,8 & --    & --    \\ \cline{2-10}
                 & Печенга           & 2005       & СГЛ & 3  & 1/30 & 767  & 218,6 & --    & --    \\ \hline
Кольский Залив   & Северное Нагорное & 2005       & СГЛ & 2  & 1/30 & 390  & 90    & --    & --    \\ \cline{2-10}
                 & Абрам-мыс         & 2005       & СГЛ & 2  & 1/30 & 3350 & 520   & --    & --    \\
                 &                   & 2008       & СГЛ & 5  & 1/20 & 540  & 208,5 & 123,1 & 41,12 \\
                 &                   & 2008       & НГЛ & 5  & 1/20 & 1804 & 78,6  & 216,5 & 54,99 \\ \cline{2-10}
                 & Ретинское         & 2005       & СГЛ & 2  & 1/30 & 660  & 300   & --    & --    \\ \cline{2-10}
                 & Пала-губа         & 2007       & СГЛ & 16 & 1/30 & 936  & 76,4  & 35,8  & 4,02  \\
                 &                   & 2007 осень & НГЛ & 36 & 1/30 & 790  & 61,7  & 172   & 13,02 \\
                 &                   & 2008 зима  & СГЛ & 11 & 1/20 & 864  & 154,4 & 77,3  & 13,09 \\
                 &                   & 2008       & НГЛ & 10 & 1/30 & 1644 & 192,5 & 193,2 & 29,14 \\ \hline
Восточный Мурман & Гаврилово         & 2008       & СГЛ & 5  & 1/30 & 99   & 24,5  & 119,9 & 33,26 \\
                 &                   & 2008       & НГЛ & 5  & 1/30 & 74   & 26,3  & 13,02 & 6,89  \\ \cline{2-10}
                 & Ярнышная          & 2007       & СГЛ & 36 & 1/30 & 70   & 9,6   & 24,5  & 5,62  \\
                 &                   & 2008       & ВГЛ & 5  & 1/30 & 219  & 97,6  & 116,9 & 20,92 \\
                 &                   & 2008       & НГЛ & 5  & 1/30 & 387  & 109,1 & 41,1  & 21,99 \\ \cline{2-10}
                 & Даль\-не-Зе\-ле\-нец\-кая  & 2002       & СГЛ & 43 & 1/30 & 52   & 7     & --    & --    \\
                 &                   & 2003       & СГЛ & 48 & 1/30 & 34   & 6,6   & --    & --    \\
                 &                   & 2004       & СГЛ & 44 & 1/30 & 32   & 5,3   & --    & --    \\
                 &                   & 2005       & СГЛ & 30 & 1/30 & 30   & 4,5   & --    & --    \\
                 &                   & 2006       & СГЛ & 28 & 1/30 & 39   & 6     & --    & --    \\
                 &                   & 2007       & СГЛ & 33 & 1/30 & 72   & 6,6   & 34,4  & 5,57  \\
                 &                   & 2008       & СГЛ & 72 & 1/30 & 72   & 5,5   & --    & --    \\
                 &                   & 2008       & ВГЛ & 10 & 1/30 & 30   & 8,9   & --    & --    \\
                 &                   & 2008       & НГЛ & 5  & 1/30 & 42   & 7,3   & 43    & 4,93  \\ \cline{2-10}
                 & Шельпино          & 2008       & ВГЛ & 5  & 1/30 & 36   & 17,5  & 14,6  & 8,02  \\
                 &                   & 2008       & СГЛ & 5  & 1/30 & 54   & 11,2  & 23,5  & 10,15 \\ \cline{2-10}
                 & Порчниха          & 2007       & СГЛ & 32 & 1/30 & 87   & 10,8  & 26,8  & 5,57  \\
                 &                   & 2008       & СГЛ & 5  & 1/30 & 48   & 15,7  & --    & --    \\ \cline{2-10}
                 & Ивановская        & 2008       & ВСЛ & 5  & 1/20 & 1208 & 72,8  & 75,2  & 1,94  \\ \hline
	\multicolumn{10}{p{16cm}}{Примечания: градации мареографического уровня: ВГЛ --- верхний горизонт литорали, СГЛ --- средний горизонт литорали, НГЛ --- нижний горизонт литорали, ВСЛ --- верхняя сублитораль. 

	$N$, экз./м$^2$ --- средняя численность {\it M.~balthica},
	$SEM_N$ --- ошибка среднего для численности,
	$B$, г/м$^2$ --- средняя биомасса {\it M.~balthica},
	$SEM_B$ --- ошибка среднего для биомассы.

	В обозначении числа повторностей индекс ''и'' означает интегральную пробу, в этом случае в графе площадь учета указано сколько проб какой площади объединялись в одну. Прочерк в ячейке --- отсутствие данных.}
	\end{longtable}
\end{center}
	\end{footnotesize}


%списки видов
\section{Таксономический состав сообществ макробентоса на исследованных участках}
\label{app:species}

Примечание: горизонты литорали: В --- верхний, С --- средний, Н --- нижний, ноль --- ноль глубин, ВСЛ --- верхняя сублитораль.

\begin{footnotesize}
\begin{longtable}{|p{2.2cm}|p{1.2cm}|*{3}{p{0.4cm}}p{0.5cm}|*{3}{p{1.2cm}|}*{4}{p{0.4cm}}|}
\caption{{\normalsizeСостав сообществ на исследованный участках литорали Белого моря}}
\label{tab:White_species}
\\ \hline
участок                            & За\-пад\-ная Ряш\-ко\-ва сал\-ма & \multicolumn{4}{c|}{о. Горелый} & Эс\-туа\-рий р. Лу\-вень\-ги & Южная губа о.~Ряш\-кова & Лом\-ниш\-ный   & \multicolumn{4}{c|}{материк (Лувеньга)} \\ \hline
горизонт литорали                  & C                    & В        & С                  & Н & ноль & С                & ноль & ноль & В & С & Н & ноль \\ \hline
\endfirsthead
	\hline
	\multicolumn{13}{|c|}{продолжение таблицы \ref{tab:White_species}} \\ \hline
участок                            & За\-пад\-ная Ряш\-ко\-ва сал\-ма & \multicolumn{4}{c|}{о. Горелый} & Эс\-туа\-рий р. Лу\-вень\-ги & Южная губа о.~Ряш\-кова & Лом\-ниш\-ный   & \multicolumn{4}{c|}{материк (Лувеньга)} \\ \hline
горизонт литорали                  & C                    & В        & С                  & Н & ноль & С                & ноль & ноль & В & С & Н & ноль \\ \hline
	\\ \hline \endhead
	\hline 
	\multicolumn{13}{|c|}{продолжение таблицы \ref{tab:White_species} на следующей странице}
	\\ \hline \endfoot
	 \endlastfoot
\multicolumn{13}{|c|}{Nemertini} \\ \hline
Nemertini indet.                   &                     &         &                   &                   &          &                 &\textbf{+}         &\textbf{+}         &  &  &  &          \\ \hline
\multicolumn{13}{|c|}{Priapulida} \\ \hline
{\it Halicryptus spinulosus}             &                     &         &                   &                   &          &                 &\textbf{+}         &\textbf{+}         &  &  &  &          \\ \hline
{\it Priapulus caudatus}                 &                     &         &                   &                   &          &                 &          &\textbf{+}         &  &  &\textbf{+} &\textbf{+}         \\ \hline
\multicolumn{13}{|c|}{Oligochaeta} \\ \hline
{\it Clitellio arenarius}                &                     &         &                   &                   &          &                 &\textbf{+}         &          &  &  &  &          \\ \hline
Enchytraeidae gen. sp.             &                     &\textbf{+}        &                   &\textbf{+}                  &          &\textbf{+}                &\textbf{+}         &          &  &  &  &          \\ \hline
Oligochaeta varia                  &                     &         &                   &                   &          &                 &\textbf{+}         &          &  &  &  &          \\ \hline
{\it Paranais littoralis}                &\textbf{+}                    &         &                   &                   &          &\textbf{+}                &          &\textbf{+}         &  &  &  &          \\ \hline
{\it Tubifex costatus}                   &\textbf{+}                    &         &                   &                   &          &                 &\textbf{+}         &\textbf{+}         &  &  &  &          \\ \hline
\multicolumn{13}{|c|}{Polychaeta} \\ \hline
{\it Alitte virens}                      &                     &         &                   &                   &          &                 &          &          &  &  &  &          \\ \hline
{\it Arenicola marina}                   &\textbf{+}                    &         &                   &\textbf{+}                  &          &\textbf{+}                &\textbf{+}         &\textbf{+}         &  &  &  &          \\ \hline
{\it Capitella capitata}                 &                     &         &                   &                   &          &                 &\textbf{+}         &          &  &  &  &          \\ \hline
{\it Eteone longa}                       &                     &         &\textbf{+}                  &\textbf{+}                  &\textbf{+}         &                 &\textbf{+}         &\textbf{+}         &\textbf{+} &\textbf{+} &\textbf{+} &          \\ \hline
{\it Fabricia sabella}                   &\textbf{+}                    &\textbf{+}        &\textbf{+}                  &\textbf{+}                  &          &\textbf{+}                &\textbf{+}         &\textbf{+}         &\textbf{+} &\textbf{+} &\textbf{+} &\textbf{+}         \\ \hline
{\it Harmathoe imbricata}                &                     &         &                   &                   &          &                 &\textbf{+}         &          &  &  &  &          \\ \hline
{\it Micronephthys minuta}               &                     &         &                   &                   &          &                 &\textbf{+}         &          &  &  &  &          \\ \hline
{\it Microspio theli}                    &\textbf{+}                    &         &                   &                   &          &                 &\textbf{+}         &\textbf{+}         &  &  &  &          \\ \hline
{\it Nephthys sp.}                       &                     &         &                   &                   &          &                 &\textbf{+}         &          &  &  &  &          \\ \hline
{\it Ophelia limacina}                   &                     &         &                   &                   &          &                 &\textbf{+}         &\textbf{+}         &  &  &  &          \\ \hline
{\it Pectinaria sp.}                     &                     &         &                   &                   &          &                 &\textbf{+}         &          &  &  &  &          \\ \hline
{\it Phyllodoce groenlandica}            &                     &         &                   &                   &          &                 &\textbf{+}         &\textbf{+}         &  &  &  &          \\ \hline
{\it Polydora quadrilobata}              &                     &         &                   &                   &          &                 &\textbf{+}         &\textbf{+}         &  &  &  &\textbf{+}         \\ \hline
{\it Pygospio elegans}                   &\textbf{+}                    &         &                   &                   &\textbf{+}         &\textbf{+}                &\textbf{+}         &\textbf{+}         &\textbf{+} &\textbf{+} &\textbf{+} &\textbf{+}         \\ \hline
{\it Scalibregma inflatum}               &                     &         &                   &                   &          &                 &\textbf{+}         &\textbf{+}         &  &  &  &          \\ \hline
{\it Scoloplos armiger}                  &                     &         &                   &                   &          &                 &\textbf{+}         &\textbf{+}         &  &  &  &          \\ \hline
{\it Spio filicornis}                    &                     &         &                   &                   &          &                 &\textbf{+}         &          &  &  &  &          \\ \hline
Spionidae gen. sp.                 &                     &         &                   &                   &          &                 &\textbf{+}         &          &  &  &  &          \\ \hline
{\it Travisia forbesii}                   &                     &         &                   &                   &          &                 &\textbf{+}         &\textbf{+}         &  &  &  &          \\ \hline
{\it Tubificoides benedeni}              &\textbf{+}                    &         &                   &\textbf{+}                  &\textbf{+}         &                 &\textbf{+}         &\textbf{+}         &  &  &  &          \\ \hline
{\it Nereimyra punctata}      &                     &         &                   &                   &          &                 &          &\textbf{+}         &  &  &  &          \\ \hline
{\it Chaetozone setosa}                  &                     &         &                   &                   &          &                 &          &\textbf{+}         &  &  &  &          \\ \hline
\multicolumn{13}{|c|}{Isopoda} \\ \hline
{\it Jaera sp.}                           &                     &\textbf{+}        &                   &                   &          &                 &          &          &  &  &  &          \\ \hline
\multicolumn{13}{|c|}{Amphipoda} \\ \hline
{\it Atylus carinatus}                   &                     &         &                   &                   &          &                 &\textbf{+}         &\textbf{+}         &  &  &  &          \\ \hline
{\it Classicorophium bonelli} &                     &         &                   &                   &          &                 &\textbf{+}         &\textbf{+}         &  &  &  &          \\ \hline
{\it Gammarus sp.}                        &\textbf{+}                    &\textbf{+}        &\textbf{+}                  &\textbf{+}                  &          &\textbf{+}                &\textbf{+}         &          &\textbf{+} &\textbf{+} &\textbf{+} &\textbf{+}         \\ \hline
{\it Monoculodes sp.}                    &                     &         &                   &                   &          &\textbf{+}                &\textbf{+}         &\textbf{+}         &  &  &\textbf{+} &\textbf{+}         \\ \hline
{\it Pontoporea affinis}                 &                     &         &                   &                   &          &\textbf{+}                &          &          &  &  &  &          \\ \hline
{\it Pseudalibrotus littoralis}          &\textbf{+}                    &         &                   &                   &          &\textbf{+}                &\textbf{+}         &          &  &\textbf{+} &\textbf{+} &\textbf{+}         \\ \hline
{\it Priscillina armata}                 &                     &         &                   &                   &          &                 &          &\textbf{+}         &  &  &  &          \\ \hline
{\it Pontoporea femorata}                &                     &         &                   &                   &          &                 &          &          &  &  &\textbf{+} &\textbf{+}         \\ \hline
\multicolumn{13}{|c|}{Cumacea} \\ \hline
{\it Diastylis sulcata}                  &                     &         &                   &                   &          &                 &\textbf{+}         &          &  &  &\textbf{+} &\textbf{+}         \\ \hline
\multicolumn{13}{|c|}{Decapoda} \\ \hline
{\it Crangon crangon}                    &                     &         &                   &                   &          &                 &\textbf{+}         &\textbf{+}         &  &  &  &          \\ \hline
\multicolumn{13}{|c|}{Diptera} \\ \hline
Chironomidae larvae                &                     &\textbf{+}        &                   &\textbf{+}                  &          &\textbf{+}                &\textbf{+}         &\textbf{+}         &  &  &  &          \\ \hline
Dolichopodidae larvae              &                     &\textbf{+}        &                   &                   &          &                 &\textbf{+}         &          &  &  &  &          \\ \hline
\multicolumn{13}{|c|}{Gastropoda} \\ \hline
{\it Cylichna alba}                      &                     &         &                   &                   &          &                 &\textbf{+}         &\textbf{+}         &  &  &  &          \\ \hline
{\it Cylichna occulta}                   &                     &         &                   &                   &          &                 &\textbf{+}         &\textbf{+}         &  &  &  &          \\ \hline
{\it Epheria vincta}                     &                     &         &                   &                   &          &                 &\textbf{+}         &          &  &  &  &          \\ \hline
{\it Hydrobia ulvae}                     &\textbf{+}                    &\textbf{+}        &\textbf{+}                  &\textbf{+}                  &\textbf{+}         &\textbf{+}                &\textbf{+}         &\textbf{+}         &  &  &  &          \\ \hline
{\it Limaponlia cocksi}                  &                     &\textbf{+}        &                   &                   &          &                 &          &          &  &  &  &          \\ \hline
{\it Littorina littorea}                 &                     &         &                   &                   &          &                 &\textbf{+}         &\textbf{+}         &  &  &  &          \\ \hline
{\it Littorina gr. obtusata}                 &                     &         &                   &                   &          &                 &\textbf{+}         &\textbf{+}         &  &  &  &          \\ \hline
{\it Littorina gr. saxatilis}                &\textbf{+}                    &\textbf{+}        &                   &\textbf{+}                  &          &\textbf{+}                &\textbf{+}         &\textbf{+}         &  &  &  &          \\ \hline
{\it Skeneopsis planorbis}               &                     &         &                   &                   &          &                 &          &\textbf{+}         &  &  &  &          \\ \hline
\multicolumn{13}{|c|}{Bivalvia} \\ \hline
{\it Macoma balthica}                    &\textbf{+}                    &\textbf{+}        &\textbf{+}                  &\textbf{+}                  &\textbf{+}         &\textbf{+}                &\textbf{+}         &\textbf{+}         &\textbf{+} &\textbf{+} &\textbf{+} &\textbf{+}         \\ \hline
{\it Mya arenaria}                       &                     &         &                   &                   &\textbf{+}         &                 &          &          &  &  &  &          \\ \hline
{\it Mytilus edulis}                     &\textbf{+}                    &\textbf{+}        &\textbf{+}                  &\textbf{+}                  &          &\textbf{+}                &          &          &  &  &  &          \\ \hline
{\it Serripes groenlandica}              &                     &         &                   &                   &          &                 &          &\textbf{+}         &  &  &  &          \\ \hline
\end{longtable}
\end{footnotesize}

%%%%%%%%%%%%%%%%%%%%%%%%%%%%

\begin{footnotesize}
\begin{longtable}{|p{2.1cm}|p{0.38cm}p{0.38cm}|p{0.38cm}p{0.38cm}|p{0.38cm}p{0.38cm}|p{0.35cm}p{0.35cm}p{0.35cm}|p{1cm}|p{0.5cm}p{0.5cm}|p{1cm}|p{1cm}|}
\caption{{\normalsizeСостав сообществ на исследованный участках литорали Баренцева моря}}
\label{tab:Barents_species}
\\ \hline
участок                     & \multicolumn{2}{c|}{Абрам-мыс} & \multicolumn{2}{c|}{Пала-губа} & \multicolumn{2}{c|}{Гав\-ри\-ло\-во} & \multicolumn{3}{c|}{Яр\-ныш\-ная} & Дальне\-зе\-ле\-нец\-кая & \multicolumn{2}{c|}{Шель\-пи\-но} & Порч\-ни\-ха & Ива\-нов\-ская \\ \hline
горизонт литорали & С       & Н       & С       & Н       & С       & Н   & В    & С    & Н    & С           & В    & С    & С    & ВСЛ        \\ \hline \endfirsthead
	\hline
	\multicolumn{15}{|c|}{продолжение таблицы \ref{tab:Barents_species}} \\ \hline
участок                     & \multicolumn{2}{c|}{Абрам-мыс} & \multicolumn{2}{c|}{Пала-губа} & \multicolumn{2}{c|}{Гав\-ри\-ло\-во} & \multicolumn{3}{c|}{Яр\-ныш\-ная} & Дальне\-зе\-ле\-нец\-кая & \multicolumn{2}{c|}{Шель\-пи\-но} & Порч\-ни\-ха & Ива\-нов\-ская \\ \hline
горизонт литорали & С       & Н       & С       & Н       & С       & Н   & В    & С    & Н    & С           & В    & С    & С    & ВСЛ        \\ \hline
	\\ \hline \endhead
	\hline 
	\multicolumn{15}{|c|}{продолжение таблицы \ref{tab:Barents_species} на следующей странице}
	\\ \hline \endfoot
	 \endlastfoot
\multicolumn{15}{|c|}{Turbellaria} \\ \hline
 Turbellaria varia         &           &           &           &           &           &           &          &          &          &                 & \textbf{+}       & \textbf{+}       &          &            \\ \hline
\multicolumn{15}{|c|}{Nemertini} \\ \hline
{\it Amphiporus lactiflorens}   &           &           &           &           &           &           &          & \textbf{+}       &          &                 &          &          &          &            \\  \hline
{\it Lineus gesserensis}        &           &           & \textbf{+}        &           &           &           &          &          &          &                 &          &          & \textbf{+}       &            \\  \hline
{\it Lineus ruber}              &           &           &           &           &           &           &          &          &          &                 &          &          & \textbf{+}       &            \\  \hline
Nemertini varia           &           & \textbf{+}        &           &           & \textbf{+}        & \textbf{+}        & \textbf{+}       & \textbf{+}       &          & \textbf{+}              &          & \textbf{+}       & \textbf{+}       &            \\ \hline
\multicolumn{15}{|c|}{Priapulida} \\ \hline
{\it Priapulus caudatus}        &           &           &           & \textbf{+}        &           &           &          &          &          & \textbf{+}              &          &          & \textbf{+}       &            \\ \hline
\multicolumn{15}{|c|}{Oligochaeta} \\ \hline
{\it Capitella capitata}        & \textbf{+}        &           & \textbf{+}        & \textbf{+}        &           & \textbf{+}        &          &          &          & \textbf{+}              &          &          & \textbf{+}       &            \\  \hline
Enchytraeidae varia       &           &           & \textbf{+}        &           & \textbf{+}        & \textbf{+}        & \textbf{+}       &          & \textbf{+}       & \textbf{+}              & \textbf{+}       &          & \textbf{+}       &            \\  \hline
{\it Nais sp.}                  &           &           &           &           &           &           &          &          &          &                 & \textbf{+}       & \textbf{+}       &          &            \\  \hline
Oligochaeta gen. sp.      &           &           &           &           &           &           &          &          &          & \textbf{+}              &          &          &          &            \\  \hline
{\it Paranais littoralis}       &           &           &           &           &           &           &          & \textbf{+}       &          & \textbf{+}              &          &          &          &            \\  \hline
{\it Tubifex costatus}          & \textbf{+}        & \textbf{+}        & \textbf{+}        &           & \textbf{+}        &           &          & \textbf{+}       & \textbf{+}       & \textbf{+}              &          &          &          & \textbf{+}         \\  \hline
Tubificidae varia         & \textbf{+}        &           &           &           &           &           &          &          &          &                 &          &          &          &            \\  \hline
{\it Tubificoides benedeni}     &           &           & \textbf{+}        & \textbf{+}        & \textbf{+}        &           &          & \textbf{+}       &          & \textbf{+}              &          &          & \textbf{+}       & \textbf{+}         \\ \hline
\multicolumn{15}{|c|}{Polychaeta} \\ \hline
{\it Alitta virens}             &           & \textbf{+}        &           &           &           &           &          &          &          &                 &          &          &          &            \\  \hline
{\it Arenicola marina}          &           &           &           &           &           &           &          & \textbf{+}       &          & \textbf{+}              & \textbf{+}       & \textbf{+}       &          &            \\  \hline
{\it Clitellio arenarius}       &           & \textbf{+}        &           &           & \textbf{+}        & \textbf{+}        & \textbf{+}       & \textbf{+}       &          & \textbf{+}              &          & \textbf{+}       & \textbf{+}       &            \\  \hline
{\it Eteone longa}              &           &           & \textbf{+}        & \textbf{+}        &           &           &          &          &          &                 &          &          &          &            \\  \hline
{\it Fabricia sabella}          &           & \textbf{+}        & \textbf{+}        &           &           & \textbf{+}        & \textbf{+}       & \textbf{+}       &          & \textbf{+}              & \textbf{+}       & \textbf{+}       &          & \textbf{+}         \\  \hline
{\it Nainereis quadricuspida}   &           &           &           &           &           &           &          &          &          & \textbf{+}              &          &          & \textbf{+}       &            \\  \hline
{\it Nereis pelagica}           &           &           &           & \textbf{+}        &           &           &          &          &          &                 &          &          &          &            \\  \hline
{\it Nereis sp.}                &           &           & \textbf{+}        & \textbf{+}        &           &           &          &          &          &                 &          &          &          &            \\  \hline
{\it Pectinaria koreni}         &           &           &           & \textbf{+}        &           &           &          &          &          &                 &          &          &          &            \\  \hline
{\it Phyllodoce groenlandica}   &           &           &           & \textbf{+}        &           &           &          &          &          & \textbf{+}              &          &          &          &            \\  \hline
{\it Polydora quadrilobata}     &           &           &           &           &           &           &          &          & \textbf{+}       &                 &          &          &          &            \\  \hline
{\it Pygospio elegans}          & \textbf{+}        &           & \textbf{+}        & \textbf{+}        & \textbf{+}        & \textbf{+}        &          & \textbf{+}       &          & \textbf{+}              & \textbf{+}       & \textbf{+}       & \textbf{+}       &            \\  \hline
Sabellidae varia          &           &           & \textbf{+}        & \textbf{+}        &           &           &          &          &          &                 &          &          &          &            \\  \hline
{\it Scalibregma infundibulum}  &           &           &           &           &           &           &          &          & \textbf{+}       &                 &          &          &          &            \\  \hline
{\it Scoloplos armiger}         & \textbf{+}        &           &           &           & \textbf{+}        &           &          &          & \textbf{+}       & \textbf{+}              &          &          & \textbf{+}       &            \\  \hline
{\it Spio sp.}                  &           &           &           &           &           &           &          &          &          &                 &          &          &          & \textbf{+}         \\  \hline
{\it Travisia forbesii}         &           &           &           &           &           &           &          & \textbf{+}       & \textbf{+}       &                 &          &          &          &            \\ \hline
\multicolumn{15}{|c|}{Isopoda} \\ \hline
{\it Jaera sp.}                 &           &           &           &           &           &           &          & \textbf{+}       &          &                 & \textbf{+}       &          &          &            \\ \hline
\multicolumn{15}{|c|}{Amphipoda} \\ \hline
{\it Gammarus sp.}              & \textbf{+}        & \textbf{+}        & \textbf{+}        & \textbf{+}        &           &           & \textbf{+}       & \textbf{+}       &          & \textbf{+}              &          &          &          &            \\  \hline
{\it Hyale prevosti}            &           &           &           & \textbf{+}        &           &           &          &          &          &                 &          &          &          &            \\  \hline
{\it Pseudolibrotus littoralis} &           &           &           &           &           &           &          &          &          & \textbf{+}              &          &          &          &            \\ \hline
\multicolumn{15}{|c|}{Decapoda} \\ \hline
{\it Crangon crangon}           &           &           &           & \textbf{+}        &           &           &          &          &          &                 &          &          &          &            \\ \hline
\multicolumn{15}{|c|}{Diptera} \\ \hline
Chironomidae varia        & \textbf{+}        & \textbf{+}        &           & \textbf{+}        &           & \textbf{+}        & \textbf{+}       & \textbf{+}       & \textbf{+}       & \textbf{+}              & \textbf{+}       & \textbf{+}       & \textbf{+}       &            \\ \hline
\multicolumn{15}{|c|}{Gastropoda} \\ \hline
{\it Epheria vincta}            &           &           &           & \textbf{+}        &           &           &          &          &          &                 &          &          &          &            \\  \hline
{\it Hydrobia ulvae}            &           & \textbf{+}        & \textbf{+}        & \textbf{+}        &           &           &          & \textbf{+}       &          &                 &          &          & \textbf{+}       &            \\  \hline
{\it Littorina gr. obtusata}    &           &           &           &           &           &           &          &          &          &                 &          &          &          &            \\  \hline
{\it Littorina gr. saxatilis}   &           & \textbf{+}        & \textbf{+}        & \textbf{+}        &           &           &          & \textbf{+}       &          &                 & \textbf{+}       &          &          &            \\  \hline
{\it Onoba aculeas}             &           &           &           & \textbf{+}        &           &           &          & \textbf{+}       &          &                 &          &          &          &            \\  \hline
{\it Skineopsis planorbis}      &           &           &           &           &           &           &          & \textbf{+}       &          &                 &          &          &          &            \\ \hline
\multicolumn{15}{|c|}{Bivalvia} \\ \hline
{\it Cerastoderma edule}        &           &           & \textbf{+}        & \textbf{+}        &           &           &          & \textbf{+}       &          & \textbf{+}              &          &          & \textbf{+}       &            \\  \hline
{\it Macoma balthica}           & \textbf{+}        & \textbf{+}        & \textbf{+}        & \textbf{+}        & \textbf{+}        & \textbf{+}        & \textbf{+}       & \textbf{+}       & \textbf{+}       & \textbf{+}              & \textbf{+}       & \textbf{+}       & \textbf{+}       & \textbf{+}         \\  \hline
{\it Mya arenaria}              &           &           &           &           &           &           & \textbf{+}       & \textbf{+}       &          & \textbf{+}              &          &          & \textbf{+}       & \textbf{+}         \\  \hline
{\it Mytilus edulis}            & \textbf{+}        & \textbf{+}        & \textbf{+}        & \textbf{+}        & \textbf{+}        &           & \textbf{+}       & \textbf{+}       & \textbf{+}       & \textbf{+}              & \textbf{+}       & \textbf{+}       & \textbf{+}       &            \\  \hline
{\it Turtonia minuta}           &           &           &           &           &           &           &          &          &          &                 &          &          & \textbf{+}       &           \\ \hline
\end{longtable}
\end{footnotesize}

%%%%%%%%%%%%%%%%%%%%%



%коррелограммы и пузырьковые диаграммы с отдельными возрастами макомы в Пале
%\section{Приложение. Распределение особей {\it Macoma balthica} разного возраста на нижнем горизонте литорали Пала-губы (Кольский залиы, Баренцево море)}
\label{app:Pala_MoranI_ages}


	\begin{figure}[h]
	\begin{minipage}[b]{\linewidth}
	\begin{center}
		Моллюски возрастом 1+
	\end{center}
	\end{minipage}

	\begin{minipage}[b]{.46\linewidth}
%Фигурка в первом ряду слева размер отведенный под весь этот объект \textendash 0.46 от ширины строки
%Параметр [b] означает, что выравнивание этих министраниц будет по нижнему краю
	\begin{center}
		\includegraphics[width=65mm]{../Barenc_Sea/distribution_Moran/Pala_macoma_age_N1_.pdf}
	\end{center}
	\end{minipage}
%
	\hfil %Это пружинка отодвигающая рисунки друг от друга
%
	\begin{minipage}[b]{.46\linewidth}
%Следующий рисунок - первый ряд справа %DUNGEON S_4 \ AB
	\begin{center}
		\includegraphics[width=65mm]{../Barenc_Sea/distribution_Moran/Pala_macoma_age_bubb_N1_.pdf}
	\end{center}
	\end{minipage}

	\begin{minipage}[b]{\linewidth}
	\begin{center}
		Моллюски возрастом 2+
	\end{center}
	\end{minipage}

	\begin{minipage}[b]{.46\linewidth}
%Фигурка в первом ряду слева размер отведенный под весь этот объект \textendash 0.46 от ширины строки
%Параметр [b] означает, что выравнивание этих министраниц будет по нижнему краю
	\begin{center}
		\includegraphics[width=65mm]{../Barenc_Sea/distribution_Moran/Pala_macoma_age_N2_.pdf}
	\end{center}
	\end{minipage}
%
	\hfil %Это пружинка отодвигающая рисунки друг от друга
%
	\begin{minipage}[b]{.46\linewidth}
%Следующий рисунок - первый ряд справа %DUNGEON S_4 \ AB
	\begin{center}
		\includegraphics[width=65mm]{../Barenc_Sea/distribution_Moran/Pala_macoma_age_bubb_N2_.pdf}
	\end{center}
	\end{minipage}

%	\caption{Микрораспределение макробентоса на литорали Пала-губы}
%	\label{ris:moransI_Pala}
	\end{figure}




	\begin{figure}[h]

	\begin{minipage}[b]{\linewidth}
	\begin{center}
		Моллюски возрастом 3+
	\end{center}
	\end{minipage}
	
	\begin{minipage}[b]{.46\linewidth}
	%Фигурка в первом ряду слева размер отведенный под весь этот объект \textendash 0.46 от ширины строки
	%Параметр [b] означает, что выравнивание этих министраниц будет по нижнему краю
	\begin{center}
%	{\small N~{\it Cerastoderma edule}}
		\includegraphics[width=65mm]{../Barenc_Sea/distribution_Moran/Pala_macoma_age_N3_.pdf}
	\end{center}
	\end{minipage}
	%
	\hfil %Это пружинка отодвигающая рисунки друг от друга
	%
	\begin{minipage}[b]{.46\linewidth}
%Следующий рисунок - первый ряд справа %DUNGEON S_4 \ AB
	\begin{center}
		\includegraphics[width=65mm]{../Barenc_Sea/distribution_Moran/Pala_macoma_age_bubb_N3_.pdf}
	\end{center}
	\end{minipage}

	\begin{minipage}[b]{\linewidth}
	\begin{center}
		Моллюски возрастом 4+
	\end{center}
	\end{minipage}

	\begin{minipage}[b]{.46\linewidth}
%Фигурка в первом ряду слева размер отведенный под весь этот объект \textendash 0.46 от ширины строки
%Параметр [b] означает, что выравнивание этих министраниц будет по нижнему краю
	\begin{center}
		\includegraphics[width=65mm]{../Barenc_Sea/distribution_Moran/Pala_macoma_age_N4_.pdf}
	\end{center}
	\end{minipage}
%
	\hfil %Это пружинка отодвигающая рисунки друг от друга
%
	\begin{minipage}[b]{.46\linewidth}
%Следующий рисунок - первый ряд справа %DUNGEON S_4 \ AB
	\begin{center}
		\includegraphics[width=65mm]{../Barenc_Sea/distribution_Moran/Pala_macoma_age_bubb_N4_.pdf}
	\end{center}
	\end{minipage}

	\begin{minipage}[b]{\linewidth}
	\begin{center}
		Моллюски возрастом 5+
	\end{center}
	\end{minipage}
	
	\begin{minipage}[b]{.46\linewidth}
	%Фигурка в первом ряду слева размер отведенный под весь этот объект \textendash 0.46 от ширины строки
	%Параметр [b] означает, что выравнивание этих министраниц будет по нижнему краю
	\begin{center}
%	{\small N~{\it Cerastoderma edule}}
		\includegraphics[width=65mm]{../Barenc_Sea/distribution_Moran/Pala_macoma_age_N5_.pdf}
	\end{center}
	\end{minipage}
	%
	\hfil %Это пружинка отодвигающая рисунки друг от друга
	%
	\begin{minipage}[b]{.46\linewidth}
%Следующий рисунок - первый ряд справа %DUNGEON S_4 \ AB
	\begin{center}
		\includegraphics[width=65mm]{../Barenc_Sea/distribution_Moran/Pala_macoma_age_bubb_N5_.pdf}
	\end{center}
	\end{minipage}

%	\caption{Микрораспределение макробентоса на литорали Пала-губы}
%	\label{ris:moransI_Pala}
	\end{figure}




	\begin{figure}[h]

	\begin{minipage}[b]{\linewidth}
	\begin{center}
		Моллюски возрастом 6+
	\end{center}
	\end{minipage}
	
	\begin{minipage}[b]{.46\linewidth}
	%Фигурка в первом ряду слева размер отведенный под весь этот объект \textendash 0.46 от ширины строки
	%Параметр [b] означает, что выравнивание этих министраниц будет по нижнему краю
	\begin{center}
%	{\small N~{\it Cerastoderma edule}}
		\includegraphics[width=65mm]{../Barenc_Sea/distribution_Moran/Pala_macoma_age_N6_.pdf}
	\end{center}
	\end{minipage}
	%
	\hfil %Это пружинка отодвигающая рисунки друг от друга
	%
	\begin{minipage}[b]{.46\linewidth}
%Следующий рисунок - первый ряд справа %DUNGEON S_4 \ AB
	\begin{center}
		\includegraphics[width=65mm]{../Barenc_Sea/distribution_Moran/Pala_macoma_age_bubb_N6_.pdf}
	\end{center}
	\end{minipage}

	\begin{minipage}[b]{\linewidth}
	\begin{center}
		Моллюски возрастом 7+
	\end{center}
	\end{minipage}

	\begin{minipage}[b]{.46\linewidth}
%Фигурка в первом ряду слева размер отведенный под весь этот объект \textendash 0.46 от ширины строки
%Параметр [b] означает, что выравнивание этих министраниц будет по нижнему краю
	\begin{center}
		\includegraphics[width=65mm]{../Barenc_Sea/distribution_Moran/Pala_macoma_age_N7_.pdf}
	\end{center}
	\end{minipage}
%
	\hfil %Это пружинка отодвигающая рисунки друг от друга
%
	\begin{minipage}[b]{.46\linewidth}
%Следующий рисунок - первый ряд справа %DUNGEON S_4 \ AB
	\begin{center}
		\includegraphics[width=65mm]{../Barenc_Sea/distribution_Moran/Pala_macoma_age_bubb_N7_.pdf}
	\end{center}
	\end{minipage}

	\begin{minipage}[b]{\linewidth}
	\begin{center}
		Моллюски возрастом 8+
	\end{center}
	\end{minipage}
	
	\begin{minipage}[b]{.46\linewidth}
	%Фигурка в первом ряду слева размер отведенный под весь этот объект \textendash 0.46 от ширины строки
	%Параметр [b] означает, что выравнивание этих министраниц будет по нижнему краю
	\begin{center}
%	{\small N~{\it Cerastoderma edule}}
		\includegraphics[width=65mm]{../Barenc_Sea/distribution_Moran/Pala_macoma_age_N8_.pdf}
	\end{center}
	\end{minipage}
	%
	\hfil %Это пружинка отодвигающая рисунки друг от друга
	%
	\begin{minipage}[b]{.46\linewidth}
%Следующий рисунок - первый ряд справа %DUNGEON S_4 \ AB
	\begin{center}
		\includegraphics[width=65mm]{../Barenc_Sea/distribution_Moran/Pala_macoma_age_bubb_N8_.pdf}
	\end{center}
	\end{minipage}

%	\caption{Микрораспределение макробентоса на литорали Пала-губы}
%	\label{ris:moransI_Pala}
	\end{figure}




	\begin{figure}[h]

	\begin{minipage}[b]{\linewidth}
	\begin{center}
		Моллюски возрастом 9+
	\end{center}
	\end{minipage}
	
	\begin{minipage}[b]{.46\linewidth}
	%Фигурка в первом ряду слева размер отведенный под весь этот объект \textendash 0.46 от ширины строки
	%Параметр [b] означает, что выравнивание этих министраниц будет по нижнему краю
	\begin{center}
%	{\small N~{\it Cerastoderma edule}}
		\includegraphics[width=65mm]{../Barenc_Sea/distribution_Moran/Pala_macoma_age_N9_.pdf}
	\end{center}
	\end{minipage}
	%
	\hfil %Это пружинка отодвигающая рисунки друг от друга
	%
	\begin{minipage}[b]{.46\linewidth}
%Следующий рисунок - первый ряд справа %DUNGEON S_4 \ AB
	\begin{center}
		\includegraphics[width=65mm]{../Barenc_Sea/distribution_Moran/Pala_macoma_age_bubb_N9_.pdf}
	\end{center}
	\end{minipage}

	\begin{minipage}[b]{\linewidth}
	\begin{center}
		Моллюски возрастом 10+
	\end{center}
	\end{minipage}

	\begin{minipage}[b]{.46\linewidth}
%Фигурка в первом ряду слева размер отведенный под весь этот объект \textendash 0.46 от ширины строки
%Параметр [b] означает, что выравнивание этих министраниц будет по нижнему краю
	\begin{center}
		\includegraphics[width=65mm]{../Barenc_Sea/distribution_Moran/Pala_macoma_age_N10_.pdf}
	\end{center}
	\end{minipage}
%
	\hfil %Это пружинка отодвигающая рисунки друг от друга
%
	\begin{minipage}[b]{.46\linewidth}
%Следующий рисунок - первый ряд справа %DUNGEON S_4 \ AB
	\begin{center}
		\includegraphics[width=65mm]{../Barenc_Sea/distribution_Moran/Pala_macoma_age_bubb_N10_.pdf}
	\end{center}
	\end{minipage}

	\begin{minipage}[b]{\linewidth}
	\begin{center}
		Моллюски возрастом 11+
	\end{center}
	\end{minipage}
	
	\begin{minipage}[b]{.46\linewidth}
	%Фигурка в первом ряду слева размер отведенный под весь этот объект \textendash 0.46 от ширины строки
	%Параметр [b] означает, что выравнивание этих министраниц будет по нижнему краю
	\begin{center}
%	{\small N~{\it Cerastoderma edule}}
		\includegraphics[width=65mm]{../Barenc_Sea/distribution_Moran/Pala_macoma_age_N11_.pdf}
	\end{center}
	\end{minipage}
	%
	\hfil %Это пружинка отодвигающая рисунки друг от друга
	%
	\begin{minipage}[b]{.46\linewidth}
%Следующий рисунок - первый ряд справа %DUNGEON S_4 \ AB
	\begin{center}
		\includegraphics[width=65mm]{../Barenc_Sea/distribution_Moran/Pala_macoma_age_bubb_N11_.pdf}
	\end{center}
	\end{minipage}

%	\caption{Микрораспределение макробентоса на литорали Пала-губы}
%	\label{ris:moransI_Pala}
	\end{figure}




	\begin{figure}[h]

	\begin{minipage}[b]{\linewidth}
	\begin{center}
		Моллюски возрастом 12+
	\end{center}
	\end{minipage}
	
	\begin{minipage}[b]{.46\linewidth}
	%Фигурка в первом ряду слева размер отведенный под весь этот объект \textendash 0.46 от ширины строки
	%Параметр [b] означает, что выравнивание этих министраниц будет по нижнему краю
	\begin{center}
%	{\small N~{\it Cerastoderma edule}}
		\includegraphics[width=65mm]{../Barenc_Sea/distribution_Moran/Pala_macoma_age_N12_.pdf}
	\end{center}
	\end{minipage}
	%
	\hfil %Это пружинка отодвигающая рисунки друг от друга
	%
	\begin{minipage}[b]{.46\linewidth}
%Следующий рисунок - первый ряд справа %DUNGEON S_4 \ AB
	\begin{center}
		\includegraphics[width=65mm]{../Barenc_Sea/distribution_Moran/Pala_macoma_age_bubb_N12_.pdf}
	\end{center}
	\end{minipage}

	\begin{minipage}[b]{\linewidth}
	\begin{center}
		Моллюски возрастом 13+
	\end{center}
	\end{minipage}

	\begin{minipage}[b]{.46\linewidth}
%Фигурка в первом ряду слева размер отведенный под весь этот объект \textendash 0.46 от ширины строки
%Параметр [b] означает, что выравнивание этих министраниц будет по нижнему краю
	\begin{center}
		\includegraphics[width=65mm]{../Barenc_Sea/distribution_Moran/Pala_macoma_age_N13_.pdf}
	\end{center}
	\end{minipage}
%
	\hfil %Это пружинка отодвигающая рисунки друг от друга
%
	\begin{minipage}[b]{.46\linewidth}
%Следующий рисунок - первый ряд справа %DUNGEON S_4 \ AB
	\begin{center}
		\includegraphics[width=65mm]{../Barenc_Sea/distribution_Moran/Pala_macoma_age_bubb_N13_.pdf}
	\end{center}
	\end{minipage}

	\begin{minipage}[b]{\linewidth}
	\begin{center}
		Моллюски возрастом 14+
	\end{center}
	\end{minipage}
	
	\begin{minipage}[b]{.46\linewidth}
	%Фигурка в первом ряду слева размер отведенный под весь этот объект \textendash 0.46 от ширины строки
	%Параметр [b] означает, что выравнивание этих министраниц будет по нижнему краю
	\begin{center}
%	{\small N~{\it Cerastoderma edule}}
		\includegraphics[width=65mm]{../Barenc_Sea/distribution_Moran/Pala_macoma_age_N14_.pdf}
	\end{center}
	\end{minipage}
	%
	\hfil %Это пружинка отодвигающая рисунки друг от друга
	%
	\begin{minipage}[b]{.46\linewidth}
%Следующий рисунок - первый ряд справа %DUNGEON S_4 \ AB
	\begin{center}
		\includegraphics[width=65mm]{../Barenc_Sea/distribution_Moran/Pala_macoma_age_bubb_N14_.pdf}
	\end{center}
	\end{minipage}

%	\caption{Микрораспределение макробентоса на литорали Пала-губы}
%	\label{ris:moransI_Pala}
	\end{figure}

 
%картинки с размерными структурами
\section{Размерная структура {\it Macoma balthica} в исследованных поселениях Кандалакшского залива Белого моря}
\label{app:White_sizestr_hist}

На всех графиках абсцисса --- длина раковины, мм; ордината --- численность особей, экз./м$^2$. Указано средняя численность особей определенного размера $\pm$ ошибка средней.

%Эстуарий Лувеньги
	\begin{figure}[hp]

	\begin{minipage}[b]{.3\linewidth}
	\begin{center}
	\includegraphics[width=60mm]{../White_Sea/Estuatiy_Luvenga/sizestr2_1992_.pdf}	
	\end{center}
	\end{minipage}
	%
	\hfil %Это пружинка отодвигающая рисунки друг от друга
	%
	\begin{minipage}[b]{.3\linewidth}
	\begin{center}
	\includegraphics[width=60mm]{../White_Sea/Estuatiy_Luvenga/sizestr2_1995_.pdf}
	\end{center}
	\end{minipage}
	%
	\hfil %Это пружинка отодвигающая рисунки друг от друга
	%
	\begin{minipage}[b]{.3\linewidth}
	\begin{center}
\includegraphics[width=60mm]{../White_Sea/Estuatiy_Luvenga/sizestr2_1998_.pdf}
	\end{center}
	\end{minipage}
	%
	\begin{minipage}[b]{.3\linewidth}
	\begin{center}
	\includegraphics[width=60mm]{../White_Sea/Estuatiy_Luvenga/sizestr2_1993_.pdf}
	\end{center}
	\end{minipage}
	%
	\hfil %Это пружинка отодвигающая рисунки друг от друга
	%
	\begin{minipage}[b]{.3\linewidth}
	\begin{center}
	\includegraphics[width=60mm]{../White_Sea/Estuatiy_Luvenga/sizestr2_1996_.pdf}
	\end{center}
	\end{minipage}
	%
	\hfil %Это пружинка отодвигающая рисунки друг от друга
	%
	\begin{minipage}[b]{.3\linewidth}
	\begin{center}
	\includegraphics[width=60mm]{../White_Sea/Estuatiy_Luvenga/sizestr2_1999_.pdf}
	\end{center}
	\end{minipage}
	%


	\begin{minipage}[b]{.3\linewidth}
	\begin{center}
\includegraphics[width=60mm]{../White_Sea/Estuatiy_Luvenga/sizestr2_1994_.pdf}
	\end{center}
	\end{minipage}
	%
	\hfill
	%
	\begin{minipage}[b]{.3\linewidth}
	\begin{center}
	\includegraphics[width=60mm]{../White_Sea/Estuatiy_Luvenga/sizestr2_1997_.pdf}
	\end{center}
	\end{minipage}	
	%
	\hfill
	%
	\begin{minipage}[b]{.3\linewidth}
	\begin{center}
	\includegraphics[width=60mm]{../White_Sea/Estuatiy_Luvenga/sizestr2_2000_.pdf}
	\end{center}
	\end{minipage}
%\smallskip
	%
	\caption{Размерная структура {\it Macoma balthica} в СГЛ эстуария р. Лувеньги}
	\label{ris:size_str_estuary_Luv}
	\end{figure}

%%%% вторая страница картинок с РС
	\begin{figure}[hp]

	\begin{minipage}[b]{.3\linewidth}
	\begin{center}
	\includegraphics[width=60mm]{../White_Sea/Estuatiy_Luvenga/sizestr2_2001_.pdf}
	\end{center}
	\end{minipage}
	%
	\hfill
	%
	\begin{minipage}[b]{.3\linewidth}
	\begin{center}
	\includegraphics[width=60mm]{../White_Sea/Estuatiy_Luvenga/sizestr2_2005_.pdf}
	\end{center}
	\end{minipage}
	%
	\hfill
	%
	\begin{minipage}[b]{.3\linewidth}
	\begin{center}
	\includegraphics[width=60mm]{../White_Sea/Estuatiy_Luvenga/sizestr2_2009_.pdf}
	\end{center}
	\end{minipage}
	%
	%
	\begin{minipage}[b]{.3\linewidth}
	\begin{center}
	\includegraphics[width=60mm]{../White_Sea/Estuatiy_Luvenga/sizestr2_2002_.pdf}
	\end{center}
	\end{minipage}
	%
	\hfill	
	%
	\begin{minipage}[b]{.3\linewidth}
	\begin{center}
	\includegraphics[width=60mm]{../White_Sea/Estuatiy_Luvenga/sizestr2_2006_.pdf}
	\end{center}
	\end{minipage}
	%
	\hfill
	%
	\begin{minipage}[b]{.3\linewidth}
	\begin{center}
	\includegraphics[width=60mm]{../White_Sea/Estuatiy_Luvenga/sizestr2_2010_.pdf}
	\end{center}
	\end{minipage}
	%
	%
	\begin{minipage}[b]{.3\linewidth}
	\begin{center}
	\includegraphics[width=60mm]{../White_Sea/Estuatiy_Luvenga/sizestr2_2003_.pdf}
	\end{center}
	\end{minipage}
	%
	\hfill
	%
	\begin{minipage}[b]{.3\linewidth}
	\begin{center}
	\includegraphics[width=60mm]{../White_Sea/Estuatiy_Luvenga/sizestr2_2007_.pdf}
	\end{center}
	\end{minipage}
	%
	\hfill
	%
	\begin{minipage}[b]{.3\linewidth}
	\begin{center}
	\includegraphics[width=60mm]{../White_Sea/Estuatiy_Luvenga/sizestr2_2011_.pdf}
	\end{center}
	\end{minipage}
	%

	\begin{minipage}[b]{.3\linewidth}
	\begin{center}
	\includegraphics[width=60mm]{../White_Sea/Estuatiy_Luvenga/sizestr2_2004_.pdf}
	\end{center}
	\end{minipage}
	%
	\hfill
	%
	\begin{minipage}[b]{.3\linewidth}
	\begin{center}
	\includegraphics[width=60mm]{../White_Sea/Estuatiy_Luvenga/sizestr2_2008_.pdf}
	\end{center}
	\end{minipage}
	%
	\hfill
	%
	\begin{minipage}[b]{.3\linewidth}
	\begin{center}
	\includegraphics[width=60mm]{../White_Sea/Estuatiy_Luvenga/sizestr2_2012_.pdf}
	\end{center}
	\end{minipage}
	\begin{center}
	Рис. \ref{ris:size_str_estuary_Luv} (продолжение). Размерная структура {\it Macoma balthica} в СГЛ эстуария р. Лувеньги
	\end{center}
	\end{figure}


%%%%%%%%%%%%%%%%%%%%%%%%%%%%%%
%Горелый верх
	\begin{figure}[hp]

	\begin{minipage}[b]{.3\linewidth}
	\begin{center}
	\includegraphics[width=60mm]{../White_Sea/Luvenga_Goreliy/high2_1992_.pdf}	
	\end{center}
	\end{minipage}
	%
	\hfil %Это пружинка отодвигающая рисунки друг от друга
	%
	\begin{minipage}[b]{.3\linewidth}
	\begin{center}
	\includegraphics[width=60mm]{../White_Sea/Luvenga_Goreliy/high2_1996_.pdf}
	\end{center}
	\end{minipage}
	%
	\hfil %Это пружинка отодвигающая рисунки друг от друга
	%
	\begin{minipage}[b]{.3\linewidth}
	\begin{center}
\includegraphics[width=60mm]{../White_Sea/Luvenga_Goreliy/high2_2000_.pdf}
	\end{center}
	\end{minipage}
	%
	\begin{minipage}[b]{.3\linewidth}
	\begin{center}
	\includegraphics[width=60mm]{../White_Sea/Luvenga_Goreliy/high2_1993_.pdf}
	\end{center}
	\end{minipage}
	%
	\hfil %Это пружинка отодвигающая рисунки друг от друга
	%
	\begin{minipage}[b]{.3\linewidth}
	\begin{center}
	\includegraphics[width=60mm]{../White_Sea/Luvenga_Goreliy/high2_1997_.pdf}
	\end{center}
	\end{minipage}
	%
	\hfil %Это пружинка отодвигающая рисунки друг от друга
	%
	\begin{minipage}[b]{.3\linewidth}
	\begin{center}
	\includegraphics[width=60mm]{../White_Sea/Luvenga_Goreliy/high2_2001_.pdf}
	\end{center}
	\end{minipage}
	%


	\begin{minipage}[b]{.3\linewidth}
	\begin{center}
\includegraphics[width=60mm]{../White_Sea/Luvenga_Goreliy/high2_1994_.pdf}
	\end{center}
	\end{minipage}
	%
	\hfill
	%
	\begin{minipage}[b]{.3\linewidth}
	\begin{center}
	\includegraphics[width=60mm]{../White_Sea/Luvenga_Goreliy/high2_1998_.pdf}
	\end{center}
	\end{minipage}	
	%
	\hfill
	%
	\begin{minipage}[b]{.3\linewidth}
	\begin{center}
	\includegraphics[width=60mm]{../White_Sea/Luvenga_Goreliy/high2_2002_.pdf}
	\end{center}
	\end{minipage}
%\smallskip

	\begin{minipage}[b]{.3\linewidth}
	\begin{center}
	\includegraphics[width=60mm]{../White_Sea/Luvenga_Goreliy/high2_1995_.pdf}
	\end{center}
	\end{minipage}
	%
	\hfill
	%
	\begin{minipage}[b]{.3\linewidth}
	\begin{center}
	\includegraphics[width=60mm]{../White_Sea/Luvenga_Goreliy/high2_1999_.pdf}
	\end{center}
	\end{minipage}
	%
	\hfill
	%
	\begin{minipage}[b]{.3\linewidth}
	\begin{center}
	\includegraphics[width=60mm]{../White_Sea/Luvenga_Goreliy/high2_2003_.pdf}
	\end{center}
	\end{minipage}
	%
	\caption{Размерная структура {\it Macoma balthica} в ВГЛ о. Горелого}
	\label{ris:size_str_Goreliy_high}
	\end{figure}

%%%% вторая страница картинок с РС
	\begin{figure}[hp]

	\begin{minipage}[b]{.3\linewidth}
	\begin{center}
	\includegraphics[width=60mm]{../White_Sea/Luvenga_Goreliy/high2_2004_.pdf}
	\end{center}
	\end{minipage}
	%
	\hfill
	%
	\begin{minipage}[b]{.3\linewidth}
	\begin{center}
	\includegraphics[width=60mm]{../White_Sea/Luvenga_Goreliy/high2_2007_.pdf}
	\end{center}
	\end{minipage}
	%
	\hfill
	%
	\begin{minipage}[b]{.3\linewidth}
	\begin{center}
	\includegraphics[width=60mm]{../White_Sea/Luvenga_Goreliy/high2_2010_.pdf}
	\end{center}
	\end{minipage}
	%
	%
	\begin{minipage}[b]{.3\linewidth}
	\begin{center}
	\includegraphics[width=60mm]{../White_Sea/Luvenga_Goreliy/high2_2005_.pdf}
	\end{center}
	\end{minipage}
	%
	\hfill	
	%
	\begin{minipage}[b]{.3\linewidth}
	\begin{center}
	\includegraphics[width=60mm]{../White_Sea/Luvenga_Goreliy/high2_2008_.pdf}
	\end{center}
	\end{minipage}
	%
	\hfill
	%
	\begin{minipage}[b]{.3\linewidth}
	\begin{center}
	\includegraphics[width=60mm]{../White_Sea/Luvenga_Goreliy/high2_2011_.pdf}
	\end{center}
	\end{minipage}
	%
	%
	\begin{minipage}[b]{.3\linewidth}
	\begin{center}
	\includegraphics[width=60mm]{../White_Sea/Luvenga_Goreliy/high2_2006_.pdf}
	\end{center}
	\end{minipage}
	%
	\hfill
	%
	\begin{minipage}[b]{.3\linewidth}
	\begin{center}
	\includegraphics[width=60mm]{../White_Sea/Luvenga_Goreliy/high2_2009_.pdf}
	\end{center}
	\end{minipage}
	%
	\hfill
	%
	\begin{minipage}[b]{.3\linewidth}
	\begin{center}
	\includegraphics[width=60mm]{../White_Sea/Luvenga_Goreliy/high2_2012_.pdf}
	\end{center}
	\end{minipage}
	%
	\begin{center}
	Рис. \ref{ris:size_str_Goreliy_high} (продолжение). Размерная структура {\it Macoma balthica} в ВГЛ о. Горелого
	\end{center}
	\end{figure}


%%%%%%%%%%%%%%%%%%%%%%%%%%%%%%
%Горелый середина
	\begin{figure}[hp]

	\begin{minipage}[b]{.3\linewidth}
	\begin{center}
	\includegraphics[width=60mm]{../White_Sea/Luvenga_Goreliy/middle2_1992_.pdf}	
	\end{center}
	\end{minipage}
	%
	\hfil %Это пружинка отодвигающая рисунки друг от друга
	%
	\begin{minipage}[b]{.3\linewidth}
	\begin{center}
	\includegraphics[width=60mm]{../White_Sea/Luvenga_Goreliy/middle2_1996_.pdf}
	\end{center}
	\end{minipage}
	%
	\hfil %Это пружинка отодвигающая рисунки друг от друга
	%
	\begin{minipage}[b]{.3\linewidth}
	\begin{center}
\includegraphics[width=60mm]{../White_Sea/Luvenga_Goreliy/middle2_2000_.pdf}
	\end{center}
	\end{minipage}
	%
	\begin{minipage}[b]{.3\linewidth}
	\begin{center}
	\includegraphics[width=60mm]{../White_Sea/Luvenga_Goreliy/middle2_1993_.pdf}
	\end{center}
	\end{minipage}
	%
	\hfil %Это пружинка отодвигающая рисунки друг от друга
	%
	\begin{minipage}[b]{.3\linewidth}
	\begin{center}
	\includegraphics[width=60mm]{../White_Sea/Luvenga_Goreliy/middle2_1997_.pdf}
	\end{center}
	\end{minipage}
	%
	\hfil %Это пружинка отодвигающая рисунки друг от друга
	%
	\begin{minipage}[b]{.3\linewidth}
	\begin{center}
	\includegraphics[width=60mm]{../White_Sea/Luvenga_Goreliy/middle2_2001_.pdf}
	\end{center}
	\end{minipage}
	%


	\begin{minipage}[b]{.3\linewidth}
	\begin{center}
	\includegraphics[width=60mm]{../White_Sea/Luvenga_Goreliy/middle2_1994_.pdf}
	\end{center}
	\end{minipage}
	%
	\hfill
	%
	\begin{minipage}[b]{.3\linewidth}
	\begin{center}
	\includegraphics[width=60mm]{../White_Sea/Luvenga_Goreliy/middle2_1998_.pdf}
	\end{center}
	\end{minipage}	
	%
	\hfill
	%
	\begin{minipage}[b]{.3\linewidth}
	\begin{center}
	\includegraphics[width=60mm]{../White_Sea/Luvenga_Goreliy/middle2_2002_.pdf}
	\end{center}
	\end{minipage}
%\smallskip

	\begin{minipage}[b]{.3\linewidth}
	\begin{center}
	\includegraphics[width=60mm]{../White_Sea/Luvenga_Goreliy/middle2_1995_.pdf}
	\end{center}
	\end{minipage}
	%
	\hfill
	%
	\begin{minipage}[b]{.3\linewidth}
	\begin{center}
	\includegraphics[width=60mm]{../White_Sea/Luvenga_Goreliy/middle2_1999_.pdf}
	\end{center}
	\end{minipage}
	%
	\hfill
	%
	\begin{minipage}[b]{.3\linewidth}
	\begin{center}
	\includegraphics[width=60mm]{../White_Sea/Luvenga_Goreliy/middle2_2003_.pdf}
	\end{center}
	\end{minipage}
	%
\caption{Размерная структура {\it Macoma balthica} в СГЛ о. Горелого}
\label{ris:size_str_Goreliy_mid}
\end{figure}

%%%% вторая страница картинок с РС
	\begin{figure}[hp]

	\begin{minipage}[b]{.3\linewidth}
	\begin{center}
	\includegraphics[width=60mm]{../White_Sea/Luvenga_Goreliy/middle2_2004_.pdf}
	\end{center}
	\end{minipage}
	%
	\hfill
	%
	\begin{minipage}[b]{.3\linewidth}
	\begin{center}
	\includegraphics[width=60mm]{../White_Sea/Luvenga_Goreliy/middle2_2007_.pdf}
	\end{center}
	\end{minipage}
	%
	\hfill
	%
	\begin{minipage}[b]{.3\linewidth}
	\begin{center}
	\includegraphics[width=60mm]{../White_Sea/Luvenga_Goreliy/middle2_2010_.pdf}
	\end{center}
	\end{minipage}
	%
	%
	\begin{minipage}[b]{.3\linewidth}
	\begin{center}
	\includegraphics[width=60mm]{../White_Sea/Luvenga_Goreliy/middle2_2005_.pdf}
	\end{center}
	\end{minipage}
	%
	\hfill	
	%
	\begin{minipage}[b]{.3\linewidth}
	\begin{center}
	\includegraphics[width=60mm]{../White_Sea/Luvenga_Goreliy/middle2_2008_.pdf}
	\end{center}
	\end{minipage}
	%
	\hfill
	%
	\begin{minipage}[b]{.3\linewidth}
	\begin{center}
	\includegraphics[width=60mm]{../White_Sea/Luvenga_Goreliy/middle2_2011_.pdf}
	\end{center}
	\end{minipage}
	%
	%
	\begin{minipage}[b]{.3\linewidth}
	\begin{center}
	\includegraphics[width=60mm]{../White_Sea/Luvenga_Goreliy/middle2_2006_.pdf}
	\end{center}
	\end{minipage}
	%
	\hfill
	%
	\begin{minipage}[b]{.3\linewidth}
	\begin{center}
	\includegraphics[width=60mm]{../White_Sea/Luvenga_Goreliy/middle2_2009_.pdf}
	\end{center}
	\end{minipage}
	%
	\hfill
	%
	\begin{minipage}[b]{.3\linewidth}
	\begin{center}
	\includegraphics[width=60mm]{../White_Sea/Luvenga_Goreliy/middle2_2012_.pdf}
	\end{center}
	\end{minipage}
	%
	\begin{center}
Рис. \ref{ris:size_str_Goreliy_mid} (продолжение). Размерная структура {\it Macoma balthica} в СГЛ о. Горелого
\end{center}
\end{figure}

%%%%%%%%%%%%%%%%%%%%%%%%%%%%%%
%Горелый низ
	\begin{figure}[hp]

	\begin{minipage}[b]{.3\linewidth}
	\begin{center}
	\includegraphics[width=60mm]{../White_Sea/Luvenga_Goreliy/midlow2_1992_.pdf}	
	\end{center}
	\end{minipage}
	%
	\hfil %Это пружинка отодвигающая рисунки друг от друга
	%
	\begin{minipage}[b]{.3\linewidth}
	\begin{center}
	\includegraphics[width=60mm]{../White_Sea/Luvenga_Goreliy/midlow2_1996_.pdf}
	\end{center}
	\end{minipage}
	%
	\hfil %Это пружинка отодвигающая рисунки друг от друга
	%
	\begin{minipage}[b]{.3\linewidth}
	\begin{center}
\includegraphics[width=60mm]{../White_Sea/Luvenga_Goreliy/midlow2_2000_.pdf}
	\end{center}
	\end{minipage}
	%
	\begin{minipage}[b]{.3\linewidth}
	\begin{center}
	\includegraphics[width=60mm]{../White_Sea/Luvenga_Goreliy/midlow2_1993_.pdf}
	\end{center}
	\end{minipage}
	%
	\hfil %Это пружинка отодвигающая рисунки друг от друга
	%
	\begin{minipage}[b]{.3\linewidth}
	\begin{center}
	\includegraphics[width=60mm]{../White_Sea/Luvenga_Goreliy/midlow2_1997_.pdf}
	\end{center}
	\end{minipage}
	%
	\hfil %Это пружинка отодвигающая рисунки друг от друга
	%
	\begin{minipage}[b]{.3\linewidth}
	\begin{center}
	\includegraphics[width=60mm]{../White_Sea/Luvenga_Goreliy/midlow2_2001_.pdf}
	\end{center}
	\end{minipage}
	%


	\begin{minipage}[b]{.3\linewidth}
	\begin{center}
\includegraphics[width=60mm]{../White_Sea/Luvenga_Goreliy/midlow2_1994_.pdf}
	\end{center}
	\end{minipage}
	%
	\hfill
	%
	\begin{minipage}[b]{.3\linewidth}
	\begin{center}
	\includegraphics[width=60mm]{../White_Sea/Luvenga_Goreliy/midlow2_1998_.pdf}
	\end{center}
	\end{minipage}	
	%
	\hfill
	%
	\begin{minipage}[b]{.3\linewidth}
	\begin{center}
	\includegraphics[width=60mm]{../White_Sea/Luvenga_Goreliy/midlow2_2002_.pdf}
	\end{center}
	\end{minipage}
%\smallskip

	\begin{minipage}[b]{.3\linewidth}
	\begin{center}
	\includegraphics[width=60mm]{../White_Sea/Luvenga_Goreliy/midlow2_1995_.pdf}
	\end{center}
	\end{minipage}
	%
	\hfill
	%
	\begin{minipage}[b]{.3\linewidth}
	\begin{center}
	\includegraphics[width=60mm]{../White_Sea/Luvenga_Goreliy/midlow2_1999_.pdf}
	\end{center}
	\end{minipage}
	%
	\hfill
	%
	\begin{minipage}[b]{.3\linewidth}
	\begin{center}
	\includegraphics[width=60mm]{../White_Sea/Luvenga_Goreliy/midlow2_2003_.pdf}
	\end{center}
	\end{minipage}
	%
\caption{Размерная структура {\it Macoma balthica} в НГЛ о. Горелого}
\label{ris:size_str_Goreliy_midlow}
\end{figure}

%%%% вторая страница картинок с РС
	\begin{figure}[hp]

	\begin{minipage}[b]{.3\linewidth}
	\begin{center}
	\includegraphics[width=60mm]{../White_Sea/Luvenga_Goreliy/midlow2_2004_.pdf}
	\end{center}
	\end{minipage}
	%
	\hfill
	%
	\begin{minipage}[b]{.3\linewidth}
	\begin{center}
	\includegraphics[width=60mm]{../White_Sea/Luvenga_Goreliy/midlow2_2007_.pdf}
	\end{center}
	\end{minipage}
	%
	\hfill
	%
	\begin{minipage}[b]{.3\linewidth}
	\begin{center}
	\includegraphics[width=60mm]{../White_Sea/Luvenga_Goreliy/midlow2_2010_.pdf}
	\end{center}
	\end{minipage}
	%
	%
	\begin{minipage}[b]{.3\linewidth}
	\begin{center}
	\includegraphics[width=60mm]{../White_Sea/Luvenga_Goreliy/midlow2_2005_.pdf}
	\end{center}
	\end{minipage}
	%
	\hfill	
	%
	\begin{minipage}[b]{.3\linewidth}
	\begin{center}
	\includegraphics[width=60mm]{../White_Sea/Luvenga_Goreliy/midlow2_2008_.pdf}
	\end{center}
	\end{minipage}
	%
	\hfill
	%
	\begin{minipage}[b]{.3\linewidth}
	\begin{center}
	\includegraphics[width=60mm]{../White_Sea/Luvenga_Goreliy/midlow2_2011_.pdf}
	\end{center}
	\end{minipage}
	%
	%
	\begin{minipage}[b]{.3\linewidth}
	\begin{center}
	\includegraphics[width=60mm]{../White_Sea/Luvenga_Goreliy/midlow2_2006_.pdf}
	\end{center}
	\end{minipage}
	%
	\hfill
	%
	\begin{minipage}[b]{.3\linewidth}
	\begin{center}
	\includegraphics[width=60mm]{../White_Sea/Luvenga_Goreliy/midlow2_2009_.pdf}
	\end{center}
	\end{minipage}
	%
	\hfill
	%
	\begin{minipage}[b]{.3\linewidth}
	\begin{center}
	\includegraphics[width=60mm]{../White_Sea/Luvenga_Goreliy/midlow2_2012_.pdf}
	\end{center}
	\end{minipage}
	%
\begin{center}
Рис. \ref{ris:size_str_Goreliy_midlow} (продолжение). Размерная структура {\it Macoma balthica} в НГЛ о. Горелого
\end{center}
\end{figure}




%%%%%%%%%%%%%%%%%%%%%%%%%%%%%%
%Горелый ноль глубин
	\begin{figure}[hp]

	\begin{minipage}[b]{.3\linewidth}
	\begin{center}
	\includegraphics[width=60mm]{../White_Sea/Luvenga_Goreliy/low2_1992_.pdf}	
	\end{center}
	\end{minipage}
	%
	\hfil %Это пружинка отодвигающая рисунки друг от друга
	%
	\begin{minipage}[b]{.3\linewidth}
	\begin{center}
	\includegraphics[width=60mm]{../White_Sea/Luvenga_Goreliy/low2_1996_.pdf}
	\end{center}
	\end{minipage}
	%
	\hfil %Это пружинка отодвигающая рисунки друг от друга
	%
	\begin{minipage}[b]{.3\linewidth}
	\begin{center}
\includegraphics[width=60mm]{../White_Sea/Luvenga_Goreliy/low2_2000_.pdf}
	\end{center}
	\end{minipage}
	%
	\begin{minipage}[b]{.3\linewidth}
	\begin{center}
	\includegraphics[width=60mm]{../White_Sea/Luvenga_Goreliy/low2_1993_.pdf}
	\end{center}
	\end{minipage}
	%
	\hfil %Это пружинка отодвигающая рисунки друг от друга
	%
	\begin{minipage}[b]{.3\linewidth}
	\begin{center}
	\includegraphics[width=60mm]{../White_Sea/Luvenga_Goreliy/low2_1997_.pdf}
	\end{center}
	\end{minipage}
	%
	\hfil %Это пружинка отодвигающая рисунки друг от друга
	%
	\begin{minipage}[b]{.3\linewidth}
	\begin{center}
	\includegraphics[width=60mm]{../White_Sea/Luvenga_Goreliy/low2_2001_.pdf}
	\end{center}
	\end{minipage}
	%


	\begin{minipage}[b]{.3\linewidth}
	\begin{center}
	\includegraphics[width=60mm]{../White_Sea/Luvenga_Goreliy/low2_1994_.pdf}
	\end{center}
	\end{minipage}
	%
	\hfill
	%
	\begin{minipage}[b]{.3\linewidth}
	\begin{center}
	\includegraphics[width=60mm]{../White_Sea/Luvenga_Goreliy/low2_1998_.pdf}
	\end{center}
	\end{minipage}	
	%
	\hfill
	%
	\begin{minipage}[b]{.3\linewidth}
	\begin{center}
	\includegraphics[width=60mm]{../White_Sea/Luvenga_Goreliy/low2_2002_.pdf}
	\end{center}
	\end{minipage}
%\smallskip

	\begin{minipage}[b]{.3\linewidth}
	\begin{center}
	\includegraphics[width=60mm]{../White_Sea/Luvenga_Goreliy/low2_1995_.pdf}
	\end{center}
	\end{minipage}
	%
	\hfill
	%
	\begin{minipage}[b]{.3\linewidth}
	\begin{center}
	\includegraphics[width=60mm]{../White_Sea/Luvenga_Goreliy/low2_1999_.pdf}
	\end{center}
	\end{minipage}
	%
	\hfill
	%
	\begin{minipage}[b]{.3\linewidth}
	\begin{center}
	\includegraphics[width=60mm]{../White_Sea/Luvenga_Goreliy/low2_2003_.pdf}
	\end{center}
	\end{minipage}
	%
\caption{Размерная структура {\it Macoma balthica} в районе нуля глубин о. Горелого}
\label{ris:size_str_Goreliy_low}
\end{figure}

%%%% вторая страница картинок с РС
	\begin{figure}[hp]

	\begin{minipage}[b]{.3\linewidth}
	\begin{center}
	\includegraphics[width=60mm]{../White_Sea/Luvenga_Goreliy/low2_2004_.pdf}
	\end{center}
	\end{minipage}
	%
	\hfill
	%
	\begin{minipage}[b]{.3\linewidth}
	\begin{center}
	\includegraphics[width=60mm]{../White_Sea/Luvenga_Goreliy/low2_2007_.pdf}
	\end{center}
	\end{minipage}
	%
	\hfill
	%
	\begin{minipage}[b]{.3\linewidth}
	\begin{center}
	\includegraphics[width=60mm]{../White_Sea/Luvenga_Goreliy/low2_2010_.pdf}
	\end{center}
	\end{minipage}
	%
	%
	\begin{minipage}[b]{.3\linewidth}
	\begin{center}
	\includegraphics[width=60mm]{../White_Sea/Luvenga_Goreliy/low2_2005_.pdf}
	\end{center}
	\end{minipage}
	%
	\hfill	
	%
	\begin{minipage}[b]{.3\linewidth}
	\begin{center}
	\includegraphics[width=60mm]{../White_Sea/Luvenga_Goreliy/low2_2008_.pdf}
	\end{center}
	\end{minipage}
	%
	\hfill
	%
	\begin{minipage}[b]{.3\linewidth}
	\begin{center}
	\includegraphics[width=60mm]{../White_Sea/Luvenga_Goreliy/low2_2011_.pdf}
	\end{center}
	\end{minipage}
	%
	%
	\begin{minipage}[b]{.3\linewidth}
	\begin{center}
	\includegraphics[width=60mm]{../White_Sea/Luvenga_Goreliy/low2_2006_.pdf}
	\end{center}
	\end{minipage}
	%
	\hfill
	%
	\begin{minipage}[b]{.3\linewidth}
	\begin{center}
	\includegraphics[width=60mm]{../White_Sea/Luvenga_Goreliy/low2_2009_.pdf}
	\end{center}
	\end{minipage}
	%
	\hfill
	%
	\begin{minipage}[b]{.3\linewidth}
	\begin{center}
	\includegraphics[width=60mm]{../White_Sea/Luvenga_Goreliy/low2_2012_.pdf}
	\end{center}
	\end{minipage}
	%
\begin{center}
Рис. \ref{ris:size_str_Goreliy_low} (продолжение). Размерная структура {\it Macoma balthica} у нуля глубин  о. Горелого
\end{center}
\end{figure}



%%%%%%%%%%%%%%%%%%%%%%%%%%%%%%
%2 разрез верхний пляж
	\begin{figure}[hp]

	\begin{minipage}[b]{.3\linewidth}
	\begin{center}
	\includegraphics[width=60mm]{../White_Sea/Luvenga_II_razrez/high_beatch2_1992_.pdf}	
	\end{center}
	\end{minipage}
	%
	\hfil %Это пружинка отодвигающая рисунки друг от друга
	%
	\begin{minipage}[b]{.3\linewidth}
	\begin{center}
	\includegraphics[width=60mm]{../White_Sea/Luvenga_II_razrez/high_beatch2_1996_.pdf}
	\end{center}
	\end{minipage}
	%
	\hfil %Это пружинка отодвигающая рисунки друг от друга
	%
	\begin{minipage}[b]{.3\linewidth}
	\begin{center}
\includegraphics[width=60mm]{../White_Sea/Luvenga_II_razrez/high_beatch2_2000_.pdf}
	\end{center}
	\end{minipage}
	%
	\begin{minipage}[b]{.3\linewidth}
	\begin{center}
	\includegraphics[width=60mm]{../White_Sea/Luvenga_II_razrez/high_beatch2_1993_.pdf}
	\end{center}
	\end{minipage}
	%
	\hfil %Это пружинка отодвигающая рисунки друг от друга
	%
	\begin{minipage}[b]{.3\linewidth}
	\begin{center}
	\includegraphics[width=60mm]{../White_Sea/Luvenga_II_razrez/high_beatch2_1997_.pdf}
	\end{center}
	\end{minipage}
	%
	\hfil %Это пружинка отодвигающая рисунки друг от друга
	%
	\begin{minipage}[b]{.3\linewidth}
	\begin{center}
	\includegraphics[width=60mm]{../White_Sea/Luvenga_II_razrez/high_beatch2_2002_.pdf}
	\end{center}
	\end{minipage}
	%


	\begin{minipage}[b]{.3\linewidth}
	\begin{center}
	\includegraphics[width=60mm]{../White_Sea/Luvenga_II_razrez/high_beatch2_1994_.pdf}
	\end{center}
	\end{minipage}
	%
	\hfill
	%
	\begin{minipage}[b]{.3\linewidth}
	\begin{center}
	\includegraphics[width=60mm]{../White_Sea/Luvenga_II_razrez/high_beatch2_1998_.pdf}
	\end{center}
	\end{minipage}	
	%
	\hfill
	%
	\begin{minipage}[b]{.3\linewidth}
	\begin{center}
	\includegraphics[width=60mm]{../White_Sea/Luvenga_II_razrez/high_beatch2_2004_.pdf}
	\end{center}
	\end{minipage}
%\smallskip

	\begin{minipage}[b]{.3\linewidth}
	\begin{center}
	\includegraphics[width=60mm]{../White_Sea/Luvenga_II_razrez/high_beatch2_1995_.pdf}
	\end{center}
	\end{minipage}
	%
	\hfill
	%
	\begin{minipage}[b]{.3\linewidth}
	\begin{center}
	\includegraphics[width=60mm]{../White_Sea/Luvenga_II_razrez/high_beatch2_1999_.pdf}
	\end{center}
	\end{minipage}
	%
	\hfill
	%
	\begin{minipage}[b]{.3\linewidth}
	\begin{center}

	\end{center}
	\end{minipage}
	%
\caption{Размерная структура {\it Macoma balthica} на верхнем пляже материковой литорали в районе пос.~ Лувеньга}
\label{ris:size_str_2razrez_high}
\end{figure}


%%%%%%%%%%%%%%%%%%%%%%%%%%%%%%
%2 разрез фукоиды
	\begin{figure}[hp]

	\begin{minipage}[b]{.3\linewidth}
	\begin{center}
	\includegraphics[width=60mm]{../White_Sea/Luvenga_II_razrez/fucus_zone2_1992_.pdf}	
	\end{center}
	\end{minipage}
	%
	\hfil %Это пружинка отодвигающая рисунки друг от друга
	%
	\begin{minipage}[b]{.3\linewidth}
	\begin{center}
	\includegraphics[width=60mm]{../White_Sea/Luvenga_II_razrez/fucus_zone2_1996_.pdf}
	\end{center}
	\end{minipage}
	%
	\hfil %Это пружинка отодвигающая рисунки друг от друга
	%
	\begin{minipage}[b]{.3\linewidth}
	\begin{center}
\includegraphics[width=60mm]{../White_Sea/Luvenga_II_razrez/fucus_zone2_2000_.pdf}
	\end{center}
	\end{minipage}
	%
	\begin{minipage}[b]{.3\linewidth}
	\begin{center}
	\includegraphics[width=60mm]{../White_Sea/Luvenga_II_razrez/fucus_zone2_1993_.pdf}
	\end{center}
	\end{minipage}
	%
	\hfil %Это пружинка отодвигающая рисунки друг от друга
	%
	\begin{minipage}[b]{.3\linewidth}
	\begin{center}
	\includegraphics[width=60mm]{../White_Sea/Luvenga_II_razrez/fucus_zone2_1997_.pdf}
	\end{center}
	\end{minipage}
	%
	\hfil %Это пружинка отодвигающая рисунки друг от друга
	%
	\begin{minipage}[b]{.3\linewidth}
	\begin{center}
	\includegraphics[width=60mm]{../White_Sea/Luvenga_II_razrez/fucus_zone2_2002_.pdf}
	\end{center}
	\end{minipage}
	%


	\begin{minipage}[b]{.3\linewidth}
	\begin{center}
	\includegraphics[width=60mm]{../White_Sea/Luvenga_II_razrez/fucus_zone2_1994_.pdf}
	\end{center}
	\end{minipage}
	%
	\hfill
	%
	\begin{minipage}[b]{.3\linewidth}
	\begin{center}
	\includegraphics[width=60mm]{../White_Sea/Luvenga_II_razrez/fucus_zone2_1998_.pdf}
	\end{center}
	\end{minipage}	
	%
	\hfill
	%
	\begin{minipage}[b]{.3\linewidth}
	\begin{center}
	\includegraphics[width=60mm]{../White_Sea/Luvenga_II_razrez/fucus_zone2_2004_.pdf}
	\end{center}
	\end{minipage}
%\smallskip

	\begin{minipage}[b]{.3\linewidth}
	\begin{center}
	\includegraphics[width=60mm]{../White_Sea/Luvenga_II_razrez/fucus_zone2_1995_.pdf}
	\end{center}
	\end{minipage}
	%
	\hfill
	%
	\begin{minipage}[b]{.3\linewidth}
	\begin{center}
	\includegraphics[width=60mm]{../White_Sea/Luvenga_II_razrez/fucus_zone2_1999_.pdf}
	\end{center}
	\end{minipage}
	%
	\hfill
	%
	\begin{minipage}[b]{.3\linewidth}
	\begin{center}

	\end{center}
	\end{minipage}
	%
\caption{Размерная структура {\it Macoma balthica} в поясе фукоидов материковой литорали в районе пос. Лувеньга}
\label{ris:size_str_2razrez_fucus}
\end{figure}


%%%%%%%%%%%%%%%%%%%%%%%%%%%%%%
%2 разрез зостера
	\begin{figure}[hp]

	\begin{minipage}[b]{.3\linewidth}
	\begin{center}
	\includegraphics[width=60mm]{../White_Sea/Luvenga_II_razrez/zostera_zone2_1992_.pdf}	
	\end{center}
	\end{minipage}
	%
	\hfil %Это пружинка отодвигающая рисунки друг от друга
	%
	\begin{minipage}[b]{.3\linewidth}
	\begin{center}
	\includegraphics[width=60mm]{../White_Sea/Luvenga_II_razrez/zostera_zone2_1996_.pdf}
	\end{center}
	\end{minipage}
	%
	\hfil %Это пружинка отодвигающая рисунки друг от друга
	%
	\begin{minipage}[b]{.3\linewidth}
	\begin{center}
\includegraphics[width=60mm]{../White_Sea/Luvenga_II_razrez/zostera_zone2_2000_.pdf}
	\end{center}
	\end{minipage}
	%
	\begin{minipage}[b]{.3\linewidth}
	\begin{center}
	\includegraphics[width=60mm]{../White_Sea/Luvenga_II_razrez/zostera_zone2_1993_.pdf}
	\end{center}
	\end{minipage}
	%
	\hfil %Это пружинка отодвигающая рисунки друг от друга
	%
	\begin{minipage}[b]{.3\linewidth}
	\begin{center}
	\includegraphics[width=60mm]{../White_Sea/Luvenga_II_razrez/zostera_zone2_1997_.pdf}
	\end{center}
	\end{minipage}
	%
	\hfil %Это пружинка отодвигающая рисунки друг от друга
	%
	\begin{minipage}[b]{.3\linewidth}
	\begin{center}
	\includegraphics[width=60mm]{../White_Sea/Luvenga_II_razrez/zostera_zone2_2002_.pdf}
	\end{center}
	\end{minipage}
	%


	\begin{minipage}[b]{.3\linewidth}
	\begin{center}
	\includegraphics[width=60mm]{../White_Sea/Luvenga_II_razrez/zostera_zone2_1994_.pdf}
	\end{center}
	\end{minipage}
	%
	\hfill
	%
	\begin{minipage}[b]{.3\linewidth}
	\begin{center}
	\includegraphics[width=60mm]{../White_Sea/Luvenga_II_razrez/zostera_zone2_1998_.pdf}
	\end{center}
	\end{minipage}	
	%
	\hfill
	%
	\begin{minipage}[b]{.3\linewidth}
	\begin{center}
	\includegraphics[width=60mm]{../White_Sea/Luvenga_II_razrez/zostera_zone2_2004_.pdf}
	\end{center}
	\end{minipage}
%\smallskip

	\begin{minipage}[b]{.3\linewidth}
	\begin{center}
	\includegraphics[width=60mm]{../White_Sea/Luvenga_II_razrez/zostera_zone2_1995_.pdf}
	\end{center}
	\end{minipage}
	%
	\hfill
	%
	\begin{minipage}[b]{.3\linewidth}
	\begin{center}
	\includegraphics[width=60mm]{../White_Sea/Luvenga_II_razrez/zostera_zone2_1999_.pdf}
	\end{center}
	\end{minipage}
	%
	\hfill
	%
	\begin{minipage}[b]{.3\linewidth}
	\begin{center}

	\end{center}
	\end{minipage}
	%
\caption{Размерная структура {\it Macoma balthica} в поясе взморника {\it Zostera marina} материковой литорали в районе пос. Лувеньга}
\label{ris:size_str_2razrez_zostera}
\end{figure}

%%%%%%%%%%%%%%%%%%%%%%%%%%%%%%
%2 разрез нижний пляж
	\begin{figure}[hp]

	\begin{minipage}[b]{.3\linewidth}
	\begin{center}
	\includegraphics[width=60mm]{../White_Sea/Luvenga_II_razrez/low_beatch2_1992_.pdf}	
	\end{center}
	\end{minipage}
	%
	\hfil %Это пружинка отодвигающая рисунки друг от друга
	%
	\begin{minipage}[b]{.3\linewidth}
	\begin{center}
	\includegraphics[width=60mm]{../White_Sea/Luvenga_II_razrez/low_beatch2_1996_.pdf}
	\end{center}
	\end{minipage}
	%
	\hfil %Это пружинка отодвигающая рисунки друг от друга
	%
	\begin{minipage}[b]{.3\linewidth}
	\begin{center}
\includegraphics[width=60mm]{../White_Sea/Luvenga_II_razrez/low_beatch2_2000_.pdf}
	\end{center}
	\end{minipage}
	%
	\begin{minipage}[b]{.3\linewidth}
	\begin{center}
	\includegraphics[width=60mm]{../White_Sea/Luvenga_II_razrez/low_beatch2_1993_.pdf}
	\end{center}
	\end{minipage}
	%
	\hfil %Это пружинка отодвигающая рисунки друг от друга
	%
	\begin{minipage}[b]{.3\linewidth}
	\begin{center}
	\includegraphics[width=60mm]{../White_Sea/Luvenga_II_razrez/low_beatch2_1997_.pdf}
	\end{center}
	\end{minipage}
	%
	\hfil %Это пружинка отодвигающая рисунки друг от друга
	%
	\begin{minipage}[b]{.3\linewidth}
	\begin{center}
	\includegraphics[width=60mm]{../White_Sea/Luvenga_II_razrez/low_beatch2_2002_.pdf}
	\end{center}
	\end{minipage}
	%


	\begin{minipage}[b]{.3\linewidth}
	\begin{center}
	\includegraphics[width=60mm]{../White_Sea/Luvenga_II_razrez/low_beatch2_1994_.pdf}
	\end{center}
	\end{minipage}
	%
	\hfill
	%
	\begin{minipage}[b]{.3\linewidth}
	\begin{center}
	\includegraphics[width=60mm]{../White_Sea/Luvenga_II_razrez/low_beatch2_1998_.pdf}
	\end{center}
	\end{minipage}	
	%
	\hfill
	%
	\begin{minipage}[b]{.3\linewidth}
	\begin{center}

	\end{center}
	\end{minipage}
%\smallskip

	\begin{minipage}[b]{.3\linewidth}
	\begin{center}
	\includegraphics[width=60mm]{../White_Sea/Luvenga_II_razrez/low_beatch2_1995_.pdf}
	\end{center}
	\end{minipage}
	%
	\hfill
	%
	\begin{minipage}[b]{.3\linewidth}
	\begin{center}
	\includegraphics[width=60mm]{../White_Sea/Luvenga_II_razrez/low_beatch2_1999_.pdf}
	\end{center}
	\end{minipage}
	%
	\hfill
	%
	\begin{minipage}[b]{.3\linewidth}
	\begin{center}

	\end{center}
	\end{minipage}
	%
\caption{Размерная структура {\it Macoma balthica} на нижнем пляже материковой литорали в районе пос. Лувеньга}
\label{ris:size_str_2razrez_low}
\end{figure}

%%%%%%%%%%%%%%%%%%%%%%%%%%%%%%%%%%%%%%%%%55
%ЮГ Ряшков
	\begin{figure}[hp]

	\begin{minipage}[b]{.3\linewidth}
	\begin{center}
	\includegraphics[width=60mm]{../White_Sea/Ryashkov_YuG/YuG2_2001_.pdf}
	\end{center}
	\end{minipage}
	%
	\hfill
	%
	\begin{minipage}[b]{.3\linewidth}
	\begin{center}
	\includegraphics[width=60mm]{../White_Sea/Ryashkov_YuG/YuG2_2005_.pdf}
	\end{center}
	\end{minipage}
	%
	\hfill
	%
	\begin{minipage}[b]{.3\linewidth}
	\begin{center}
	\includegraphics[width=60mm]{../White_Sea/Ryashkov_YuG/YuG2_2009_.pdf}
	\end{center}
	\end{minipage}
	%
	%
	\begin{minipage}[b]{.3\linewidth}
	\begin{center}
	\includegraphics[width=60mm]{../White_Sea/Ryashkov_YuG/YuG2_2002_.pdf}
	\end{center}
	\end{minipage}
	%
	\hfill	
	%
	\begin{minipage}[b]{.3\linewidth}
	\begin{center}
	\includegraphics[width=60mm]{../White_Sea/Ryashkov_YuG/YuG2_2006_.pdf}
	\end{center}
	\end{minipage}
	%
	\hfill
	%
	\begin{minipage}[b]{.3\linewidth}
	\begin{center}
	\includegraphics[width=60mm]{../White_Sea/Ryashkov_YuG/YuG2_2010_.pdf}
	\end{center}
	\end{minipage}
	%
	%
	\begin{minipage}[b]{.3\linewidth}
	\begin{center}
	\includegraphics[width=60mm]{../White_Sea/Ryashkov_YuG/YuG2_2003_.pdf}
	\end{center}
	\end{minipage}
	%
	\hfill
	%
	\begin{minipage}[b]{.3\linewidth}
	\begin{center}
	\includegraphics[width=60mm]{../White_Sea/Ryashkov_YuG/YuG2_2007_.pdf}
	\end{center}
	\end{minipage}
	%
	\hfill
	%
	\begin{minipage}[b]{.3\linewidth}
	\begin{center}
	\includegraphics[width=60mm]{../White_Sea/Ryashkov_YuG/YuG2_2011_.pdf}
	\end{center}
	\end{minipage}
	%

	\begin{minipage}[b]{.3\linewidth}
	\begin{center}
	\includegraphics[width=60mm]{../White_Sea/Ryashkov_YuG/YuG2_2004_.pdf}
	\end{center}
	\end{minipage}
	%
	\hfill
	%
	\begin{minipage}[b]{.3\linewidth}
	\begin{center}
	\includegraphics[width=60mm]{../White_Sea/Ryashkov_YuG/YuG2_2008_.pdf}
	\end{center}
	\end{minipage}
	%
	\hfill
	%
	\begin{minipage}[b]{.3\linewidth}
	\begin{center}
	\includegraphics[width=60mm]{../White_Sea/Ryashkov_YuG/YuG2_2012_.pdf}
	\end{center}
	\end{minipage}
\caption{Размерная структура {\it Macoma balthica} у нуля глубин в Южной губе о. Ряшкова}
\label{ris:size_str_YuG}
\end{figure}



%%%%%%%%%%%%%%%%%%%%%%%%%%%%%%
%ЗРС Ряшков
	\begin{figure}[hp]

	\begin{minipage}[b]{.3\linewidth}
	\begin{center}
	\includegraphics[width=60mm]{../White_Sea/Ryashkov_ZRS/zrs2_1994_.pdf}	
	\end{center}
	\end{minipage}
	%
	\hfil %Это пружинка отодвигающая рисунки друг от друга
	%
	\begin{minipage}[b]{.3\linewidth}
	\begin{center}
	\includegraphics[width=60mm]{../White_Sea/Ryashkov_ZRS/zrs2_1998_.pdf}
	\end{center}
	\end{minipage}
	%
	\hfil %Это пружинка отодвигающая рисунки друг от друга
	%
	\begin{minipage}[b]{.3\linewidth}
	\begin{center}
\includegraphics[width=60mm]{../White_Sea/Ryashkov_ZRS/zrs2_2002_.pdf}
	\end{center}
	\end{minipage}
	%
	\begin{minipage}[b]{.3\linewidth}
	\begin{center}
	\includegraphics[width=60mm]{../White_Sea/Ryashkov_ZRS/zrs2_1995_.pdf}
	\end{center}
	\end{minipage}
	%
	\hfil %Это пружинка отодвигающая рисунки друг от друга
	%
	\begin{minipage}[b]{.3\linewidth}
	\begin{center}
	\includegraphics[width=60mm]{../White_Sea/Ryashkov_ZRS/zrs2_1999_.pdf}
	\end{center}
	\end{minipage}
	%
	\hfil %Это пружинка отодвигающая рисунки друг от друга
	%
	\begin{minipage}[b]{.3\linewidth}
	\begin{center}
	\includegraphics[width=60mm]{../White_Sea/Ryashkov_ZRS/zrs2_2003_.pdf}
	\end{center}
	\end{minipage}
	%


	\begin{minipage}[b]{.3\linewidth}
	\begin{center}
\includegraphics[width=60mm]{../White_Sea/Ryashkov_ZRS/zrs2_1996_.pdf}
	\end{center}
	\end{minipage}
	%
	\hfill
	%
	\begin{minipage}[b]{.3\linewidth}
	\begin{center}
	\includegraphics[width=60mm]{../White_Sea/Ryashkov_ZRS/zrs2_2000_.pdf}
	\end{center}
	\end{minipage}	
	%
	\hfill
	%
	\begin{minipage}[b]{.3\linewidth}
	\begin{center}
	\includegraphics[width=60mm]{../White_Sea/Ryashkov_ZRS/zrs2_2004_.pdf}
	\end{center}
	\end{minipage}
%\smallskip

	\begin{minipage}[b]{.3\linewidth}
	\begin{center}
	\includegraphics[width=60mm]{../White_Sea/Ryashkov_ZRS/zrs2_1997_.pdf}
	\end{center}
	\end{minipage}
	%
	\hfill
	%
	\begin{minipage}[b]{.3\linewidth}
	\begin{center}
	\includegraphics[width=60mm]{../White_Sea/Ryashkov_ZRS/zrs2_2001_.pdf}
	\end{center}
	\end{minipage}
	%
	\hfill
	%
	\begin{minipage}[b]{.3\linewidth}
	\begin{center}
	\includegraphics[width=60mm]{../White_Sea/Ryashkov_ZRS/zrs2_2005_.pdf}
	\end{center}
	\end{minipage}
	%
\caption{Размерная структура {\it Macoma balthica} в СГЛ Западной Ряшковой салмы}
\label{ris:size_str_ZRS}
\end{figure}

%%%% вторая страница картинок с РС
	\begin{figure}[hp]

	\begin{minipage}[b]{.3\linewidth}
	\begin{center}
	\includegraphics[width=60mm]{../White_Sea/Ryashkov_ZRS/zrs2_2006_.pdf}
	\end{center}
	\end{minipage}
	%
	\hfill
	%
	\begin{minipage}[b]{.3\linewidth}
	\begin{center}
	\includegraphics[width=60mm]{../White_Sea/Ryashkov_ZRS/zrs2_2009_.pdf}
	\end{center}
	\end{minipage}
	%
	\hfill
	%
	\begin{minipage}[b]{.3\linewidth}
	\begin{center}
	\includegraphics[width=60mm]{../White_Sea/Ryashkov_ZRS/zrs2_2012_.pdf}
	\end{center}
	\end{minipage}
	%
	%
	\begin{minipage}[b]{.3\linewidth}
	\begin{center}
	\includegraphics[width=60mm]{../White_Sea/Ryashkov_ZRS/zrs2_2007_.pdf}
	\end{center}
	\end{minipage}
	%
	\hfill	
	%
	\begin{minipage}[b]{.3\linewidth}
	\begin{center}
	\includegraphics[width=60mm]{../White_Sea/Ryashkov_ZRS/zrs2_2010_.pdf}
	\end{center}
	\end{minipage}
	%
	\hfill
	%
	\begin{minipage}[b]{.3\linewidth}
	\begin{center}
%	\includegraphics[width=60mm]{../White_Sea/Ryashkov_ZRS/zrs2_2011_.pdf}
	\end{center}
	\end{minipage}
	%
	%
	\begin{minipage}[b]{.3\linewidth}
	\begin{center}
	\includegraphics[width=60mm]{../White_Sea/Ryashkov_ZRS/zrs2_2008_.pdf}
	\end{center}
	\end{minipage}
	%
	\hfill
	%
	\begin{minipage}[b]{.3\linewidth}
	\begin{center}
	\includegraphics[width=60mm]{../White_Sea/Ryashkov_ZRS/zrs2_2011_.pdf}
	\end{center}
	\end{minipage}
	%
	\hfill
	%
	\begin{minipage}[b]{.3\linewidth}
	\begin{center}

	\end{center}
	\end{minipage}
	%
\begin{center}
Рис.\ref{ris:size_str_ZRS} (продолжение). Размерная структура {\it Macoma balthica} в СГЛ Западной Ряшковой салмы
\end{center}
\end{figure}

%%%%%%%%%%%%%%%%%%%%%%%%%%%
%Ломнишный

\begin{figure}[hp]


	\begin{minipage}[b]{.46\linewidth}
	\begin{center}
	\includegraphics[width=65mm]{../White_Sea/Lomnishniy/Lomnishniy_2007_.pdf}
	\end{center}
	\end{minipage}
	%
	\hfill
	%
	\begin{minipage}[b]{.46\linewidth}
	\begin{center}
	\includegraphics[width=65mm]{../White_Sea/Lomnishniy/Lomnishniy_2010_.pdf}
	\end{center}
	\end{minipage}
%
	\begin{minipage}[b]{.46\linewidth}
	\begin{center}
	\includegraphics[width=65mm]{../White_Sea/Lomnishniy/Lomnishniy_2008_.pdf}
	\end{center}
	\end{minipage}
	%
	\hfill
	%
	\begin{minipage}[b]{.46\linewidth}
	\begin{center}
	\includegraphics[width=65mm]{../White_Sea/Lomnishniy/Lomnishniy_2011_.pdf}
	\end{center}
	\end{minipage}
%
	\begin{minipage}[b]{.46\linewidth}
	\begin{center}
	\includegraphics[width=65mm]{../White_Sea/Lomnishniy/Lomnishniy_2009_.pdf}
	\end{center}
	\end{minipage}
	%
	\hfill
	%
	\begin{minipage}[b]{.46\linewidth}
	\begin{center}
	\includegraphics[width=65mm]{../White_Sea/Lomnishniy/Lomnishniy_2012_.pdf}
	\end{center}
	\end{minipage}
%
\caption{Размерная структура {\it Macoma balthica} у нуля глубин литорали о.Ломнишный}
\label{ris:size_str_Lomnishniy}
\end{figure}




\section{Размерная структура {\it Macoma balthica} в исследованных поселениях Баренцева моря} 
\label{app:Barents_sizestr_hist}
На всех графиках абсцисса --- длина раковины, мм; ордината --- численность особей, экз./м$^2$. Указано средняя численность особей определенного размера $\pm$ ошибка средней.
%%%%%%%%%%%%%%%%%%%%%%%%%%%%%%%%%%%%%
%Баренцево море

	\begin{figure}[hp]
	
	\begin{minipage}[b]{.46\linewidth}
	\begin{center}
	{\footnotesize Абрам-мыс, СГЛ}
		\includegraphics[width=55mm]{../Barenc_Sea/Abram-mys/middle_2008_.pdf}
	\end{center}
	\end{minipage}
	%
	\hfil %Это пружинка отодвигающая рисунки друг от друга
	%
	\begin{minipage}[b]{.46\linewidth}
	\begin{center}
	{\footnotesize Абрам-мыс, НГЛ}
		\includegraphics[width=55mm]{../Barenc_Sea/Abram-mys/low_2008_.pdf}
	\end{center}
	\end{minipage}
	%\smallskip
	\begin{minipage}[b]{.46\linewidth}
	\begin{center}
	{\footnotesize Пала-губа, СГЛ}
	\includegraphics[width=55mm]{../Barenc_Sea/Pala/middle_2007_.pdf}
	\end{center}
	\end{minipage}
	%	
	\hfil %Это пружинка отодвигающая рисунки друг от друга
	%
	\begin{minipage}[b]{.46\linewidth}
	\begin{center}	
	{\footnotesize Пала-губа, НГЛ}
	\includegraphics[width=55mm]{../Barenc_Sea/Pala/low_2007_.pdf}
	\end{center}
	\end{minipage}
	%\smallskip
	\begin{minipage}[b]{.46\linewidth}
	\begin{center}
	{\footnotesize Гаврилово, СГЛ}
	\includegraphics[width=55mm]{../Barenc_Sea/Gavrilovo/middle_2008_.pdf}
	\end{center}
	\end{minipage}
	%
	\hfil %Это пружинка отодвигающая рисунки друг от друга
	%
	\begin{minipage}[b]{.46\linewidth}
	\begin{center}
	{\footnotesize Порчниха, СГЛ}
	\includegraphics[width=55mm]{../Barenc_Sea/Porchnikha/sizestr2007.pdf}
	\end{center}
	\end{minipage}
	%\smallskip
\caption{Размерная структура {\it Macoma balthica} в поселениях Мурманского побережья Баренцева моря}
\label{ris:Barents_sizestr}
\end{figure}


%страница 2
	\begin{figure}[h]
	
	\begin{minipage}[b]{.46\linewidth}
	\begin{center}
	{\footnotesize Ярнышная, ВГЛ}
	\includegraphics[width=55mm]{../Barenc_Sea/Yarnyshnaya/high_2008_.pdf}
	\end{center}
	\end{minipage}
	%
	\hfil %Это пружинка отодвигающая рисунки друг от друга
	%
	\begin{minipage}[b]{.46\linewidth}
	\begin{center}
	{\footnotesize Ярнышная, СГЛ}
	\includegraphics[width=55mm]{../Barenc_Sea/Yarnyshnaya/middle_2007_.pdf}
	\end{center}
	\end{minipage}
	%\smallskip
	\begin{minipage}[b]{.46\linewidth}

	\begin{center}
	{\footnotesize Ярнышная, НГЛ}
	\includegraphics[width=55mm]{../Barenc_Sea/Yarnyshnaya/low_2008_.pdf}
	\end{center}
	\end{minipage}
	%	
	\hfil %Это пружинка отодвигающая рисунки друг от друга
	%
	\begin{minipage}[b]{.46\linewidth}
	\begin{center}	
	{\footnotesize Ивановская, ВСЛ}
	\includegraphics[width=55mm]{../Barenc_Sea/Ivanovskaya/sizestr2008.pdf}
	\end{center}
	\end{minipage}
	%\smallskip
	\begin{minipage}[b]{.46\linewidth}
	\begin{center}
	{\footnotesize Шельпино, ВГЛ}
	\includegraphics[width=55mm]{../Barenc_Sea/Shel'pino/high_2008_.pdf}
	\end{center}
	\end{minipage}
	%
	\hfil %Это пружинка отодвигающая рисунки друг от друга
	%
	\begin{minipage}[b]{.46\linewidth}
	\begin{center}
	{\footnotesize Шельпино, СГЛ}
	\includegraphics[width=55mm]{../Barenc_Sea/Shel'pino/middle_2008_.pdf}
	\end{center}
	\end{minipage}
	%\smallskip
\begin{center}
Рис. \ref{ris:Barents_sizestr} (продолжение). Размерная структура {\it Macoma balthica} в поселениях Мурманского побережья Баренцева моря
\end{center}
\end{figure}

%%%%%%%%%%%%%%%%%%
% Дальний пляж
	\begin{figure}[hp]

	\begin{minipage}[b]{.3\linewidth}
	\begin{center}
	\includegraphics[width=60mm]{../Barenc_Sea/Dalnezeleneckaya/DZ2_2002_.pdf}	
	\end{center}
	\end{minipage}
	%
	\hfil %Это пружинка отодвигающая рисунки друг от друга
	%
	\begin{minipage}[b]{.3\linewidth}
	\begin{center}
	\includegraphics[width=60mm]{../Barenc_Sea/Dalnezeleneckaya/DZ2_2005_.pdf}
	\end{center}
	\end{minipage}
	%	
	\hfill
	%
	\begin{minipage}[b]{.3\linewidth}
	\begin{center}
	\includegraphics[width=60mm]{../Barenc_Sea/Dalnezeleneckaya/DZ2_2008_.pdf}
	\end{center}
	\end{minipage}
	%
	%
	\begin{minipage}[b]{.3\linewidth}
	\begin{center}
	\includegraphics[width=60mm]{../Barenc_Sea/Dalnezeleneckaya/DZ2_2003_.pdf}
	\end{center}
	\end{minipage}
	%
	\hfill
	%
	\begin{minipage}[b]{.3\linewidth}
	\begin{center}
	\includegraphics[width=60mm]{../Barenc_Sea/Dalnezeleneckaya/DZ2_2006_.pdf}
	\end{center}
	\end{minipage}
	%
	\hfill
	%
	\begin{minipage}[b]{.3\linewidth}
	\begin{center}
	
	\end{center}
	\end{minipage}
	%
	%
	\begin{minipage}[b]{.3\linewidth}
	\begin{center}
	\includegraphics[width=60mm]{../Barenc_Sea/Dalnezeleneckaya/DZ2_2004_.pdf}
	\end{center}
	\end{minipage}	
	%
	\hfill
	%
	\begin{minipage}[b]{.3\linewidth}
	\begin{center}
	\includegraphics[width=60mm]{../Barenc_Sea/Dalnezeleneckaya/DZ2_2007_.pdf}
	\end{center}
	\end{minipage}
	%
	\hfill
	%
	\begin{minipage}[b]{.3\linewidth}
	\begin{center}

	\end{center}
	\end{minipage}

\caption{Размерная структура {\it Macoma balthica} на Дальнем пляже губы Дальнезеленецкая}
\label{ris:size_str_DZ}
\end{figure}



% треугольные матрицы с возрастной структурой

\section{Приложение. Ростовые характеристики {\it Macoma balthica} на Мурманском побережьи Баренцева моря}
\label{app:growth_matrix}

В   таблицах данного приложения   приведены средние длины колец остановки роста у моллюсков разных возрастов.

\vspace{5em}

Обозначения во таблицах:\\[2em]
$N$ --- количество  особей  данного возраста, экз.;\\
$L min$  ---  минимальная   длина  особей   данного   возраста,   мм;\\
$L max$   ---   максимальная   длина   особей   данного   возраста,   мм;\\
$L aver$ --- средняя длина моллюсков данного возраста, мм;\\
$m_L$ --- ошибка средней,\\
1к -- 13к --- длина колец остановки роста;\\
$L_k aver$ --- средняя длина данного кольца остановки роста, мм;\\
$m_{L_k}$ --- ошибка средней;\\
$L_k min$ --- минимальная длина данного кольца остановки роста, мм; \\
$L_k   max$   --   максимальная   длина   данного   кольца   остановки   роста.   \\



\begin{landscape}

\begin{table}[h]
\caption{Возрастная структура {\it M.~balthica} в среднем горизонте литорали в районе Абрам-мыса }
\label{tab:Abram_sgl_growth_matrix}
\begin{tabular}{|c|c|cc|cc|ccccccccccc|}
\hline
возраст & $N$  & $L min$ & $L max$ & $L aver$ & $m\_L$    & 1 к & 2к  & 3к  & 4к  & 5к  & 6к  & 7к  & 8к  & 9к   & 10к  & 11к  \\ \hline
0+      & 0  &       &       &         &         &     &     &     &     &     &     &     &     &      &      &      \\
1+      & 0  &       &       &         &         &     &     &     &     &     &     &     &     &      &      &      \\
2+      & 0  &       &       &         &         &     &     &     &     &     &     &     &     &      &      &      \\
3+      & 1  & 4,5   & 4,5   & 4,5     &         & 0,7 & 2,0 & 3,5 &     &     &     &     &     &      &      &      \\
4+      & 4  & 5,3   & 8,5   & 6,2     & 0,4     & 1,4 & 2,6 & 4,3 & 5,1 &     &     &     &     &      &      &      \\
5+      & 8  & 5,3   & 8,0   & 6,8     & 0,4     & 1,0 & 2,0 & 3,3 & 4,4 & 5,5 &     &     &     &      &      &      \\
6+      & 5  & 6,6   & 8,0   & 7,1     & 0,3     & 1,4 & 2,7 & 3,4 & 4,4 & 5,2 & 6,1 &     &     &      &      &      \\
7+      & 11 & 7,1   & 11,4  & 9,0     & 0,3     & 1,3 & 2,0 & 3,3 & 4,6 & 5,6 & 6,8 & 7,9 &     &      &      &      \\
8+      & 11 & 8,8   & 11,8  & 10,0    & 0,3     & 1,0 & 2,1 & 3,2 & 4,4 & 5,7 & 6,9 & 8,1 & 9,0 &      &      &      \\
9+      & 6  & 9,6   & 12,7  & 10,8    & 0,5     & 1,0 & 2,2 & 3,6 & 4,3 & 5,4 & 6,7 & 7,8 & 8,9 & 9,9  &      &      \\
10+     & 6  & 10,2  & 12,8  & 11,4    & 0,4     & 1,2 & 2,3 & 3,5 & 4,2 & 5,1 & 6,1 & 7,4 & 8,4 & 9,4  & 10,4 &      \\
11+     & 3  & 12,5  & 14,5  & 13,2    & 0,6     &     &     & 3,5 & 4,6 & 5,6 & 6,3 & 7,0 & 8,6 & 10,0 & 11,2 & 12,2 \\ \hline
        &    &       &       &         & $L_k aver$ & 1,1 & 2,2 & 3,5 & 4,5 & 5,4 & 6,5 & 7,6 & 8,7 & 9,7  & 10,8 & 12,2 \\
        &    &       &       &         & $m_{L_k}$  & 0,1 & 0,1 & 0,1 & 0,1 & 0,1 & 0,1 & 0,2 & 0,1 & 0,2  & 0,4  &      \\
        &    &       &       &         & $L_k min$  & 0,7 & 2,0 & 3,2 & 4,2 & 5,1 & 6,1 & 7,0 & 8,4 & 9,4  & 10,4 & 12,2 \\
        &    &       &       &         & $L_k max$  & 1,4 & 2,7 & 4,3 & 5,1 & 5,7 & 6,9 & 8,1 & 9,0 & 10,0 & 11,2 & 12,2 \\ \hline
\end{tabular}
\end{table}



\begin{table}[h]
\caption{Возрастная структура {\it M.~balthica} в нижнем горизонте литорали в районе Абрам-мыса }
\label{tab:Abram_ngl_growth_matrix}
\begin{tabular}{|c|c|cc|cc|ccccccccccc|}
\hline
возраст & $N$  & $L min$ & $L max$ & $L aver$ & $m_L$   & 1 к & 2к  & 3к  & 4к  & 5к  & 6к  & 7к  & 8к   & 9к   & 10к  & 11к  \\ \hline
0+      & 0  &       &       &         &         &     &     &     &     &     &     &     &      &      &      &      \\
1+      & 12 & 1,5   & 2,3   & 1,9     & 0,1     & 0,9 &     &     &     &     &     &     &      &      &      &      \\
2+      & 1  & 3,4   & 3,4   & 3,4     &         & 1,3 & 2,4 &     &     &     &     &     &      &      &      &      \\
3+      & 7  & 3,9   & 5,1   & 4,6     & 0,2     & 1,4 & 2,6 & 3,6 &     &     &     &     &      &      &      &      \\
4+      & 6  & 5,2   & 6,5   & 5,8     & 0,2     & 1,0 & 1,9 & 3,2 & 4,5 &     &     &     &      &      &      &      \\
5+      & 6  & 7,1   & 8,0   & 7,6     & 0,2     & 1,1 & 2,9 & 4,0 & 5,4 & 6,4 &     &     &      &      &      &      \\
6+      & 5  & 7,3   & 8,5   & 8,0     & 0,2     & 1,4 & 2,3 & 3,3 & 4,8 & 5,9 & 6,9 &     &      &      &      &      \\
7+      & 4  & 8,7   & 11,5  & 9,7     & 0,6     & 1,0 & 2,6 & 4,2 & 5,1 & 6,4 & 7,5 & 8,5 &      &      &      &      \\
8+      & 4  & 9,8   & 12,3  & 11,3    & 0,6     &     & 2,9 & 4,4 & 5,4 & 6,6 & 8,0 & 9,3 & 10,3 &      &      &      \\
9+      & 3  & 11,7  & 12,2  & 12,0    & 0,1     & 1,2 & 2,4 & 4,0 & 5,4 & 6,2 & 8,0 & 9,1 & 10,2 & 11,2 &      &      \\
10+     & 4  & 11,2  & 12,6  & 11,9    & 0,3     &     & 3,0 & 4,1 & 5,2 & 6,3 & 7,7 & 8,6 & 9,5  & 10,4 & 11,1 &      \\
11+     & 1  & 13,0  & 13,0  & 13,0    &         &     &     &     &     &     & 6,1 & 7,5 & 8,9  & 9,7  & 11,2 & 12,0 \\\hline
        &    &       &       &         & $L_k aver$ & 1,2 & 2,5 & 3,8 & 5,1 & 6,3 & 7,4 & 8,6 & 9,7  & 10,4 & 11,2 & 12,0 \\
        &    &       &       &         & $m_{L_k}$  & 0,1 & 0,1 & 0,1 & 0,1 & 0,1 & 0,3 & 0,3 & 0,3  & 0,4  & 0,0  &      \\
        &    &       &       &         & $L_k min$  & 0,9 & 1,9 & 3,2 & 4,5 & 5,9 & 6,1 & 7,5 & 8,9  & 9,7  & 11,1 & 12,0 \\
        &    &       &       &         & $L_k max$  & 1,4 & 3,0 & 4,4 & 5,4 & 6,6 & 8,0 & 9,3 & 10,3 & 11,2 & 11,2 & 12,0 \\ \hline
\end{tabular}
\end{table}

\begin{table}[h]
\caption{Возрастная структура {\it M.~balthica} в среднем горизонте литорали Пала-губы }
\label{tab:Pala_sgl_growth_matrix}
\begin{tabular}{|c|c|cc|cc|cccccccc|}
    \hline
возраст & $N$   & $L min$ & $L max$ & $L aver$ & $m_L$   & 1 к & 2к  & 3к  & 4к  & 5к  & 6к  & 7к   & 8к   \\ \hline
0+      & 0   &       &       &         &         &     &     &     &     &     &     &      &      \\
1+      & 22  & 1,0   & 2,5   & 1,7     & 0,1     & 0,6 &     &     &     &     &     &      &      \\
2+      & 346 & 1,7   & 15,0  & 3,0     & 0,0     & 0,6 & 1,7 &     &     &     &     &      &      \\
3+      & 70  & 3,1   & 7,3   & 4,4     & 0,1     & 0,6 & 1,6 & 2,8 &     &     &     &      &      \\
4+      & 15  & 4,6   & 9,2   & 7,3     & 0,4     & 0,7 & 1,7 & 3,2 & 5,3 &     &     &      &      \\
5+      & 3   & 7,2   & 9,2   & 8,2     & 0,6     & 0,8 & 1,6 & 3,4 & 4,6 & 6,4 &     &      &      \\
6+      & 1   &     &     & 9,7     &         &   & 1,5 & 2,6 & 3,5 & 5,5 & 8,4 &      &      \\
7+      & 5   & 9,4   & 11,5  & 10,1    & 0,4     & 0,7 & 2,4 & 3,6 & 4,9 & 6,3 & 8,0 & 9,5  &      \\
8+      & 3   & 12,7  & 13,9  & 13,3    & 0,6     &   & 2,2 & 4,4 & 6,8 & 7,9 & 8,9 & 10,4 & 11,8 \\ \hline
        &     &       &       &         & $L_k aver$ & 0,7 & 1,8 & 3,3 & 5,0 & 6,5 & 8,4 & 9,9  & 11,8 \\
        &     &       &       &         & $m_{L_k}$  & 0,0 & 0,1 & 0,3 & 0,5 & 0,5 & 0,3 & 0,4  &      \\
        &     &       &       &         & $L_k min$  & 0,6 & 1,5 & 2,6 & 3,5 & 5,5 & 8,0 & 9,5  & 11,8 \\
        &     &       &       &         & $L_k max$  & 0,8 & 2,4 & 4,4 & 6,8 & 7,9 & 8,9 & 10,4 & 11,8 \\ \hline
\end{tabular}
\end{table}

\begin{table}[h]
\caption{Возрастная структура {\it M.~balthica} в нижнем горизонте литорали Пала-губы }
\label{tab:Pala_ngl_growth_matrix}
\begin{tabular}{|c|c|cc|cc|ccccccccc|}
    \hline
возраст & $N$  & $L min$ & $L max$ & $L aver$ & $m_L$   & 1 к & 2к  & 3к  & 4к  & 5к  & 6к  & 7к  & 8к   & 9к   \\ \hline
0+      & 0  &       &       &         &         &     &     &     &     &     &     &     &      &      \\
1+      & 9  & 1,8   & 2,5   & 2,2     & 0,1     & 1,1 &     &     &     &     &     &     &      &      \\
2+      & 76 & 1,6   & 7,9   & 3,1     & 0,1     & 0,7 & 2,0 &     &     &     &     &     &      &      \\
3+      & 40 & 2,1   & 5,8   & 3,8     & 0,1     & 0,7 & 1,8 & 2,9 &     &     &     &     &      &      \\
4+      & 34 & 2,1   & 8,5   & 5,4     & 0,2     & 0,7 & 1,8 & 3,1 & 4,6 &     &     &     &      &      \\
5+      & 37 & 3,5   & 9,8   & 6,8     & 0,2     & 0,8 & 1,9 & 3,1 & 4,6 & 6,2 &     &     &      &      \\
6+      & 44 & 4,6   & 11,5  & 8,2     & 0,2     & 0,8 & 1,8 & 2,9 & 4,1 & 5,5 & 7,3 &     &      &      \\
7+      & 48 & 7,4   & 12    & 9,9     & 0,2     & 0,9 & 2,1 & 3,3 & 4,6 & 6,0 & 7,7 & 9,1 &      &      \\
8+      & 61 & 8     & 13,7  & 10,6    & 0,1     & 0,7 & 2,0 & 3,4 & 4,6 & 6,1 & 7,5 & 8,9 & 9,9  &      \\
9+      & 44 & 8,6   & 14,2  & 11,1    & 0,2     &   &   & 3,4 & 4,7 & 6,5 & 8,2 & 9,7 & 10,5 & 11,4 \\
10+     & 39 & 10,3  & 15,3  & 12,6    & 0,2     &     &     &     &     &     &     &     &      &      \\
11+     & 7  & 12    & 15,2  & 13,2    & 0,5     &     &     &     &     &     &     &     &      &      \\
12+     & 5  & 14,4  & 18    & 16,1    & 0,6     &     &     &     &     &     &     &     &      &      \\
13+     & 3  & 13,9  & 16,8  & 15,4    & 0,8     &     &     &     &     &     &     &     &      &      \\
14+     & 1  &     &     & 17,8    &         &     &     &     &     &     &     &     &      &      \\\hline
        &    &       &       &         & $L_k aver$ & 0,8 & 1,9 & 3,1 & 4,5 & 6,0 & 7,7 & 9,2 & 10,2 & 11,4 \\
        &    &       &       &         & $m_{L_k}$  & 0,0 & 0,0 & 0,1 & 0,1 & 0,2 & 0,2 & 0,3 & 0,4  &      \\
        &    &       &       &         & $L_k min$  & 0,7 & 1,8 & 2,9 & 4,1 & 5,5 & 7,3 & 8,9 & 9,9  &      \\
        &    &       &       &         & $L_k max$  & 1,1 & 2,1 & 3,4 & 4,7 & 6,5 & 8,2 & 9,7 & 10,5 &      \\ \hline
\end{tabular}
\end{table}

\begin{table}[h]
\caption{Возрастная структура {\it M.~balthica} в среднем горизонте литорали губы Гаврилово}
\label{tab:Gavrilovo_sgl_growth_matrix}
\begin{tabular}{|c|c|cc|cc|ccccccccccccccc|}
    \hline
возраст & $N$ & $L min$ & $L max$ & $L aver$ & $m_L$   & 1 к & 2к  & 3к  & 4к  & 5к  & 6к  & 7к   & 8к   & 9 к  & 10 к & 11 к & 12 к & 13 к & 14к  & 15к  \\ \hline
0+      & 0 &       &       &         &         &     &     &     &     &     &     &      &      &      &      &      &      &      &      &      \\
1+      & 1 & 2,3   & 2,3   & 2,3     &         & 0,8 &     &     &     &     &     &      &      &      &      &      &      &      &      &      \\
2+      & 1 & 2,7   & 2,7   & 2,7     &         & 0,7 & 1,4 &     &     &     &     &      &      &      &      &      &      &      &      &      \\
3+      & 1 & 3,2   & 3,2   & 3,2     &         & 0,7 & 1,6 & 2,3 &     &     &     &      &      &      &      &      &      &      &      &      \\
4+      & 0 &       &       &         &         &     &     &     &     &     &     &      &      &      &      &      &      &      &      &      \\
5+      & 0 &       &       &         &         &     &     &     &     &     &     &      &      &      &      &      &      &      &      &      \\
6+      & 1 & 6,2   & 6,2   & 6,2     &         & 0,9 & 1,9 & 2,7 & 3,8 & 5,4 & 6,5 &      &      &      &      &      &      &      &      &      \\
7+      & 0 &       &       &         &         &     &     &     &     &     &     &      &      &      &      &      &      &      &      &      \\
8+      & 1 & 10,0  & 10,0  & 10,0    &         & 0,7 & 1,5 & 2,7 & 3,5 & 4,5 & 6,0 & 8,0  & 9,1  &      &      &      &      &      &      &      \\
9+      & 0 &       &       &         &         &     &     &     &     &     &     &      &      &      &      &      &      &      &      &      \\
10+     & 1 & 15,0  & 15,0  & 15,0    &         & 1,2 & 2,5 & 3,2 & 4,9 & 7,3 & 8,3 & 9,7  & 11,1 & 12,7 & 14,0 &      &      &      &      &      \\
11+     & 2 & 15,0  & 17,9  & 16,5    & 1,5     & 1,3 & 2,6 & 3,5 & 5,8 & 7,3 & 9,4 & 10,4 & 12,1 & 13,5 & 14,6 & 15,4 &      &      &      &      \\
12+     & 2 & 17,2  & 17,5  & 17,4    & 0,2     & 0,7 & 1,6 & 3,7 & 5,1 & 6,7 & 8,1 & 9,9  & 11,5 & 13,0 & 14,7 & 15,9 & 16,7 &      &      &      \\
13+     & 2 & 16,6  & 18,2  & 17,4    & 0,8     & 1,1 & 2,1 &     & 5,2 & 6,6 & 7,7 & 8,5  & 9,9  & 10,9 & 12,3 & 13,4 & 15,1 & 16,5 &      &      \\
14+     & 5 & 14,3  & 18,4  & 16,9    & 0,8     & 1,0 & 2,0 & 3,6 & 4,8 & 5,9 & 7,5 & 8,4  & 9,6  & 11,1 & 12,4 & 13,6 & 14,7 & 15,6 & 16,3 &      \\
15+     & 4 & 16,2  & 18,8  & 17,1    & 0,6     & 0,7 & 1,9 & 3,1 & 4,4 & 5,7 & 6,6 & 7,6  & 8,7  & 9,8  & 10,9 & 12,2 & 13,5 & 14,8 & 15,6 & 16,4 \\\hline
        &   &       &       &         & $L_k aver$ & 0,9 & 1,9 & 3,1 & 4,7 & 6,2 & 7,5 & 8,9  & 10,3 & 11,8 & 13,1 & 14,1 & 15,0 & 15,6 & 15,9 & 16,4 \\
        &   &       &       &         & $m_{L_k}$  & 0,1 & 0,1 &     & 0,3 & 0,3 & 0,4 & 0,4  & 0,5  &      &      &      &      &      &      &      \\
        &   &       &       &         & $L_k min$  & 0,7 & 1,4 & 2,3 & 3,5 & 4,5 & 6,0 & 7,6  & 8,7  & 9,8  & 10,9 & 12,2 & 13,5 & 14,8 & 15,6 & 16,4 \\
        &   &       &       &         & $L_k max$  & 1,3 & 2,6 & 3,7 & 5,8 & 7,3 & 9,4 & 10,4 & 12,1 & 13,5 & 14,7 & 15,9 & 16,7 & 16,5 & 16,3 & 16,4 \\ \hline
\end{tabular}
\end{table}

\begin{table}[h]
\caption{Возрастная структура {\it M.~balthica} в нижнем горизонте литорали губы Гаврилово}
\label{tab:Gavrilovo_ngl_growth_matrix}
\begin{tabular}{|c|c|cc|cc|cccccccccccc|}
    \hline
возраст & $N$ & $L min$ & $L max$ & $L aver$ & $m_L$   & 1 к & 2к  & 3к  & 4к  & 5к  & 6к   & 7к   & 8к   & 9 к  & 10 к & 11 к & 12 к \\ \hline
0+      & 0 &       &       &         &         &     &     &     &     &     &      &      &      &      &      &      &      \\
1+      & 0 &       &       &         &         &     &     &     &     &     &      &      &      &      &      &      &      \\
2+      & 0 &       &       &         &         &     &     &     &     &     &      &      &      &      &      &      &      \\
3+      & 1 & 4,8   & 4,8   & 4,8     &         & 0,8 & 2,1 & 4,3 &     &     &      &      &      &      &      &      &      \\
4+      & 0 &       &       &         &         &     &     &     &     &     &      &      &      &      &      &      &      \\
5+      & 0 &       &       &         &         &     &     &     &     &     &      &      &      &      &      &      &      \\
6+      & 0 &       &       &         &         &     &     &     &     &     &      &      &      &      &      &      &      \\
7+      & 0 &       &       &         &         &     &     &     &     &     &      &      &      &      &      &      &      \\
8+      & 1 & 11,8  & 11,8  & 11,8    &         & 1,0 & 1,6 & 3,9 & 6,0 & 7,0 & 7,8  & 9,0  & 10,3 &      &      &      &      \\
9+      & 1 & 14,8  & 14,8  & 14,8    &         & 1,3 & 3,1 & 4,5 & 6,7 & 8,8 & 10,1 & 12,0 & 13,0 & 13,8 &      &      &      \\
10+     & 0 &       &       &         &         &     &     &     &     &     &      &      &      &      &      &      &      \\
11+     & 0 &       &       &         &         &     &     &     &     &     &      &      &      &      &      &      &      \\
12+     & 1 & 17,9  & 17,9  & 17,9    &         & 1,5 &     & 4,9 & 7,1 & 9,2 & 10,8 & 12,2 & 13,3 & 14,9 & 15,6 & 16,5 & 17,1 \\ \hline
        &   &       &       &         & $L_k aver$ & 1,2 & 2,3 & 4,4 & 6,6 & 8,3 & 9,6  & 11,1 & 12,2 & 14,4 & 15,6 & 16,5 & 17,1 \\
        &   &       &       &         & $m_{L_k}$  & 0,2 & 0,4 &     & 0,3 & 0,7 & 0,9  & 1,0  & 1,0  &      &      &      &      \\
        &   &       &       &         & $L_k min$  & 0,8 & 1,6 & 3,9 & 6,0 & 7,0 & 7,8  & 9,0  & 10,3 & 13,8 & 15,6 & 16,5 & 17,1 \\
        &   &       &       &         & $L_k max$  & 1,5 & 3,1 & 4,9 & 7,1 & 9,2 & 10,8 & 12,2 & 13,3 & 14,9 & 15,6 & 16,5 & 17,1 \\ \hline
\end{tabular}
\end{table}

\begin{table}[h]
\caption{Возрастная структура {\it M.~balthica} в верхнем горизонте литорали губы Ярнышная}
\label{tab:Yarnyshnaya_vgl_growth_matrix}
\begin{tabular}{|c|c|cc|cc|ccccccccccccc|}
    \hline
возраст & $N$  & $L min$ & $L max$ & $L aver$ & $m_L$   & 1 к & 2к  & 3к  & 4к  & 5к  & 6к   & 7к   & 8к   & 9 к  & 10 к & 11 к & 12 к & 13 к \\ \hline
0+      & 0  &       &       &         &         &     &     &     &     &     &      &      &      &      &      &      &      &      \\
1+      & 0  &       &       &         &         &     &     &     &     &     &      &      &      &      &      &      &      &      \\
2+      & 2  & 3,1   & 3,3   & 3,2     & 0,1     & 1,3 & 2,5 &     &     &     &      &      &      &      &      &      &      &      \\
3+      & 17 & 3,4   & 5,5   & 4,4     & 0,1     & 1,1 & 2,3 & 3,4 &     &     &      &      &      &      &      &      &      &      \\
4+      & 33 & 4,2   & 6,1   & 5,2     & 0,1     & 1,2 & 2,2 & 3,3 & 4,2 &     &      &      &      &      &      &      &      &      \\
5+      & 1  & 5,6   & 5,6   & 5,6     &         & 0,7 & 1,5 & 2,7 & 3,9 & 4,5 &      &      &      &      &      &      &      &      \\
6+      & 0  &       &       &         &         &     &     &     &     &     &      &      &      &      &      &      &      &      \\
7+      & 0  &       &       &         &         &     &     &     &     &     &      &      &      &      &      &      &      &      \\
8+      & 0  &       &       &         &         &     &     &     &     &     &      &      &      &      &      &      &      &      \\
9+      & 0  &       &       &         &         &     &     &     &     &     &      &      &      &      &      &      &      &      \\
10+     & 2  & 10,3  & 16,1  & 13,2    & 2,9     &     &     & 3,3 & 4,7 & 5,5 & 6,7  & 7,3  & 7,9  & 9,0  & 9,9  &      &      &      \\
11+     & 1  & 16,4  & 16,4  & 16,4    &         &     &     &     &     &     & 10,7 & 12,5 & 13,5 & 14,0 & 14,8 & 15,5 &      &      \\
12+     & 4  & 16,4  & 20,2  & 17,9    & 0,9     &     & 2,7 & 4,0 & 6,0 & 7,3 & 9,2  & 11,7 & 13,2 & 14,2 & 15,1 & 15,9 & 16,6 &      \\
13+     & 2  & 19,8  & 20,3  & 20,1    & 0,3     &     &     &     &     &     &      &      &      & 15,4 & 16,7 & 17,5 & 18,5 & 19,3 \\ \hline
        &    &       &       &         & $L_k aver$ & 1,1 & 2,2 & 3,3 & 4,7 & 5,8 & 8,9  & 10,5 & 11,5 & 13,2 & 14,1 & 16,3 & 17,5 & 19,3 \\
        &    &       &       &         & $m_{L_k}$  & 0,1 & 0,2 &     & 0,5 & 0,8 & 1,2  & 1,6  & 1,8  &      &      &      &      &      \\
        &    &       &       &         & $L_k min$  & 0,7 & 1,5 & 2,7 & 3,9 & 4,5 & 6,7  & 7,3  & 7,9  & 9,0  & 9,9  & 15,5 & 16,6 & 19,3 \\
        &    &       &       &         & $L_k max$  & 1,3 & 2,7 & 4,0 & 6,0 & 7,3 & 10,7 & 12,5 & 13,5 & 15,4 & 16,7 & 17,5 & 18,5 & 19,3 \\ \hline
\end{tabular}
\end{table}

\begin{table}[h]
\caption{Возрастная структура {\it M.~balthica} в среднем горизонте литорали губы Ярнышная}
\label{tab:Yarnyshnaya_sgl_growth_matrix}
\begin{tabular}{|c|c|cc|cc|cccccccc|}
    \hline
возраст & $N$  & $L min$ & $L max$ & $L aver$ & $m_L$   & 1 к  & 2к   & 3к   & 4к   & 5к   & 6к   & 7к   & 8к   \\ \hline
0+      &    &       &       &         &         &      &      &      &      &      &      &      &      \\
1+      & 16 & 2,3   & 4,8   & 3,4     & 0,17    & 1,1  &      &      &      &      &      &      &      \\
2+      & 18 & 3,1   & 6,3   & 4,7     & 0,19    & 1,0  & 2,5  &      &      &      &      &      &      \\
3+      & 4  & 4,2   & 9,4   & 6,4     & 1,09    & 2,4  & 5,7  & 7,5  &      &      &      &      &      \\
4+      & 10 & 7,3   & 10,8  & 8,7     & 0,35    & 0,9  & 2,0  & 4,1  & 6,5  &      &      &      &      \\
5+      & 9  & 8,3   & 17,1  & 13,1    & 0,88    & 2,5  & 5,8  & 9,4  & 11,9 & 13,1 &      &      &      \\
6+      & 6  & 11,9  & 17,7  & 14,8    & 0,79    & 1,9  & 4,2  & 7,3  & 10,0 & 12,1 & 13,9 &      &      \\
7+      & 7  & 14,6  & 17,3  & 15,9    & 0,43    & 1,7  & 3,8  & 7,0  & 9,6  & 12,0 & 14,3 & 15,9 &      \\
8+      & 6  & 14,8  & 19,5  & 16,7    & 0,69    & 2,0  & 4,4  & 6,3  & 8,9  & 11,7 & 12,9 & 14,7 & 16,2 \\
9+      & 1  &     &     & 16,8    &         &    &    &    &    &    &    &    &    \\
10+     & 3  & 17,7  & 18    & 17,8    & 0,09    &    &    &    &    &    &    &    &    \\
11+     & 1  &     &     & 17,6    &         &    &    &    &    &    &    &    &    \\ \hline
        &    &       &       &         & $L_k aver$ & 1,7  & 4,1  & 6,9  & 9,4  & 12,2 & 13,7 & 15,3 & 16,2 \\
        &    &       &       &         & $m_{L_k}$  & 0,22 & 0,55 & 0,70 & 0,87 & 0,31 & 0,41 & 0,59 &      \\
        &    &       &       &         & $L_k min$  & 0,9  & 2,0  & 4,1  & 6,5  & 11,7 & 12,9 & 14,7 &    \\
        &    &       &       &         & $L_k max$  & 2,5  & 5,8  & 9,4  & 11,9 & 13,1 & 14,3 & 15,9 &    \\ \hline
\end{tabular}
\end{table}

\begin{table}[h]
\caption{Возрастная структура {\it M.~balthica} в нижнем горизонте литорали губы Ярнышная}
\label{tab:Yarnyshnaya_ngl_growth_matrix}
\begin{tabular}{|c|c|cc|cc|ccccccccccccc|}
\hline]
возраст & $N$  & $L min$ & $L max$ & $L aver$ & $m_L$   & 1 к  & 2к   & 3к  & 4к   & 5к   & 6к   & 7к   & 8к   & 9 к  & 10 к & 11 к & 12 к & 13 к \\ \hline
0+      & 0  &       &       &         &         &      &      &     &      &      &      &      &      &      &      &      &      &      \\
1+      & 0  &       &       &         &         &      &      &     &      &      &      &      &      &      &      &      &      &      \\
2+      & 7  & 2,5   & 3,5   & 3,0     & 0,14    & 0,8  & 1,8  &     &      &      &      &      &      &      &      &      &      &      \\
3+      & 38 & 3,0   & 4,8   & 3,7     & 0,07    & 1,0  & 2,0  & 3,0 &      &      &      &      &      &      &      &      &      &      \\
4+      & 1  & 4,8   & 4,8   & 4,8     &         & 0,9  & 1,7  & 2,8 & 3,8  &      &      &      &      &      &      &      &      &      \\
5+      & 2  & 5,6   & 6,0   & 5,8     & 0,20    & 0,9  & 2,2  & 3,2 & 4,4  & 5,2  &      &      &      &      &      &      &      &      \\
6+      & 2  & 7,5   & 7,7   & 7,6     & 0,10    & 1,2  & 2,2  & 3,3 & 4,4  & 5,3  & 6,5  &      &      &      &      &      &      &      \\
7+      & 0  &       &       &         &         &      &      &     &      &      &      &      &      &      &      &      &      &      \\
8+      & 2  & 9,7   & 11,2  & 10,5    & 0,75    & 1,0  & 2,6  & 3,9 & 5,0  & 6,1  & 7,1  & 8,4  & 9,6  &      &      &      &      &      \\
9+      & 0  &       &       &         &         &      &      &     &      &      &      &      &      &      &      &      &      &      \\
10+     & 0  &       &       &         &         &      &      &     &      &      &      &      &      &      &      &      &      &      \\
11+     & 3  & 12,5  & 18,0  & 14,8    & 1,64    &      &      &     &      & 5,5  & 6,5  & 7,7  & 8,9  & 10,3 & 11,1 & 12,0 &      &      \\
12+     & 5  & 13,5  & 17,6  & 15,7    & 0,68    &      & 2,5  & 3,8 & 5,1  & 6,2  & 7,2  & 8,5  & 10,1 & 11,8 & 12,7 & 13,9 & 14,0 &      \\
13+     & 1  & 14,4  & 14,4  & 14,4    &         &      &      &     &      & 5,0  & 6,5  & 8,0  & 9,3  & 10,1 & 11,4 & 12,7 & 13,5 & 14,0 \\ \hline
        &    &       &       &         & $L_k aver$ & 1,0  & 2,1  & 3,3 & 4,5  & 5,5  & 6,8  & 8,2  & 9,5  & 10,7 & 11,7 & 12,9 & 13,7 & 14,0 \\
        &    &       &       &         & $m_{L_k}$  & 0,05 & 0,13 &     & 0,24 & 0,20 & 0,16 & 0,19 & 0,26 &      &      &      &      &      \\
        &    &       &       &         & $L_k min$  & 0,8  & 1,7  & 2,8 & 3,8  & 5,0  & 6,5  & 7,7  & 8,9  & 10,1 & 11,1 & 12,0 & 13,5 & 14,0 \\
        &    &       &       &         & $L_k max$  & 1,2  & 2,6  & 3,9 & 5,1  & 6,2  & 7,2  & 8,5  & 10,1 & 11,8 & 12,7 & 13,9 & 14,0 & 14,0 \\ \hline
\end{tabular}
\end{table}

\begin{table}[h]
\caption{Возрастная структура {\it M.~balthica} в верхнем горизонте литорали Дальнего пляжа губа Дальнезеленецкая}
\label{tab:DP_vgl_growth_matrix}
\begin{tabular}{|c|c|cc|cc|ccccccccccccc|}
    \hline
возраст & $N$ & $L min$ & $L max$ & $L aver$ & $m_L$   & 1 к  & 2к   & 3к  & 4к   & 5к   & 6к   & 7к   & 8к   & 9 к  & 10 к & 11 к & 12 к &  \\ \hline
0+      & 0 &       &       &         &         &      &      &     &      &      &      &      &      &      &      &      &      &  \\
1+      & 2 & 2,2   & 3,5   & 2,9     & 0,65    & 1,1  &      &     &      &      &      &      &      &      &      &      &      &  \\
2+      & 1 &       &       & 4,0     &         & 1,1  & 1,8  &     &      &      &      &      &      &      &      &      &      &  \\
3+      & 1 &       &       & 5,8     &         & 1,5  & 2,5  & 3,5 &      &      &      &      &      &      &      &      &      &  \\
4+      & 0 &       &       &         &         &      &      &     &      &      &      &      &      &      &      &      &      &  \\
5+      & 0 &       &       &         &         &      &      &     &      &      &      &      &      &      &      &      &      &  \\
6+      & 0 &       &       &         &         &      &      &     &      &      &      &      &      &      &      &      &      &  \\
7+      & 0 &       &       &         &         &      &      &     &      &      &      &      &      &      &      &      &      &  \\
8+      & 0 &       &       &         &         &      &      &     &      &      &      &      &      &      &      &      &      &  \\
9+      & 1 &       &       & 14,9    &         &      &      &     & 5,1  & 7,7  & 10,0 & 11,6 & 13,0 & 14,3 &      &      &      &  \\
10+     & 1 &       &       & 15,1    &         &      &      &     & 5,5  & 8,3  & 9,9  & 11,5 & 13,3 & 14,0 & 14,5 &      &      &  \\
11+     & 1 &       &       & 16,2    &         &      &      &     &      & 7,5  & 10,0 & 11,7 & 13,2 & 14,5 & 15,5 & 16,0 &      &  \\
12+     & 1 &       &       & 16,6    &         &      &      &     &      & 8,0  & 9,3  & 11,0 & 12,6 & 13,4 & 14,6 & 15,7 & 16,3 &  \\ \hline
        &   &       &       &         & $L_k aver$ & 1,2  & 2,2  & 3,5 & 5,3  & 7,9  & 9,8  & 11,5 & 13,0 & 14,1 & 14,9 & 15,9 &      &  \\
        &   &       &       &         & $m_{L_k}$  & 0,13 & 0,35 &     & 0,20 & 0,18 & 0,17 & 0,16 & 0,15 & 0,24 & 0,32 &      &      &  \\
        &   &       &       &         & $L_k min$  & 1,1  & 1,8  & 3,5 & 5,1  & 7,5  & 9,3  & 11,0 & 12,6 & 13,4 & 14,5 & 15,7 &      &  \\
        &   &       &       &         & $L_k max$  & 1,5  & 2,5  & 3,5 & 5,5  & 8,3  & 10,0 & 11,7 & 13,3 & 14,5 & 15,5 & 16,0 &      &  \\ \hline
\end{tabular}
\end{table}

\begin{table}[h]
\caption{Возрастная структура {\it M.~balthica} в среднем горизонте литорали Дальнего пляжа губа Дальнезеленецкая}
\label{tab:DP_sgl_growth_matrix}
\begin{tabular}{|c|c|cc|cc|ccccccccccccc|}
    \hline
возраст & $N$  & $L min$ & $L max$ & $L aver$ & $m_L$   & 1 к & 2к  & 3к  & 4к   & 5к   & 6к   & 7к   & 8к   & 9к   & 10к  & 11к  & 12к  &  \\ \hline
0+      & 0  &       &       &         &         &     &     &     &      &      &      &      &      &      &      &      &      &  \\
1+      & 3  & 2,5   & 5,8   & 3,8     & 1,0     & 1,1 &     &     &      &      &      &      &      &      &      &      &      &  \\
2+      & 17 & 2,1   & 9,8   & 7,2     & 0,6     & 1,3 & 4,7 &     &      &      &      &      &      &      &      &      &      &  \\
3+      & 1  & 10,2  & 10,2  & 10,2    &         & 1,5 & 4,0 & 7,0 &      &      &      &      &      &      &      &      &      &  \\
4+      & 4  & 9,4   & 15,2  & 13,0    & 1,3     & 1,2 & 5,2 & 9,4 & 11,4 &      &      &      &      &      &      &      &      &  \\
5+      & 6  & 12,4  & 16,5  & 14,9    & 0,6     &     & 4,5 & 8,9 & 11,7 & 13,3 &      &      &      &      &      &      &      &  \\
6+      & 14 & 6,8   & 17,6  & 14,8    & 0,7     & 3,0 & 4,9 & 7,9 & 10,3 & 12,1 & 13,4 &      &      &      &      &      &      &  \\
7+      & 7  & 13,7  & 18,4  & 16,8    & 0,6     & 2,0 & 5,0 & 7,6 & 10,6 & 12,5 & 14,2 & 15,4 &      &      &      &      &      &  \\
8+      & 3  & 9,0   & 17,7  & 13,5    & 2,5     & 1,1 & 4,1 & 5,9 & 8,1  & 9,8  & 11,0 & 12,0 & 12,7 &      &      &      &      &  \\
9+      & 2  & 13,0  & 13,8  & 13,4    & 0,4     &     &     & 4,1 & 5,7  & 7,6  & 8,9  & 10,4 & 11,7 & 12,7 &      &      &      &  \\
10+     & 1  & 15,0  & 15,0  & 15,0    &         & 1,0 & 2,6 & 5,5 & 7,7  & 9,4  & 10,5 & 11,7 & 12,5 & 13,2 & 14,3 &      &      &  \\
11+     & 1  & 16,5  & 16,5  & 16,5    &         &     &     & 4,5 & 6,5  & 7,8  & 8,8  & 9,8  & 10,8 & 13,0 & 14,9 & 15,9 &      &  \\
12+     & 1  & 16,5  & 16,5  & 16,5    &         &     &     & 4,7 & 7,5  & 8,5  & 9,8  & 10,6 & 12,4 & 13,7 & 14,5 & 15,5 & 16,0 &  \\ \hline
        &    &       &       &         & $L_k aver$ & 1,5 & 4,4 & 6,5 & 8,8  & 10,1 & 10,9 & 11,6 & 12,0 & 13,1 & 14,6 & 15,7 &      &  \\
        &    &       &       &         & $m_{L_k}$  & 0,2 & 0,3 & 0,6 & 0,7  & 0,8  & 0,8  & 0,8  & 0,4  & 0,2  & 0,2  &      &      &  \\
        &    &       &       &         & $L_k min$  & 1,0 & 2,6 & 4,1 & 5,7  & 7,6  & 8,8  & 9,8  & 10,8 & 12,7 & 14,3 & 15,5 &      &  \\
        &    &       &       &         & $L_k max$  & 3,0 & 5,2 & 9,4 & 11,7 & 13,3 & 14,2 & 15,4 & 12,7 & 13,7 & 14,9 & 15,9 &      &  \\ \hline
\end{tabular}
\end{table}

\begin{table}[h]
\caption{Возрастная структура {\it M.~balthica} в верхнем горизонте литорали губы Шельпино}
\label{tab:Shelpino_vgl_growth_matrix}
\begin{tabular}{|c|c|cc|cc|ccccccccccccc|}
    \hline
возраст & $N$ & $L min$ & $L max$ & $L aver$ & $m_L$   & 1 к  & 2к   & 3к  & 4к   & 5к   & 6к   & 7к   & 8к   & 9 к  & 10 к & 11 к & 12 к &  \\ \hline
0+      & 0 &       &       &         &         &      &      &     &      &      &      &      &      &      &      &      &      &  \\
1+      & 0 &       &       &         &         &      &      &     &      &      &      &      &      &      &      &      &      &  \\
2+      & 0 &       &       &         &         &      &      &     &      &      &      &      &      &      &      &      &      &  \\
3+      & 0 &       &       &         &         &      &      &     &      &      &      &      &      &      &      &      &      &  \\
4+      & 0 &       &       &         &         &      &      &     &      &      &      &      &      &      &      &      &      &  \\
5+      & 1 & 8,2   & 8,2   & 8,2     &         & 0,7  & 4    & 5,1 & 6,4  & 7,3  &      &      &      &      &      &      &      &  \\
6+      & 0 &       &       &         &         &      &      &     &      &      &      &      &      &      &      &      &      &  \\
7+      & 1 & 10,9  & 10,9  & 10,9    &         & 1,1  & 2,2  & 4,3 & 6,2  & 7,4  & 8,6  & 9,3  &      &      &      &      &      &  \\
8+      & 1 & 10,1  & 10,1  & 10,1    &         &      & 4,1  & 5,0 & 6,0  & 6,8  & 7,7  & 8,5  & 9,4  &      &      &      &      &  \\
9+      & 0 &       &       &         &         &      &      &     &      &      &      &      &      &      &      &      &      &  \\
10+     & 0 &       &       &         &         &      &      &     &      &      &      &      &      &      &      &      &      &  \\
11+     & 0 &       &       &         &         &      &      &     &      &      &      &      &      &      &      &      &      &  \\
12+     & 2 & 14,9  & 15,4  & 15,2    & 0,25    &      &      &     &      & 7,5  & 8,1  & 9,0  & 10,3 & 11,5 & 12,3 & 13,3 & 14,0 &  \\ \hline
        &   &       &       &         & $L_k aver$ & 0,9  & 3,4  & 4,8 & 6,2  & 7,3  & 8,1  & 8,9  & 9,9  & 11,5 & 12,3 & 13,3 & 14,0 &  \\
        &   &       &       &         & $m_{L_k}$  & 0,20 & 0,62 &     & 0,12 & 0,16 & 0,26 & 0,23 & 0,45 &      &      &      &      &  \\
        &   &       &       &         & $L_k min$  & 0,7  & 2,2  & 4,3 & 6,0  & 6,8  & 7,7  & 8,5  & 9,4  & 11,5 & 12,3 & 13,3 & 14,0 &  \\
        &   &       &       &         & $L_k max$  & 1,1  & 4,1  & 5,1 & 6,4  & 7,5  & 8,6  & 9,3  & 10,3 & 11,5 & 12,3 & 13,3 & 14,0 &  \\ \hline
\end{tabular}
\end{table}

\begin{table}[h]
\caption{Возрастная структура {\it M.~balthica} в среднем горизонте литорали губы Шельпино}
\label{tab:Shelpino_sgl_growth_matrix}
\begin{tabular}{|c|c|cc|cc|ccccccccccccc|}
    \hline
возраст & $N$ & $L min$ & $L max$ & $L aver$ & $m_L$   & 1 к & 2к  & 3к  & 4к       & 5к  & 6к  & 7к   & 8к   & 9 к  & 10 к &  &  &  \\ \hline
0+      & 0 &       &       &         &         &     &     &     &          &     &     &      &      &      &      &  &  &  \\
1+      & 0 &       &       &         &         &     &     &     &          &     &     &      &      &      &      &  &  &  \\
2+      & 1 &       &       & 5,8     &         & 1,8 & 3   &     &          &     &     &      &      &      &      &  &  &  \\
3+      & 1 &       &       & 8,6     &         & 1,2 & 3,6 & 6,7 &          &     &     &      &      &      &      &  &  &  \\
4+      & 0 &       &       &         &         &     &     &     &          &     &     &      &      &      &      &  &  &  \\
5+      & 1 &       &       & 7       &         & 0,7 & 1,6 & 2,5 & 4,3      & 5,6 &     &      &      &      &      &  &  &  \\
6+      & 0 &       &       &         &         &     &     &     &          &     &     &      &      &      &      &  &  &  \\
7+      & 0 &       &       &         &         &     &     &     &          &     &     &      &      &      &      &  &  &  \\
8+      & 0 &       &       &         &         &     &     &     &          &     &     &      &      &      &      &  &  &  \\
9+      & 1 &       &       & 14,6    &         &     &     &     &          &     & 8,9 & 10,1 & 12,0 & 13,5 &      &  &  &  \\
10+     & 1 &       &       & 14,3    &         &     &     &     &          & 7,5 & 8,8 & 10,2 & 12,3 & 13,2 & 13,8 &  &  &  \\ \hline
        &   &       &       &         & $L_k aver$ & 1,2 & 2,7 & 4,6 & 4,3      & 6,6 & 8,9 & 10,2 & 12,2 & 13,4 & 13,8 &  &  &  \\
        &   &       &       &         & $m_{L_k}$  & 0,3 & 0,6 &     &       & 1,0 & 0,0 & 0,0  & 0,2  &      &      &  &  &  \\
        &   &       &       &         & $L_k min$  & 0,7 & 1,6 & 2,5 & 4,3      & 5,6 & 8,8 & 10,1 & 12,0 & 13,2 & 13,8 &  &  &  \\
        &   &       &       &         & $L_k max$  & 1,8 & 3,6 & 6,7 & 4,3      & 7,5 & 8,9 & 10,2 & 12,3 & 13,5 & 13,8 &  &  &  \\ \hline
\end{tabular}
\end{table}

\begin{table}[h]
\caption{Возрастная структура {\it M.~balthica} в среднем горизонте литорали губы Порчниха}
\label{tab:Porchnikha_sgl_growth_matrix}
\begin{tabular}{|c|c|cc|cc|ccccccccc|}
    \hline
    возраст & $N$  & $L min$ & $L max$ & $L aver$ & $m_L$    & 1 к  & 2к   & 3к   & 4к   & 5к   & 6к   & 7к   & 8к   & 9к   \\ \hline
0+      & 0  &       &       &         &         &      &      &      &      &      &      &      &      &      \\
1+      & 2  & 3,4   & 3,6   & 3,5     & 0,10    & 1,5  &      &      &      &      &      &      &      &      \\
2+      & 24 & 3,2   & 6,9   & 4,7     & 0,21    & 1,1  & 3,2  &      &      &      &      &      &      &      \\
3+      & 29 & 4,5   & 13,3  & 7,5     & 0,48    & 1,4  & 3,8  & 5,8  &      &      &      &      &      &      \\
4+      & 12 & 5,4   & 15,1  & 9,3     & 0,80    & 1,4  & 3,9  & 5,7  & 7,9  &      &      &      &      &      \\
5+      & 10 & 6,8   & 18,9  & 14,5    & 1,19    & 1,9  & 4,6  & 8,0  & 10,8 & 12,8 & 13,6 &      &      &      \\
6+      & 6  & 16,8  & 20,5  & 18,5    & 0,57    & 2,1  & 4,8  & 8,9  & 13,0 & 15,3 & 17,2 &      &      &      \\
7+      & 1  &     &     & 18,5    &         &    & 5,5  & 9,4  & 12,2 & 14,8 & 16,4 & 17,8 &      &      \\
8+      & 0  &       &       &         &         &      &      &      &      &      &      &      &      &      \\
9+      & 1  &     &     & 19,4    &         &    &    & 7,2  & 10,6 & 13,1 & 15,0 & 16,8 & 17,5 & 18,0 \\
10+     & 1  &     &     & 19,0    &         &    &    &    &    &    &    &    &    &    \\ \hline
        &    &       &       &         & $L_k aver$ & 1,6  & 4,3  & 7,5  & 10,9 & 14,0 & 15,5 & 17,3 & 17,5 & 18,0 \\ 
        &    &       &       &         & $m_{L_k}$  & 0,14 & 0,34 & 0,63 & 0,87 & 0,62 & 0,79 & 0,50 &      &      \\
        &    &       &       &         & $L_k min$  & 1,1  & 3,2  & 5,7  & 7,9  & 12,8 & 13,6 & 16,8 &    &    \\
        &    &       &       &         & $L_k max$  & 2,1  & 5,5  & 9,4  & 13,0 & 15,3 & 17,2 & 17,8 &    &    \\ \hline
\end{tabular}
\end{table}

\end{landscape}



%обилие маком по литературным и авторским источникам
\section{Обилие {\it Macoma balthica} в европейской части ареала}
\label{app:NB_areal}
	\begin{footnotesize}
    \begin{center}
	\begin{longtable}{|p{3cm}p{2cm}|*{2}{p{1cm}}|*{3}{p{0.9cm}}|p{0.9cm}|p{2cm}|}
	\caption{Обилие {\it Macoma balthica} в различных частях ареала по собственным и литературным данным} \label{tab:NB_areal}\\
	\hline
место                                     & море            & широта      & долгота      & $N_{min}$         & $N_{max}$            & $N_{mean}$          & $B_{mean}$  & источник                                    \\ \hline \endfirsthead
	\hline
	\multicolumn{9}{|c|}{продолжение таблицы \ref{tab:NB_areal}} \\ \hline
участок                                     & акватория            & широта      & долгота      & $N_{min}$         & $N_{max}$            & $N_{mean}$          & $B_{mean}$  & источник                                    \\ \hline \endhead
	\hline 
	\multicolumn{9}{|c|}{продолжение таблицы \ref{tab:NB_areal} на следующей странице}
	\\ \hline \endfoot
	 \endlastfoot
Gironde                                  & Бискайский залив & 45,55     & -1,05     &             &                & 500            &       & \cite{Bachelet_1986}                            \\ \hline
St John s Lake                           & Ла-Манш      & 50,37 & -4,21     &             & 36              & 36             &       & \cite{Warwick_Price_1975}                      \\ \hline
Lynher estuary                           & Ла-Манш      & 50,38     & -4,30     &             &                & 48,7           & 0,337  & \cite{Warwick_Price_1975}                      \\ \hline
river Exe                                & Ла-Манш      & 50,625134 & -3,43     &             &                & 24             &       & \cite{Warwick_Price_1975}                      \\ \hline
Wash                                     & Северное          & 52,85     & 0,42      & 48 & 1667 & 693 &       & \cite{Reading_1979}                    \\ \hline
Balgzand                                 & Ваттово море         & 52,93     & 4,83      &             &                &               &       & \cite{Beukema_1979}                             \\ \hline
between Den Helder and Delfzijl          & Ваттово море         & 52,93     & 4,83      &             &                & 113            &       & \cite{Beukema_1976}                             \\ \hline
river Clwyd                              & Ирландское море          & 53,31 & -3,51     & 2            & 184             & 74             &       & \cite{Parsons_Thomas_1979}                     \\ \hline
Friesland                                & Ваттово море         & 53,42     & 6,07      &             & 300             & 250            &       & \cite{Zwarts_Wanink_1993}                      \\ \hline
Кали\-нин\-град\-ский залив                          & Балтийское         & 54,61     & 20,00     &             &                & 460            & 40,85  & \cite{Gusev_et_al_2013}                       \\ \hline
Юго-Запад Балтийского моря                    & Балтийское         & 54,61     & 20,00     & 20           & 1000            & 650            & 80     & \cite{Gusev_2010}                               \\ \hline
K\"onigshafen of Sylt                     & Ваттово         & 55,04     & 8,40      & 5            & 265             & 81             &       & \cite{Reise_et_al_1994}                       \\ \hline
Ho Bay                                   & Северное          & 55,49     & 8,40      &             &                & 254            &       & \cite{Madsen_Jensen_1987}                      \\ \hline
Skallingen                               & Ваттово         & 55,52     & 8,29      &             &                & 241            &       & \cite{Jensen_Jensen_1985}                      \\ \hline
Budle Bay                                & Северное          & 55,62 & -1,76 & 2            & 554             & 122            &       & \cite{Brady_1943}                               \\ \hline
Black Middens                            & Северное          & 56,01 & -2,59 & 4            & 102             & 21,4           &       & \cite{Brady_1943}                               \\ \hline
Aberlady bay                             & Северное          & 56,01 & -2,86     &             &                & 200            &       & \cite{Stephen_1931}                             \\ \hline
Ythan river estuary                      & Северное          & 57,30     & -1,90     &             &                & 687            &       & \cite{Chambers_Milne_1975}                      \\ \hline
Копорская и Лужская губа Финского залива & Балтийское         & 59,71     & 28,31     &             &                &               & 43,2   & \cite{Maximov_et_al_2009}                     \\ \hline
Tvarminne                                & Балтийское         & 59,83 & 23,17     &             &                & 855            &       & \cite{Segerstrale_1969}                         \\ \hline
Tvarminne                                & Балтийское         & 59,83     & 23,17     & 62           & 1084            & 321            & 42,2   & \cite{Aschan_1988}                              \\ \hline
Tvarminne                                & Балтийское         & 59,83     & 23,17     & 353          & 1078            & 715            &       & \cite{Segerstrale_1933b} по \cite{Rousi_et_al_2013} \\ \hline
Tvarminne                                & Балтийское         & 59,83 & 23,17     & 190          & 990             & 590            &       & \cite{Laine_et_al_2013}                       \\ \hline
Aland Islands                            & Балтийское         & 60,17     & 20,53     & 600          & 1200            & 850            &       & \cite{Bostrom_Bonsdorff_2000}                   \\ \hline
Aland Islands                            & Балтийское         & 60,38     & 19,64     &             &                & 1360           & 230    & \cite{Bonsdorff_et_al_1995}                    \\ \hline
Skjellvika, Oydegard                     & Норвежское         & 63,01     & 8,00      &             &                & 121            &       & \cite{Sneli_1968}                               \\ \hline
Borgenfjord, Sund                        & Норвежское      & 63,86 & 11,31     & 8            & 270             & 109            &       & \cite{Stromgren_et_al_1973}                   \\ \hline
Borgenfjord, Lorvikleiret                & Норвежское      & 63,88 & 11,37     & 0            & 139             & 64             &       & \cite{Stromgren_et_al_1973}                   \\ \hline
Borgenfjord, Korsen                      & Норвежское      & 63,95 & 11,38     & 62           & 370             & 207            &       & \cite{Stromgren_et_al_1973}                   \\ \hline
Долгая губа, Соловки                     & Белое          & 65,06     & 35,75     &             &                & 1556           &       & \cite{Khaitov_et_al_2007}                     \\ \hline
Lakselvvatn                              & Норвежское      & 65,91     & 13,1      &             & 142             & 77             &       & \cite{Jensen_et_al_1985}                      \\ \hline
пролив Подпахта                                 & Белое          & 66,30     & 33,62     & 372          & 688             & 530            & 1,8    & authors data                              \\ \hline
бухта Клющиха (о.~Кереть)               & Белое          & 66,31     & 33,78     & 362          & 1136            & 647,3          & 1,1    & authors data                              \\ \hline
бухта Клющиха (о.~Кереть)               & Белое          & 66,31     & 33,78     & 130          & 1607            & 678            &       & \cite{Gerasimova_Maximovich_2013}              \\ \hline
Сухая Салма (о.~Кереть)                & Белое          & 66,31     & 33,65     & 992          & 1165            & 1096           & 6,2    & authors data                              \\ \hline
Сухая Салма (о.~Кереть)                & Белое          & 66,31     & 33,65     & 22           & 1114            & 410            &       & \cite{Gerasimova_Maximovich_2013}              \\ \hline
Лисья губа                                & Белое          & 66,31     & 33,57     & 1006         & 2832            & 1728           & 1,9    & authors data                              \\ \hline
губа Сельдяная                         & Белое          & 66,34     & 33,62     & 42           & 2089            & 669            & 20,25  & \cite{Varfolomeeva_Naumov_2013}                \\ \hline
Круглая губа, Картеш                    & Белое          & 66,34     & 33,64     &             &                & 545            &       & \cite{Khaitov_et_al_2007}                     \\ \hline
Медвежья губа                          & Белое          & 66,35     & 33,60     & 53           & 3300            & 618            & 41,96  & \cite{Varfolomeeva_Naumov_2013}                \\ \hline
губа Подволочье                           & Белое          & 66,52     & 33,20     & 120          & 1240            &               &       & \cite{Semenova_1974}                            \\ \hline
Ермолинская губа                         & Белое          & 66,55     & 33,05     & 75           & 400             &               &       & \cite{Semenova_1974}                            \\ \hline
ББС МГУ                                 & Белое          & 66,55     & 33,10     & 80           & 2760            &               &       & \cite{Semenova_1974}                            \\ \hline
губа Лобаниха (о.~Великий)             & Белое          & 66,56     & 33,20     & 20           & 810             &               &       & \cite{Semenova_1974}                            \\ \hline
Пеккелинская губа                        & Белое          & 66,59     & 32,96     & 43           & 980             &               &       & \cite{Semenova_1974}                            \\ \hline
остров Ломнишный                        & Белое          & 66,98     & 32,62     & 378          & 1530            & 713          & 25,03  & authors data                              \\ \hline
Южная губа о.~Ряшкова                & Белое          & 67,01     & 32,57     & 142          & 1913            & 1082           & 20,42  & authors data                              \\ \hline
Фукусовая губа о.~Ряшкова                 & Белое          & 67,01     & 32,58     &             &                & 285            &       & \cite{Khaitov_et_al_2007}                     \\ \hline
Западная Ряшкова салма, о.~Ряшков      & Белое          & 67,01     & 32,54     & 220          & 8530            & 1811           & 106,67 & authors data                              \\ \hline
о.Горелый, Лувеньга                   & Белое          & 67,09     & 32,68     & 13           & 2740            & 1079           & 68,83  & authors data                              \\ \hline
материковая литораль, Лувеньга                                  & Белое          & 67,10     & 32,71     & 94           & 7240            & 1504           & 85,13  & authors data                              \\ \hline
эстуарий р.~Лувеньги                    & Белое          & 67,10     & 32,69     & 55           & 3330            & 1449           & 81,49  & authors data                              \\ \hline
губа Ивановская                          & Баренцево        & 68,29     & 38,71     & 1208         & 1208            & 1208           & 75     & authors data                              \\ \hline
Печорская губа                              & Печорская губа    & 68,59     & 55,22     &             &                & 654            & 267,84 & \cite{Denisenko_et_al_2003}                   \\ \hline
Северное Нагорное, Мурманск                                 & Баренцево        & 68,9      & 33,06     & 390          & 390             & 390            &       & authors data                              \\ \hline
Абрам-мыс                               & Баренцево        & 68,98     & 33,03     & 540          & 3350            & 1898           & 197    & authors data                              \\ \hline
губа Порчниха                           & Баренцево        & 69,08     & 36,25     & 60           & 87              & 73,5           & 27     & authors data                              \\ \hline
губа Ярнышная                          & Баренцево        & 69,09     & 36,05     & 70           & 414             & 281,3          & 57,7   & authors data                              \\ \hline
губа Шельпино                             & Баренцево        & 69,10     & 36,21     & 36           & 54              & 45             & 19,5   & authors data                              \\ \hline
Ретинское                                & Баренцево        & 69,11     & 33,38     & 660          & 660             & 660            &       & authors data                              \\ \hline
Дальне-Зеленецкая губа                     & Баренцево        & 69,11     & 36,10     & 30           & 72              & 44,78          & 24,6   & authors data                              \\ \hline
губа Гаврилово                            & Баренцево        & 69,17     & 35,86     & 24           & 138             & 81             & 54,5   & authors data                              \\ \hline
Пала-губа                                 & Баренцево        & 69,19     & 33,37     & 790          & 1644            & 1058           & 104    & authors data                              \\ \hline
Ура-губа                                  & Баренцево        & 69,32     & 32,82     & 1267         & 1267            & 1267           &       & authors data                              \\ \hline
Печенга                             & Баренцево        & 69,58     & 31,27     & 767          & 767             & 767            &       & authors data                              \\ \hline
Tromso                                   & Норвежское         & 69,64     & 18,87     & 10           & 3360            & 890            &       & \cite{Oug_2001}                                 \\ \hline                                
	\end{longtable}
\end{center}
	\end{footnotesize}


\section{Ростовые характеристики {\it Macoma balthica} в европейской части ареала}
\label{app:growth_omega}
	\begin{footnotesize}
    \begin{center}
	\begin{longtable}{|p{3cm}p{2cm}|p{1.2cm}|*{3}{p{1cm}}|p{2cm}|}
	\caption{Обилие {\it Macoma balthica} в различных частях ареала по собственным и литературным данным} \label{tab:omega}\\
	\hline
место               & море            & широта      & $L_{\infty}$         & $k$            & $\omega$          & источник     \\ \hline \endfirsthead
	\hline
	\multicolumn{7}{|c|}{продолжение таблицы \ref{tab:omega}} \\ \hline
место               & море            & широта      & $L_{infty}$         & $k$            & $\omega$          & источник     \\ \hline \endhead
	\hline 
	\multicolumn{7}{|c|}{продолжение таблицы \ref{tab:omega} на следующей странице}
	\\ \hline \endfoot
	 \endlastfoot
Gironde Estuary       & Бискайский залив & 45,55     & 17,15  & 0,4354 & 7,47  & \cite{Bachelet_1980}              \\
Gironde Estuary       & Бискайский залив & 45,55     & 15,95  & 0,3644 & 5,81  & \cite{Bachelet_1980}              \\
Gironde Estuary       & Бискайский залив & 45,55     & 15,92  & 0,3769 & 6,00  & \cite{Bachelet_1980}              \\
Lynher estuary        & Ла-Манш          & 50,38     & 14,33  & 0,5109 & 7,32  & \cite{Warwick_Price_1975}        \\
Wash                  & Северное море    & 52,85     & 18,03  & 0,3447 & 6,22  & \cite{Reading_1979}               \\
Der Helder            & Ваттово море     & 53        & 25,56  & 0,3382 & 8,65  & \cite{Lammens_1967}               \\
Clwyd                 & Ирландское море  & 53,3 & 42,37  & 0,1272 & 5,39  & \cite{Parsons_Thomas_1979}       \\
Гданьский залив       & Балтийское море  & 54,5      & 21,80  & 0,2852 & 6,22  & \cite{Wenne_Klusek_1985}         \\
Гданьский залив       & Балтийское море  & 54,5      & 28,69  & 0,1895 & 5,44  & \cite{Wenne_Klusek_1985}         \\
Гданьский залив       & Балтийское море  & 54,5      & 27,45  & 0,2049 & 5,63  & \cite{Wenne_Klusek_1985}         \\
List, Sylt, Nordernay & Северное море    & 54,5      & 25,66  & 0,2634 & 6,76  & \cite{Vogel_1959}                 \\
Калининградский залив & Балтийское море  & 54,61     & 23,99  & 0,1293 & 3,10  & \cite{Gusev_JurgensMarkina_2012} \\
Калининградский залив & Балтийское море  & 54,61     & 20,61  & 0,1813 & 3,74  & \cite{Gusev_JurgensMarkina_2012} \\
Budle Bay             & Северное море    & 55,62 & 27,18  & 0,2815 & 7,65  & \cite{Brady_1943}                 \\
Ythan estuary         & Северное море    & 57,3      & 15,62  & 0,4372 & 6,83  & \cite{Chambers_Milne_1975}        \\
Tvaren Bay            & Балтийское море  & 59        & 32,92  & 0,0520 & 1,71  & \cite{Bergh_1974}                 \\
Tvarminne             & Балтийское море  & 60        & 64,38  & 0,0446 & 2,87  & \cite{Segerstrale_1960}           \\
Tvarminne             & Балтийское море  & 60        & 9,72   & 0,8819 & 8,57  & \cite{Vogel_1959}                 \\
б. Клющиха            & Белое море       & 66,31 & 45,24  & 0,0490 & 2,22  & \cite{Maximovich_et_al_1992}    \\
б. Клющиха            & Белое море       & 66,31 & 475,00 & 0,0058 & 2,77  & \cite{Maximovich_et_al_1992}    \\
Сухая салма           & Белое море       & 66,31 & 22,22  & 0,1369 & 3,04  & \cite{Maximovich_et_al_1992}    \\
Сухая салма           & Белое море       & 66,31 & 351,00 & 0,0062 & 2,18  & \cite{Maximovich_et_al_1992}    \\
Ермолинская губа      & Белое море       & 66,55     & 18,61  & 0,0556 & 1,03  & \cite{Semenova_1970}              \\
ББС МГУ               & Белое море       & 66,55     & 22,57  & 0,0424 & 0,96  & \cite{Semenova_1970}              \\
Пеккелинская губа     & Белое море       & 66,59     & 22,20  & 0,0473 & 1,05  & \cite{Semenova_1970}              \\
Абрам-мыс             & Баренцево море   & 68,98     & 30,17  & 0,0468 & 1,41  & наши данные                 \\
Порчниха              & Баренцево море   & 69,08     & 33,03  & 0,0970 & 3,21  & наши данные                 \\
Ярнышная              & Баренцево море   & 69,09     & 193,00 & 0,0061 & 1,17  & наши данные                 \\
Ярнышная              & Баренцево море   & 69,09     & 49,50  & 0,0520 & 2,57  & наши данные                 \\
Шельпино              & Баренцево море   & 69,1      & 26,55  & 0,0619 & 1,64  & наши данные                 \\
Дальне-Зеленецкая     & Баренцево море   & 69,11     & 352,89 & 0,0044 & 1,55  & наши данные                 \\
Дальне-Зеленецкая     & Баренцево море   & 69,11     & 20,09  & 0,1284 & 2,58  & наши данные                 \\
Гаврилово             & Баренцево море   & 69,17     & 75,80  & 0,0223 & 1,69  & наши данные                 \\
Гаврилово             & Баренцево море   & 69,17     & 55,27  & 0,0252 & 1,39  & наши данные                 \\ \hline
	\end{longtable}
\end{center}
	\end{footnotesize}


\section{Источники данных о росте {\it Macoma balthica} в европейской части ареала}
\label{app:growth_sources}
	\begin{footnotesize}
    \begin{center}
	\begin{longtable}{|l*{3}{p{4.3cm}}|} \hline
код   & участок               & акватория                 & источник                 \\ \hline \endfirsthead
	\hline
	\multicolumn{4}{|c|}{продолжение приложения \ref{app:growth_sources}} \\ \hline
код   & участок               & акватория                 & источник                 \\ \hline \endhead
	\hline 
	\multicolumn{4}{|c|}{продолжение приложения \ref{app:growth_sources} на следующей странице}
	\\ \hline \endfoot
	 \endlastfoot
Балт1 & IPB                   & Балтийское море           & \cite{Wenne_Klusek_1985}      \\
Балт2 & GN                    & Балтийское море           & \cite{Wenne_Klusek_1985}      \\
Балт3 & H75                   & Балтийское море           & \cite{Wenne_Klusek_1985}      \\
Балт4 & Tvaren Bay            & Балтийское море           & \cite{Bergh_1974}              \\
Балт6 & Finland Gulf 6m       & Балтийское море           & \cite{Segerstrale_1960}        \\
Балт9 & Tvarminne area        & Балтийское море           & \cite{Vogel_1959}              \\ 
Бар1  & Абрам-мыс СГЛ         & Баренцево море            & авторские данные         \\
Бар2  & Абрам-мыс НГЛ         & Баренцево море            & авторские данные         \\
Бар3  & Дальнезеленецкая ВГЛ  & Баренцево море            & авторские данные         \\
Бар4  & Дальнезеленецкая СГЛ  & Баренцево море            & авторские данные         \\
Бар5  & Гаврилово СГЛ         & Баренцево море            & авторские данные         \\
Бар6  & Гаврилово НГЛ         & Баренцево море            & авторские данные         \\
Бар7  & Шельпино ВГЛ          & Баренцево море            & авторские данные         \\
Бар8  & Шельпино СГЛ          & Баренцево море            & авторские данные         \\
Бар9  & Ярнышная ВГЛ          & Баренцево море            & авторские данные         \\
Бар10 & Ярнышная СГЛ          & Баренцево море            & авторские данные         \\
Бар11 & Ярнышнвя НГЛ          & Баренцево море            & авторские данные         \\
Бар12 & Пала-губа СГЛ         & Баренцево море            & авторские данные         \\
Бар13 & Пала-губа НГЛ         & Баренцево море            & авторские данные         \\
Бар14 & Порчниха СГЛ          & Баренцево море            & авторские данные         \\
Бар15 & Ивановская ВСЛ       & Баренцево море            & авторские данные         \\
Бел1  & Пеккелинская губа     & Белое море                & \cite{Semenova_1970}           \\
Бел2  & ББС МГУ               & Белое море                & \cite{Semenova_1970}           \\
Бел3  & Ермолинская губа      & Белое море                & \cite{Semenova_1970}           \\
Бел4  & Сухая салма СГЛ       & Белое море                & \cite{Maximovich_et_al_1992} \\
Бел5  & Сухая салма НГЛ       & Белое море                & \cite{Maximovich_et_al_1992} \\
Бел6  & Сухая салма Zostera   & Белое море                & \cite{Maximovich_et_al_1992} \\
Бел7  & Клющиха СГЛ           & Белое море                & \cite{Maximovich_et_al_1992} \\
Бел8  & Клющиха НГЛ           & Белое море                & \cite{Maximovich_et_al_1992} \\
Бел9  & Клющиха Zostera       & Белое море                & \cite{Maximovich_et_al_1992} \\
Сев   & List, Sylt, Nordernay & Северное море             & \cite{Vogel_1959}              \\
Биск1 & Le Verdon high        & Жиронда, Бискайский залив & \cite{Bachelet_1980}           \\
Биск2 & Phare de Richard      & Жиронда, Бискайский залив & \cite{Bachelet_1980}           \\ \hline
	\end{longtable}
\end{center}
	\end{footnotesize}


\end{document}
