%из Жениного диссера

Снижение численности особей в популяциях на границе ареала, по сравнению с расположенными вблизи его географического центра представляет собой один из ключевых постулатов современной биогеографии (Sagarin, Gaines, 2002б; Sorte, Hofmann, 2004; Sagarin et al., 2006). 
Большинство гипотез, сформулированных для объяснения причин снижения размера популяции вида от центра его ареала к периферии, предполагают, что в центре ареала вида наблюдается наиболее благоприятное сочетание факторов и условий среды, и протяженность ареала зависит от толерантности вида по отношению к изменению всего комплекса факторов (Andrewartha, Birch, 1954; Brown, 1984). 
Таким образом, обилие в конкретном районе можно считать комплексным отражением требований к многомерному пространству факторов, формирующему экологическую нишу данного вида (температурная толерантность, конкурентоспособность и устойчивость популяции к воздействию хищников и т.д.) (Brown, 1984; Gaston, 2003).
