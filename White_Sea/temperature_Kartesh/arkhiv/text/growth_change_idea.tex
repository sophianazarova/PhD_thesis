
\documentclass[12pt]{article}
\usepackage[english,russian]{babel}
\usepackage[warn]{mathtext}
\usepackage[T2A]{fontenc}
\usepackage[utf8]{inputenc}

\usepackage{color}
\usepackage{amssymb,amsmath}

\textheight=24cm           % высота текста
\textwidth=16cm            % ширина текста
\oddsidemargin=0pt         % отступ от левого края
\topmargin=-1.5cm          % отступ от верхнего края
\parindent=24pt            % абзацный отступ
\parskip=0pt               % интервал между абзацами
\tolerance=2000            % терпимость к "жидким" строкам
\flushbottom               % выравнивание высоты страниц
%\def\baselinestretch{1.5} % печать с большим интервалом

%\title{}
%\author{\copyright~~С.А.~Назарова \thanks{e-mail:~sophia.nazarova@gmail.com}}
%\date{}

\begin{document}
%\maketitle
Откуда взялась исходная идея - все говорят про изменение климата. И действительно есть ощущение, что было несколько теплых зим. \\
Известно, что температура влияет на скорость химических реакций, а, значит, на метаболизм пойкилотермных организмов, т.е. в том числе и на рост. \\
\\

Идея - посмотреть, влияет ли изменение температуры на рост {\it Macoma balthica}. \\
Для этого хочется выделить <<холодные>> и <<теплые>> периоды (если таковые реально были).\\
\\

Что есть?\\
Есть мониторинг маком в 2х точках на литорали за 30 лет.\\
Есть с нескольких горизонтов (0, 5, 10 м) температура воды с декадной станции ЗИН РАН (от исследованных участков литорали она находится на расстоянии около 1-1,5 км в центре пролива перед Картешем) за те же годы. Не везде данные полные, особенно зимние. \\
\\

Вопросы:\\
\begin{enumerate}
\item Можно ли по температуре в губе оценивать условия в конкретный год на литорали (в ситуации если хочется не выводить функциональную зависимость, а именно разделить на группы <<холодные>> и <<теплые>> года)?\\

\item Для более адекватной оценки условий на литорали какие данные стоит использовать - об изменении поверхностной температуры или брать усреднение по нескольким горизонтам (0-5 или 0-10 м)?

\item Насколько варьирует температура воды в течении 10 дней? Имеет ли смысл считать накопленное тепло (сумму градусодней), если нет температуры за каждый день, а есть только декадная станция?

\item По какому показателю пытаться выделять <<холодные>> и <<теплые>> года: среднегодовая температура, среднелетняя температура (с учетом того что растут ракушки в основном летом), ход среднемесячных температур, сумма градусодней?

\item Каким методом выделять <<холодные>> и <<теплые>> годы (есть ли что-то принятое в гидрологии)?
\end{enumerate}


\end{document}
