% для компиляции в lualatex!!
%\documentclass[12pt, a4paper]{article}
\documentclass[12pt, a4paper]{disser}
\usepackage[english,russian]{babel}
\usepackage[warn]{mathtext}
%\usepackage[T2A]{fontenc}
%\usepackage[utf8]{inputenc}

\usepackage{xecyr} % Продукт Вашего покорного слуги ;)

%\setmainfont{DejaVu Serif}
\setmainfont{Liberation Serif}

\usepackage{color}
\usepackage{amssymb,amsmath}
\usepackage{graphicx}
\usepackage{multicol}

\textheight=24cm           % высота текста
\textwidth=16cm            % ширина текста
\oddsidemargin=0pt         % отступ от левого края
\topmargin=-1.5cm          % отступ от верхнего края
\parindent=24pt            % абзацный отступ
\parskip=0pt               % интервал между абзацами
\tolerance=2000            % терпимость к "жидким" строкам
\flushbottom               % выравнивание высоты страниц
%\def\baselinestretch{1.5} % печать с большим интервалом

%\title{}
%\author{\copyright~~С.А.~Назарова \thanks{e-mail:~sophia.nazarova@gmail.com}}
%\date{}


\begin{document}

	\begin{figure}[ht]
	
	\begin{minipage}[b]{.46\linewidth}
	%Фигурка в первом ряду слева размер отведенный под весь этот объект \textendash 0.46 от ширины строки
	%Параметр [b] означает, что выравнивание этих министраниц будет по нижнему краю
	\begin{center}
	{\tiny 2002}
		\includegraphics[width=65mm]{station_jaccard_2002.pdf}

	\end{center}
	\end{minipage}
	%
	\hfil %Это пружинка отодвигающая рисунки друг от друга
	%
	\begin{minipage}[b]{.46\linewidth}
%Следующий рисунок - первый ряд справа %DUNGEON S_4 \ AB
	\begin{center}
	{\tiny 2003}
		\includegraphics[width=65mm]{station_jaccard_2003.pdf}
	\end{center}
	\end{minipage}

%\smallskip


	\begin{minipage}[b]{.46\linewidth}
%Фигурка в первом ряду слева размер отведенный под весь этот объект \textendash 0.46 от ширины строки
%Параметр [b] означает, что выравнивание этих министраниц будет по нижнему краю
	\begin{center}
	{\tiny 2004}
		\includegraphics[width=65mm]{station_jaccard_2004.pdf}
	\end{center}
	\end{minipage}
%
	\hfil %Это пружинка отодвигающая рисунки друг от друга
%
	\begin{minipage}[b]{.46\linewidth}
%Следующий рисунок - первый ряд справа %DUNGEON S_4 \ AB
	\begin{center}	
	{\tiny 2005}
		\includegraphics[width=65mm]{station_jaccard_2005.pdf}
	\end{center}
	\end{minipage}

%\smallskip

	\begin{minipage}[b]{.46\linewidth}
%Фигурка в первом ряду слева размер отведенный под весь этот объект \textendash 0.46 от ширины строки
%Параметр [b] означает, что выравнивание этих министраниц будет по нижнему краю
	\begin{center}
	{\tiny 2006}
		\includegraphics[width=65mm]{station_jaccard_2006.pdf}
	\end{center}
	\end{minipage}
%
	\hfil %Это пружинка отодвигающая рисунки друг от друга
%
	\begin{minipage}[b]{.46\linewidth}
%Следующий рисунок - первый ряд справа %DUNGEON S_4 \ AB
	\begin{center}
	{\tiny 2007}
		\includegraphics[width=65mm]{station_jaccard_2007.pdf}
	\end{center}
	\end{minipage}

%\smallskip
	\caption{Сходство мониторинговых станций на Дальнем пляже по видовому разнообразию в разные годы. Мера сходства - коэффициент Жаккара. Кластеризация методом Варда.}
	\label{ris:jaccard_station}
	\end{figure}

\end{document}
