\subsection*{\Large Общая характеристика работы}
%\fontsize{14pt}{15pt}\selectfont
\underline{\textbf{Актуальность темы.}}

\underline{\textbf{Целью}} данной работы является ...

Для~достижения поставленной цели необходимо было решить следующие \underline{\textbf{задачи}}:
\begin{enumerate}
 \item Задача номер один;
 \item Задача номер два;
 \item Задача номер три;
% и так далее, если нужно 
\end{enumerate}

\underline{\textbf{Основные положения, выносимые на~защиту:}}
\begin{enumerate}
 \item Первое положение.
 \item Второе положение.
 \item Третье положение.
% и так далее, если нужно
\end{enumerate}

\underline{\textbf{Научная новизна:}}
\begin{enumerate}
 \item Впервые ... . 
 \item Впервые ... .
 \item Впервые ... . 
\end{enumerate}

\underline{\textbf{Практическая значимость}} диссертационной работы определяется ...

\underline{\textbf{Достоверность}} изложенных в работе результатов обеспечивается ...

\underline{\textbf{Апробация работы.}}
Основные результаты работы докладывались~на:
Название симпозиума (Страна, город, год),
Название конференции (Страна, город, год),
% и так далее, если нужно

Диссертационная работа была выполнена при поддержке грантов ...

\underline{\textbf{Личный вклад.}} Автор принимал активное участие ...

\underline{\textbf{Публикации.}} Основные результаты по теме диссертации изложены в ХХ печатных изданиях, Х из которых изданы в журналах, рекомендованных ВАК, ХХ --- в тезисах докладов.

\underline{\textbf{Объем и структура работы.}} Диссертация состоит из~введения, четырех глав, заключения и~приложения. Полный объем диссертации \textbf{ХХХ}~страниц текста с~\textbf{ХХ}~рисунками и~5~таблицами. Список литературы содержит \textbf{ХХX}~наименование.

\underline{\textbf{Благодарности}}
%\addcontentsline{toc}{chapter}{Благодарности}
В заключение я хочу поблагодарить администрацию Кандалакшского заповедника и лично \fbox{А.\:С.~Корякина} за поддержку наших экспедиций на Белом и Баренцевом морях.
Я благодарна администрации СПбГУ, биологического факультета и кафедры ихтиологии и гидробиологии за возможность работы на Морской биологической станции СПбГУ.

На Баренцевом море мы работали вместе с сотрудниками Мурманского морского биологического института, Мурманского государственного технического университета и Полярного научно-исследовательского института морского рыбного хозяйства и океанографии: М.В.~Макаров, С.В~Малавенда, С.\:С.~Малавенда, О.~Тюкина, И.\:П.~Прокопчук, которые оказывали нам всяческую поддержку.  

Эта работа не могла бы состоятся без моих коллег по экспедициям: Беломорской экспедиции ГИПС ЛЭМБ, студенчской Баренцевоморской экспедиции СПбГУ, Беломорской экспедиции кафежры ихтиологи и гидробиологии СПбГУ. 
Отдельное спасибо руководителям экспедиций: А.\:В.~Полоскину, И.\:А.~Коршуновой, Д.\:А.~Аристову, Е.\:А. Генельт-Яновскому, М.В.~Иванову за возможность работы в экспедиционных командах и помощь в сборе материала.

Я благодарю А.\:В.~Полоскина, Д.\:А.~Аристова, К.\:В.~Шунькину, А.\:В.~Герасимову (кафедра ихтиологии и гидробиологии СПбГУ), А.\:Д.~Наумова (ББС ЗИН РАН) за предоставленные материалы.

Постоянные обсуждения с Ю.\:Ю.~Тамберг и В.\:М.~Хайтовым значительно улучшили мои навыки в статистической обработке материала и помогло мне в работе.
На этапе обработки данных неоценимую помощь идеями и разъяснениями мне оказали В.\:М. Хайтов и Д.\:А. Аристов.


Кроме того, я не могу не поблагодарить руководителей Лаборатории экологии морского бентоса И.\:А. Коршуновой, А.\:В.~Полоскину, \fbox{Е.\:А. Нинбургу} и В.\:М. Хайтову, которые 13 лет назад убедили меня, что морская биология это очень интересно и вложили много сил в мое обучение и воспитание. 
Без них меня бы тут просто не было.

И мой низкий поклон моему научному руководителю Н.\:В. Максимовичу за конструктивную помощь на всех этапах работы, жесткие споры и долгие беседы, ехидные комментарии и  неизменно доброе отношение.

\vspace{3ex}

Данная работа выполнена при частичной финансовой поддержке грантов Санкт-Петер\-бург\-ского государственного университета (1.\:0.\:134.\:2010, 1.\:42.\:527.\:2011, 1.\:42.\:282.\:2012, 1.\:38.\:253.\:2014) и Российкого фонда фундаментальных исследований (12-04-01507, 13-04-10131 К). 

%\newpage
\subsection*{\Large Содержание работы}
Во \underline{\textbf{введении}} обосновывается актуальность исследований, проводимых в рамках данной диссертационной работы, приводится обзор научной литературы по изучаемой проблеме, формулируется цель, ставятся задачи работы, сформулированы научная новизна и практическая значимость представляемой работы.

\underline{\textbf{Первая глава}} посвящена ...

 картинку можно добавить так:
\begin{figure}[h] 
  \center
  \includegraphics [scale=0.27] {latex}
  \caption{Подпись к картинке.} 
  \label{img:latex}
\end{figure}

Формулы в строку без номера добавляются так:
$$
  \lambda_{T_s} = K_x\frac{d{x}}{d{T_s}}, \qquad
  \lambda_{q_s} = K_x\frac{d{x}}{d{q_s}},
$$

\underline{\textbf{Вторая глава}} посвящена исследованию 

\underline{\textbf{Третья глава}} посвящена исследованию 

В \underline{\textbf{четвертой главе}} приведено описание 

В \underline{\textbf{заключении}} приведены основные результаты работы, которые заключаются в следующем:
\begin{enumerate}
 \item Результат номер один.
 \item Результат номер два.
 \item Результат номер три.
% и так далее, если нужно
\end{enumerate}


%\newpage
\renewcommand{\refname}{\Large Публикации автора по теме диссертации}
\nocite{*}
\bibliography{biblio}
